%\setbeamertemplate{navigation symbols}{}
%\beamertemplatenavigationsymbolsempty

\AtBeginSection{\frame{\frametitle{Einführung}\tableofcontents[currentsection]\note{~}}}

\setcounter{part}{0}

% \showfromto{5}{5}

\begin{document}

  % ------------------------------------------------------------------------------------------
  \begin{frame}
    \titlepage

    \note{
      \textbf{8:15}
      
      \par\bigskip
    }
  \end{frame}

  % ==============================================================================================
  % ==============================================================================================
  \section{Organisatorisches}

%     % ------------------------------------------------------------------------------------------
%     \begin{frame}
%       \begin{center}
%         \begin{Huge}
%           \dblu{\textbf{Herzlich willkommen!}}
%         \end{Huge}
%       \end{center}
%     \end{frame}
    
    % ------------------------------------------------------------------------------------------
    \begin{frame}
      \frametitle{Organisatorisches}

%      \begin{Itemize}
%        \item
          \Bmph{Zeit und Ort}
          \par\smallskip
          \begin{tabular}{@{}l@{~}r@{~~}l@{~~}l@{}}
            Mo. & 12--14 & MZH 6340 \\
            Mi. &  8--10 & MZH 6340
          \end{tabular}
%          \par\smallskip
          \par\bigskip
%        \item
          \Bmph{Vortragender}
          \par\smallskip
          Thomas Schneider \\
          Cartesium, Raum 1.56 \\
          Tel. (218) 64432 \\
          \texttt{ts[ÄT]cs.uni-bremen.de}
%          \par\medskip
          \par\bigskip
%        \item
          \Bmph{Position im Curriculum}
          \par\smallskip
          Informatik: Master-Ergänzung,\\
          Modul "`Spezielle Themen der Theoretischen Informatik"'
          \par\smallskip
          Mathematik: Ergänzungsfach
%      \end{Itemize}

      \note{%
      \textbf{8:16}

        \par\bigskip
        Anfangszeiten diskutieren?%
      }
    \end{frame}
          
    % ------------------------------------------------------------------------------------------
    \begin{frame}
      \frametitle{Organisatorisches}
%      \begin{Itemize}
%        \item
          \Bmph{Form}
          \par\smallskip
          K4\quad (in der Regel 3V, 1Ü)
          \par\smallskip
          Fragen und Diskussion \Emph{jederzeit erwünscht}.
%          \par\smallskip
          \par\bigskip
%        \item
          \Bmph{Voraussetzungen}
          \par\smallskip
          Grundkenntnisse aus Theoret.\ Informatik 1+2 hilfreich,\\
          aber nicht zwingend erforderlich
%          \par\smallskip
          \par\bigskip
%        \item
          \Bmph{Vorlesungsmaterial:}
          \begin{Itemize}
            \item
              Folien und Aufgabenblätter:~ \href{http://tinyurl.com/ws1819-autom}{\texttt{tinyurl.com/ws1819-autom}}
            \item
              Folien werden online gestellt, enthalten aber nicht alle Details.\\
              (Beweise, Beispiele etc.\ von der Tafel bitte mitschreiben.)
            \item
              Skript (englisch) für den Theorie-Teil der Vorlesung in Stud.IP           
            \item
              Literatur:~ wird bei jedem Kapitel bekannt gegeben
          \end{Itemize}
%      \end{Itemize}

      \note{%
        \textbf{8:18}

        \par\smallskip
        \textbf{Vorkenntnisse:}~
        natürlich endliche Aut.\ aus ThI\,1 (werden wir hier aber wiederholen)
        sowie Berechenbarkeit und Komplexität aus ThI\,2.
        
        \par\smallskip
        Wer es nicht (mehr) parat hat,
        kann es an den entspr.\ Stellen nacharbeiten.
        
        \par\medskip
        \textbf{Studiengänge erfragen:}~ Informatik? MSc/BSc? Mathematik?
Andere Studiengänge (welche)?
        \qquad
        $\leadsto$ Im Gegenzug \textbf{paar Daten zu mir}
        
        \par\medskip
        \textbf{ThI-Skript} kann ich bei Stud.IP hochladen; ist in sich abgeschlossen \\
        (und es müssen nur wenige Abschnitte durchgearbeitet werden).~ \textbf{Bedarf?}
        
        \par\smallskip
        Bitte tragt euch also in Stud.IP für die Veranstaltung ein.
        
        \par\medskip
        \textbf{AT-Skript:}~ von Meghyn, enthält den Theorie-Teil vollständig, einschl. Beweisen,
        aber vielleicht weniger/andere Beispiele
        
        \par\smallskip
        Ich fertige \textbf{möglicherweise} ein Skript mit den Tafelmitschriften an, \\
        kann es aber noch nicht versprechen.
        
        \par
      }
    \end{frame}

    % ------------------------------------------------------------------------------------------
    \begin{frame}
      \frametitle{Prüfungsmodalitäten}
      
%      \begin{Itemize}
%        \item
          \Bmph{Übungsaufgaben \& Fachgespräch:}
          \begin{Itemize}
            \item
              Übungsaufgaben ca.\ jede zweite Woche; \\
              voraussichtlich 6 Blätter, mit Zusatzaufgaben
            \item
              Werden in Gruppen (2–3 Personen) bearbeitet, abgegeben
und korrigiert – jede\_r muss mindestens einmal vorrechnen
            \item
              Aus der erreichten Gesamtpunktzahl aller Blätter
ergibt sich die vorläufige Note für diesen Kurs
            \item
              Fachgespräche am Ende des Semesters\\
              (Prüfungsleistung, Änderung der Note möglich)\\
              Voraussetzung:  insgesamt 50\,\% der Punkte in Übungsaufgaben             
          \end{Itemize}

          \par\smallskip
          oder
          \par\smallskip
%        \item
          \Bmph{Mündliche Prüfung}
          
          \par\bigskip
          \Bmph{Wiederholungsregelungen}
          auf der nächsten Folie \dots
%      \end{Itemize}
    
      \note{%
        \begin{footnotesize}
          \textbf{8:24}

          \par\medskip
          \textbf{Sagen:}~  ich biete folgende Prüfungsmodalitäten an.
          
          \par\smallskip
          „ca. jede 2. Woche“:  erst Mo., dann Mi.\ (wegen Dies Academicus 5.12.) \\
          Liste folgt; steht auch auf Homepage
  
          \par\smallskip
          Erstes Blatt bereits online — sage ich gleich was dazu.
  
          \par\smallskip
          Gruppen:  nicht 1, nicht 4. (Hilfe anbieten)
  
          \par\smallskip
          Man muss \textbf{insgesamt} 50\,\% der Punkte haben!
  
          \par\smallskip
          Übungen besuchen: auch bei mündl. Prüfung (Teil der VL)
  
          \par\smallskip
          Gegen Ende November (Hälfte 2. Teil der VL) Genaueres zum Ablauf der Prüfungsformen
          –> Entscheidungshilfe beim Anmelden für Modul$+$Prüfungsform \\
          Für \textbf{beide} Prüfungsformen ist Üben unerlässlich \\
          (Abgabe+Korrektur ohne Verwendung der Punkte ist mgl. und erwünscht!)
  
          \par\medskip
          \textbf{Fragen/Anmerkungen/Änderungswünsche zu den Prüf.modalitäten?}
          $\leadsto$ Festgelegt.      
          \par
        \end{footnotesize}
      }
      
    \end{frame}

    % ------------------------------------------------------------------------------------------
    \begin{frame}
      \frametitle{Prüfungsmodalitäten}

      \Bmph{Wiederholungsregelungen}
      \begin{itemize}
        \item
          Fachgespräch nicht bestanden?
          \par\smallskip
          $\leadsto$ 1 Wiederholungsversuch im selben Semester möglich
          \par\smallskip
        \item
          Weitere Wiederholungsversuche (wenn nötig): \\
          \par\smallskip
          mündliche Prüfung in den folgenden 4 Semestern \\
          (je 1 Versuch pro Semester)
      \end{itemize}

      \note{%
%        \begin{footnotesize}
          \textbf{8:29}

          \par\smallskip
          \textbf{Sagen:}~  ich biete folgende Wiederholungsregelungen an.
          
          \par\medskip
          Wer 2. Versuch FG nicht besteht, kann ÜA nicht als „Teilleistung“ mitnehmen.~ %\\
          Dann ist eine mP erforderlich.
          
          \par\medskip
          Vorlesung findet \textbf{nicht} regelmäßig statt
          $\leadsto$ kann im unwsl.\ Fall des Nichtbestehens 2er Versuche
          nicht pauschal anbieten, nochmal den Übungsbetrieb zu besuchen.
          
%          \par\smallskip
%          50\,\% der Punkte nicht erreicht?  Dann mündliche Prüfung bei mir.
%          Die ist aber deutlich anspruchsvoller als FG – 
%          wenn Ihr also sowieso Probleme habt, die 50\,\% zu schaffen,
%          dann verlasst Euch nicht auf die mP!
%          Ich empfehle das Lösen der ÜA, weil sie Euch am meisten beim Lernen unterstützen.
%          
          \par\medskip
          Die üblichen Krankheitsregelungen bleiben hiervon unberührt;
          meldet Euch im Falle des Falles einfach bei mir oder beim Prüfungsamt.
          
%          \par\smallskip
%          Letzter Punkt:
%          Ebenso natürlich beim Verwenden anderer Quellen:
%          immer Quelle angeben und Eigenanteil kenntlich machen! (Nur den kann ich bewerten.)
%          Idealerweise aber stammt die gesamte Lösung von Euch allein (Gruppe).
%          
          \par\medskip
          \textbf{Fragen/Anmerkungen/Änderungswünsche zu diesen Regelungen?}
          $\leadsto$ Festgelegt.
          
          \par
%        \end{footnotesize}
                
      }
    \end{frame}

    % ------------------------------------------------------------------------------------------
    \begin{frame}
    \frametitle{Termine}
    
      \Bmph{Terminübersicht Übung (geplant)}
      
      \begin{center}
        \begin{tabular}{rlll}
          \hline\rule{0pt}{12pt}%
          Blatt & Erscheinen     & Abgabe               & Besprechung,         \\
                & (geplant)      &                      & Übungstermin         \\[1pt]
          \hline\rule{0pt}{12pt}%
              1 & ist online!    & Do.\ 1.\,11.\        & Mo.\ 5.\,11.\        \\
              2 & Do.\ 1.\,11.\  & Do.\ 15.\,11.\       & Mo.\ 19.\,11.\       \\
              3 & Do.\ 15.\,11.\ & Do.\ 29.\,11.\       & Mo.\ 3.\,12.\        \\
              4 & Mo.\ 3.\,12.\  & \Emph{Mo.} 17.\,12.\ & \Emph{Mi.} 19.\,12.\ \\
              5 & Mo.\ 17.\,12.\ & Mo.\ 14.\,1.\        & Mi.\ 16.\,1.\        \\
              6 & Mo.\ 14.\,1.\  & Mo.\ 28.\,1.\        & Mi.\ 30.\,1.\        \\[1pt]
          \hline
        \end{tabular}
      \end{center}
      
      \begin{Itemize}
        \item
          Blätter erscheinen auf Homepage der Vorlesung
        \item
          Abgabe per PDF in Stud.IP (separater Ordner, bis 23:59 Uhr)
      \end{Itemize}

      \parI
      \scalebox{.96}[1]{\Bmph{Vorlesung:} Ausfall 29.\,10., 31.\,10.\ (Reformationstag), 5.\,12.\ (Dies Acad.)}
    
      \note{
        \begin{footnotesize}
          \textbf{8:32 \quad $\to$ 8:35}

          \par\smallskip
          Übungstermine und Blätter~
          (stehen auch auf Homepage der VL)
  
          \par\smallskip        
          ¡¡Erstes Blatt bereits online — abzugeben am 29.10. (Ende nächster Wo.)
          
          \par\smallskip        
          Dann noch 1x Abgabe Sonntag; danach immer schon freitags! \\
          (Ü auf Montag wg.\ Dies Academicus 22.11.). \\
          Blatt immer knapp 2 Wochen vorher verfügbar ($\pm$ 1--2 Tage)
          
          \par\smallskip        
          Blätter werden auf Homepage hochgeladen. \\
          Abgabe: elektronisch in Stud.IP bis jeweils 23:59. \\
          Ich richte Ordner ein, in den man nur schreiben, nicht lesen kann. \\
          $\leadsto$ wer hat Einwände?  (ggf. Postfach o.\,Ä.\ anbieten)
          
          \par\smallskip        
          Stud.IP:  bitte für diese Veranstaltung eintragen, sofern noch nicht geschehen!
          
          \par\smallskip
          Grund für diese Vorgehensweise:
          Ich möchte die ÜS vor der Besprechung korrigieren,
          um in der Übung auf Schwierigkeiten einzugehen und den Fokus nach Eurem Bedarf zu setzen.
          Ich habe aber viel zu tun und brauche ein paar Tage „Puffer“ zum Korrigieren.
          Kann auch noch nicht versprechen, dass das immer so klappt
          (ist in AT das 2.\ Mal, dass ich das so mache.)
          
          \par
        \end{footnotesize}
      }
    \end{frame}

  % ==============================================================================================
  % ==============================================================================================
  \section{Vorlesungsüberblick}

    % ------------------------------------------------------------------------------------------
    \begin{frame}
      \frametitle{Ursprünge der Automatentheorie}
      
      \Bmph{Automaten als Berechnungsmodelle, zur Definition formaler Sprachen}
      
      \begin{Enumerate}
        \item[\Bmph{(3)}]
          (Nicht-)deterministische endliche Automaten (NEA/DEA) \\
          \Lit{[McCulloch \& Pitts 1943;~ Kleene 1956]}
          \par\smallskip
        \item[\Bmph{(2)}]
          Kellerautomaten (pushdown automata, PDA) \\
          \Lit{[Newell, Shaw, Simon 1959]}
        \item[\Bmph{(1)}]
          Linear beschränkte Automaten (LBA) \\
          \Lit{[Myhill 1960;~ Kuroda 1964]}
        \item[\Bmph{(0)}]
          Turingmaschinen (TM) \\
          \Lit{[Turing 1936]}
      \end{Enumerate}
    
      \note{
        \textbf{8:35}

        \par\medskip
        Die kennt Ihr (hoffentlich) alle aus Theorie 1.\\
        Hier nach Chomsky-Typ geordnet, nicht historisch.
        
        \par\smallskip
        N/DEAs:~ Ideen 1943 zur Modellierung Nervensysteme, \\
        präzise definiert 1956 von Kleene
        
        \par\smallskip
        LBAs:~ determ.\ Variante von Myhill, nichtdet.\ von Kuroda
        
        \par\smallskip
        TMs:~ dazu muss ich sicherlich nichts mehr sagen \dots
      }
    \end{frame}
  
    % ------------------------------------------------------------------------------------------
    \begin{frame}
    \frametitle{Ursprünge der Automatentheorie}
    
      \Bmph{Varianten endlicher Automaten} \\
      zum Lösen von Entscheidungsproblemen
      
      \begin{Itemize}
        \item
          \Bmph{Baumautomaten} \\
          $=$ endliche Automaten auf Bäumen (statt auf Wörtern)
          \par\smallskip
          ursprünglich für Schaltkreisverifikation \\
          \Lit{[Church, 50er/60er]}
          \par\smallskip
        \item
          \Bmph{Büchi-Automaten} \\
          $=$ endliche Automaten auf unendlichen Wörtern
          \par\smallskip
          ursprünglich zum Entscheiden logischer Theorien \\
          \Lit{[Büchi 1962]}
          \par\smallskip
        \item
          \Bmph{alternierende Automaten} \\
          (Alternierung $=$ Verallgemeinerung des Nichtdeterminismus)
          \par\smallskip
          \Lit{[Chandra, Kozen, Stockmeyer 1981]}
          \par\smallskip
         \item
           und viele weitere \dots
      \end{Itemize}
    
      \note{
        \textbf{8:37}

        \par\medskip
        Hier wird der Begriff des (nichtdet.) endl.\ Automaten
        auf verschiedene Weisen erweitert
      }
    \end{frame}

    % ------------------------------------------------------------------------------------------
    \begin{frame}
      \frametitle{Moderne Anwendungen der Automatentheorie}
    
      Automaten werden in der Informatik angewendet z.\,B.\ für
      
      \begin{Itemize}
        \item
          Validierung semistrukturierter Daten (XML)
        \item
          Verifikation von Hard- und Software
        \item
          Komplexitätstheorie (Definition Komplexitätsklassen)
        \item
          Entscheidungsverfahren \\
          z.\,B.\ für Logiken (aus der KI, Verifikation und mehr)
        \item
          etc.\
      \end{Itemize}
    
      \par\medskip
      Es besteht eine enge Verbindung zwischen Automaten und Logik.
      
      \par\medskip
      Automaten haben die Entwicklung der Informatik entscheidend mitbestimmt.

      \note{
        \textbf{8:39}

        \par\medskip
      }
    \end{frame}

    % ------------------------------------------------------------------------------------------
    \begin{frame}
      \frametitle{Fallbeispiel 1: XML}
      
      \Bmph{XML-Schema und Validierung von XML-Dokumenten} \\
      können als Automatenprobleme verstanden werden:
      
      \begin{Itemize}
        \item
          XML-Dokument $\approx$ \Emph{Baum}
        \item
          XML-Schema beschreibt Menge der gültigen XML-Dokumente \\
          $\approx$ \Emph{formale Sprache} (Menge von Bäumen, i.\,d.\,R.\ unendlich)
        \item
          Formale Sprache kann man durch \Emph{endlichen Baumautomaten} beschreiben.
      \end{Itemize}
    
      \par\medskip
      Dann entspricht \dots
      
      \begin{Itemize}
        \item
          Validität eines XML-Dokuments $\hat=$ \Emph{Wortproblem}
        \item
          Konsistenz des XML-Schemas $\hat=$ \Emph{Leerheitsproblem}
        \item
          \dots
      \end{Itemize}
    
    \note{
      \textbf{8:41}

      \par\medskip
      "`$\approx$ Baum"':~
      Struktur kann als Baum aufgeschrieben werden, \\
      wenn man konkrete Datenwerte vernachlässigt
    }
    \end{frame}

    % ------------------------------------------------------------------------------------------
    \begin{frame}
      \frametitle{Fallbeispiel 2: Verifikation}
      
      \Bmph{Verifikation:}~ nachweisen, dass ein Chip/Programm eine gewünschte
      Spezifikation erfüllt (z.\,B.\ keine Division durch 0, keine Deadlocks)
      
      \par\medskip
      Manche Systeme sollen $\infty$ lange laufen (keine Terminierung): \\
      Betriebssysteme, Bankautomaten, Flugsicherungssysteme
      
      \par\medskip
      Wichtige Technik:~ \Emph{Model checking} -- oft automatenbasiert:
      
      \begin{Itemize}
        \item
          Lauf des Systems $=$ \Emph{unendliches Wort}
        \item
          System $=$ \Emph{formale Sprache} $L_1$ \\
          (Menge aller Läufe, i.\,d.\,R.\ unendlich)
        \item
          erlaubtes Verhalten $=$ \Emph{formale Sprache} $L_2$ \\
          (Menge aller erlaubten Läufe, i.\,d.\,R.\ unendlich)
        \item
          Beschreiben $L_1$ und $L_2$ durch \Emph{Büchi-Automaten} \\
          (endliche Automaten auf unendlichen Wörtern)
        \item[$\leadsto$]
          Model checking ~$\hat=$~ "`$L_1 \subseteq L_2\,?$"'
          ~$\approx$~ \Emph{Äquivalenzproblem}
      \end{Itemize}
    
      \note{
        \textbf{8:43}

        \par\medskip
        Bei diesen kritischen Systemen wäre Terminierung ein Fehler, \\
        oft sogar ein fataler.
        
        \par\medskip
        Letzte Zeile:~
        das ist eigentlich nichts anderes als das Äquivalenzproblem,
        wie wir sehen werden. \\
        (genau genommen braucht man da noch gewisse Abschlusseig.\ \dots)
      }
    \end{frame}
    
    % ------------------------------------------------------------------------------------------
    \begin{frame}
      \frametitle{Ziele der Vorlesung}
      
      \Bmph{Einführung in grundlegende Automatenbegriffe}
      \begin{Itemize}
        \item
          auf endlichen Bäumen
        \item
          auf unendlichen Wörtern
        \item
          auf unendlichen Bäumen
      \end{Itemize}
    
      \par\bigskip
      \Bmph{Untersuchung der zugehörigen Sprachklassen}
      \begin{Itemize}
        \item
          Abschlusseigenschaften, Determinisierung, Charakterisierungen, Entscheidungsprobleme
        \item
          teils einfach, teils anspruchsvoll
        \item
          interessante Techniken: Safra-Konstruktion, Paritätsspiele
      \end{Itemize}
      
      \par\bigskip
      \Bmph{Herstellung von Bezügen zu Anwendungen}
      \par\smallskip
      Einsatz dieser Automaten z.\,B.\ in XML-Validierung und Verifikation

      \note{
        \textbf{8:46}

        \par\medskip
      }
    \end{frame}

    \newlength{\teileins}
    \settowidth{\teileins}{\Bmph{Teil 1:}}
    % ------------------------------------------------------------------------------------------
    \begin{frame}
      \frametitle{Übersicht Vorlesung}
      
      Einführung \YES
      
      \par\bigskip
      \Bmph{Teil 1:}~ Endliche Automaten auf endlichen Wörtern \\
      \hspace*{\teileins}~ (Kurzwiederholung und Anwendungen, ca.\ 2 Sitzungen)
    
      \par\bigskip
      \Bmph{Teil 2:}~ Endliche Automaten auf endlichen Bäumen
      
      \par\bigskip
      \Bmph{Teil 3:}~ Endliche Automaten auf unendlichen Wörtern
      
      \par\bigskip
      \Bmph{Teil 4:}~ Endliche Automaten auf unendlichen Bäumen
      
      \note{
        \textbf{8:48\quad  $\to$ 8:50}

        \par\medskip
        \textbf{5\,min Pause;}~ Übungsgruppen zusammenfinden lassen!
      }
    \end{frame}

%    % ------------------------------------------------------------------------------------------
%    \begin{frame}
%      \frametitle{Teil 1:~ Endliche Automaten auf endlichen Wörtern}
%      
%      \Bmph{überwiegend Wiederholung (ca.\ 2 Sitzungen)}
%      \begin{Itemize}
%        \item
%          Grundbegriffe: (nicht)deterministische EA
%        \item
%          Determinisierung
%        \emphitem
%          \emph{Anwendung: Textsuche}
%          \par\bigskip
%        \item
%          Abschlusseigenschaften
%        \item
%          Reguläre Ausdrücke und der Satz von Kleene
%        \emphitem
%          \emph{Anwendungen: Patternsuche, Textersetzung}
%          \par\bigskip
%        \item
%          Satz von Myhill-Nerode
%        \item
%          Pumping-Lemma für reguläre Sprachen
%        \item
%          Entscheidungsprobleme
%      \end{Itemize}
%    \end{frame}
%
%    % ------------------------------------------------------------------------------------------
%    \begin{frame}
%      \frametitle{Teil 2:~ Endliche Automaten auf endlichen Bäumen}
%      \begin{Itemize}
%        \item
%          Grundbegriffe: Bottom-up-Baumautomaten
%        \item
%          Determinisierung
%        \emphitem
%          \emph{Anwendung: Termersetzungssysteme}
%          \par\bigskip
%        \item 
%          Pumping-Lemma für reguläre Baumsprachen
%        \item 
%          Abschlusseigenschaften regulärer Baumsprachen
%        \item
%          Top-down-Baumautomaten
%        \item
%          Entscheidungsprobleme
%          \par\bigskip
%        \emphitem
%          \emph{Anwendung: XML}
%          \begin{Itemize}
%            \emphitem
%              \emph{Vernetzung von Kenntnissen über Baumautomaten und reguläre Ausdrücke}
%          \end{Itemize}
%      \end{Itemize}
%    \end{frame}
%
%    % ------------------------------------------------------------------------------------------
%    \begin{frame}
%      \frametitle{Teil 3:~ Endliche Automaten auf unendlichen Wörtern}
%      \begin{Itemize}
%        \item
%          Grundbegriffe: Büchi-Automaten
%        \item
%          Abschlusseigenschaften
%        \item
%          Charakterisierung mittels regulärer Sprachen
%        \emphitem
%          Determinisierung
%          \begin{Itemize}
%            \item
%              Deterministische Büchi-Automaten
%            \item
%              Müller-, Rabin- und Streett-Automaten %\\
%%               (und Gleichmächtigkeit mit BA)
%            \emphitem
%              Safras Tricks
%          \end{Itemize}
%          \par\bigskip
%        \item
%          Entscheidungsprobleme
%        \emphitem
%          \emph{Anwendung: Verifikation und (Temporal-)Logik}
%      \end{Itemize}
%    \end{frame}
%
%    % ------------------------------------------------------------------------------------------
%    \begin{frame}
%      \frametitle{Teil 4:~ Endliche Automaten auf unendlichen Bäumen}
%      \begin{Itemize}
%        \item
%          Grundbegriffe
%        \item
%          Zusammenhang zwischen den Baumautomatenmodellen
%        \emphitem
%          Komplementierung: Reduktion zu Paritätsspielen
%          \par\bigskip
%        \item
%          \emph{Anwendung: Verifikation und (Temporal-)Logik}
%      \end{Itemize}
%    \end{frame}

% 
%     % ------------------------------------------------------------------------------------------
%     \begin{frame}
%       \frametitle{\dots}
% 
%     \end{frame}

  

%     % ------------------------------------------------------------------------------------------
%     \begin{frame}
%       \frametitle{Ausblick}
% 
%       \dots
%       
%       \par\bigskip
%       \uncover<2>{%
%         \begin{center}
%           \begin{Huge}
%             \dblu{\textbf{Thank you.}}
%           \end{Huge}
%         \end{center}
%       }
%     \end{frame}

%   % ==============================================================================================
%   % ==============================================================================================
%   \appendix
%   
%     % ------------------------------------------------------------------------------------------
%     \begin{frame}
%       \frametitle{\dots}
%       \dots
%     \end{frame}

\end{document}
