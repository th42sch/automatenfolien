%\setbeamertemplate{navigation symbols}{}
%\beamertemplatenavigationsymbolsempty

\setcounter{part}{1}

%%% \showfromto{5}{5}

\begin{document}

  \defverbatim{\RAi}{\begin{verbatim}    straße|str\.|weg|gasse\end{verbatim}}
  \defverbatim{\RAii}{\begin{verbatim}    [A-Z][a-z]*(straße|str\.|weg|gasse) [0-9]*\end{verbatim}}
  \defverbatim{\RAiii}{\begin{verbatim}    [A-Z][a-z]*(straße|str\.|weg|gasse)\end{verbatim}}
  \defverbatim{\RAiv}{\begin{verbatim}                    ([0-9]*[A-Za-z]?-)?[0-9]*[A-Za-z]?\end{verbatim}}

  \defverbatim{\lexbsp}{%
         \begin{verbatim}
else
  {return(ELSE);}
  
[A-Za-z][A-za-z0-9]*
  {<Trage gefundenen Bezeichner in Symboltabelle ein>;
  return(ID);}
  
>=
  {return(GE);}
  
=
  {return(EQ);}
        \end{verbatim}%
  }


  % ------------------------------------------------------------------------------------------
  \begin{frame}
    \titlepage
    \note{
%      \textbf{{\boldmath $\to$ F.\,40 (Wdhlg.\ bis einschl.\ F.\,44)}}
%            
%      \par
      \textbf{8:55}
      
      \par
    }
  \end{frame}

  % ------------------------------------------------------------------------------------------
  \begin{frame}
    \frametitle{Vorlesungsübersicht}
    
    \Bmph{Kapitel 1:~ endliche Automaten auf endlichen Wörtern}
    \par\smallskip
    Kapitel 2:~ endliche Automaten auf endlichen Bäumen
    \par\smallskip
    Kapitel 3:~ endliche Automaten auf unendlichen Wörtern
    \par\smallskip
    Kapitel 4:~ endliche Automaten auf unendlichen Bäumen
  
    \note{
      \textbf{8:55}

      \par\medskip
      Hier nochmal die 4 Kapitel \text{(kurz!)}
    }
  \end{frame}

  % ------------------------------------------------------------------------------------------
  \begin{frame}
    \frametitle{Ziel dieses Kapitels}
    
    \begin{Itemize}
      \item
        Wiederholung der Definitionen \& Resultate zu endlichen Automaten
        aus "`Theoretische Informatik 1"'
        
        \par\bigskip
      \item
        Kennenlernen zweier Anwendungen endlicher Automaten
    \end{Itemize}
  
    \note{
      \textbf{8:55}

      \par\medskip
      Das wird ein kurzer, leichter Teil.~
      Ca.\ 2 Sitzungen.
      
      \par\bigskip
      (1) Wdhlg.\ mache ich knapp; wenn Euch auf"|fällt, 
      was Ihr nicht mehr parat habt, dann schlagt es im Skript nach.
      
      \par\medskip
      (2) Die Anwendungen sind (hoffentlich) neu für Euch.
    }
  \end{frame}

  % ------------------------------------------------------------------------------------------
  \begin{frame}
    \frametitle{Überblick}
    \tableofcontents
    \note{
      \textbf{8:56}

      \par\medskip
    }
  \end{frame}

  % ==============================================================================================
  % ==============================================================================================
  \section{Grundbegriffe}
  
%%%  \mysectioncontent{%
    % ------------------------------------------------------------------------------------------
    \begin{frame}
      \frametitle{Wörter, Sprachen, \dots}
      
      \begin{Itemize}
        \item
          \Bmph{Symbole} $a,b,\dots$
        \item
          \Bmph{Alphabet} $\Sigma$: endliche nichtleere Menge von Symbolen
        \item
          \Bmph{(endliches) Wort} $w$ über $\Sigma$:\\
          endliche Folge $w = a_1a_2\dots a_n$ von Symbolen $a_i \in \Sigma$
        \item
          \Bmph{leeres Wort} $\varepsilon$
        \item
          \Bmph{Wortlänge} $|a_1a_2\dots a_n| = n$,\quad $|\varepsilon| = 0$
        \item
          \Bmph{Menge aller Wörter} über $\Sigma$:~ $\Sigma^*$
        \item
          \Bmph{Sprache} $L$ über $\Sigma$: Teilmenge $L \subseteq \Sigma^*$ von Wörtern
        \item
          \Bmph{Sprachklasse} $\mathcalreg{L}$: Menge von Sprachen
      \end{Itemize}

      \note{
      \textbf{8:56}

      \par\medskip
      }
    \end{frame}

    % ------------------------------------------------------------------------------------------
    \begin{frame}
      \frametitle{Endliche Automaten}
      
      \begin{Definition}
        Ein \Bmph{nichtdeterministischer endlicher Automat (NEA)}
        über einem Alphabet $\Sigma$ ist ein 5-$\!$Tupel
        $\Aut{A} = (Q, \Sigma, \Delta, I, F)$, wobei
        \begin{Itemize}
          \item
            $Q$ eine endliche nichtleere \Bmph{Zustandsmenge} ist,
          \item
            $\Sigma$ ein Alphabet ist,
          \item
            $\Delta \subseteq Q \times \Sigma \times Q$ die \Bmph{Überführungsrelation} ist,\qquad $(*)$
          \item
            $I \subseteq Q$ die Menge der \Bmph{Anfangszustände} ist,
          \item
            $F \subseteq Q$ die Menge der \Bmph{akzeptierenden Zustände} ist.
        \end{Itemize}
      \end{Definition}

      \begin{Itemize}
        \item<2->
          $(*)$ bedeutet: \\ 
          $\Delta$ besteht aus Tripeln $(q,a,q')$ mit $q,q'\in Q$ und $a \in \Sigma$
        \item<2->
          $(q,a,q') \in \Delta$ bedeutet intuitiv: \\
          ist $\Aut{A}$ in Zustand $q$ und liest ein $a$,
          geht er in Zustand $q'$ über.
      \end{Itemize}
      \note{%
        \textbf{8:57}

        \par\medskip
        Einziger Unterschied zur Def.\ aus Theorie 1:~
        mehrere Anfangszustände \\
        (lassen sich aber immer auf 1 reduzieren)
        
        \par\bigskip
        Außerdem steht $\Delta$ vor $I$ -- das ist aber nur Festlegungssache.
        
        \par\bigskip
        Akz.\ Zustände werden oft Endzustände genannt (en: final states $\leadsto$ $F$). \\
        Das kann aber für Verwirrung sorgen, denn die Berechnung muss beim Erreichen
        eines solchen Zustandes noch nicht enden. \\
        Deshalb benutze ich "`akz.\ Zustände"'.
      }
    \end{frame}

    % ------------------------------------------------------------------------------------------
    \begin{frame}
      \frametitle{Beispiel und graphische Repräsentation von NEAs}
      
      \begin{exampleblock}{}
        Betrachte $\Aut{A} =$\\
        \scalebox{.98}[1]{$(\{q_0,q_1\},~ \{a,b\},~ \{(q_0,a,q_0), (q_0,b,q_1), (q_1,b,q_1)\},~ \{q_0\}, \{q_1\})$}
      \end{exampleblock}

      
      \begin{Itemize}
        \item
          Zustände: $q_0, q_1$
          \hfill
          \uncover<2->{\raisebox{-2mm}{\Fig{0}}}%
          \vspace*{1.4mm}%
        \item
          Alphabet $\{a,b\}$
          \vspace*{-6mm}
        \item
          Übergänge: von $q_0$ mittels $a$ zu $q_0$, \dots
          \hfill
          \uncover<2->{\raisebox{-2mm}{\Fig{1}}}
        \item
          Anfangszustand $q_0$
          \hfill
          \uncover<2->{\raisebox{-2mm}{\Fig{2}}}%
          \vspace*{-1mm}%
        \item
          einziger akzeptierender Zustand $q_1$
          \hfill
          \uncover<2->{\raisebox{-2mm}{\Fig{3}}}
      \end{Itemize}

      \vspace*{-4mm}
      \begin{center}
        \uncover<2->{\Aut{A}: \raisebox{-3mm}{\Fig{4}}}
      \end{center}

      \note{
        \textbf{9:00}

        \par\medskip
      }
    \end{frame}

    % ------------------------------------------------------------------------------------------
    \begin{frame}
      \frametitle{Berechnungen und Akzeptanz}
      
      \begin{Definition}
        Sei $\Aut{A} = (Q,\Sigma,\Delta,I,F)$ ein NEA.
        \begin{Itemize}
          \item
            Ein \Bmph{Run} von \Aut{A} auf $w = a_1a_2\dots a_n$
            ist eine Folge
%            \vspace*{-.2\baselineskip}
            \[
              r = q_0q_1q_2\dots q_{n},
            \]
%            \vspace*{-.4\baselineskip}%
            so dass für alle $i=0,\dots,n-1$ gilt: $(q_i,a_{i+1},q_{i+1}) \in \Delta$.
            \par
            Man sagt/schreibt: $w$ \Bmph{überführt} $q_0$ in $q_n$,~ $q_0 \vdash^{w}_{\!\Aut{A}} q_n$
            \parI
          \item<2->
            Ein Run $r = q_0q_1q_2\dots q_{n}$ von \Aut{A} auf $w$ ist \Bmph{erfolgreich,} \\
            wenn $q_0 \in I$ und $q_n \in F$.
            \parI
          \item<3->
            \Aut{A} \Bmph{akzeptiert} $w$, wenn es einen erfolgreichen Run von \Aut{A} auf $w$ gibt.
            \parI
          \item<4->
            Die von \Aut{A} \Bmph{erkannte Sprache} ist
            $\ddblu{L(\Aut{A})} = \{w \in \Sigma^* \mid \text{\Aut{A} akzeptiert $w$}\}$.
        \end{Itemize}
      \end{Definition}

      \note{%
        \textbf{9:02}

        \par\medskip
        $\vdash^w_{\!{A}}$ ~war in Theorie~1~ $\stackrel{w}{\Longrightarrow}_{\!\Aut{A}}$
      }
    \end{frame}

    % ------------------------------------------------------------------------------------------
    \begin{frame}
      \frametitle{Beispiele}
      \label{frame:automatenbsp}
      
      \begin{exampleblock}{}
%         \begin{tabular}{ll}
%           \hspace*{12mm} \Aut{A}                  & $L(\Aut{A})$                                                              \\
%           \raisebox{-3mm}{\Fig{4}}                & \uncover<2->{$\{a^nb^m \mid n \geqslant 0, m \geqslant 1\}$}              \\[4mm]
%           \uncover<3->{\raisebox{-3mm}{\Fig{10}}} & \uncover<4->{$\{w \in \{a,b\}^* \mid \text{$w$ enthält Teilwort $ab$}\}$}
%         \end{tabular}
        $\Aut{A}_1$:\quad \raisebox{-3mm}{\Fig{4}} 
        \hfill
        $L(\Aut{A}_1) = \uncover<2->{\{a^nb^m \mid n \geqslant 0, m \geqslant 1\}}$

        \par\bigskip
        \uncover<3->{$\Aut{A}_2$: \raisebox{-3mm}{\Fig{10}}}
        \par\bigskip
        $\uncover<3->{L(\Aut{A}_2) = }\uncover<4->{\{w \in \{a,b\}^* \mid \text{$w$ enthält Teilwort $ab$}\}}$

        \par\bigskip
        \uncover<5->{$\Aut{A}_3$: \raisebox{-3mm}{\Fig{11}}}
        \par\bigskip
        $\uncover<5->{L(\Aut{A}_3) = }\uncover<6->{\{w \in \{a,b\}^* \mid \text{$w$ endet auf $ab$}\}}$
      \end{exampleblock}

    \note{
        \textbf{9:04}

        \par\medskip
    }
    \end{frame}

    % ------------------------------------------------------------------------------------------
    \begin{frame}
      \frametitle{Erkennbare Sprache}
      
      \begin{Definition}
        Eine Sprache $L \subseteq \Sigma^*$ ist \Bmph{(NEA-)erkennbar}, \\
        wenn es einen NEA \Aut{A} gibt mit $L = L(\Aut{A})$.
      \end{Definition}

    \note{
      \textbf{9:08}

      \par\medskip
    }
    \end{frame}

    % ------------------------------------------------------------------------------------------
    \begin{frame}
      \frametitle{Determinismus}
      
      \begin{Definition}
        Sei $\Aut{A} = (Q,\Sigma,\Delta,I,F)$ ein NEA.
        \par\smallskip
        Enthält $\Delta$ für jedes $q \in Q$ u.\ jedes $a \in \Sigma$
        \Emph{genau 1} Tripel $(q,a,q')$
        \par
        und enthält $I$ \Emph{genau 1} Zustand,
        \par\smallskip
        dann ist \Aut{A} ein \Bmph{deterministischer endlicher Automat (DEA)}.
      \end{Definition}
      
      \begin{Itemize}
        \item<2->[$\leadsto$]
          Nachfolgezustand für jedes Paar $(q,a)$ eindeutig bestimmt
          \par\bigskip
        \item<3->
          Jeder DEA ist ein NEA,\\
          aber nicht umgekehrt %\\
%           \hspace*{38.5mm} 
          {\small (z.\,B.\ $\Aut{A}_1, \Aut{A}_3$ auf Folie \ref{frame:automatenbsp})}.
        \par\smallskip
        \item<4->
          Auf Folie \ref{frame:automatenbsp} ist nur $\Aut{A}_2$ ein DEA;\\
          $\Aut{A}_1$ kann mittels \emph{Papierkorbzustand} zum DEA werden;  \Tafel \\
          bei $\Aut{A}_3$ genügt auch das nicht.
      \end{Itemize}

%       \vspace*{-1.2pt}
%       \uncover<3->{%
%         \begin{alertblock}{Frage}
%           Sind DEAs und NEAs gleichmächtig? \Tafel
%         \end{alertblock}
% % 
% %         (Was bedeutet diese Frage genau?) \Tafel%
%       }
    \note{
      \textbf{9:08}

      \par\medskip
      Tafelanschrieb: nur Bsp.\ \textbf{ganz kurz}
    }
    \end{frame}

    % ------------------------------------------------------------------------------------------
    \begin{frame}
      \frametitle{Potenzmengenkonstruktion}

%       \begin{exampleblock}{Antwort}
%         Ja, DEAs und NEAs sind gleichmächtig.
%       \end{exampleblock}

      \Emph{Frage:}
      \parbox[t]{.9\textwidth}{%
        Sind DEAs und NEAs gleichmächtig? % \\
%        {\small D.\,h.: $\{L \mid L \text{ ist DEA-erkennbar}\} = \{L \mid L \text{ ist NEA-erkennbar}\}$\,?}%
      }

      \par\medskip
      \uncover<2->{%
        \Gmph{Antwort:} Ja!
      }

      \par\bigskip
      \uncover<3->{%
        \begin{Satz}[Rabin, Scott 1959]
          Für jeden NEA \Aut{A} gibt es einen DEA $\Aut{A}^d$ mit $L(\Aut{A}^d) = L(\Aut{A})$.
          \label{thm:potenzmengenkonstruktion}%
        \end{Satz}
      }
    
      \uncover<4->{%
        \par\smallskip
        \Bmph{Beweisskizze:}~
        Sei $\Aut{A} = (Q,\Sigma,\Delta,I,F)$.
        \par\smallskip
        Wir konstruieren $\Aut{A}^d = (Q^d,\Sigma,\Delta^d,I^d,F^d)$ wie folgt.
        
        \begin{Itemize}
          \item
            $Q^d = 2^Q$\qquad (Potenzmenge der Zustandsmenge)
          \item
            $I^d = \{I\}$
          \item
            $(S,a,S') \in \Delta^d$ \quad gdw.\quad
            $S' = \{q' \mid \exists q \in S : (q,a,q') \in \Delta\}$
          \item
            $F^d = \{S \subseteq Q \mid S \cap F \neq \emptyset\}$
            \Tafel
        \end{Itemize}
      }

      \par\bigskip
      \uncover<5->{%
        Im schlimmsten Fall kann $\Aut{A}^d$ im Vergleich zu $\Aut{A}$\\
        exponentiell viele Zustände haben {\small (s.\ Hopcroft et al.\ 2001, S. 65)}.%
      }

    \note{
      \textbf{9:11}

      \par\medskip
      Klären:~ was heißt "`gleichmächtig"'? \\
      ($L$ DEA-erkennbar ~gdw.~ $L$ NEA-erkennbar. Hinrichtung trivial.)
      
      \par\bigskip
      Hier nur \textbf{kurz} die Konstruktion und ein Beispiel. \\
      Vollständiger Beweis siehe Theorie 1. \\
      (Wir werden fast dieselbe Konstruktion später an Baumautomaten genauer ausführen.)
    }
    \end{frame}
%%%  }

  % ==============================================================================================
  % ==============================================================================================
  \section[\protect\emph{Textsuche}]{\protect\emph{Anwendung: Textsuche}}

%%%  \mysectioncontent{%
    % ------------------------------------------------------------------------------------------
    \begin{frame}
      \frametitle{Stichwortsuche}
      
      \begin{exampleblock}{Typisches Problem aus dem Internetzeitalter}
        Gegeben sind \Gmph{Stichwörter $w_1,\dots,w_n \in \Sigma^*$} \\
        und \Gmph{Dokumente} $D_1,\dots,D_M \in \Sigma^*$.
        \par\smallskip
        Finde alle $j$, so dass $D_j$ mindestens ein (alle) $w_i$ als Teilwort hat.
      \end{exampleblock}
    
      \begin{Itemize}
        \item
          relevant z.B. für \Bmph{Suchmaschinen}
        \item
          übliche Technologie: \Bmph{invertierter Index}
          \par\smallskip
          speichert für jedes im Internet auftretende $w_i$ \\
          eine Liste aller Dokumente $D_j$, die $w_i$ enthalten
        \item
          invertierte Indizes sind zeitaufwändig zu erstellen\\
          und setzen voraus, dass die $D_j$ sich nur langsam ändern
      \end{Itemize}
    \note{
      \textbf{9:15}

      \par\medskip
    }
    \end{frame}


    % ------------------------------------------------------------------------------------------
    \begin{frame}
      \frametitle{Stichwortsuche ohne invertierte Indizes?}
      
      \Emph{Invertierte Indizes versagen, wenn}
      \begin{Itemize}
        \item
          die (relevanten) Dokumente sich schnell ändern:
          \begin{Itemize}
            \item
              Suche in tagesaktuellen Nachrichtenartikeln
            \item
              Einkaufshelfer sucht nach bestimmten Artikeln \\
              in aktuellen Seiten von Online-Shops
          \end{Itemize}
        \item
          die Dokumente nicht katalogisiert werden können:
          \begin{Itemize}
            \item
              Online-Shops wie Amazon generieren oft Seiten für ihre Artikel
              nur auf Anfragen hin.
          \end{Itemize}
      \end{Itemize}
      %
      $\leadsto$ Wie kann man dennoch Stichwortsuche implementieren?

    \note{
      \textbf{9:17}

      \par\medskip
    }
    \end{frame}


    % ------------------------------------------------------------------------------------------
    \begin{frame}
      \frametitle{Ein Fall für endliche Automaten!}
      
%      \begin{exampleblock}{Wdhlg.: Typisches Problem aus dem Internetzeitalter}
      \begin{exampleblock}{}
        Gegeben sind \Gmph{Stichwörter $w_1,\dots,w_n \in \Sigma^*$} \\
        und \Gmph{Dokumente} $D_1,\dots,D_M \in \Sigma^*$.
        \par\smallskip
        Finde alle $j$, so dass $D_j$ mindestens ein $w_i$ als Teilwort hat.
      \end{exampleblock}
      
      \bigskip
      \Emph{Ziel:} konstruiere NEA \Aut{A}, der
      \begin{Itemize}
        \item
          ein $D_j$ zeichenweise liest und
        \item
          in einen Endzustand geht ~gdw.~ er eins der $w_i$ findet
      \end{Itemize}
    
      Der Einfachheit halber legen wir fest, dass $\Aut{A}$ ein Wort $w$ akzeptiert, \\
      \scalebox{.95}[1]{wenn $\Aut{A}$ bereits nach Lesen eines \emph{Teilworts} einen akz.\ Zustand erreicht.}
      
      \medskip
      \uncover<2->{%
        \label{frame:web_ebay_example}%
        \begin{Beispiel}
%          Seien 
          $w_1 = \texttt{web}$ und $w_2 = \texttt{ebay}$
%          \par\smallskip
%          Wie muss \Aut{A} aussehen?
        \end{Beispiel}%
      
        \Tafel
      }


    \note{
      \textbf{9:18}

      \par\medskip
    }
    \end{frame}


    % ------------------------------------------------------------------------------------------
    \begin{frame}
      \frametitle{Implementation des NEAs \Aut{A}}
      
      \uncover<1->{%
        \Bmph{Eine Möglichkeit:}
        \begin{Enumerate}
          \item
            Determinisierung (Potenzmengenkonstruktion)
          \item
            Simulation des resultierenden DEA $\Aut{A}^d$
        \end{Enumerate}

        \par\bigskip
        \Emph{Wird $\Aut{A}^d$ nicht zu groß?}\\
        {\small ($2^{27} > 134$\,Mio.\ Zustände bei Stichw.\ "`Binomialkoeffizient"', "`Polynom"')}
      }
      
      \par\bigskip
      \uncover<2->{%
        \label{frame:web_ebay_example_2}%
        \Gmph{Nein,}
        \begin{Itemize}
          \item
            mit der leicht geänderten Definition von Akzeptanz
          \item
            und unserer Variante der Potenzmengenkonstruktion
        \end{Itemize}
        wird $\Aut{A}^d$ genauso viele Zustände haben wie $\Aut{A}$\,!
        
%        \par\bigskip
%        Beispiel: siehe Tafel 
        \Tafel
      }
    \note{
      \textbf{9:22 bis 9:30}

      \par\medskip
    }
    \end{frame}

%    % ------------------------------------------------------------------------------------------
%    \begin{frame}
%      \frametitle{Zum Nachdenken}
%      
%      \Bmph{Übung:}
%      \begin{Itemize}
%        \item
%          Konstruiere den DEA $\Aut{A}^d$ für \Aut{A} von Folie \ref{frame:web_ebay_example}.
%        \item
%          Beschreibe die Konstruktion von $\Aut{A}^d$ allgemein, \\
%          wenn $w_1,\dots,w_n$ gegeben sind, \\
%          mit $w_i = a_{i1}\dots a_{i\ell_i}$ für jedes $i=1,\dots,n$.
%      \end{Itemize}
%    \note{~}
%    \end{frame}
%%%  }

  % ==============================================================================================
  % ==============================================================================================
  \section[Abschlusseig.]{Abschlusseigenschaften}

  \newlength{\leftbox}
  \settowidth{\leftbox}{\textbf{Konkatenation}}

%%%  \mysectioncontent{%
    % ------------------------------------------------------------------------------------------
    \begin{frame}
      \frametitle{Operationen auf Sprachen sind Operationen auf Mengen}
      
      \begin{alertblock}{}
        Wie können (NEA-erkennbare) Sprachen kombiniert werden?
      \end{alertblock}

      \begin{Itemize}
        \item<2->
          Boolesche Operationen
          \begin{Itemize}
            \item[]
              \parbox{\leftbox}{\Bmph{Vereinigung}}
              $L_1 \cup L_2 = \{w \mid w \in L_1 \text{~oder~} w \in L_2\}$
%               \par\medskip
            \item[]
              \parbox{\leftbox}{\Bmph{Schnitt}}
              $L_1 \cap L_2 = \{w \mid w \in L_1 \text{~und~} w \in L_2\}$
%               \par\medskip
            \item[]
              \parbox{\leftbox}{\Bmph{Komplement}}
              $\overline{L} = \{w \in \Sigma^* \mid w \notin L\}$
          \end{Itemize}
          \par\medskip
        \item<3->
          Wortoperationen
          \begin{Itemize}
            \item[]
              \parbox{\leftbox}{\Bmph{Konkatenation}}
              $L_1 \cdot L_2 = \{vw \mid v \in L_1 \text{~und~} w \in L_2\}$
%               \par\medskip
            \item[]
              \parbox{\leftbox}{\Bmph{Kleene-Hülle}}
              $L^* = \displaystyle\bigcup_{i \geqslant 0} L^i$,
              \par\smallskip
              \hspace*{\fill}%
              wobei
              $L^0 = \{\varepsilon\}$ und $L^{i+1} = L^i \cdot L$ für alle $i \geqslant 0$
          \end{Itemize}
      \end{Itemize}

      \note{
        \uz{9:30}
        
        \par
      }
    \end{frame}


    % ------------------------------------------------------------------------------------------
    \begin{frame}
      \frametitle{Abgeschlossenheit}
      
      Die Menge der erkennbaren Sprachen heißt
      \Bmph{abgeschlossen unter} \dots
      %
      \begin{Itemize}
        \item
          \Bmph{Vereinigung}, falls gilt:
          \par\smallskip
          Falls $L_1,L_2$ erkennbar, so auch $L_1 \cup L_2$.
        \item
          \Bmph{Komplement}, falls gilt:
          \par\smallskip
          Falls $L$ erkennbar, so auch $\overline{L}$.
        \item
          \Bmph{Schnitt}, falls gilt:
          \par\smallskip
          Falls $L_1,L_2$ erkennbar, so auch $L_1 \cap L_2$.
        \item
          \Bmph{Konkatenation}, falls gilt:
          \par\smallskip
          Falls $L_1,L_2$ erkennbar, so auch $L_1 \cdot L_2$.
        \item
          \Bmph{Kleene-Stern}, falls gilt:
          \par\smallskip
          Falls $L$ erkennbar, so auch $L^*$.
      \end{Itemize}

      \uncover<2->{%      
        \Emph{Unter welchen Op.\ sind die NEA-erkennbaren Sprachen abgeschlossen?}
      }

      \note{
        \uz{9:32}
  
        \par\medskip
        \textbf{Fragen:}~ Wer weiß es noch?
          
        \par\medskip
        Gemeinsam durchgehen \& rekapitulieren:
        %
        \begin{itemize}
          \item
            Vereinigungsautomat
          \item
            Produktautomat
          \item
            Det.+Vertauschen aZ/nicht-aZ
          \item
            Hintereinanderhängen
          \item
            Schleife aZ$\to$AZ
        \end{itemize}
        
        \par
      }
    \end{frame}


    % ------------------------------------------------------------------------------------------
    \begin{frame}
      \frametitle{Abgeschlossenheit}
      
      \begin{Satz}
        Die Menge der NEA-erkennbaren Sprachen ist abgeschlossen unter den Operationen
        $\cup,\cap,\overline{\phantom{o}},\cdot,{}^\ast$.
%        \par\smallskip
%        Das heißt:
%        \par\smallskip
%        Wenn $L,L_1,L_2$ NEA-erkennbar sind, dann sind auch 
%        \begin{Itemize}
%          \item
%            $L_1 \cup L_2$
%          \item
%            $L_1 \cap L_2$
%          \item
%            $\overline{L}$
%          \item
%            $L_1 \cdot L_2$
%          \item
%            $L^*$
%        \end{Itemize}
%        NEA-erkennbar.
        \label{thm:abgeschlossenheit}
      \end{Satz}

      \par\medskip
      \Bmph{Beweis:}
      Siehe ThI\,1.%

      \note{
        \uz{9:36}
        
        \par
      }
    \end{frame}

%    % ------------------------------------------------------------------------------------------
%    \begin{frame}[t]
%      \frametitle{Abgeschlossenheit unter Vereinigung}
%      
%      \begin{Lemma}
%        Seien $\Aut{A}_1,\Aut{A}_2$ NEAs über $\Sigma$.\\
%        Dann gibt es einen NEA $\Aut{A}_3$ mit $L(\Aut{A}_3) = L(\Aut{A}_1) \cup L(\Aut{A}_2)$.
%        \label{lem:abgeschlossenheit_vereinigung}%
%      \end{Lemma}
%      
%      \par\bigskip
%      \uncover<2->{
%        \Bmph{Beweis:}~
%        Seien $\Aut{A}_i = (Q_i, \Sigma, \Delta_i, I_i, F_i)$ für $i=1,2$.
%        \par
%        O.\,B.\,d.\,A.\ gelte $Q_1 \cap Q_2 = \emptyset$.
%        \par\smallskip
%        Konstruieren $\Aut{A}_3 = (Q_3, \Sigma, \Delta_3, I_3, F_3)$ wie folgt.
%        \uncover<3->{%
%          \begin{Itemize}
%            \emphitem
%              \emph{Idee: vereinige $\Aut{A}_1$ und $\Aut{A}_2$.}
%            \item
%              $Q_3 = Q_1 \cup Q_2$
%            \item
%              $\Delta_3 = \Delta_1 \cup \Delta_2$
%            \item
%              $I_3 = I_1 \cup I_2$
%            \item
%              $F_3 = F_1 \cup F_2$ \hfill (Beispiel siehe Tafel) \Tafel~~~~~
%          \end{Itemize}
%          \par\smallskip
%          Dann gilt $L(\Aut{A}_3) = L(\Aut{A}_1) \cup L(\Aut{A}_2)$.\quad (Tafel) \Tafel~~~~
%          \par\vspace*{-\baselineskip}
%          \qed
%          \par\vspace*{-7pt}~
%        }
%      }
%
%    \note{~}
%    \end{frame}
%
%%     % ------------------------------------------------------------------------------------------
%%     \begin{frame}[t]
%%       \frametitle{Abgeschlossenheit unter Schnitt}
%%       
%%       \begin{Lemma}
%%         Seien $\Aut{A}_1,\Aut{A}_2$ NEAs über $\Sigma$.\\
%%         Dann gibt es einen NEA $\Aut{A}_3$ mit $L(\Aut{A}_3) = L(\Aut{A}_1) \cap L(\Aut{A}_2)$.
%%       \end{Lemma}
%%       
%%       \par\bigskip
%%       \uncover<2->{
%%         \Bmph{Beweis:}~
%%         Seien $\Aut{A}_i = (Q_i, \Sigma, \Delta_i, I_i, F_i)$ für $i=1,2$.
%%         \par\smallskip
%%         Konstruieren $\Aut{A}_3 = (Q_3, \Sigma, \Delta_3, I_3, F_3)$ wie folgt.
%%         \begin{Itemize}
%%           \emphitem
%%             \emph{Idee: lasse $\Aut{A}_1$ und $\Aut{A}_2$ "`gleichzeitig"' auf Eingabewort laufen.}
%%           \item
%%             $Q_3 = Q_1 \times Q_2$
%%           \item
%%             $\Delta_3 = \{((p,p'),a,(q,q')) \mid (p,a,q) \!\in\! \Delta_1 ~\&~ (p',a,q') \!\in\! \Delta_2\}$
%%           \item
%%             $I_3 = I_1 \times I_2$
%%           \item
%%             $F_3 = F_1 \times F_2$ \hfill (Beispiel siehe Tafel) \Tafel
%%         \end{Itemize}
%%         \par\smallskip
%%         Dann gilt $L(\Aut{A}_3) = L(\Aut{A}_1) \cap L(\Aut{A}_2)$.\quad (Übung)\qed%
%%       }
%% 
%%     \end{frame}
%
%    % ------------------------------------------------------------------------------------------
%    \begin{frame}[t]
%      \frametitle{Abgeschlossenheit unter Komplement}
%      
%      \begin{Lemma}
%        Sei $\Aut{A}$ ein NEA über $\Sigma$.\\
%        Dann gibt es einen NEA $\Aut{A}^c$ mit $L(\Aut{A}^c) = \overline{L(\Aut{A})}$.
%      \end{Lemma}
%      
%      \par\bigskip
%      \uncover<2->{
%        \Bmph{Beweis:}
%        \begin{Itemize}
%          \emphitem
%            \emph{Idee:} Vertausche End- und Nicht-Endzustände (im DEA!)
%        \end{Itemize}
%
%        \par\smallskip
%        O.\,B.\,d.\,A.\ sei $\Aut{A} = (Q, \Sigma, \Delta, I, F)$ ein DEA (Satz \ref{thm:potenzmengenkonstruktion}).
%        \par\smallskip
%        Dann erkennt $\Aut{A} = (Q, \Sigma, \Delta, I, Q\setminus F)$ die Sprache $\overline{L(\Aut{A})}$.\qed%
%      }
%
%%       \par\bigskip
%%       \uncover<3->{%
%%         \centerline{$
%%           \left(
%%             \begin{array}{@{}l@{}}
%%               \text{Jetzt erhält man Abgeschlossenheit unter $\cap$ auch mittels} \\
%%               L_1 \cap L_2 = \overline{\overline{L_1} \cup \overline{L_2}}).
%%             \end{array}
%%           \right)
%%         $}
%%       }
%
%
%      \par\bigskip
%      \uncover<3->{%
%        \begin{Folgerung}
%          Seien $\Aut{A}_1,\Aut{A}_2$ NEAs über $\Sigma$.\\
%          Dann gibt es einen NEA $\Aut{A}_3$ mit $L(\Aut{A}_3) = L(\Aut{A}_1) \cap L(\Aut{A}_2)$.
%        \end{Folgerung}
%
%        \par\smallskip
%        Gilt wegen $L_1 \cap L_2 = \overline{\overline{L_1} \cup \overline{L_2}}$.
%      }
%
%    \note{~}
%    \end{frame}
%
%    % ------------------------------------------------------------------------------------------
%    \begin{frame}[t]
%      \frametitle{Abgeschlossenheit unter Verkettung}
%      
%      \begin{Lemma}
%        Seien $\Aut{A}_1,\Aut{A}_2$ NEAs über $\Sigma$.\\
%        Dann gibt es einen NEA $\Aut{A}_3$ mit $L(\Aut{A}_3) = L(\Aut{A}_1) \cdot L(\Aut{A}_2)$.
%        \label{lem:abgeschlossenheit_verkettung}%
%      \end{Lemma}
%      
%      \par\bigskip
%      \uncover<2->{
%        \Bmph{Beweis:}~
%        Seien $\Aut{A}_i = (Q_i, \Sigma, \Delta_i, I_i, F_i)$ für $i=1,2$.
%        \par
%        O.\,B.\,d.\,A.\ gelte $Q_1 \cap Q_2 = \emptyset$ und $I_i = \{q_{0i}\}$.
%        \par\smallskip
%        Konstruieren $\Aut{A}_3 = (Q_3, \Sigma, \Delta_3, I_3, F_3)$ wie folgt.
%        \begin{Itemize}
%          \emphitem
%            \emph{Idee:} "`Hintereinanderhängen"' von $\Aut{A}_1$ und $\Aut{A}_2$. \Tafel
%%           \item
%%             $Q_3 = Q_1 \cup Q_2$
%%           \item
%%             $\Delta_3 = \Delta_1 \cup \Delta_2 \cup \{(q,a,q') \mid q \in F_1 \text{~und~} (q_{02},a,q') \in \Delta_2\}$
%%           \item
%%             $I_3 = I_1$,\quad
%% %           \item
%%             $F_3 = 
%%             \begin{cases}
%%               F_2 \cup F_1 & \text{falls~} q_{02} \in F_2 \\
%%               F_2          & \text{sonst} 
%%             \end{cases}$
%%             \hfill (Bsp.\ s.\ Tafel) \Tafel
%        \end{Itemize}
%        \par\smallskip
%%         Dann gilt $L(\Aut{A}_3) = L(\Aut{A}_1) \cdot L(\Aut{A}_2)$.\quad (Übung)\qed%
%        (Details siehe ThI1 -- nicht wichtig für diese Vorlesung)\qed%
%      }
%
%    \note{~}
%    \end{frame}
%
%    % ------------------------------------------------------------------------------------------
%    \begin{frame}[t]
%      \frametitle{Abgeschlossenheit unter Kleene-Hülle}
%      
%      \begin{Lemma}
%        Sei $\Aut{A}$ ein NEA über $\Sigma$.\\
%        Dann gibt es einen NEA $\Aut{A}^k$ mit $L(\Aut{A}^k) = {L(\Aut{A})}^*$.
%        \label{lem:abgeschlossenheit_kleene}%
%      \end{Lemma}
%      
%      \par\bigskip
%      \uncover<2->{
%        \Bmph{Beweis:}~
%        Sei $\Aut{A} = (Q, \Sigma, \Delta, I, F)$.
%        Konstruieren $\Aut{A}^k = (Q^k, \Sigma, \Delta^k, I^k, F^k)$ wie folgt.
%        \begin{Itemize}
%          \emphitem
%            \emph{Idee:} Lege "`Schleife"' um $\Aut{A}$;\\
%            Löse $\varepsilon$-Kanten auf wie im $\cdot$\,-Fall \Tafel
%%           \item
%%             $Q^k = Q \cup \{q_0\}$\quad (für ein $q_0 \notin Q$)
%%           \item
%%             $\Delta^k = \Delta \cup \{(q_0,\varepsilon,q) \mid q \in I\}
%%                                \cup \{(q,\varepsilon,q_0) \mid q \in F\}$
%%             \par\smallskip
%%             \hspace*{20mm}{\small ($\varepsilon$-Kanten werden aufgelöst wie im $\cdot$\,-Fall.)}
%%           \item
%%             $I^k = F^k = \{q_0\}$ \hfill (Beispiel siehe Tafel) \Tafel
%        \end{Itemize}
%        \par\smallskip
%%         Dann gilt $L(\Aut{A}^k) = L(\Aut{A})^*$.\quad (Übung)\qed%
%        (Details siehe ThI1 -- nicht wichtig für diese Vorlesung)\qed%
%      }
%
%    \note{~}
%    \end{frame}
%%%  }

  % ==============================================================================================
  % ==============================================================================================
  \section[Reguläre Ausdrücke]{Reguläre Ausdrücke und \protect\emph{Anwendungen}}

%%%  \mysectioncontent{%
    % ------------------------------------------------------------------------------------------
    \begin{frame}
      \frametitle{Reguläre Ausdrücke und Anwendungen}
      
      Reguläre Ausdrücke sind \dots
      
      \begin{Itemize}
        \item
          bequeme Charakterisierung NEA-erkennbarer Sprachen
        \item
          besonders praktisch für Anwendungen
      \end{Itemize}
      
      \note{
        \uz{9:37}
  
        \par
      }
    \end{frame}

    % ------------------------------------------------------------------------------------------
    \begin{frame}
      \frametitle{Reguläre Sprachen}
      
      \begin{Definition}
        Eine Sprache $L \subseteq \Sigma^*$ ist \Bmph{regulär}, falls gilt:
        \begin{Itemize}
          \item
            $L=\emptyset$\quad oder
          \item
            $L=\{\varepsilon\}$\quad oder
          \item
            $L=\{a\}$, $a \in \Sigma$,\quad oder
          \item
            $L$ lässt sich durch (endlichmaliges) Anwenden der Operatoren
            $\cup,\cdot,{}^*$ aus den vorangehenden Fällen konstruieren.
            \label{def:reg_sprache}%
        \end{Itemize}
        
      \end{Definition}

      \par\bigskip
      \uncover<2->{%
        \Gmph{Beispiele:}
        \begin{Itemize}
          \item[]
            $(\{a\} \cup \{b\})^* \cdot \{a\} \cdot \{b\}$\quad (siehe $\Aut{A}_3$ auf Folie \ref{frame:automatenbsp})
          \item[]
            $\{b\}^* \cdot \{a\} \cdot \{a\}^* \cdot \{b\} \cdot (\{a\} \cup \{b\})^*$\quad (s. $\Aut{A}_2$ auf Folie \ref{frame:automatenbsp})
        \end{Itemize}
      }

      \note{
        \uz{9:37}
  
        \par
      }
    \end{frame}

    \settowidth{\leftbox}{$r = a$, für $a \in \Sigma$,}
    % ------------------------------------------------------------------------------------------
    \begin{frame}
      \frametitle{Reguläre Ausdrücke}
      
      \begin{Definition}%
        Ein \Bmph{regulärer Ausdruck (RA)} $r$ über $\Sigma$
        und die \Bmph{zugehörige Sprache} $\Bmph{$L(r)$} \subseteq \Sigma^*$
        werden induktiv wie folgt definiert.
        \begin{Itemize}
          \item
            \parbox{\leftbox}{$r = \emptyset$}
            ~ist ein RA mit~ $L(r) = \emptyset$
          \item
            \parbox{\leftbox}{$r = \varepsilon$}
            ~ist ein RA mit~ $L(r) = \{\varepsilon\}$
          \item
            \parbox{\leftbox}{$r = a$, für $a \in \Sigma$,}
            ~ist ein RA mit~ $L(r) = \{a\}$
          \item
            \parbox{\leftbox}{$r = (r_1 + r_2)$}
            ~ist ein RA mit~ $L(r) = L(r_1) \cup L(r_2)$
          \item
            \parbox{\leftbox}{$r = (r_1r_2)$}
            ~ist ein RA mit~ $L(r) = L(r_1) \cdot L(r_2)$
          \item
            \parbox{\leftbox}{$r = (r_1)^*$}
            ~ist ein RA mit~ $L(r^*) = (L(r))^*$
            \label{def:reg_ausdruck}%
        \end{Itemize}
      \end{Definition}

      \par\bigskip
      \uncover<2->{%
        \Gmph{Beispiele:}\quad (wir lassen Klammern weg soweit eindeutig)
        \begin{Itemize}
          \item[]
            $(a+b)^*ab$\quad (siehe $\Aut{A}_3$ auf Folie \ref{frame:automatenbsp})
          \item[]
            $b^*aa^*b(a+b)^*$\quad (siehe $\Aut{A}_2$ auf Folie \ref{frame:automatenbsp})
        \end{Itemize}%
      }

      \note{
        \uz{9:39}
  
        \par
      }
    \end{frame}

    % ------------------------------------------------------------------------------------------
    \begin{frame}
      \frametitle{Reguläre und NEA-erkennbare Sprachen}
      
      \begin{Satz}[Kleene 1956]
        Sei $L \subseteq \Sigma^*$ eine Sprache.
        \begin{Enumerate}
          \item
            $L$ ist regulär gdw.\ es einen RA $r$ gibt mit $L=L(r)$.
          \item
            $L$ ist regulär gdw.\ $L$ NEA-erkennbar ist.
        \end{Enumerate}
      \end{Satz}
      
      \par\bigskip
      \uncover<2->{%
        \Bmph{Beweis.}
        \begin{Enumerate}
          \item
            Folgt offensichtlich aus Def.\ \ref{def:reg_sprache}, \ref{def:reg_ausdruck}.
          \item
            Benutze Punkt 1.
            \begin{Itemize}
              \item["`$\Rightarrow$"':]
                Induktion über Aufbau von $r$.
                \par\smallskip
                IA: gib Automaten an, die $\emptyset, \{\varepsilon\}, \{a\}$ erkennen.
                \par
                IS: benutze Abschlusseigenschaften
%                -- Lemmas \ref{lem:abgeschlossenheit_vereinigung}, \ref{lem:abgeschlossenheit_verkettung}, \ref{lem:abgeschlossenheit_kleene}
                (Satz~\ref{thm:abgeschlossenheit})
              \item["`$\Leftarrow$"':]
%                 siehe Tafel.\Tafel
                siehe Theoretische Informatik 1.
                \qed%
            \end{Itemize}
        \end{Enumerate}
      }

      \note{
        \uz{9:41}
  
        \par
      }
    \end{frame}

    % ------------------------------------------------------------------------------------------
    \begin{frame}
      \frametitle{Anwendungen regulärer Ausdrücke}
      
      \begin{Itemize}
        \item
          RAs werden verwendet, um "`Muster"' von zu suchendem Text zu beschreiben.
          \begin{Itemize}
            \item[]
              z.\,B.: suche alle Vorkommen von "`PLZ Ort"':\\
              $(0+\dots+9)^5\texttt{\textvisiblespace}(A+\dots+Z)(a+\dots+z)^*$
          \end{Itemize}
          \par\medskip
        \item
          Programme zum Suchen von Mustern im Text\\
          übersetzen RAs in NEAs/DEAs und simulieren diese.  
          \par\medskip
        \item
          wichtige Klassen von Anwendungen:\\
          \Bmph{lexikalische Analyse}, \Bmph{Textsuche}
      \end{Itemize}

      \note{
        \uzz{8:30}{9:43 bis 9:44, 1\,min Reserve}
        
        $\uparrow$ Zeitplanung ist Quatsch; die Sitzung war von 12:15 bis 13:45.
        
        \parIII
        \textbf{Ankündigen:}~ Terminfindung (Mo. 12--14 klappt bei 2 TN nicht). In Pause.
  
        \par
      }
    \end{frame}

    % ------------------------------------------------------------------------------------------
    \begin{frame}
      \frametitle{Komfortablere Syntax regulärer Ausdrücke}
      
      \begin{Itemize}
        \item
          UNIX und andere Anwendungen erweitern Syntax von RAs
        \item
          Hier: nur "`syntaktischer Zucker"' -- die Erweiterungen,\\
          die nicht aus den regulären Sprachen herausführen
      \end{Itemize}
      
      \par\bigskip
      \begin{Itemize}
        \item<2->
          Alphabet $\Sigma$: alle ASCII-Zeichen
        \item<2->
          RA \texttt{.} mit $L(\texttt{.}) = \Sigma$
        \item<2->
          RA \texttt{[}$a_1a_2\dots a_k$\texttt{]}, Abkürzung für $a_1 + a_2 + \dots + a_k$
        \item<2->
          RAs für Bereiche:
          z.\,B. \texttt{[a-z0-9]}, Abkü.\ für \texttt{[ab$\dots$z01$\dots$9]}
        \item<2->
          Operator \texttt{|} anstelle $+$
        \item<2->
          Operator \texttt{?}: $r\texttt{?}$ steht für $\varepsilon + r$
        \item<2->
          Operator \texttt{+}: $r\texttt{+}$ steht für $rr^*$
        \item<2->
          Operator $\{n\}$: $r\{5\}$ steht für $rrrrr$
        \item<2->
          Klammern und $*$ wie gehabt
      \end{Itemize}
      
      \vspace*{-3\baselineskip}%
      \hspace*{.655\textwidth}%
      \begin{minipage}{.345\textwidth}
        \begin{exampleblock}<3->{}
          \begin{small}
            PLZ-Ort-Beispiel: \\
            \texttt{[0-9]$\{$5$\}$\textvisiblespace[A-Z][a-z]*}
            \par
          \end{small}
         \end{exampleblock}
     \end{minipage}
      \note{
        \uz{8:32}

        \par
      }
    \end{frame}

    % ------------------------------------------------------------------------------------------
    \begin{frame}
      \frametitle{Anwendung: lexikalische Analyse}
      
      \begin{Itemize}
        \item
          \Bmph{Lexer} (auch: Tokenizer) durchsucht Quellcode nach \Bmph{Token}:
          \par%\smallskip
          {\small zusammengehörende Zeichenfolgen, z.\,B.\ Kennwörter, Bezeichner}
          \par\smallskip
        \item
          Ausgabe des Lexers: Token-Liste,\\
          wird an Parser weitergegeben
          \par\smallskip
        \item
          Mit RAs: Lexer leicht programmier- und modifizierbar
          \par\smallskip
        \item<2->
          UNIX-Kommandos \texttt{lex} und \texttt{flex} generieren Lexer
          \begin{Itemize}
            \item
              Eingabe: Liste von Einträgen RA\,$+$\,Code
            \item
              Code beschreibt Ausgabe des Lexers für das jeweilige Token
            \item
              generierter Lexer wandelt alle RAs in \Emph{einen DEA} um,\\
              um Vorkommen der Tokens zu finden (siehe Folie \ref{frame:web_ebay_example})
            \item
              anhand des Zustands des DEAs lässt sich bestimmen,\\
              \emph{welches} Token gefunden wurde
          \end{Itemize}
      \end{Itemize}

      \note{
        \uz{8:35}
        
        \par\medskip
        \textbf{Lexer:~}~ kurz für "`lexikalischer Scanner"', auch Tokenizer
        
        \par\medskip
        \textbf{Lex:}~ a computer program that generates lexical analyzers"' [Wikipedia]
        
        \par\medskip
        \textbf{Flex:}~ ``fast lexical analyzer generator'', ``a free and open-source software alternative to lex'' [Wikipedia]

        \par
      }
    \end{frame}

    % ------------------------------------------------------------------------------------------
    \begin{frame}[containsverbatim]
      \frametitle{Beispieleingabe für \texttt{lex}}
      
      \begin{exampleblock}{}
        \vspace*{-4mm}
        \lexbsp
        \vspace*{-8mm}
      \end{exampleblock}

      (Lexer-Generator muss Prioritäten beachten:\\
      \texttt{else} wird auch vom 2. RA erkannt, ist aber reserviert)

      \note{
        \uz{8:37}

        \par\medskip
        \textbf{ZF:}~ Der Lexer-Generator dient dazu, Lexer zu erzeugen. \\
        Diese Tabelle hier gibt an, wie das passiert. \\
        Der Vorteil ist die leichte Änderbarkeit, wenn sich mal 
        ein Token oder dessen Beschreibung (RA!) ändert.
        
        \par
      }
    \end{frame}

    % ------------------------------------------------------------------------------------------
    \begin{frame}
      \frametitle{Anwendung: Finden von Mustern im Text}
      
      \Gmph{Beispiel:} Suchen von Adressen (Str.\ $+$ Hausnr.) in Webseiten
      \par\smallskip
      Solche Angaben sollen gefunden werden:
%       \begin{center}
%         \begin{ttfamily}
%           Parkstraße 5\\
%           Enrique-Schmdt-Straße 12a \\
%           Breitenweg 12--14 \\
%           Knochenhauergasse 42
%         \end{ttfamily}
%       \end{center}
      \begin{itemize}
        \item[]
          \texttt{Parkstraße 5}
          \par\vspace{-1mm}
        \item[]
          \texttt{Enrique-Schmidt-Straße 12a}
          \par\vspace{-1mm}
        \item[]
          \texttt{Breitenweg 24A}
          \par\vspace{-1mm}
        \item[]
          \texttt{Knochenhauergasse 30--32}
      \end{itemize}
      \uncover<2->{%
        aber auch solche:
  %       \begin{center}
  %         \begin{ttfamily}
  %           Straße des 17.\ Juni 17\\
  %           ...boulevard, ...allee, ... \\
  %           Postfach 330 440 \\
  %           Am Wall 8
  %         \end{ttfamily}
  %       \end{center}
        \begin{itemize}
          \item[]
            \texttt{Straße des 17.\ Juni 17}
            \par\vspace{-1mm}
          \item[]
            \dots\texttt{boulevard}, \dots\texttt{allee}, \dots\texttt{platz}, \dots
            \par\vspace{-1mm}
          \item[]
            \texttt{Postfach 330 440}
            \par\vspace{-1mm}
          \item[]
            \texttt{Am Wall 8}
        \end{itemize}%
      }
      \par\bigskip
      \uncover<3->{%
        \begin{Itemize}
          \item[$\leadsto$]
            Ausmaß der Variationen erst während der Suche deutlich
          \item[$\leadsto$]
            Gesucht: einfach modifizierbare Beschreibung der Muster
        \end{Itemize}%
      }
      \note{
        \uz{8:40}
  
        \par
      }
    \end{frame}

    % ------------------------------------------------------------------------------------------
    \begin{frame}
      \frametitle{Mustersuche mit regulären Ausdrücken}
      
      Mögliches Vorgehen:
      \begin{Enumerate}
        \item[\Bmph{(1)}]
          Beschreibung des Musters mit einem einfachen RA
        \item[\Bmph{(2)}]
          Umwandlung des RA in einen NEA
        \item[\Bmph{(3)}]
          Implementation des DEA wie auf Folie \ref{frame:web_ebay_example}+\ref{frame:web_ebay_example_2}
        \item[\Bmph{(4)}]
          Test
        \item[\Bmph{(5)}]
          Wenn nötig, RA erweitern/ändern und Sprung zu Schritt 2
      \end{Enumerate}

      \note{
        \uz{8:42}

        \par\medskip
        Dies ist ein ganz banales, heuristisches Vorgehen. \\
        Keine tiefgründigen Techniken. \\
        Dient nur zur Demonstration der einfachen Erweiterbarkeit, \\
        wenn man RAs benutzt.
        
        \par
      }
    \end{frame}

    % ------------------------------------------------------------------------------------------
    % RAi - RAiv sind oben definiert
    \begin{frame}
      \frametitle{Adresssuche mit regulären Ausdrücken}
      
      So kann sich der RA entwickeln:
      \begin{Itemize}
        \item
          Vorkommen von "`straße"' etc.:\footnote{%
            Weil der UNIX-RA \texttt{.} für $\Sigma$ reserviert ist,
            steht \texttt{\textbackslash.} für $\{.\}$%
          }
          \par\vspace*{-2mm}
          \RAi
          \par\vspace*{-2mm}
        \item<2->
          Plus Name der Straße und Hausnummer:
          \par\vspace*{-1mm}
          \uncover<3->{%
            \RAii
          }
          \par\vspace*{-2mm}
        \item<4->
          Hausnummern mit Buchstaben (12a), -bereiche (30--32):
          \par\vspace*{-2mm}
          \uncover<5->{%
            \RAiii
            \par\vspace*{-6mm}
            \RAiv
          }
          \par\vspace*{-6mm}
        \item<6->
          und mehr:
          \begin{itemize}
            \item
              Straßennamen mit Bindestrichen
            \item
              "`Straße"' etc.\ am Anfang
            \item
              Plätze, Boulevards, Alleen etc.
            \item
              Postfächer
            \item
              \dots
          \end{itemize}          
      \end{Itemize}

      \note{
        \uz{8:43 bis 8:46}

        \par
      }
    \end{frame}
%%%  }

  % ==============================================================================================
  % ==============================================================================================
  \section{Charakterisierungen}

%%%  \mysectioncontent{%
    % ------------------------------------------------------------------------------------------
    \begin{frame}
      \frametitle{Pumping-Lemma}
      
      Wie zeigt man, dass $L$ \Emph{nicht} NEA-erkennbar (regulär) ist?
      
      \par\bigskip
      \uncover<2->{%
        \begin{Satz}[Pumping-Lemma]
          Sei $L$ eine NEA-erkennbare Sprache.
          \par\smallskip
          Dann gibt es eine Konstante $p \geqslant 0$,
          \par
          so dass für alle Wörter $w \in L$ mit $|w| \geqslant p$ gilt:
          \par\smallskip
          Es gibt eine Zerlegung $w=xyz$ mit $y \neq \varepsilon$ und $|xy| \leqslant p$,\\
          \par
          so dass $xy^iz \in L$ für alle $i \geqslant 0$.
          \label{thm:pumping_lemma}
        \end{Satz}
%      }

      \par\bigskip
%      \uncover<3->{%
        \Bmph{Beweis:}~ siehe ThI\,1. % \Tafel \par~\qed%
      }

      \note{
        \textbf{8:46}

        \par\medskip
        \textbf{Fragen:}~ Wer kann sich an die 2 Werkzeuge aus ThI\,1 erinnern?
        
        \par\medskip
        Am Ende:~ \textbf{Fragebogen F.\,2a}
      }
    \end{frame}

    % ------------------------------------------------------------------------------------------
    \begin{frame}
      \frametitle{Anwendung des Pumping-Lemmas}
      
      Benutzen Kontraposition:

      \par\bigskip
      \begin{block}{}
        \Emph{Wenn} es \Emph{für alle} Konstanten $p \geqslant 0$
        \par
        \quad ein Wort $w \in L$ mit $|w| \geqslant p$ \Emph{gibt}, so dass es
        \par\smallskip
        \quad \Emph{für alle} Zerlegungen $w=xyz$ mit $y \neq \varepsilon$ und $|xy| \leqslant p$\\
        \par
        \quad ein $i \geqslant 0$ \Emph{gibt} mit $xy^iz \mathbin{\ddred{\notin}} L$,
        \par\smallskip
        \Emph{dann} ist $L$ \Emph{keine} NEA-erkennbare Sprache.
      \end{block}

      \par\bigskip
%      \uncover<2->{%
%        \ddgre{$\blacktriangleleft\!$~ Bsp.: s.\ Tafel}
        \Tafel
%      }

      \note{
        \uz{8:48 bis 8:55}

        \par
      }
    \end{frame}

    % ------------------------------------------------------------------------------------------
    \begin{frame}
      \frametitle{Bemerkungen zum Pumping-Lemma}
      
      Die Bedingung in Satz \ref{thm:pumping_lemma} ist \dots
      
      \begin{Itemize}
        \item
          \Emph{notwendig} dafür, dass $L$ NEA-erkennbar ist
          \par\smallskip
        \item
          \Emph{nicht hinreichend} \\
          Bsp.:~ $\{a^nb^kc^k \mid n,k \geqslant 1\} \cup \{b^nc^k \mid n,k \geqslant 0\}$
      \end{Itemize}

      \par\medskip
      $\leadsto$
      Pumping-L.\ nur zum \Emph{Widerlegen} von Erkennbarkeit verwendbar,\\
      nicht zum Beweisen, dass $L$ regulär ist

      \par\bigskip
      (Notwendige und hinreichende Variante: Jaffes Pumping-Lemma)

      \note{
        \textbf{8:55}

        \par\medskip
      }
    \end{frame}

    % ------------------------------------------------------------------------------------------
    \begin{frame}
      \frametitle{Der Satz von Myhill-Nerode}
      
      \Bmph{Ziel:} notwendige \Emph{und} hinreichende Bedingung für Erkennbarkeit
      
      \par\medskip
      \begin{Definition}
        Sei $L \subseteq \Sigma^*$ eine Sprache.
        \par\smallskip
        Zwei Wörter $u,v \in \Sigma^*$ sind \Bmph{$L$-äquivalent}
        (Schreibweise: $u \sim_L v$),
        \par
        wenn für alle $w \in \Sigma^*$ gilt:
        \[
          uw \in L \quad\text{genau dann, wenn}\quad vw \in L
        \]
      \end{Definition}

      \par\bigskip
      \uncover<2->{%
        $\sim_L$ heißt \Bmph{Nerode-Rechtskongruenz} und ist Äquivalenzrelation \\
        (Reflexivität, Symmetrie, Transitivität sind offensichtlich)%
%      }
      \par\bigskip
%      \uncover<3->{%
        \Bmph{Index} von $\sim_L$: Anzahl der Äquivalenzklassen%
      }

      \note{
        \textbf{8:57}

        \par\medskip
        Satz von Myhill-Nerode liefert eine "`echte"' Charakterisierung der erkennbaren Sprachen
        (und eine weitere nützliche Information)
        
        \par
      }
    \end{frame}

    % ------------------------------------------------------------------------------------------
    \begin{frame}
      \frametitle{Der Satz von Myhill-Nerode}
    
      \begin{Satz}[Myhill-Nerode]
        $L \subseteq \Sigma^*$ is NEA-erkennbar gdw.\ $\sim_L$ endlichen Index hat.
      \end{Satz}

      \par\bigskip
      \Bmph{Beweis:}~ siehe ThI\,1. \Tafel
      
      \par\bigskip\bigskip
      \uncover<2->{%
        Interessantes \Bmph{"`Nebenprodukt"'} des Beweises:
        \par\smallskip
        Endlicher Index $n$ von $\sim_L$ \\
        $=$ minimale Anzahl von Zuständen in einem DEA, der $L$ erkennt.
      }
      \note{
        \textbf{8:59}

        \par\medskip
        {\textgray\textbf{Fragebogen F.\,2b}}
        
        \par\medskip
        \textbf{13\,min Pause: Terminfindung $\leadsto$ 9:15}
        
        (Ist aber eigentlich zu zeitig für Pause)
        
        \par
      }
    \end{frame}
%%%  }

  % ==============================================================================================
  % ==============================================================================================
  \section[Entscheidungsprobleme]{Entscheidungsprobleme}

%%%  \mysectioncontent{%
%     % ------------------------------------------------------------------------------------------
%     \begin{frame}[t]
%       \frametitle{Grundbegriffe}
%       
%       \Bmph{(Entscheidungs-)Problem}
%       \begin{Itemize}
%         \item
%           \dots\ ist eine Teilmenge $P \subseteq M$
%         \item
%           Beispiele:
%           \begin{Itemize}
%             \item
%               $P = \text{Menge aller Primzahlen}$, $M = \mathbb{N}$
%             \item
%               $\,P = \text{Menge aller NEAs \Aut{A} mit $L(\Aut{A}) \neq \emptyset$}$, \\
%               $M = \text{Menge aller NEAs}$
%           \end{Itemize}
%         \item
%           man stelle sich eine Blackbox vor mit
%           \begin{Itemize}
%             \item
%               Eingabe $m \in M$
%             \item
%               Ausgabe \emph{ja}, falls $m \in P$, \emph{nein} sonst\quad (jede Berech.\ terminiert)
%           \end{Itemize}
%       \end{Itemize}
% 
%       \par\bigskip
%       \uncover<2->{%
%         \Bmph{Entscheidbarkeit:} $P$ ist entscheidbar,\\
%         wenn es einen Algorithmus $A$ gibt, der die Blackbox implementiert.
%         \[
%           \left(
%             \begin{array}{@{}l@{}}
%               \text{Programmiersprache u.\ Rechnermodell sind relativ unerheblich:} \\
%               \text{erweiterte Churchsche These}
%             \end{array}
%           \right)
%         \]
%       }      
%       \note{~}
%    \end{frame}
% 
%     % ------------------------------------------------------------------------------------------
%     \begin{frame}[t]
%       \frametitle{Komplexität}
%       
%       \Bmph{(Entscheidungs-)Problem}
%       \begin{Itemize}
%         \item
%           \dots\ ist eine Teilmenge $P \subseteq M$
%         \item
%           man stelle sich eine Blackbox vor mit
%           \par
%           \begin{minipage}{.3\textwidth}
%             \begin{Itemize}
%               \item
%                 Eingabe $m \in M$
%             \end{Itemize}%
%           \end{minipage}
%           \begin{minipage}{.6\textwidth}
%             \begin{Itemize}
%               \item
%                 Ausgabe \emph{ja}, falls $m \in P$, \emph{nein} sonst
%             \end{Itemize}%
%           \end{minipage}
%       \end{Itemize}
% 
%       \par\medskip
%       \Bmph{Entscheidbarkeit:}
%       gibt es Alg.\ $A$, der die Blackbox implementiert?
% 
%       \par\bigskip
%       \uncover<2->{%
%         \Bmph{Komplexität}:\\
%         zusätzliche Anforderungen an Zeit-/Speicherplatzbedarf von $A$%
%         \begin{Itemize}
%           \item
%             \Bmph{Polynomialzeit:}
%             Anzahl Rechenschritte von $A$ ist $\leqslant pol(|m|)$,\\
%             $|m|:$ Länge der Eingabe;\quad $pol$ ist ein festes \Bmph{Polynom}
%         \end{Itemize}
%       }
%     \end{frame}
% 
%     % ------------------------------------------------------------------------------------------
%     \begin{frame}[t]
%       \frametitle{Reduktion}
%       
%       \Bmph{(Entscheidungs-)Problem}
%       \begin{Itemize}
%         \item
%           \dots\ ist eine Teilmenge $P \subseteq M$
%         \item
%           man stelle sich eine Blackbox vor mit
%           \par
%           \begin{minipage}{.3\textwidth}
%             \begin{Itemize}
%               \item
%                 Eingabe $m \in M$
%             \end{Itemize}%
%           \end{minipage}
%           \begin{minipage}{.6\textwidth}
%             \begin{Itemize}
%               \item
%                 Ausgabe \emph{ja}, falls $m \in P$, \emph{nein} sonst
%             \end{Itemize}%
%           \end{minipage}
%       \end{Itemize}
% 
%       \par\medskip
% %       \Bmph{Entscheidbarkeit:}
% %       gibt es Alg.\ $A$, der die Blackbox implementiert?
% % 
% %       \par\bigskip
%       \Bmph{Polynomialzeit:}
%       Anzahl der Rechenschritte ist $\leqslant pol(|m|)$,\\
%       \quad $|m|:$ Länge der Eingabe;\quad $pol$ ist ein festes \Bmph{Polynom}
%       
%       \par\bigskip
%       \uncover<2->{%
%         \Bmph{(Polynomielle) Reduktion} von $P \subseteq M$ nach $P' \subseteq M'$ \\
%         ist eine (in Polynomialzeit berechenbare) Funktion $\pi$ mit 
%         \begin{Itemize}
%           \item
%             $\pi : M \to M'$
%           \item
%             $m \in P$ gdw.\ $\pi(m) \in P'$
%           \item
%             Skizze/Bsp.: siehe Tafel \Tafel
%         \end{Itemize}
%       }
%       
%       \par\medskip
%       \uncover<3->{%
%         Wenn $P$ zu $P'$ reduzierbar, dann ist $P$ \Bmph{höchstens so schwer} wie $P'$.
%       }
%     \end{frame}

    % ------------------------------------------------------------------------------------------
    \begin{frame}[t]
      \frametitle{Entscheidbarkeit}
      
      \Bmph{(Entscheidungs-)Problem}
      \begin{Itemize}
        \item
          \dots\ ist eine Teilmenge $X \subseteq M$
          \begin{itemize}
            \item
              Eingabe:~ $m \in M$;\quad Frage: $m \in X$\,?
            \item
              $m \in X$\,: \Bmph{Ja-Instanzen;}\quad $m \in M \setminus X$\,: \Bmph{Nein-Instanzen}
          \end{itemize}
          \par\smallskip
        \item<2->
          Beispiele:
          \begin{Itemize}
            \item
              $X = \text{Menge aller Primzahlen}$, $M = \mathbb{N}$
            \item
              $\,X = \text{Menge aller NEAs \Aut{A} mit $L(\Aut{A}) \neq \emptyset$}$, \\
              $M = \text{Menge aller NEAs}$
          \end{Itemize}
          \par\smallskip
        \item<3->
          man stelle sich eine Blackbox vor:
%           \begin{Itemize}
%             \item
%               Eingabe $m \in M$
%             \item
%               Ausgabe \emph{ja}, falls $m \in P$, \emph{nein} sonst\quad (jede Berech.\ terminiert)
%           \end{Itemize}
          \begin{center}
            \Fig{20}
          \end{center}
      \end{Itemize}

      \par\smallskip
      \uncover<4->{%
        \Bmph{Entscheidbarkeit:} $X$ ist entscheidbar,\\
        wenn es einen Algorithmus $A$ gibt, der die Blackbox implementiert.
%        \[
%          \left(
%            \begin{array}{@{}l@{}}
%              \text{} \\
%              \text{}
%            \end{array}
%          \right)
%        \]
      }
%      \vspace*{-3pt}

      \note{
        \textbf{9:15}
%        \textbf{8:45 kurz}

        \par\medskip
        Entscheidungsprobleme sind zentral für diese Vorlesung,
        denn wir werden sehen, dass algorithmische Probleme in Anwendungen
        auf Entscheidungsprobleme gewisser Automatenmodelle
        zurückzuführen sind.
        
        \par\smallskip
        Z.\,B.:~ Validierung eines XML-Dokuments entspricht dem Wortproblem
        für gewisse Baumautomaten.
        
        \par\smallskip
        Durch das Studium dieser Entscheidungsprobleme bekommen wir also
        einen prinzipiellen Ansatz zur Lösung der Anwendungsprobleme.
        
        \par\bigskip
        Zunächst ein kurzer Abriss von Entscheidbarkeit und Komplexität;
        mehr dazu im ThI2-Skript.
        
        \par\bigskip
        "`Algorithmus"':~ Prog.sprache \& Rechnermodell sind relativ unerheblich \\
        ((erweiterte) Church-Turing-These)
        
        \par
      }
    \end{frame}

    % ------------------------------------------------------------------------------------------
    \begin{frame}
      \frametitle{Komplexität}
      
      \begin{center}
        \Fig{20}
      \end{center}

      \Bmph{Komplexität}:\\
      zusätzliche Anforderungen an Zeit-/Speicherplatzbedarf von $A$%
      \begin{Itemize}
        \item<2->
          \Bmph{Polynomialzeit:}
          Anzahl Rechenschritte von $A$ ist $\leqslant |m|^k$,\\
          $|m|:$ Länge der Eingabe;\quad $k:$ beliebige Konstante
          \par\smallskip
        \item<3->
          \Bmph{Polynomieller Platz:}
          von $A$ benötigter Speicherplatz $\leqslant |m|^k$
          \par\smallskip
        \item<4->
          \Bmph{Exponentialzeit:}
          Anzahl Rechenschritte von $A$ ist $\leqslant 2^{|m|^k}$
          \par\smallskip
        \item<5->
          \dots
      \end{Itemize}
      \note{
        \uz{9:18}
%        \textbf{8:46 kurz}

        \par\medskip
        \textbf{Wichtig:}~
        Es geht dabei immer um eine "`Worst-Case-Analyse"', \\
        also: wie viele Ressourcen braucht ein Algorithmus/
        der beste Algorithmus \textbf{im schlechtesten Fall}\,?
        
        \par
      }
    \end{frame}

    % ------------------------------------------------------------------------------------------
    \begin{frame}
      \frametitle{Einige übliche Komplexitätsklassen}
      
%        \par\vspace*{-10pt}
%      \begin{block}{}
%        \begin{center}
        \hspace{-5mm}%
          \scalebox{.94}[1]{%
            \begin{tabular}{@{}lll@{}}
              \hline\stab
              Name            & Bedeutung                     & Beispiel-Problem           \\
              \hline\stab
              \LS             & logarithm.\ Speicherplatz     & Erreichbarkeit, ungerichtete Graphen \\
              \NL             & nichtdetermin.\ log.\ Platz   & Erreichbarkeit, gerichtete Graphen \\
  %             \leftthumbsup
              \PT             & Polynomialzeit                & Primzahlen                 \\[5pt]
              \hdashline[5pt/2pt]
              \rule{0pt}{15pt}%
  %             \leftthumbsdown
              \NP             & nichtdeterminist.\ Polyzeit   & Erfüllbarkeit Aussagenlogik  \\
              \PS             & polynom.\ Speicherplatz       & Erfüllbarkeit QBF  \\[10pt]
              \EXP            & Exponentialzeit               & Gewinnstrategie $n\!\times\!n$-Schach
              \\
  %             \multirow{6}{3cm}{%
  %               $\left.\rule{0pt}{3.05\baselineskip}\right\}$
  %               \parbox[t]{2.5cm}{HL SAT \\[1.3\baselineskip]FOL SAT}%
  %             }                                                                                              \\
              \NEXP           & nichtdet. Exponentialzeit     & Clique f.\ schaltkreiscodierte Graphen             \\
              \EXPSPACE       & exponentieller Platz          & Äquiv.\ regulärer Ausdrücke mit "`\,$\cdot^2$\,"'  \\[-2pt]
              \qquad \vdots   & \qquad \vdots                 &                            \\%[10pt]
  %             $\skull$
                              & unentscheidbar                & Erfüllbarkeit Prädikatenlogik \\
              \hline
            \end{tabular}
            \hspace*{-10mm}
          }
        
%        \end{center}
      
        \par\bigskip
        \hspace{-5mm}%
        Komplementklassen:~ \coNL, \coNP etc.\
%      \end{block}
      \note{
        \uz{9:21}
%        \textbf{8:47 kurz}

        \par\medskip
        \textbf{Klassen so vorlesen:}~
        "`\dots ist die Menge aller Probleme, die sich mit einer \dots TM in \dots\ Zeit/Platz lösen lassen"'
        
        \par\medskip
        Beispiele nur am Rande erwähnen
        
        \par
      }
    \end{frame}

    % ------------------------------------------------------------------------------------------
    \begin{frame}
      \frametitle{Reduktion}
      
      \begin{center}
        \Fig{20}
      \end{center}

      \Bmph{(Polynomielle) Reduktion} von $X \subseteq M$ nach $X' \subseteq M'$ \\
      ist eine (in Polyzeit) berechenbare Funktion $\pi : M \to M'$ mit 
%       \begin{Itemize}
%         \item
%           $\pi : M \to M'$
%         \item
%           $m \in X$ gdw.\ $\pi(m) \in X'$
% %         \item
% %           Skizze/Bsp.: siehe Tafel \Tafel
%       \end{Itemize}
      \[
        m \in X\qquad \text{gdw.}\qquad \pi(m) \in X'
      \]
%         \item
%           Skizze/Bsp.: siehe Tafel \Tafel

      \uncover<2->{%
        \begin{center}
          \Fig{21}
        \end{center}
%        \vspace*{-2\baselineskip}
%        \Tafel
      }
      
      \par\medskip
      \uncover<3->{%
        Schreibweise: \Bmph{$X \leq X'$} bzw.\ \Bmph{$X \leq_{\text{P}} X'$} ($X$ auf $X'$ \Bmph{reduzierbar})
        \Tafel\label{tafel:Reduktion}
      }
      
      \par\medskip
      \uncover<4->{%
        Wenn alle Probleme aus Komplexitätsklasse $\mathcalreg{C}$ auf $X$ reduzierbar,\\
        dann ist $X$ \Bmph{schwer für} $\mathcalreg{C}$.
      }
      \note{
        \uz{9:26 bis 9:38}
%        \textbf{8:48 -- hier wieder einsteigen}

        \par\medskip
        Reduktion:~ wichtiges Hilfsmittel zum genauen Bestimmen der Komplexität
        
        \par\bigskip
        Intuition:~
        Wenn $X \leq X'$, dann ist $X$ \emph{höchstens so schwer wie} $X'$.
        
        \par
      }
    \end{frame}

    % ------------------------------------------------------------------------------------------
    \begin{frame}
      \frametitle{Bestimmung der Komplexität}
      
      Normalerweise zeigt man, dass ein Problem $X \subseteq M$ \dots
      \begin{Itemize}
        \item
          \Bmph{in} einer Komplexitätsklasse $\mathcalreg{C}$ liegt, indem man
          \begin{Itemize}
            \item
              einen Algorithmus $A$ findet, der $X$ löst
            \item
              zeigt, dass $A$ korrekt ist (\emph{ja}/\emph{nein}-Antworten) und terminiert
            \item
              zeigt, dass $A$ für jedes $m\in M$ \Emph{höchstens} die $\mathcalreg{C}$-Ressourcen braucht
            \item[]
              \dots\ $A$ kann z.\,B.\ eine Reduktion zu einem Problem aus $\mathcalreg{C}$ sein
          \end{Itemize}
          \par\smallskip
        \item<2->
          \Bmph{schwer (hard) für} $\mathcalreg{C}$ ist, indem man
          \begin{Itemize}
            \item
              ein Problem $X' \subseteq M'$ findet, dass \Emph{schwer für $\mathcalreg{C}$} ist
            \item
              und eine Reduktion von $X'$ nach $X$ angibt
          \end{Itemize}
          \par\smallskip
        \item<3->
          \Bmph{vollständig für} $\mathcalreg{C}$ ist, indem man zeigt, dass es
          \begin{Itemize}
            \item
              \Emph{in $\mathcalreg{C}$} liegt und
            \item
              \Emph{schwer für $\mathcalreg{C}$} ist
          \end{Itemize}
      \end{Itemize}
      \note{
        \uz{9:38}

        \par\medskip
        {\textgray\textbf{Fragebogen F.\,1 $\leadsto$ 9:02}}
        
        \par
      }
    \end{frame}

    % ------------------------------------------------------------------------------------------
    \begin{frame}
      \frametitle{Entscheidungsprobleme für endliche Automaten}
      
      \begin{Itemize}
        \item
          Betrachten wesentliche Eigenschaften von Sprachen\\
          (Sprachen repräsentiert durch NEAs oder reguläre Ausdr.)
          \begin{Itemize}
            \item
              Ist eine gegebene Sprache leer?
            \item
              Ist ein gegebenes Wort $w$ in einer Sprache $L$?
            \item
              Beschreiben zwei Repräsentationen einer Sprache \\
              tatsächlich dieselbe Sprache?
          \end{Itemize}
          \par\smallskip
        \item<2->
          Wichtig für Anwendungen (siehe Einführung)
          \par\smallskip
        \item<3->
          Art der Repräsentation spielt manchmal eine Rolle: \\
          NEA, DEA, regulärer Ausdruck, Typ-3-Grammatik etc.\
          
          \par\smallskip
          Wir betrachten im Folgenden \Emph{NEAs und DEAs.}
      \end{Itemize}

      \note{
        \textbf{9:42}

        \par\medskip
        Umwandlung \dots
        \begin{Itemize}
          \item
            DEA $\to$ NEA:\quad konstante Zeit \dgre{\blacksmiley}
          \item
            NEA $\to$ DEA:\quad Exponentialzeit \dred{\frownie}
          \item
            reg.\ Ausdr.\ $\to$ NEA:\quad Polynomialzeit \dgre{\blacksmiley}
          \item
            NEA $\to$ reg.\ Ausdr.:\quad Exponentialzeit \dred{\frownie}
          \item
            NEA $\leftrightarrow$ Typ-3-Gramm.:\quad Polynomialzeit \dgre{\blacksmiley}
        \end{Itemize}
      }
    \end{frame}

    % ------------------------------------------------------------------------------------------
    \begin{frame}
      \frametitle{Das Leerheitsproblem}
      
      \Bmph{Eingabe:}~ NEA (oder DEA) \Aut{A}
      
      \par\smallskip
      \Bmph{Frage:}~ Ist $L(\Aut{A}) = \emptyset$\,?
      
      \par\medskip
      \begin{tabular}{@{}l@{~~}l@{}}
        d.\,h. & $\Bmph{$\text{LP}_\text{NEA}$} = \{\Aut{A} \mid \Aut{A} \text{~NEA,~} L(\Aut{A}) = \emptyset\}$, \\
               & $\Bmph{$\text{LP}_\text{DEA}$} = \{\Aut{A} \mid \Aut{A} \text{~DEA,~} L(\Aut{A}) = \emptyset\}$
      \end{tabular}
      
      \par\bigskip
      \begin{Satz}<2->
%        Das Leerheitsproblem für NEAs bzw.\ DEAs ist entscheidbar \\
%        und \NL-vollständig.
        $\text{LP}_\text{NEA}$ und $\text{LP}_\text{DEA}$ sind entscheidbar
        und \coNL-vollständig.
      \end{Satz}
      
      \par\medskip
      \uncover<3->{%
        \Bmph{Beweis.}
        %
        \begin{Itemize}
          \item
            Entscheidbarkeit (in Polyzeit):~ siehe ThI\,1
          \item
            \coNL-Zugehörigkeit:\\
            Reduktion zu Erreichbarkeit in gerichteten Graphen,~ siehe T\arabic{part}.\ref{tafel:Reduktion}
          \item
            \coNL-Härte:\\
            Reduktion \emph{von} Erreichbarkeit, analog \qed
        \end{Itemize}
    }      

      \note{%
        \textbf{9:44}\quad
%
%        \par\medskip
        Def.\ der Probleme:~ Eingabe ist üblicherweise ein Wort, \\
        also braucht man eine geeignete Kodierung von NEAs/DEAs (s.\, ThI\,2).
        
        \par\smallskip
        Überprüfung, ob Eingabe wohlgeformt ist, ist üblicherweise billig, \\
        macht also keinen Unterschied, ob sie mit in Komplexität zählt oder nicht.
        
        \par\bigskip
        \textbf{co}NL, weil das Komplement (Nichtleerheit) NL-vollst. ist.
        
        \par\bigskip
        {\textgray
          \textbf{coNL-Härte:}~  bei DEAs aufpassen -- für jede ausgehende Kante e.\ Knotens braucht man ein neues Zeichen,
          also Alphabetgröße $=$ max.\ Ausgangsgrad.%

          \par\smallskip
          Außerdem Papierkorbzustände einbauen.
        }
        
        
        \par\bigskip
        (Nicht-)Leerheit von NEAs/DEAs ist also nichts anderes als Wegsuche in gerichteten Graphen.
        
        \par\medskip
        {\textgray\textbf{Fragebogen F.\,2:} Tabelle, während der nächsten Folien vervollständigen}
        
        \par
      }
    \end{frame}

    % ------------------------------------------------------------------------------------------
    \begin{frame}
      \frametitle{Das Wortproblem}
      
      \Bmph{Eingabe:}~ NEA (oder DEA) \Aut{A},~ Wort $w \in \Sigma^*$

      \par\smallskip
      \Bmph{Frage:}~ Ist $w \in L(\Aut{A})$\,?
      
      \par\medskip
      \begin{tabular}{@{}l@{~~}l@{}}
        d.\,h. & $\Bmph{$\text{WP}_\text{NEA}$} = \{(\Aut{A},w) \mid \Aut{A} \text{~NEA,~} w \in L(\Aut{A})\}$, \\
               & $\Bmph{$\text{WP}_\text{DEA}$} = \{(\Aut{A},w) \mid \Aut{A} \text{~DEA,~} w \in L(\Aut{A})\}$
      \end{tabular}
      
      \par\bigskip
      \begin{Satz}<2->
%        Das Wortproblem für NEAs bzw.\ DEAs ist entscheidbar \\
%        und \NL-vollständig.
        $\text{WP}_\text{NEA}$ und $\text{WP}_\text{DEA}$ sind entscheidbar.\\
        $\text{WP}_\text{NEA}$ ist \NL-vollständig;~ $\text{WP}_\text{DEA}$ ist \LS-vollständig.
      \end{Satz}
      
      \par\medskip
      \uncover<3->{%
        \Bmph{Beweis.}
        %
        \begin{Itemize}
          \item
            Entscheidbarkeit (in Polyzeit):~ siehe ThI\,1 \\
            (Reduktion zum LP:~ $w \in L(\Aut{A})$ ~gdw.~ $L(\Aut{A}) \cap L(\Aut{A}_w) \neq \emptyset$)
          \item
            \NL-Vollst.:~ $\approx$ Erreichbarkeit in gerichteten Graphen 
%          \item
%            \L-Vollständigkeit:~ \dots
            \qed
        \end{Itemize}
      }      

      \note{
        \textbf{9:48}\quad
%
%        \par\medskip
        Reduktion zum LP: \\
        $\bullet$ $\Aut{A}_w$ ist ein N/DEA, der nur $w$ akzeptiert (einfach zu bauen) \\
        $\bullet$ nutzt Abgeschlossenheit unter Schnitt (Konstr.\ Produktautomat) \\
        $\bullet$ liefert nur "`in P"'
      
        \par\bigskip
        obere Schranke:~ Rate Weg der Länge $\leq |Q|$ ab $q_0$, \\
        so dass im $i$-ten Schritt das $i$-te Zeichen von $w$ gelesen wird. \\
        Akzeptiere, wenn am Ende ein akz.\ Zustand erreicht ist.
        
        \par\smallskip
        Für DEAs muss nicht geraten werden.
        
        \par\bigskip
        NL-Härte:~ Wandle gegebenen $G=(V,E)$ in NEA, so dass
        alle Kanten mit $a$ beschriftet sind. Anfangs-/akz.\ Zustand: $s,t$.
        Zusätzlich $a$-Schleife an $t$. Dann frage nach $w = a^{|V|}$.
        
        \par\bigskip
        L-Härte:~ Führt hier zu weit; braucht speziellen Reduktionsbegriff \\
        (weak red., siehe auch Holzer \& Kutrib, Inf \& Comp.\ 2011)
        
        \par
        
      }
    \end{frame}


    % ------------------------------------------------------------------------------------------
    \begin{frame}
      \frametitle{Das Äquivalenzproblem}
      
      \Bmph{Eingabe:}~ NEAs (oder DEAs) $\Aut{A}_1,\Aut{A}_2$

      \par\smallskip
      \Bmph{Frage:}~ Ist $L(\Aut{A}_1) = L(\Aut{A}_2)$\,?
      
      \par\medskip
      \begin{tabular}{@{}l@{~~}l@{}}
        d.\,h. & $\Bmph{$\text{ÄP}_\text{NEA}$} = \{(\Aut{A}_1,\Aut{A}_2) \mid \Aut{A}_1,\Aut{A}_2 \text{~NEAs,~} L(\Aut{A}_1) = L(\Aut{A}_2)\}$, \\
               & $\Bmph{$\text{ÄP}_\text{DEA}$} = \{(\Aut{A}_1,\Aut{A}_2) \mid \Aut{A}_1,\Aut{A}_2 \text{~DEAs,~} L(\Aut{A}_1) = L(\Aut{A}_2)\}$
      \end{tabular}
      
      \par\bigskip
      \begin{Satz}<2->
        %        Das Wortproblem für NEAs bzw.\ DEAs ist entscheidbar \\
        %        und \NL-vollständig.
        $\text{ÄP}_\text{NEA}$ und $\text{ÄP}_\text{DEA}$ sind entscheidbar.\\
        $\text{ÄP}_\text{NEA}$ ist \PSPACE-vollständig;~ $\text{ÄP}_\text{DEA}$ ist \NL-vollständig.
      \end{Satz}
      
      \par\medskip
      \uncover<3->{%
        \Bmph{Beweis.}
        %
        \begin{Itemize}
          \item
            Entscheidbarkeit:~ siehe ThI\,1 \\
            (Red.\ zum LP:~ $L(\Aut{A}_1) = L(\Aut{A}_2)$ ~gdw.~ $L(\Aut{A}_1) \vartriangle L(\Aut{A}_2) = \emptyset$)
          \item
            Komplexität:~ Automat für $L(\Aut{A}_1) \vartriangle L(\Aut{A}_2)$ ist exponentiell in
            der Größe der Eingabe-NEAs;  polynomiell im Fall von DEAs
            
            \par\smallskip
            Details: siehe \cite{HK11}
            \qed
        \end{Itemize}
      }      

      \note{
        \uz{9:54}

        \par\medskip
        $\vartriangle$ $=$ symmetrische Differenz zweier Mengen; \\
        ausdrückbar mittels $\cup,\cap,\overline{\cdot}$\quad $\leadsto$ Abschlusseigenschaften!
        
        \par
      }
    \end{frame}

    % ------------------------------------------------------------------------------------------
    \begin{frame}
      \frametitle{Das Universalitätsproblem}
      
      \Bmph{Eingabe:}~ NEA (oder DEA) \Aut{A}

      \par\smallskip
      \Bmph{Frage:}~ Ist $L(\Aut{A}) = \Sigma^*$\,?
      
      \par\medskip
      \begin{tabular}{@{}l@{~~}l@{}}
        d.\,h. & $\Bmph{$\text{UP}_\text{NEA}$} = \{\Aut{A} \mid \Aut{A} \text{~NEA,~} L(\Aut{A}) = \Sigma^*\}$, \\
               & $\Bmph{$\text{UP}_\text{DEA}$} = \{\Aut{A} \mid \Aut{A} \text{~DEA,~} L(\Aut{A}) = \Sigma^*\}$
      \end{tabular}
      
      \par\bigskip
      \begin{Satz}<2->
        %        Das Leerheitsproblem für NEAs bzw.\ DEAs ist entscheidbar \\
        %        und \NL-vollständig.
        $\text{UP}_\text{NEA}$ und $\text{UP}_\text{DEA}$ sind entscheidbar.
      \end{Satz}
      
      \par\medskip
      \uncover<2->{%
        \Bmph{Beweis:}~ Übungsaufgabe
      }      

      \note{
        \textbf{9:57}

        \par\medskip
        Reduktion vom/zum Komplement LP: \\
        $L(\Aut{A}) = \Sigma^*$ ~gdw.~ $\overline{L(\Aut{A})} = \emptyset$
        
        \par\medskip
        Für DEAs NL-v., analog zu Wegsuche:\\ ist ein nicht-akz.\ Zustand erreichbar?
        
        \par\medskip
        Für NEAs PSPACE-v., Wegsuche im Potenz-DEA on-the-fly
      }
    \end{frame}

    \newlength{\sternchen}
    \settowidth{\sternchen}{${}^*$}
    \newcommand{\stNOst}{\hspace*{\sternchen}\NO{}${}^*$}
    % ------------------------------------------------------------------------------------------
    \begin{frame}
      \frametitle{Überblick Entscheidungsprobleme für NEAs/DEAs}
      
%      \begin{center}
        \begin{tabular}{cccc}
          \hline\stab
                  &               & für DEAs          & für NEAs          \\
          Problem & entscheidbar? & effizient lösbar? & effizient lösbar? \\
          \hline\stab
          LP      & \YES          & \YES              & \YES              \\
          WP      & \YES          & \YES              & \YES              \\
          ÄP      & \YES          & \YES              & \stNOst           \\
          UP      & \YES          & \YES              & \stNOst           \\
          \hline
        \end{tabular}

        \par\bigskip
        \begin{tabular}{@{}l@{\,}l@{}}
          ${}^*$ & unter den üblichen komplexitätstheoretischen Annahmen \\
                 & (z.\,B.\ $\PSPACE \neq \PT$)
        \end{tabular}
%      \end{center}

      \note{
        \textbf{9:58 bis 10:00, Ende}

        \par\medskip
        Hier nochmal für Euch als Überblick \\
%        (und zum Vergleich von Aufgabe 2 auf Fragebogen)
        
%        \par\medskip
%        \textbf{Nächstes Mal (Mi.):}
%        \begin{itemize}
%          \item
%            Beginn mit Baumautomaten (dann Neues, mehr Tafel, \dots)
%          \item
%            Abstimmung über Verschiebung des Montagstermins \\
%            (bitte macht Euch Gedanken, welche Zeiten möglich wären \& welche nicht)
%        \end{itemize}

        \par\medskip
        Zum \textbf{Literaturverzeichnis} blättern
        
        \par\medskip
%        \textbf{5\,min Pause $\to$ bis 9:26}
      }
    \end{frame}
%%%  }

  \AtBeginSection{\relax}
  % ==============================================================================================
  % ==============================================================================================
  \section*{}
    \begin{frame}
      \frametitle{Literatur für diesen Teil (Basis)}
      \begin{small}
        \begin{thebibliography}{99}
          \bibitem{HMU01}
            John E.\ Hopcroft, Rajeev Motwani, Jeffrey D.\ Ullman.
            \newblock
            Introduction to Automata Theory, Languages and Computation.
            \newblock
            2. Auf"|lage, Addison-Wesley, 2001.
            \newblock
            Kapitel 1,2.
            \newblock
            Verfügbar in SUUB (verschiedene Auf"|lagen, auch auf Deutsch)
          \bibitem{Bie09}
            Meghyn Bienvenu.
            \newblock
            Automata on Infinite Words and Trees.
            \newblock
            Vorlesungsskript, Uni Bremen, WS 2009/10.
            \newblock
            \url{http://www.informatik.uni-bremen.de/tdki/lehre/ws09/automata/automata-notes.pdf}
        \end{thebibliography}
        \par
      \end{small}
      \note{
        "`Basis"':~ um die Inhalte der Vorlesung nachzulesen

        \par
      }
    \end{frame}

    \begin{frame}
      \frametitle{Literatur für diesen Teil (weiterführend)}
      \begin{small}
        \begin{thebibliography}{99}
          \bibitem[Holzer \& Kutrib 2011]{HK11}
          Markus Holzer, Martin Kutrib.
          \newblock
          Descriptional and computational complexity of finite automata – A survey.
          \newblock
          Information and Computation 209:456--470, 2011.
          \newblock
          Kapitel 3:~
          sehr umfassender Überblick über Entscheidungsprobleme für endliche Automaten
          und deren Komplexität;~ viel Literatur
          \newblock
          Verfügbar in SUUB (elektronisch) \\
          \url{https://doi.org/10.1016/j.ic.2010.11.013}
        \end{thebibliography}
        \par
      \end{small}
      \note{
        "`weiterführend"':~ um bestimmte Details zu vertiefen, die in der Vorlesung nur am Rande angesprochen werden
        
        \par
      }
      \end{frame}

%     \begin{frame}
%       \frametitle{~}
% %       \par\bigskip
%       \uncover<1->{%
%         \begin{center}
%           \begin{Huge}
%             \Bmph{Vielen Dank.}
%           \end{Huge}
%         \end{center}
%       }
%     \end{frame}


%   \AtBeginSection{\frame{\frametitle{Und nun \dots}\tableofcontents[currentsection]}}

\mode<presentation>{
  {   
    \setbeamercolor{background canvas}{bg=black}
    \begin{frame}<handout:0>[plain]{}
      \note{~}
    \end{frame}
  }
}


%     % ------------------------------------------------------------------------------------------
%     \begin{frame}
%       \frametitle{Ausblick}
% 
%       \dots
%       
%       \par\bigskip
%       \uncover<2>{%
%         \begin{center}
%           \begin{Huge}
%             \dblu{\textbf{Thank you.}}
%           \end{Huge}
%         \end{center}
%       }
%     \end{frame}

%   % ==============================================================================================
%   % ==============================================================================================
%   \appendix
%   
%     % ------------------------------------------------------------------------------------------
%     \begin{frame}
%       \frametitle{\dots}
%       \dots
%     \end{frame}

\end{document}
