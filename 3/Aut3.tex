%\setbeamertemplate{navigation symbols}{}
%\beamertemplatenavigationsymbolsempty

\setcounter{part}{3}

%%% \showfromto{5}{5}

\begin{document}

  % ------------------------------------------------------------------------------------------
  \begin{frame}
    \titlepage
    \note{
      \uzz{Di:~ S.\ 74\quad Mi:~ S.\ 111}{8:40}
      
      \parII
      \textbf{TODO:} 
      Es ist nicht günstig, die ganzen anderen Automatentypen \\
      (Müller etc.) im Abschnitt "`Determinisierung"' zu haben.
      
      \parI
      $\leadsto$ Kapitel neu aufteilen, Tafelanschriebe umordnen, Nummern+Verweise im Skript anpassen:
      %
      \par\vspace*{-4pt}
      \begin{enumerate}
        \item
          Motivation
          \par\vspace*{-4pt}
        \item
          Büchi-Automaten (mit Def. DBA am Ende)
          \par\vspace*{-4pt}
        \item 
          Abschlusseig. und Chrakterisierungen (einschl. Charakt. DBAs und deren "`Schwäche"')
          \par\vspace*{-4pt}
        \item 
          Alternative Akzeptanzbedingungen
          \par\vspace*{-4pt}
        \item 
          Determinisierung
          \par\vspace*{-4pt}
        \item 
          Entscheidungsprobleme
          \par\vspace*{-4pt}
        \item 
          Model-Checking
      \end{enumerate}
      %
      \textbf{Außerdem:}~ Safra-Konstruktion anders einführen/beweisen, s.\ TODO.txt
      
      \par
    }
  \end{frame}

  % ------------------------------------------------------------------------------------------
  \begin{frame}
    \frametitle{Überblick}
    \tableofcontents
    \note{
      \uz{8:40}
      
      \par
    }
  \end{frame}

  % ==============================================================================================
  % ==============================================================================================
  \section[Motiv.]{\protect\emph{Motivation}}
  
    % ------------------------------------------------------------------------------------------
    \begin{frame}
      \frametitle{Terminierung}

      \begin{block}{}
        \Bmph{Terminierung} von Algorithmen ist wichtig für Problemlösung.
      \end{block}

      \par\bigskip
      \Bmph{Übliches Szenario:}
      \begin{Itemize}
        \item
          Eingabe: endliche Menge von Daten
        \item
          Lasse Programm $P$ laufen, bis es \Emph{terminiert}
        \item
          Ausgabe: Ergebnis, das durch $P$ berechnet wurde
      \end{Itemize}
      Um Ausgabe zu erhalten, \Emph{muss $P$ für jede Eingabe terminieren}.

      \par\bigskip
      \uncover<2->{%
        \begin{exampleblock}{}
          \Gmph{Beispiel:} Validierung von XML-Dokumenten für gegebenes Schema
          \begin{Itemize}
            \item
              Konstruiere Automaten für Schema und Dokument \Emph{(terminiert)}
            \item
              Reduziere auf Leerheitsproblem \Emph{(terminiert)}
            \item
%               Löse Leerheitsprob.\ (sammle erreichb.\ Zustände -- \Emph{terminiert})
                Löse Leerheitsproblem\\
                \hspace*{\fill} (sammle erreichbare Zustände -- \Emph{terminiert})
          \end{Itemize}
        \end{exampleblock}
      }
      \note{
        \uz{8:41}
        
        \par
      }
    \end{frame}

  % ------------------------------------------------------------------------------------------
    \begin{frame}
      \frametitle{Terminierung unerwünscht}

      \begin{block}{}
        Von manchen Systemen/Programmen fordert man,\\
        dass sie \Bmph{nie terminieren}.
      \end{block}

      \par\bigskip
      \begin{exampleblock}{}
        \Gmph{Beispiele:}
        \par\vspace*{-6pt}
        \begin{Itemize}
          \item
            (Mehrbenutzer-)Betriebssysteme\\
            {\small sollen beliebig lange laufen ohne abzustürzen, egal was Benutzer tun}
          \item
            Bankautomaten, Flugsicherungssysteme, Netzwerkkommunikationssysteme, \dots
        \end{Itemize}
      \end{exampleblock}

      \par\bigskip
      \uncover<2->{%
        \Bmph{Gängiges Berechnungsmodell:}
        \begin{Itemize}
          \item
            endliche Automaten mit nicht-terminierenden Berechnungen
          \item
            Terminierung wird als Nicht-Akzeptanz angesehen
          \item
            ursprünglich durch Büchi entwickelt (1960)\\
            {\small Ziel: Algorithmen zur Entscheidung mathematischer Theorien}
%             z.\,B.\ bestimmte Fragmente der Arithmetik
        \end{Itemize}
      }
      \note{
        \uz{8:43}
        
        \par
      }
    \end{frame}

  % ------------------------------------------------------------------------------------------
    \begin{frame}
      \frametitle{Ziel und Vorgehen dieses Kapitels}

      \Bmph{Ziel}
      \par\smallskip
      Beschreibung von Automatenmodellen mit \Emph{unendlichen} Eingaben \\
      und \Emph{nicht-terminierenden} Berechnungen

      \par\bigskip\noindent
      \Bmph{Vorgehen}
      \begin{Itemize}
        \item
          Theorie: ausgiebiges Studium von Büchi-Automaten\\
          und der von ihnen erkannten Sprachen
          \begin{Itemize}
            \item
              Definition, Abschlusseigenschaften
            \item
              Charakterisierung mittels regulärer Sprachen
            \item
              Determinisierung
            \item
              Entscheidungsprobleme
          \end{Itemize}
        \item
          Anwendung von Büchi-Automaten:\\
          Spezifikation \& Verifikation in Linearer Temporallogik (LTL)\\
      \end{Itemize}

      \note{
        \uz{8:45}
        
        \par
      }
    \end{frame}

  % ------------------------------------------------------------------------------------------
    \begin{frame}
      \frametitle{Beispiel: Philosophenproblem \hfill {\normalsize (Dining Philosophers Problem)}}

      Erläutert Nebenläufigkeit und Verklemmung von Prozessen

      \par\medskip
      Demonstriert auch unendliche Berechnungen

      \par\medskip
      Hier: einfachste Version mit 3 Philosophen

      \begin{exampleblock}<2->{Philosophenproblem}
%         \begin{Itemize}
%           \item
%             3 Philosophen $P_1,P_2,P_3$
%           \item
%             Für alle $i$ gilt: entweder denkt $P_i$, oder $P_i$ isst.
%           \item
%             Alle $P_i$ sitzen um einen runden Tisch.
%           \item
%             Jeder $P_i$ hat einen Teller mit Essen vor sich.
%           \item
%             Zwischen je zwei Tellern liegt ein Essstäbchen.
%           \item
%             Um zu essen, muss $P_i$ beide Stäbchen neben seinem Teller benutzen.
%             \par
%             \hfill $\Rightarrow$ Keine zwei $P_i,P_j$ können gleichzeitig essen.
%         \end{Itemize}
        3 Philosophen $P_1,P_2,P_3$

        \par\medskip
        Für alle $i$ gilt: entweder denkt $P_i$, oder $P_i$ isst.

        \par\medskip
        Alle $P_i$ sitzen um einen runden Tisch.

        \par\medskip
        Jeder $P_i$ hat einen Teller mit Essen vor sich.

        \par\medskip
        Zwischen je zwei Tellern liegt ein Essstäbchen.

        \par\medskip
        Um zu essen, benötigt $P_i$ beide Stäbchen neben seinem Teller.

        \par\medskip
        \uncover<3->{%
          \hfill $\Rightarrow$ Keine zwei $P_i,P_j$ können gleichzeitig essen.%
        }
      \end{exampleblock}
      \note{
        \uz{8:46}
        
        \parI
        Weil dieses Beispiel so schräg und archaisch ist, \\
        belasse ich es mal bei der männlichen Form.
        
        \par
      }
    \end{frame}

  % ------------------------------------------------------------------------------------------
    \begin{frame}
      \frametitle{Skizze zum Philosophenproblem}
      \begin{exampleblock}{Zusammenfassung}
        \begin{Itemize}
          \item
            Für alle $i$: entweder denkt $P_i$, oder $P_i$ isst.
          \item
            Keine zwei $P_i,P_j$ können gleichzeitig essen.
        \end{Itemize}
      \end{exampleblock}

%       \par\bigskip
      \uncover<2->{%
        \begin{center}
          \begin{minipage}{.48\textwidth}
            \begin{exampleblock}{}
              \begin{center}
                \Fig{0}
                \par\medskip
                \begin{small}
                  $P_1,P_2,P_3$ denken.
                  \par
                \end{small}
              \end{center}
            \end{exampleblock}
          \end{minipage}
          \hfill
          \begin{minipage}{.48\textwidth}
            \begin{exampleblock}{}
              \begin{center}
                \Fig{1}
                \par\medskip
                \begin{small}
                  $P_1,P_3$ denken; $P_2$ isst.
                  \par
                \end{small}
              \end{center}
            \end{exampleblock}
          \end{minipage}
        \end{center}

      }
      \note{
        \uz{8:48}
        
        \par
      }
    \end{frame}

  % ------------------------------------------------------------------------------------------
    \begin{frame}
      \frametitle{Modellierung durch endliches Transitionssystem}

      \Bmph{Annahmen}
      \begin{Itemize}
        \item
          Am Anfang denken \Bmph{(d)} alle $P_i$.
          \par\smallskip
        \item
          Reihum können sich $P_1,P_2,P_3$ entscheiden,\\
          ob sie denken oder essen \Bmph{(e)} wollen.
      \end{Itemize}

      \par\bigskip
      \Bmph{Zustände des Systems}
      \begin{Itemize}
        \item
%           Anfangszustand $(d,d,d,1)$:
          Anfangszustand ddd1:
          \par%\smallskip
          alle $P_i$ denken, und $P_1$ trifft nächste Entscheidung.
          \par\smallskip
        \item
          alle zulässigen Zustände:
          \begin{center}
            \begin{tabular}{@{}l@{~~~}l@{~~~}l@{~~~}l@{}}
%               $(d,d,d,1)$ & $(e,d,d,1)$ & $(d,e,d,1)$ & $(d,d,e,1)$ \\
%               $(d,d,d,2)$ & $(e,d,d,2)$ & $(d,e,d,2)$ & $(d,d,e,2)$ \\
%               $(d,d,d,3)$ & $(e,d,d,3)$ & $(d,e,d,3)$ & $(d,d,e,3)$
              ddd1 & edd1 & ded1 & dde1 \\
              ddd2 & edd2 & ded2 & dde2 \\
              ddd3 & edd3 & ded3 & dde3
            \end{tabular}
          \end{center}

      \end{Itemize}

      \par\bigskip
      \Bmph{Zustandsüberführungen:}
      \begin{Itemize}
        \item[]
          $d$ oder $e$ -- je nach Entscheidung des $P_i$, der an der Reihe ist
      \end{Itemize}
      \note{
        \uz{8:49}
        
        \par
      }
    \end{frame}

  % ------------------------------------------------------------------------------------------
    \begin{frame}
      \frametitle{Das Transitionssystem}

      \begin{center}
        \Fig{10}
      \end{center}

      \uncover<2->{%
        \Emph{Was sind die Eingaben in das System?}

        \begin{Enumerate}
          \item<3->[]
            Endliche Zeichenketten über $\Sigma = \{d,e\}$\,?
            \par\smallskip
            \uncover<4->{Dann ist das System ein NEA.}
          \item<4->[\Gmph{$\blacktriangleright$}]
            \Gmph{Unendliche Zeichenketten über $\Sigma = \{d,e\}$\,!}
        \end{Enumerate}

      }
      \note{
        \uz{8:51}
        
        \par
      }
    \end{frame}


  % ------------------------------------------------------------------------------------------
    \begin{frame}
      \frametitle{Warum unendliche Zeichenketten?}

%       \begin{Itemize}
%         \item
%           Nehmen an, jeder $P_i$ möchte \Emph{beliebig oft} denken und essen.
%           \par\smallskip
%           Dann ist $P_i$ \Bmph{zufrieden}.
%           \par\smallskip
%         \item
%           System soll \Emph{beliebig lange} ohne Terminierung laufen.
%           \par\smallskip
%         \item[$\leadsto$]
%           \Bmph{mögliche Fragen:}
%           \begin{Enumerate}
%             \item
%               Ist es überhaupt möglich, dass das System beliebig lange läuft?
%             \item
%               Ist es zusätzlich möglich, dass $P_i$ zufrieden ist?
%             \item
%               Ist es möglich, dass $P_1,P_2$ zufrieden sind, aber $P_3$ nicht?
%             \item
%               Ist es möglich, dass alle $P_i$ zufrieden sind?
%           \end{Enumerate}
%       \end{Itemize}

      Nehmen an, jeder $P_i$ möchte \Emph{beliebig oft} denken und essen.

      \par\medskip
      System soll dazu \Emph{beliebig lange} ohne Terminierung laufen.

      \par\medskip
      Philosoph $P_i$ heißt \Bmph{zufrieden}, wenn er währenddessen \Emph{unendlich oft} denkt und isst.

      \par\bigskip
      \uncover<2->{%
        \Bmph{$\leadsto$ Mögliche Fragen:}
        \begin{Enumerate}
          \item
            Kann das System überhaupt beliebig lange laufen?
          \item
            Ist es zusätzlich möglich, dass $P_i$ zufrieden ist?
          \item
            Ist es möglich, dass $P_1,P_2$ zufrieden sind, aber $P_3$ nicht?
          \item
            Ist es möglich, dass alle $P_i$ zufrieden sind?
        \end{Enumerate}
      }

      \note{
        \uz{8:55}
        
        \par
      }
    \end{frame}

  % ------------------------------------------------------------------------------------------
    \begin{frame}[t]
      \frametitle{Frage 1}

      \begin{center}
        \Fig{10}
      \end{center}

      \Emph{Ist es überhaupt möglich, dass das System beliebig lange läuft?}%

      \par\bigskip
      \uncover<2->{%
        \begin{tabular}{@{}l@{~~}l@{}}
          \Gmph{Ja:} & jeder Zustand hat mindestens einen Nachfolgerzustand.\\
                     & $dddddd\dots$ ist ein möglicher unendlicher Lauf.
        \end{tabular}%
      }
      \note{
        \uz{8:57}
        
        \par
      }
    \end{frame}

  % ------------------------------------------------------------------------------------------
    \begin{frame}[t]
      \frametitle{Frage 2}

      \begin{center}
        \Fig{10}
      \end{center}

      \Emph{Ist es möglich, dass $P_1$ zufrieden ist?}%

      \par\bigskip
      \uncover<2->{%
        \begin{tabular}{@{}l@{~~}l@{}}
          \Gmph{Ja:} & z.\,B.\ wenn ein Lauf ddd1 und edd1 unendlich oft durchläuft:\\
                     & $ed^5ed^5\dots$
        \end{tabular}%
      }
      \note{
        \uz{8:58}
        
        \par
      }
    \end{frame}

  % ------------------------------------------------------------------------------------------
    \begin{frame}[t]
      \frametitle{Frage 3}

      \begin{center}
        \Fig{10}
      \end{center}

      \Emph{Ist es möglich, dass $P_1,P_2$ zufrieden sind, aber $P_3$ nicht?}%

      \par\bigskip
      \uncover<2->{%
        \begin{tabular}{@{}l@{~~}l@{}}
          \Gmph{Ja:} & z.\,B.\ "`ddd1, edd1, ddd2, ded2 unendlich oft, aber dde$i$ nicht"':\\
                     & $ed^3ed^4ed^3ed^4\dots$
        \end{tabular}%
      }
      \note{
        \uz{8:59}
        
        \par
      }
    \end{frame}

  % ------------------------------------------------------------------------------------------
    \begin{frame}[t]
      \frametitle{Frage 4}

      \begin{center}
        \Fig{10}
      \end{center}

      \Emph{Ist es möglich, dass alle $P_i$ zufrieden sind?}%

      \par\bigskip
      \uncover<2->{%
        \begin{tabular}{@{}l@{~~}l@{}}
          \Gmph{Ja:} & z.\,B.\ "`ddd1, edd1, ddd2, ded2, ddd3, dde3 unendlich oft"': \\
                     & $ed^3ed^3\dots$ \quad oder \quad $ed^2ed^3ed^2ed^3\dots$ \quad oder \quad $\dots$
        \end{tabular}%
      }
      \note{
        \uz{9:00}
        
        \par
      }
    \end{frame}

  % ------------------------------------------------------------------------------------------
    \begin{frame}
      \frametitle{Weiteres Beispiel}
      
      \dots\ siehe Anhang, Folie \ref{fra:anhang_bsp2} \dots
      \note{
        \uz{9:01}
        
        \par
      }
    \end{frame}



  
  % ==============================================================================================
  % ==============================================================================================
  \section[Büchi-Aut.]{Grundbegriffe und Büchi-Automaten}
  
    % ------------------------------------------------------------------------------------------
    \begin{frame}
      \frametitle{Grundbegriffe}

      \Bmph{Unendliches Wort} über \Bmph{Alphabet} $\Sigma$
      \begin{Itemize}
        \item
          ist Funktion $\Bmph{$\alpha$} : \mathbb{N} \to \Sigma$
        \item
          \Bmph{$\alpha(n)$:}~ Symbol an $n$-ter Stelle (auch: \Bmph{$\alpha_n$})
        \item
          wird oft geschrieben als $\alpha = \alpha_0\alpha_1\alpha_2\dots$
      \end{Itemize}
%       Ab jetzt: unendliche Wörter $\alpha,\dots$; endliche Wörter $w,\dots$

      \par\bigskip
      \uncover<2->{%
        \Bmph{Weitere Notation}
        \begin{Itemize}
          \item
            \Bmph{$\alpha[m,n]$:}~ endliche Teilfolge $\alpha_m\alpha_{m+1}\dots \alpha_n$
          \item
            \begin{tabular}[t]{@{}l@{~}l@{}}
              \Bmph{$\#_w(\alpha)$:}~ & Anzahl der Vorkommen von $w$ als Teilwort in $\alpha$\\
                                      & $= \#\{(m,n) \mid \alpha[m,n] = w\}$
            \end{tabular}
          \item
            \begin{tabular}[t]{@{}l@{~}l@{}}
              \Bmph{$w^\omega$:}~ & unendliche Verkettung von $w$ \\ %\hfill(sei $n := |w|$)\\
                                  & ($\alpha$ mit $\alpha[i\!\!\:\cdot\!\!\:n,(i\!\!\:+\!\!\:1)n\!\!\;-\!\!\;1] = w \text{ f.\ alle } i \geqslant 0,~ n = |w|$)
            \end{tabular}
        \end{Itemize}

      }
      \par\bigskip
      \uncover<3->{%
        \Bmph{$\Sigma^\omega$:} Menge aller unendlichen Wörter
      }

      \par\bigskip
      \uncover<3->{%
        \Bmph{$\omega$-Sprache:} $L \subseteq \Sigma^\omega$
      }
      \note{
        \uz{9:02}
        
        \par
      }
    \end{frame}

  % ------------------------------------------------------------------------------------------
    \begin{frame}
      \frametitle{Büchi-Automaten}

      \begin{Definition}
        Ein \Bmph{nichtdeterministischer Büchi-Automat (NBA)}
        über einem \Bmph{Alphabet} $\Sigma$ ist ein 5-$\!$Tupel
        $\Aut{A} = (Q, \Sigma, \Delta, I, F)$, wobei
        \begin{Itemize}
          \item
            $Q$ eine endliche nichtleere \Bmph{Zustandsmenge} ist,
          \item
            $\Sigma$ eine endliche nichtleere Menge von Zeichen ist,
          \item
            $\Delta \subseteq Q \times \Sigma \times Q$ die \Bmph{Überführungsrelation} ist,
          \item
            $I \subseteq Q$ die Menge der \Bmph{Anfangszustände} ist,
          \item
            $F \subseteq Q$ die Menge der \Bmph{akzeptierenden Zustände} ist.
        \end{Itemize}
      \end{Definition}

      \par\bigskip
      \uncover<2->{%
        Bisher kein Unterschied zu NEAs, aber \dots
      }
      
      \note{
        \uz{9:04}
        
        \par
      }
    \end{frame}

    % ------------------------------------------------------------------------------------------
    \begin{frame}
      \frametitle{Berechnungen und Akzeptanz}

      \begin{Definition}
        Sei $\Aut{A} = (Q,\Sigma,\Delta,I,F)$ ein Büchi-Automat.
        \begin{Itemize}
          \item
            Ein \Bmph{Run} von \Aut{A} auf $\omega$-Wort $\alpha$
            ist eine Folge
            \vspace*{-.4\baselineskip}
            \[
              r = q_0q_1q_2\dots,
            \]
            \par\vspace*{-.4\baselineskip}
            so dass für alle $i\geqslant0$ gilt: $(q_i,\alpha_i,q_{i+1}) \in \Delta$.
            \par\smallskip
          \item<2->
            \Bmph{Unendlichkeitsmenge $\Inf(r)$} von $r = q_0q_1q_2\dots$:\\
            Menge der Zustände, die unendlich oft in $r$ vorkommen
            \par\smallskip
          \item<3->
            \Bmph{Erfolgreicher Run} $r = q_0q_1q_2\dots$:\hfill
            $q_0 \in I$ und $\Inf(r) \cap F \neq \emptyset$
            \par\smallskip
          \item<4->
            \Aut{A} \Bmph{akzeptiert} $\alpha$,\\
            wenn es einen erfolgreichen Run von \Aut{A} auf $\alpha$ gibt.
            \par\smallskip
          \item<5->
            Die von \Aut{A} \Bmph{erkannte Sprache} ist
            $\ddblu{L_\omega(\Aut{A})} = \{\alpha \in \Sigma^\omega \mid \text{\Aut{A} akzeptiert $\alpha$}\}$.
        \end{Itemize}
      \end{Definition}

      \note{
        \uz{9:06}
        
        \par
      }
    \end{frame}

    % ------------------------------------------------------------------------------------------
    \begin{frame}
      \frametitle{Beispiele}
%       \label{frame:automatenbsp}

      \begin{exampleblock}{}
        $\Aut{A}_1$:\quad \raisebox{-3mm}{\Fig{30}}
        \hfill
        \begin{minipage}{.55\textwidth}
          $L_\omega(\Aut{A}_1) = \uncover<2->{\{a^nb^\omega \mid n \geqslant 1\}}$
        \end{minipage}

        \par\bigskip
        \uncover<3->{$\Aut{A}_2$:\quad \raisebox{-3mm}{\Fig{31}}}
        \hfill
        \begin{minipage}{.55\textwidth}
          $\uncover<3->{L_\omega(\Aut{A}_2) = }\uncover<4->{\{\alpha\!\!\:\in\!\!\:\Sigma^\omega \mid \#_a(\alpha) < \infty\}}$
        \end{minipage}

        \par\bigskip
        \uncover<5->{$\Aut{A}_3$:\quad \raisebox{-3mm}{\Fig{32}}}
        \hfill
        \begin{minipage}{.55\textwidth}
          $\uncover<5->{L_\omega(\Aut{A}_3) = }\uncover<6->{\{\alpha \in \Sigma^\omega \mid \text{s.\ unten}\}}$
        \end{minipage}
        \par\bigskip
        \uncover<6->{%
%           Zwischen zwei aufeinanderfolgenden $a$'s in $\alpha$\\
%           -- und am Anfang von $\alpha$ -- steht eine gerade Anzahl von $b$'s.%
          \hspace*{\fill} Zwischen je zwei $a$'s in $\alpha$ sowie vor dem ersten $a$ \\
          \hspace*{\fill} steht jeweils eine gerade Anzahl von $b$'s.%
        }


%         \hfill
%         $L(\Aut{A}_1) = \uncover<2->{\{a^nb^m \mid n \geqslant 0, m \geqslant 1\}}$
% 
%         \par\bigskip
%         \uncover<3->{$\Aut{A}_2$: \raisebox{-3mm}{\Fig{10}}}
%         \par\bigskip
%         $\uncover<3->{L(\Aut{A}_2) = }\uncover<4->{\{\alpha \in \{a,b\}^* \mid \text{$\alpha$ enthält Teilwort $ab$}\}}$
% 
%         \par\bigskip
%         \uncover<5->{$\Aut{A}_3$: \raisebox{-3mm}{\Fig{11}}}
%         \par\bigskip
%         $\uncover<5->{L(\Aut{A}_3) = }\uncover<6->{\{\alpha \in \{a,b\}^* \mid \text{$\alpha$ endet auf $ab$}\}}$
      \end{exampleblock}

      \note{
        \uz{9:08 bis 9:14}
        
        \par
      }
    \end{frame}

%     % ------------------------------------------------------------------------------------------
%     \begin{frame}
%       \frametitle{Mehr Beispiele}
%       
%       \begin{exampleblock}{}
%         \begin{small}
%           $\Aut{A}_4$: \raisebox{-6mm}{\Fig{40}}
%           \hfill
%           \begin{minipage}{.557\textwidth}
%             \begin{align*}
%               L_\omega(\Aut{A}_4) & = \uncover<2->{\{\alpha\!\in\!\Sigma^\omega \mid \#_{ab}(\alpha) = \infty\}} \\
%                                   & \uncover<3->{= \{\alpha \mid \#_{a}(\alpha) = \#_{b}(\alpha) = \infty\}}
%             \end{align*}
%           \end{minipage}
% 
%           \par\bigskip
%           \uncover<4->{$\Aut{A}_5$: \raisebox{-6mm}{\Fig{41}}}
%           \hfill
%           \begin{minipage}{.54\textwidth}
%             $\uncover<4->{L_\omega(\Aut{A}_5) = }\uncover<5->{\{\alpha \mid \#_a(\alpha) = \infty\}}$
%           \end{minipage}
%           \par\smallskip
%           \uncover<6->{%
%             (Idee:~ $q_1$ kann nur erreicht werden, wenn ein $a$ gelesen wird)%
%           }
% 
%           \par\bigskip
%           \uncover<7->{$\Aut{A}_6$: \raisebox{-6mm}{\Fig{42}}}
%           \hfill
%           \begin{minipage}{.54\textwidth}
%             $\uncover<7->{L_\omega(\Aut{A}_6) = }\uncover<8->{\{\alpha \mid \#_{bb}(\alpha) = \infty\}}$
%           \end{minipage}
%           \par\smallskip
%           \uncover<9->{%
%             (Idee:~ $q_0$ nur durch $bb$ erreichbar;~ jeder Teilstring $bb$ führt zu $q_0$)%
%           }
%         \par
%         \end{small}
%       \end{exampleblock}
%       \note{~}
%     \end{frame}
% 
    % ------------------------------------------------------------------------------------------
    \begin{frame}
      \frametitle{Mehr Beispiele}

      \begin{exampleblock}{}
        \begin{small}
          $\Aut{A}_4$: \raisebox{-8mm}{\Fig{50}}
          \hfill
          \begin{minipage}[t]{.54\textwidth}
            \vspace*{-1.6\baselineskip}
            \begin{align*}
              L_\omega(\Aut{A}_4) & = \uncover<2->{\{\alpha\in\Sigma^\omega \mid \#_{a}(\alpha) < \infty} \\
                                  & \uncover<2->{\qquad\qquad \text{oder } \#_b(\alpha) < \infty\}}
            \end{align*}
          \end{minipage}

          \par\bigskip
          \uncover<3->{$\Aut{A}_5$: \raisebox{-34mm}{\Fig{51}}}
%           \hspace*{-20mm}%
          \hfill
          \begin{minipage}[t]{.54\textwidth}
            \vspace*{-1.6\baselineskip}
            \begin{align*}
              \uncover<3->{L_\omega(\Aut{A}_5) =}~ & \uncover<4->{\{\alpha\in\Sigma^\omega \mid \#_{a}(\alpha) < \infty} \\
                                                   & \uncover<4->{\qquad~~ \text{oder } \#_{aa}(\alpha) = 0\}}
            \end{align*}
            \uncover<5->{%
              \begin{footnotesize}%
                \hspace*{\fill}(Letzteres heißt:~~~~\\
                \hspace*{\fill}auf jedes $a$ in $\alpha$ folgt direkt ein $b$)~~~~%
                \par
              \end{footnotesize}%
            }

          \end{minipage}
%           \par\vspace*{-\baselineskip}
%           \uncover<5->{%
%             \hspace{\fill}{\footnotesize (letzteres heißt: auf jedes $a$ in $\alpha$ folgt direkt ein $b$)}%
%           }

%           \par\bigskip
%           \uncover<7->{$\Aut{A}_6$: \raisebox{-6mm}{\Fig{42}}}
%           \hfill
%           \begin{minipage}{.54\textwidth}
%             $\uncover<7->{L_\omega(\Aut{A}_6) = }\uncover<8->{\{\alpha \mid \#_{bb}(\alpha) = \infty\}}$
%           \end{minipage}
%           \par\smallskip
%           \uncover<9->{%
%             (Idee:~ $q_0$ nur durch $bb$ erreichbar;~ jeder Teilstring $bb$ führt zu $q_0$)%
%           }
        \par
        \end{small}
      \end{exampleblock}

      \note{
        \uz{9:14 bis 9:20; 5\,min Pause}
        
        \par
      }
    \end{frame}

    % ------------------------------------------------------------------------------------------
    \begin{frame}
      \frametitle{Erkennbare Sprache}
      
      \begin{Definition}
        Eine Sprache $L \subseteq \Sigma^\omega$ ist \Bmph{Büchi-erkennbar}, \\
        wenn es einen NBA \Aut{A} gibt mit $L = L_\omega(\Aut{A})$.
      \end{Definition}

      \note{
        \uz{9:20}
        
        \par
      }
    \end{frame}

  


  % ==============================================================================================
  % ==============================================================================================
  \section[Abschlusseig.]{Abschlusseigenschaften}
  
  % ------------------------------------------------------------------------------------------
    \begin{frame}
      \frametitle{Operationen auf $\omega$-Sprachen}

      Zur Erinnerung: die Menge der Büchi-erkennbaren Sprachen heißt\\
      abgeschlossen unter
      \begin{Itemize}
        \item
          \Bmph{Vereinigung}, wenn gilt:
          \par\smallskip
          Falls $L_1,L_2$ Büchi-erkennbar, so auch $L_1 \cup L_2$.
        \item
          \Bmph{Schnitt}, wenn gilt:
          \par\smallskip
          Falls $L_1,L_2$ Büchi-erkennbar, so auch $L_1 \cap L_2$.
        \item
          \Bmph{Komplement}, wenn gilt: \hfill
%           \par\smallskip
          Falls $L$ Büchi-erkennbar, so auch $\overline{L}$.
      \end{Itemize}

      \par\smallskip
      \begin{alertblock}<2->{Quiz}
%       \uncover<2->{%
%         \Bmph{Quiz:}
        Unter welchen Operationen sind die Büchi-erkennbaren Sprachen abgeschlossen,
        und wie leicht ist das zu zeigen?
        \begin{center}
          \begin{tabular}{lll}
            Vereinigung? & \uncover<3->{\YES} & \uncover<3->{{\small (leicht)}} \\
            Schnitt?     & \uncover<4->{\YES} & \uncover<4->{{\small (mittel)}} \\
            Komplement?  & \uncover<5->{\YES} & \uncover<5->{{\small (schwer)}}
          \end{tabular}
        \end{center}
      \end{alertblock}
%       }
      \note{
        \uz{9:21}
        
        \par
      }
    \end{frame}

    % ------------------------------------------------------------------------------------------
    \begin{frame}
      \frametitle{Abgeschlossenheit}

      \begin{Satz}
        Die Menge der Büchi-erkennbaren Sprachen ist abgeschlossen unter den Operationen
        $\cup$ und $\cap$.
        \label{thm:abgeschlossenheit_v+d}
      \end{Satz}

      \par\medskip
      Direkte Konsequenz aus den folgenden Lemmata.
      
      \par\bigskip
      Abgeschlossenheit unter $\overline{\phantom{o}}$ : siehe Abschnitt "`Determinisierung"'

      \note{
        \uz{9:23}
        
        \par
      }
    \end{frame}

    % ------------------------------------------------------------------------------------------
    \begin{frame}
      \frametitle{Abgeschlossenheit unter Vereinigung}
      
      \begin{Lemma}
        Seien $\Aut{A}_1,\Aut{A}_2$ NBAs über $\Sigma$.\\
        Dann gibt es einen NBA $\Aut{A}_3$ mit $L_\omega(\Aut{A}_3) = L_\omega(\Aut{A}_1) \cup L_\omega(\Aut{A}_2)$.
        \label{lem:abgeschlossenheit_vereinigung}%
      \end{Lemma}
      
      \par\medskip
      \uncover<2->{
        \Bmph{Beweis.}~ analog zu NEAs und NEBAs:
        \par\medskip
        Seien $\Aut{A}_i = (Q_i, \Sigma, \Delta_i, I_i, F_i)$ für $i=1,2$.
        \par
        O.\,B.\,d.\,A.\ gelte $Q_1 \cap Q_2 = \emptyset$.
        \par\medskip
        Konstruieren $\Aut{A}_3 = (Q_3, \Sigma, \Delta_3, I_3, F_3)$ wie folgt.%
      }
      %
      \uncover<3->{
        \begin{Itemize}
          \item
            $Q_3 = Q_1 \cup Q_2$
          \item
            $\Delta_3 = \Delta_1 \cup \Delta_2$
          \item
            $I_3 = I_1 \cup I_2$
          \item
            $F_3 = F_1 \cup F_2$
        \end{Itemize}
      }
      
      \par\smallskip
      \uncover<4->{
        Dann gilt:~ $L_\omega(\Aut{A}_3) = L_\omega(\Aut{A}_1) \cup L_\omega(\Aut{A}_2)$
        \qed%
      }

      \note{
        \uz{9:24}
        
        \par
      }
    \end{frame}

    % ------------------------------------------------------------------------------------------
    \begin{frame}
      \frametitle{Abgeschlossenheit unter Schnitt}
      
      \Bmph{Für NEAs: Produktautomat}
      
      \par\smallskip
      Idee: lasse $\Aut{A}_1$ und $\Aut{A}_2$ "`gleichzeitig"' auf Eingabewort laufen.
      
      \par\smallskip
      Gegeben $\Aut{A}_1,\Aut{A}_2$,
      konstruiere $\Aut{A}_3$ mit $L(\Aut{A}_3) = L(\Aut{A}_1) \cap L(\Aut{A}_2)$:
      \begin{Itemize}
        \item
          $Q_3 = Q_1 \times Q_2$
        \item
          $\Delta_3 = \{((p,p'),a,(q,q')) \mid (p,a,q) \!\in\! \Delta_1 ~\&~ (p',a,q') \!\in\! \Delta_2\}$
        \item
          $I_3 = I_1 \times I_2$
        \item
          $F_3 = F_1 \times F_2$
          \Tafel
      \end{Itemize}
      
      \par\bigskip
      \uncover<2->{%
        \Bmph{Funktioniert das auch für Büchi-Automaten?}
      }
      
      \par\bigskip
      \uncover<3->{%
        \begin{tabular}[b]{@{}l@{~}l@{}}
          \Emph{Nein.} & $\Aut{A}_1$ und $\Aut{A}_2$ besuchen ihre akzeptierenden Zustände\\
                       & möglicherweise nicht synchron! %\\[4pt]
%                       & Beispiel siehe Tafel.
                         \TafelForts
        \end{tabular}
%        \Tafel
      }

      \note{
        \uz{9:25 bis 9:35?}
        
        \par
      }
    \end{frame}

    % ------------------------------------------------------------------------------------------
    \begin{frame}
      \frametitle{Abgeschlossenheit unter Schnitt}
      
      \Gmph{Neue Idee} für Schnitt-Automat $\Aut{A}$\,:
      %
      \begin{Itemize}
        \item
          $\Aut{A}$ simuliert $\Aut{A}_1,\Aut{A}_2$ nach wie vor parallel,
          aber mit \Bmph{2 Modi $1,2$}
          \par\smallskip
        \item
          Modus $i$ bedeutet:~ warte auf einen akz.\ Zustand $f$ von $\Aut{A}_i$
          \par\smallskip
        \item
          Sobald so ein $f$ erreicht ist, wechsle den Modus.
          \par\smallskip
        \item
          Run von $\Aut{A}$ ist erfolgreich, wenn er $\infty$ oft den Modus wechselt.
          \par\smallskip
        \item[$\leadsto$]
          Es werden genau die Wörter akzeptiert,\\
          für die $\Aut{A}_1,\Aut{A}_2$ jeweils einen erfolgreichen Run haben.
      \end{Itemize}
    
      \note{
        \uz{9:35}
        
        \par
      }
    \end{frame}

    \newcommand{\ONE}{\text{\Emph{1}}}
    \newcommand{\TWO}{\text{\Emph{2}}}
    \newcommand{\IN}[2]{\text{\Emph{$p#1\!\in\!F_{#2}$}}}
    \newcommand{\NIN}[2]{\text{\Emph{$p#1\!\notin\!F_{#2}$}}}
    % ------------------------------------------------------------------------------------------
    \begin{frame}
      \frametitle{Abgeschlossenheit unter Schnitt}
      
      \begin{Lemma}
        Seien $\Aut{A}_1,\Aut{A}_2$ NBAs über $\Sigma$.\\
        Dann gibt es einen NBA $\Aut{A}$ mit $L_\omega(\Aut{A}) = L_\omega(\Aut{A}_1) \cap L_\omega(\Aut{A}_2)$.
      \end{Lemma}
      
      \par\smallskip
      \uncover<2->{%
        \Bmph{Beweis:}~
        Seien $\Aut{A}_i = (Q_i, \Sigma, \Delta_i, I_i, F_i)$ NBAs für $i=1,2$.

        \par\smallskip
        Konstruieren $\Aut{A} = (Q, \Sigma, \Delta, I, F)$ wie folgt.
        %
        \begin{align*}
          Q      & = Q_1 \times Q_2 \times \{1,2\} \\
          \Delta & = \{((p,\!\!\;p'\!,\!\!\;\ONE),a,(q,\!\!\;q'\!,\!\!\;\ONE)) \mid \NIN{}{1} ~\&~ (p,\!\!\;a,\!\!\;q) \!\in\! \Delta_1 ~\&~ (p'\!,\!\!\;a,\!\!\;q') \!\in\! \Delta_2\} \\
          \uncover<3->{%
                 & \:\cup\!\!\: \{((p,\!\!\;p'\!,\!\!\;\ONE),a,(q,\!\!\;q'\!,\!\!\;\TWO)) \mid \IN{}{1} ~\&~ (p,\!\!\;a,\!\!\;q) \!\in\! \Delta_1 ~\&~ (p'\!,\!\!\;a,\!\!\;q') \!\in\! \Delta_2\} \\
          }%
          \uncover<4->{%
                 & \:\cup\!\!\: \{((p,\!\!\;p'\!,\!\!\;\TWO),a,(q,\!\!\;q'\!,\!\!\;\TWO)) \mid \NIN{'}{2} ~\&~ (p,\!\!\;a,\!\!\;q) \!\in\! \Delta_1 ~\&~ (p'\!,\!\!\;a,\!\!\;q') \!\in\! \Delta_2\} \\
          }%
          \uncover<5->{%
                 & \:\cup\!\!\: \{((p,\!\!\;p'\!,\!\!\;\TWO),a,(q,\!\!\;q'\!,\!\!\;\ONE)) \mid \IN{'}{2} ~\&~ (p,\!\!\;a,\!\!\;q) \!\in\! \Delta_1 ~\&~ (p'\!,\!\!\;a,\!\!\;q') \!\in\! \Delta_2\} \\
          }%
          \uncover<6->{%
            I    & = I_1 \times I_2 \times \{1\} \\
            F    & = Q_1 \times F_2 \times \{2\}
          }%
        \end{align*}
      }
      \par\vspace*{-2.4\baselineskip}
      \uncover<6->{%
        \Tafel
      }

      \par\medskip
      \uncover<7->{%
        Dann gilt $L_\omega(\Aut{A}) = L_\omega(\Aut{A}_1) \cap L_\omega(\Aut{A}_2)$. \hfill\hfill\TafelForts \qed%
      }

      \note{
        \uzz{16:00}{9:37 bis 9:59?}
        
        \parII
        Kurze Wdhlg.: haben Büchi-Automaten eingeführt, Abschlusseigenschaften behandelt.
        
        \parI
        Vereinigung war einfach; Schnitt ist komplizierter: \\
        2 Kopien des "`alten"' Produktaut., weil akz.\ Zust.\ asynchron auftreten können

        \parI
        Konstruktion auf Folie; jetzt noch 2.\ Richtung des Korrektheitsbeweises
        
        \parII
        \textbf{Beweis bis 16:20}
        
        \par
      }
    \end{frame}

    % ------------------------------------------------------------------------------------------
    \begin{frame}
      \frametitle{Abgeschlossenheit unter Komplement}

      \dots\ siehe Abschnitt\\
      "`Deterministische Büchi-Automaten und Determinisierung"'

      \note{
        \uzz{16:20}{9:59--10:00, Ende}

        \par
      }
    \end{frame}



  
  % ==============================================================================================
  % ==============================================================================================
  \section[Charakt.]{Charakterisierung}
  
    % ------------------------------------------------------------------------------------------
    \begin{frame}
      \frametitle{Ziel}

      \Bmph{Ziel dieses Abschnitts}
      \par\smallskip
      Charakterisierung der Büchi-erkennbaren Sprachen\\
      mittels regulärer Sprachen

      \par\bigskip
      \uncover<2->{%
        \Bmph{Etwas Notation}
        \par\smallskip
        Seien $W \subseteq \Sigma^*$ und $L \subseteq \Sigma^\omega$.
        \begin{Itemize}
          \item
            \Bmph{$W^\omega$} $=$ $\{w_0w_1w_2 \dots \mid w_i \in W\setminus\{\varepsilon\} \text{~für alle~} i \geqslant 0\}$
            \parI
            {\small (ist $\omega$-Sprache, weil $\varepsilon$ ausgeschlossen wurde)}
            \parII
          \item
            \Bmph{$WL$} $=$ $\{w\alpha \mid w \in W,~ \alpha \in L\}$
            \parI
            {\small (ist $\omega$-Sprache)}
        \end{Itemize}
      }
      \note{
        \textbf{16:20}
        
        \parI
        $W^{\omega}$ entspricht dem Kleene-Stern bei Sprachen endlicher Wörter; \\
        $WL$ entspricht der Konkatenation.

        \par
      }
    \end{frame}

  % ------------------------------------------------------------------------------------------
    \begin{frame}[t]
      \frametitle{Von regulären zu Büchi-erkennbaren Sprachen (1)}

      \begin{lemma}
        Für jede reguläre Sprache $W \subseteq \Sigma^*$ gilt:~
        $W^\omega$ ist Büchi-erkennbar.%
        \label{lem:Charakt_Buchi_hoch_omega}%
      \end{lemma}

      \parII%\bigskip
      \uncover<2->{%
        \Bmph{Beweis. (Schritt 1)}

        \par\smallskip
        Sei $\Aut{A}$ ein \Emph{NEA} mit $L(\Aut{A}) = W$.
      }
    
      \parI
      \uncover<3->{
        Dann gibt es NEA $\Aut{A}_1$ mit $L(\Aut{A}_1) = W\!\setminus\!\{\varepsilon\}$ (Abschlusseig.!)
      }
        
      \parI
      \uncover<4->{
        O.\,B.\,d.\,A.\ habe $\Aut{A}_1$ \dots
        %
        \begin{Enumerate}
          \item
            einen \Emph{einzigen} Anfangszustand $q_I$\quad und
          \item
            \Emph{keine} in $q_I$ \Emph{eingehenden} Kanten:~ keine Transitionen $(\cdot,\cdot,q_I$)
          \item
            und sei $q_I \notin F$.
        \end{Enumerate}
      }
        
      \parI
      \uncover<5->{
        Diese Form lässt sich durch Hinzufügen eines frischen Anfangs-\\
        zustandes (und der entsprechenden Transitionen) erreichen! (Ü)
      }
    
      \note{
        \textbf{16:22}
        
        \parI
        $\Big($Idee:~ Füge neuen Startzustand hinzu; dupliziere alle Kanten, die von bisherigen SZen ausgehen.
        
        \parI
        Damit sind (1) und (2) erreicht.~
        (3) ist korrekt, weil $\varepsilon \notin L(\Aut{A}_1)$.$\Big)$
        
        \parIII
        $\Big($Gegenbeispiel für die Rückrichtung:
        
        \parI
        $W = \{a^n \mid n \text{~ist prim}\}$
        
        \parI
        $\Rightarrow~ W^\omega = \{a^\omega\}$\qquad (und $W^* = \{a^n \mid n \geq 2\}$)$\Big)$

        \par        
      }
    \end{frame}

    \addtocounter{theorem}{-1}
    % ------------------------------------------------------------------------------------------
    \begin{frame}[t]
    \frametitle{Von regulären zu Büchi-erkennbaren Sprachen (1)}
    
      \begin{lemma}
        Für jede reguläre Sprache $W \subseteq \Sigma^*$ gilt:~
        $W^\omega$ ist Büchi-erkennbar.%
%        \label{lem:Charakt_Buchi_hoch_omega}
      \end{lemma}
    
      \parII
      \Bmph{Beweis. (Schritt 2a)}
      
      \parI
      Sei also $\Aut{A}_1 = (Q_1, \Sigma, \Delta_1, \{q_I\}, F)$ mit den genannten Eigenschaften
      und $L(\Aut{A}_1) = W\!\setminus\!\{\varepsilon\}$.
      
      \parII
      \uncover<2->{%
        \Bmph{Idee:}~ konstruiere \Bmph{NBA $\Amcb_2$}\,, der
        \begin{Itemize}
          \item
            $\Aut{A}_1$ simuliert, bis ein akzeptierender Zustand erreicht ist und
          \item
            dann \Emph{nichtdeterministisch entscheidet,} \\
            ob die Simulation fortgesetzt wird \\
            oder eine neue Simulation von $q_0$ aus gestartet wird
        \end{Itemize}
      }

      \note{
        \textbf{16:24}
        
        \parI
        Letztlich ist das dieselbe Idee wie bei der Kleene-Abg.\ der NEA-erkennbaren Sprachen: \\
        erzeuge Kreis, der $\infty$ oft durchlaufen werden kann.
        
        \parI
        Die kann man aber nicht so leicht auf Büchiaut.\ übertragen,
        denn sie führt $\varepsilon$-Kanten ein, und diese kann man innerhalb von Kreisen
        nicht so leicht eliminieren wie bei NEAs
        (ehemals akz.\ Zustände könnten keine ausgehenden Kanten mehr haben, also gehen erfolgr.\ Runs verloren \dots)
        

        \par
      }
    \end{frame}

    \addtocounter{theorem}{-1}
    % ------------------------------------------------------------------------------------------
    \begin{frame}[t]
      \frametitle{Von regulären zu Büchi-erkennbaren Sprachen (1)}
      
      \begin{lemma}
        Für jede reguläre Sprache $W \subseteq \Sigma^*$ gilt:~
        $W^\omega$ ist Büchi-erkennbar.%
        %        \label{lem:Charakt_Buchi_hoch_omega}
      \end{lemma}
      
      \parII
      \Bmph{Beweis. (Schritt 2b)}
      
      \parI
      Sei also $\Aut{A}_1 = (Q_1, \Sigma, \Delta_1, \{q_I\}, F)$ mit den genannten Eigenschaften
      und $L(\Aut{A}_1) = W\!\setminus\!\{\varepsilon\}$.
      
      \parII
      Definiere NBA $\Bmph{$\Amcb_2$} = (Q_1, \Sigma, \Bmph{$\Delta_2$}, \{q_I\}, \Emph{$\{q_I\}$})$ mit
      %
      \begin{center}
        \parbox{.85\linewidth}{%
          \uncover<2->{%
            $\Bmph{$\Delta_2$} \,=\, \Delta_1 \cup \{(q,a,q_I) \mid (q,a,q_f) \in \Delta_1 \text{~für ein~} q_f \in F\}$
          }%
          
          \parII
          \uncover<3->{%
            \begin{footnotesize}
              (d.\,h.\ alle Kanten, die in $\Aut{A}_1$ zu einem akz.\ Zustand führen, \\
              können in $\Aut{A}_2$ zusätzlich zu $q_I$ führen \\
              -- siehe "`nichtdeterministisch entscheidet"' auf voriger Folie!)
              \par
            \end{footnotesize}
          }
        }
      \end{center}
      %
      
      \uncover<4->{%
        \Bmph{Noch zu zeigen:}~
        $L_\omega(\Aut{A}_2) = L(\Aut{A}_1)^\omega$
        \Tafel~~~~~
        \par\vspace*{-.95\baselineskip}
        \qed
      }
      
      \note{
        \textbf{16:26 bis 16:45 $\to$ 5min Pause}
        
        \par
      }
    \end{frame}

  % ------------------------------------------------------------------------------------------
    \begin{frame}
      \frametitle{Von regulären zu Büchi-erkennbaren Sprachen (2)}

      \begin{lemma}
        \label{lem:Charakt_Buchi_Konkat}
        ~\par\vspace*{-1.4\baselineskip}
        Für jede reguläre Sprache $W \subseteq \Sigma^*$\\
        und jede Büchi-erkennbare Sprache $L \subseteq \Sigma^\omega$ gilt:
        \par\smallskip
        $WL$ ist Büchi-erkennbar.
      \end{lemma}

      \par\bigskip
      \uncover<2->{%
        \Bmph{Beweis:}
        \par\smallskip
        Wie Abgeschlossenheit der regulären Sprachen unter Konkatenation.
        \qed
      }

      \note{
        \textbf{16:50}

        \par
      }
    \end{frame}

  % ------------------------------------------------------------------------------------------
    \begin{frame}
      \frametitle{Satz von Büchi}

      \begin{Satz}
        \label{thm:Charakt_Buchi}
        ~\par\vspace*{-1.4\baselineskip}
        Eine Sprache $L \subseteq \Sigma^\omega$ ist Büchi-erkennbar
        genau dann,\\
        wenn es reguläre Sprachen $V_1,W_1,\dots,V_n,W_n$ gibt mit $n \geqslant 1$ und

        \par\medskip
        \centerline{$L = V_1W_1^\omega \cup \dots \cup V_nW_n^\omega$}
%         \[
%           L = V_1W_1^\omega \cup \dots \cup V_nW_n^\omega
%         \]
      \end{Satz}

      \par\smallskip
      \uncover<2->{%
        \Bmph{Beweisskizze:}
        \only<2|handout:0>{\hfill (Quiz: Welche der Richtungen $\Rightarrow$, $\Leftarrow$ ist leichter?)}
      }

      \par\smallskip
      \uncover<3->{%
        ``$\Leftarrow$'': folgt aus Lemmas~\ref{lem:abgeschlossenheit_vereinigung}, \ref{lem:Charakt_Buchi_hoch_omega} und~\ref{lem:Charakt_Buchi_Konkat}

        \par\medskip
        ``$\Rightarrow$'': bilden $V_i,W_i$ aus denjenigen Wörtern,
        die zum jeweils nächsten Vorkommen eines akzeptierenden Zustandes führen

        \par\medskip
        Details siehe Tafel. \Tafel~~~~
        \par\vspace*{-.95\baselineskip}
        \qed
      }
      
      \par\bigskip
      \uncover<4->{%
        \Bmph{Konsequenz:}
        \par\smallskip
        Büchi-erkennbare Sprachen durch
        \Bmph{$\omega$-reguläre Ausdrücke} darstellbar:
        \par\medskip
        \centerline{%
          $r_1s_1^\omega + \dots + r_ns_n^\omega$
          \qquad\qquad
          ($r_i,s_i$ sind reguläre Ausdrücke)%
        }%
%         \[
%           r_1s_1^\omega + \dots + r_ns_n^\omega
%           \qquad\qquad
%           \text{($r_i,s_i$ sind reguläre Ausdrücke)}%
%         \]
      }
      \note{
        \textbf{16:52 bis 17:02}
        
        \parI
        TODO: Tafelanschrieb auf Folie; hier passiert nix Anstrengendes.
        
        \par
      }
    \end{frame}




  % ==============================================================================================
  % ==============================================================================================
  \section[Determinismus]{Deterministische Büchi-Automaten und Determinisierung}
  
    % ------------------------------------------------------------------------------------------
    \begin{frame}
      \frametitle{Ziel dieses Abschnitts}
      
      \Bmph{Wollen zeigen:}
      \begin{Itemize}
        \item
          det.\ und nichtdet.\ Büchi-Automaten sind
          \Emph{nicht} gleichmächtig
          \begin{Itemize}
            \item[]
              d.\,h.: es gibt $\omega$-Sprachen, die von NBAs akzeptiert werden,\\
              aber nicht von DBAs              
          \end{Itemize}
%           \par\smallskip
        \item
          Komplement-Abgeschlossenheit gilt trotzdem
          \begin{Itemize}
            \item[]
              (der Beweis wird aber anspruchsvoll sein)
          \end{Itemize}
      \end{Itemize}

      \par\bigskip
      \begin{Definition}<2->
        Ein \Bmph{deterministischer Büchi-Automat (DBA)} ist ein NBA $\Aut{A} = (Q,\Sigma,\Delta,I,F)$ mit
        \begin{Itemize}
          \item
            $|I|=1$
          \item
            $|\{q' \mid (q,a,q') \in \Delta\}| = 1$ für alle $(q,a) \in Q \times \Sigma$
        \end{Itemize}
      \end{Definition}

%       \uncover<2->{%
%         \par\medskip
%         \Bmph{Etwas Notation:}
%         \begin{Itemize}
%           \item
%             \Bmph{Deterministischer Büchi-Automat (DBA):} NBA $\Aut{A} = (Q,\Sigma,\Delta,I,F)$ mit
%             \begin{Itemize}
%               \item
%                 $|I|=1$
%               \item
%                 $|\{q' \mid (q,a,q') \in \Delta\}| = 1$ für alle $(q,a) \in Q \times \Sigma$
%             \end{Itemize}
%           \par\smallskip
%           \item<3->
%             Sei $W \subseteq \Sigma^*$.
%             \par\smallskip
%             \Bmph{$\Vec W$} $=$ $\{w \in \Sigma^\omega \mid w[0,n] \in W \text{~für unendlich viele $n$}\}$\\
%             \hspace*{29.5mm}{\small (d.\,h.\ $w$ hat $\infty$ viele Präfixe in $W$)}
%             \par\smallskip
%             \uncover<4->{%
%               \Gmph{Beispiel:} siehe Tafel \Tafel
%             }
%         \end{Itemize}
%       }
      \note{
        \textbf{17:02}
        
        \parI
        "`nicht gleichmächtig"':~ Überraschung! :)
        
        \parIII
        Beachte: hier wieder "`genau ein"'\\
        (Papierkörbe sind wieder einfach, wie bei DEAs)
        
        \par
      }
    \end{frame}

  % ------------------------------------------------------------------------------------------
    \begin{frame}
      \frametitle{Zu Hilfe: Charakterisierung der DBA-erkennbaren Sprachen}

      Sei $W \subseteq \Sigma^*$.
      \par\smallskip
      \Bmph{$\Vec W$} $=$ $\{\alpha \in \Sigma^\omega \mid \alpha[0,n] \in W \text{~für unendlich viele $n$}\}$\\
      \hspace*{29.5mm}{\small (d.\,h.\ $\alpha$ hat $\infty$ viele Präfixe in $W$)} \Tafel

      \par\bigskip
      \begin{Satz}<2->
        Eine $\omega$-Sprache $L \subseteq \Sigma^\omega$
        ist DBA-erkennbar genau dann,\\
        wenn es eine reguläre Sprache $W \subseteq \Sigma^*$ gibt
        mit $L = \Vec W$.
        \label{thm:Charakt_det_Buchi}
      \end{Satz}

      \par\smallskip
      \uncover<3->{%
        \Bmph{Beweis.}
        Genügt zu zeigen, dass \\
        für jeden \Emph{D}EA/\Emph{D}BA $\Aut{A}=(Q,\Sigma,\Delta,\{q_I\},F)$ gilt:
        \[
%           \underbrace{L_\omega(\Aut{A})}_{\text{als DBA}} = \underbrace{\Vec{L(\Aut{A})}}_{\text{als DEA}}
          L_\omega(\Aut{A}) = \Vec{L(\Aut{A})}
        \]

        \Tafel~~~~
        \par\vspace*{-.95\baselineskip}
        \qed
      }

      \note{
        \textbf{17:04}
        
        \parI
        Dazu zunächst auch eine Charakt.\ der DBA-erkennbaren Sprachen, \\
        die uns erlauben wird, NBAs und DBAs bezüglich der Mächtigkeit zu trennen.
        
        \parIII
        \textbf{T3.5 bis 17:14}
        
        \parIII
        \textbf{T3.6 bis 17:24}

        \par
      }
    \end{frame}

  % ------------------------------------------------------------------------------------------
    \begin{frame}
      \frametitle{DBAs sind schwächer als NBAs}
      
      \begin{Satz}
        Es gibt eine Büchi-erkennbare Sprache,\\
        die nicht durch einen DBA erkannt wird.
        \label{thm:DBAs_schwaecher_als_NBAs}
      \end{Satz}

      \par\bigskip
      \uncover<2->{%
        \Bmph{Beweis.}
        \begin{Itemize}
          \item
%             Betrachte $\Sigma = \{a,b\}$ und $L = \{\alpha \in \Sigma^\omega \mid \#_b(\alpha) \text{ ist endlich}\}$
            Betrachte $L = \{\alpha \in \{a,b\}^\omega \mid \#_a(\alpha) \text{ ist endlich}\}$
          \item
            $L$ ist Büchi-erkennbar:\quad
            \uncover<3->{$L = \Sigma^*\{b\}^\omega$, wende Satz \ref{thm:Charakt_Buchi} an}
          \item<4->
            Annahme, $L$ sei DBA-erkennbar.
            \begin{Itemize}
              \item[$\Rightarrow$]
                Satz \ref{thm:Charakt_det_Buchi}: $L = \Vec W$ für eine reguläre Sprache $W$
              \item<5->[$\Rightarrow$]
                Wegen $b^\omega \in L$ gibt es ein nichtleeres Wort $b^{n_1} \in W$
                \par\smallskip
                \uncover<6->{%
                  Wegen $b^{n_1}ab^\omega \in L$ gibt es ein nichtleeres Wort $b^{n_1}ab^{n_2} \in W$
                }
                \par\smallskip
                \uncover<7->{%
                  \quad\vdots
                }
              \item<8->[$\Rightarrow$]
                $\alpha ~:=~ b^{n_1}ab^{n_2}ab^{n_3}\dots ~\in~ \Vec W$
                \qquad
                \uncover<9->{\Emph{Widerspruch:} $\alpha \notin L$ \qed}
            \end{Itemize}
        \end{Itemize}
      }
      \note{%
        \textbf{17:24 bis 17:29}
        
        \parI
        \textbf{Am Anfang fragen:}~ Ideen für so eine Sprache?
        
        \par
      }
    \end{frame}

  % ------------------------------------------------------------------------------------------
    \begin{frame}
      \frametitle{Nebenprodukt des letzten Beweises}
      
      Die DBA-erkennbaren Sprachen sind \Emph{nicht} unter Komplement abgeschlossen:
      \begin{Itemize}
        \item
          $L = \{\alpha \in \{a,b\}^\omega \mid \#_a(\alpha) \text{ ist endlich}\}$\\
          wird von keinem DBA erkannt
        \item
          aber $\overline L$ wird von einem DBA erkannt (Ü)
      \end{Itemize}

      \note{%
        \textbf{17:29 bis 17:30 $\to$ hoffentlich Punktlandung!}
        
        \par
      }
    \end{frame}

  % ------------------------------------------------------------------------------------------
    \begin{frame}
      \frametitle{Wie können wir trotzdem determinisieren?}

      \uncover<2->{%
        \Emph{Indem wir das Automatenmodell ändern!}
        \par\smallskip
        Genauer: ändern die Akzeptanzbedingung%
      }

      \par\bigskip
      \uncover<3->{%
%         \Bmph{:}
        \begin{block}{Zur Erinnerung}
          \Bmph{NBA} ist 5-$\!$Tupel $\Aut{A} = (Q, \Sigma, \Delta, I, F)$ mit
          \begin{Itemize}
            \item
              \dots
            \item
              $F \subseteq Q$\quad (Menge der akz.\ Zustände)
          \end{Itemize}
          \par\medskip
          \Bmph{Erfolgreicher Run:} $r = q_0q_1q_2\dots$ ~mit~
          $q_0 \in I$ ~und~ $\Inf(r) \cap F \neq \emptyset$
        \end{block}
        
        \par\bigskip
        \Bmph{Idee:} $r$ erfolgreich $\Leftrightarrow$ ein Zustand aus $F$ kommt $\infty$ oft in $r$ vor

        \par\bigskip\bigskip
        \begin{footnotesize}
          \hspace*{\fill}
          (Julius Richard Büchi, 1924--1984, Logiker/Mathematiker; Zürich, Lafayette)
        \end{footnotesize}

      }
      \note{
        \textbf{8:30}
        
        \parI
        Erinnerung vom letzten Mal:~ haben DBAs eingeführt und gezeigt, \\
        dass sie weniger mächtig sind als NBAs \\
        (über versch.\ Charakterisierungen mittels regulärer Sprachen)
        
        \parI
        Heute:~ wollen geänderte Automatenmodelle einführen 
        und zeigen, dass ihre deterministischen Varianten
        genauso mächtig sind wie NBAs.
        
        \par
      }
    \end{frame}

  % ------------------------------------------------------------------------------------------
    \begin{frame}[t]
      \frametitle{Muller-Automaten \hfill {\footnotesize (David E.\ Muller, 1924--2008, Math./Inf.; Illinois)}}

      \begin{Definition}{}
        Nichtdet.\ \Bmph{Muller-Automat (NMA)} ist 5-$\!$Tupel $\Aut{A} = (Q, \Sigma, \Delta, I, \Uwave{\calF})$ mit
        \begin{Itemize}
          \item
            \dots
          \item
            $\calF \subseteq 2^Q$\quad (Kollektion von Endzustandsmengen)
        \end{Itemize}
        \par\medskip
        \Bmph{Erfolgreicher Run} $r = q_0q_1q_2\dots$ ~mit~
        $q_0 \in I$ ~und~ $\UWave{\Inf(r) \in \calF}$
      \end{Definition}

      \par\bigskip
      \Bmph{Idee:} $r$ erfolgreich $\Leftrightarrow$ $\Inf(r)$ stimmt mit einer Menge aus $\calF$ überein

      \par\bigskip
%      \Gmph{Beispiel:} Siehe Tafel 
      \Tafel

      \note{
        \textbf{8:32 bis 8:44}

        \par
      }
    \end{frame}

  % ------------------------------------------------------------------------------------------
    \begin{frame}[t]
      \frametitle{Rabin-Automaten \hfill \mbox{{\footnotesize (Michael O.\ Rabin, ${}^*$1931, Inf.; Jerusalem, Princeton, Harvard)}}\hspace*{-2.5mm}}

      \begin{Definition}{}
        Nichtdet.\ \Bmph{Rabin-Automat (NRA)} ist 5-$\!$Tupel $\Aut{A} = (Q, \Sigma, \Delta, I, \Uwave{\calP})$ mit
        \begin{Itemize}
          \item
            \dots
          \item
            $\calP = \{(E_1,F_1),~\dots,~(E_n,F_n)\} \text{~~mit~~} E_i,F_i \subseteq Q$
            \par\smallskip
            (Menge "`akzeptierender Paare"')
        \end{Itemize}
        \par\medskip
        \Bmph{Erfolgreicher Run} $r = q_0q_1q_2\dots$~mit~
        $q_0 \in I$ ~und
        \[
          \exists i \in\{1,\dots,n\}
          \text{~~~mit~~~}
          \Inf(r) \cap E_i = \emptyset
          \text{~~~und~~~}
          \Inf(r) \cap F_i \neq \emptyset
        \]
      \end{Definition}

      \par\bigskip
      \Bmph{Idee:} $r$ erfolgreich $\Leftrightarrow$ es gibt Paar $(E_i,F_i)$, so dass
      \begin{Itemize}
        \item
          \Emph{mindestens ein} Zustand aus $F_i$ unendlich oft in $r$ vorkommt \&
        \item
          \Emph{alle} Zustände aus $E_i$ nur endlich oft in $r$ vorkommen
%          \hfill \Gmph{(Bsp. \Tafel)}
          \Tafel
      \end{Itemize}

      \note{
        \textbf{8:44 bis 8:54}
        
        \par
      }
    \end{frame}

  % ------------------------------------------------------------------------------------------
    \begin{frame}[t]
      \frametitle{Streett-Automaten \hfill {\footnotesize (Robert S.\ Streett, ?; Boston, Oakland)}}

      \begin{Definition}{}
        \scalebox{.96}[1]{Nichtdet.\ \Bmph{Streett-Automat (NSA)} ist 5-$\!$Tupel $\Aut{A} = (Q, \Sigma, \Delta, I, \Uwave{\calP})$ mit}
        \begin{Itemize}
          \item
            \dots
          \item
            $\calP = \{(E_1,F_1),~\dots,~(E_n,F_n)\} \text{~~mit~~} E_i,F_i \subseteq Q$
            \par\smallskip
            (Menge "`fairer Paare"')
        \end{Itemize}
        \par\medskip
        \Bmph{Erfolgreicher Run} $r = q_0q_1q_2\dots$~mit~
        $q_0 \in I$ ~und
        \[
          \Uwave{\forall} i \in\{1,\dots,n\}:
          \text{~~wenn~~}
          \Inf(r) \cap F_i \neq \emptyset,
          \text{~~dann~~}
          \Inf(r) \cap E_i \neq \emptyset
        \]
        \vspace*{-22pt}
      \end{Definition}

      \par\bigskip
      \Bmph{Idee:} $r$ erfolgreich $\Leftrightarrow$ \Emph{für alle} Paare $(E_i,F_i)$ gilt:
      \begin{Itemize}
        \item
          \Emph{wenn} ein Zustand aus $F_i$ unendlich oft in $r$ vorkommt,
        \item
          \Emph{dann} kommt ein Zustand aus $E_i$ unendlich oft in $r$ vor
%          \hfill \Gmph{(Bsp. \Tafel)}
          \Tafel
      \end{Itemize}

      \note{
        \textbf{8:54 bis 9:06}
        
        \par
      }
    \end{frame}

  % ------------------------------------------------------------------------------------------
    \begin{frame}
      \frametitle{Gleichmächtigkeit der vier Automatenmodelle}

      Für $X \in \{\text{Muller},\text{Rabin},\text{Streett}\}$ werden analog definiert:
      \begin{Itemize}
        \item
          \Bmph{$L_\omega(\Amcb)$} für (nichtdeterministische) $X$-Automaten
        \item
          \Bmph{$X$-erkennbar}
      \end{Itemize}

      \par\smallskip
      \begin{Satz}<2->
        Für jede Sprache $L \subseteq \Sigma^\omega$ sind die folgenden Aussagen äquivalent.
        \par\smallskip
%         \begin{tabular}{@{\quad}l@{~~}l@{}}
%           \ddblu{\textup{(B)}} & $L$ ist Büchi-erkennbar. \\
%           \ddblu{\textup{(M)}} & $L$ ist Muller-erkennbar. \\
%           \ddblu{\textup{(R)}} & $L$ ist Rabin-erkennbar. \\
%           \ddblu{\textup{(S)}} & $L$ ist Streett-erkennbar.
%         \end{tabular}
        \begin{tabular}{@{\quad}l@{~~}l@{\qquad}l@{~~}l@{}}
          \ddblu{\textup{(B)}} & $L$ ist Büchi-erkennbar.  & \ddblu{\textup{(R)}} & $L$ ist Rabin-erkennbar.   \\
          \ddblu{\textup{(M)}} & $L$ ist Muller-erkennbar. & \ddblu{\textup{(S)}} & $L$ ist Streett-erkennbar.
        \end{tabular}
        \label{thm:gleichmaechtigkeit}
      \end{Satz}

      \parIII
      \uncover<3->{%
        \Bmph{Beweis:}~
        Konsequenz aus Lemmas~\ref{lem:BRStoM}--\ref{lem:MtoB}.~~~ \raisebox{-1.2mm}{\turnbox{90}{$\hookleftarrow$}}
        \Tafel
        \qed
      }

      \note{
        \textbf{9:06}
        
        \par
      }
    \end{frame}
    
    % ------------------------------------------------------------------------------------------
    \begin{frame}[t]
      \frametitle{Von B-, R-, S- zu Muller-Automaten}
      
      \begin{lemma}
        \begin{Enumerate}
          \item
            Wenn $L$ Büchi-erkennbar, dann auch Muller-erkennbar.
          \item
            Wenn $L$ Rabin-erkennbar, dann auch Muller-erkennbar.
          \item
            Wenn $L$ Streett-erkennbar, dann auch Muller-erkennbar.
        \end{Enumerate}%
        \label{lem:BRStoM}%
      \end{lemma}
    
      \parII
      \uncover<2->{%
        \Bmph{Beweis.}
        
        \parI
        \Bmph{(1)}~ Sei $\Amc = (Q,\Sigma,\Delta,I,F)$ NBA.
        
        \parI
        Konstruiere NMA $\Amc' = (Q,\Sigma,\Delta,I,\Fmc)$ mit
        
        \begin{center}
          $\Fmc = \{Q' \subseteq Q \mid Q' \cap F \neq \emptyset\}$.
        \end{center}
      
        Leicht zu sehen:~ $L_\omega(\Amc') = L_\omega(\Amc)$.
      }
      
      \note{
        \textbf{9:08}
        
        \parI
        Idee:~ Kodiere $F$ in \Fmc. \\
        Die $Q'$ sind alle erlaubten Unendlichkeitsmengen $\textsf{Inf}(r)$.

        \par
      }
    \end{frame}

    % ------------------------------------------------------------------------------------------
    \begin{frame}[t]
      \frametitle{Von B-, R-, S- zu Muller-Automaten}
      
      \begin{block}{Lemma \ref{lem:BRStoM}}
        \begin{Enumerate}
          \item
          Wenn $L$ Büchi-erkennbar, dann auch Muller-erkennbar.
          \item
          Wenn $L$ Rabin-erkennbar, dann auch Muller-erkennbar.
          \item
          Wenn $L$ Streett-erkennbar, dann auch Muller-erkennbar.
        \end{Enumerate}%
        \label{lem:BRStoM_dummy}%
      \end{block}
      
      \parII
      \Bmph{Beweis.}
      
      \parI
      \Bmph{(2)}~ Sei $\Amc = (Q,\Sigma,\Delta,I,\Pmc)$ NRA.
      
      \parI
      Konstruiere NMA $\Amc' = (Q,\Sigma,\Delta,I,\Fmc)$ mit
      
      \begin{center}
        $\Fmc \,=\, \{Q' \!\!\:\subseteq\!\!\: Q \,\mid\, \exists i \!\!\:\leq\!\!\: n : Q' \!\!\:\cap\!\!\: E_i = \emptyset \text{~und~} Q' \!\!\:\cap\!\!\: F_i \neq \emptyset\}$.
      \end{center}
      
      Leicht zu sehen:~ $L_\omega(\Amc') = L_\omega(\Amc)$.

      \parII
      \uncover<2->{%
        \Bmph{(3)}~ Analog. \qed
      }
      
      \note{
        \textbf{9:11}
        
        \parI
        Dieselbe Idee:~ Kodiere \Pmc in \Fmc.
        
        \parII
        \dots\ und natürlich auch bei Streett-Automaten \dots
        
        \par
      }
    \end{frame}

    % ------------------------------------------------------------------------------------------
    \begin{frame}[t]
      \frametitle{Von Büchi- zu R- und S-Automaten}
      
      \begin{lemma}
        Wenn $L$ Büchi-erkennbar, dann auch  
        \begin{Enumerate}
          \item
            Rabin-erkennbar\quad und
          \item
            Streett-erkennbar.
        \end{Enumerate}%
        \label{lem:BtoRS}%
      \end{lemma}
      
      \parII
      \uncover<2->{%
        \Bmph{Beweis.}
        
        \parI
        \Bmph{(1)}~ Sei $\Amc = (Q,\Sigma,\Delta,I,F)$ NBA.
        
        \parI
        Konstruiere NRA $\Amc' = (Q,\Sigma,\Delta,I,\Pmc)$ mit
        
        \begin{center}
          $\Pmc = \uncover<3->{\{(\emptyset,F)\}.}$
        \end{center}
        
        \uncover<4->{%
          Leicht zu sehen:~ $L_\omega(\Amc') = L_\omega(\Amc)$.
        }
      }
      
      \parII
      \uncover<5->{%
        \Bmph{(2)}~ Analog, aber mit $\Pmc =$ \uncover<6->{$\{(F,Q)\}$. \qed}
      }

      \note{
        \textbf{9:13 bis 9:15, 5\,min Pause.}
        
        \parII
        Jeweils vorm Aufdecken von \Pmc:~ wer weiß es?
        
        \par
      }
    \end{frame}

  % ------------------------------------------------------------------------------------------
    \begin{frame}
      \frametitle{Von Muller- zu Büchi-Automaten}
      
      \begin{Lemma}
        Jede Muller-erkennbare Sprache ist Büchi-erkennbar.
        \label{lem:MtoB}%
      \end{Lemma}

      \par\smallskip
      \uncover<2->{%
        \Bmph{Beweis.}
        \begin{Itemize}
          \item
            Sei $\Aut{A} = (Q,\Sigma,\Delta,I,\calF)$ ein Muller-Automat
            \par\smallskip
          \item<3->
            Dann ist\quad $L_\omega(\Aut{A}) ~=~ \bigcup_{F \in \calF}~ L_\omega(\,(Q,\Sigma,\Delta,I,\{F\})\,)$
            \par\smallskip
          \item<4->
            Wegen $\cup$-Abgeschlossenheit genügt es zu zeigen,
            dass $L_\omega(\,(Q,\Sigma,\Delta,I,\{F\})\,)$ Büchi-erkennbar ist
            \par\smallskip
          \item<5->
            Konstruiere Büchi-Automaten $\Aut{A}' = (Q',\Sigma,\Delta',I,F')$, der
            \begin{Itemize}
              \item
                $\Aut{A}$ simuliert
              \item
                einen Zeitpunkt rät,\\ ab dem nur noch Zustände aus
%                 $F := \{q_1,\dots,q_n\}$ vorkommen
                $F$ vorkommen
              \item
                ab dort sicherstellt, dass \emph{alle} diese unendlich oft vorkommen
            \end{Itemize}
% %             \par\smallskip
%           \item<6->
%             Details: siehe Tafel \Tafel~~~~~
        \end{Itemize}
      }

      \note{
        \textbf{9:20}
        
        \par
      }
    \end{frame}

  % ------------------------------------------------------------------------------------------
    \begin{frame}[t]
      \frametitle{Von Muller- zu Büchi-Automaten}

      Sei also $\Aut{A} = (Q,\Sigma,\Delta,I,\{F\})$\quad (Muller-Automat)

      \par\smallskip
      Konstruieren NBA $\Aut{A}' = (Q',\Sigma,\Delta',I',F')$ mit
      \begin{Itemize}
        \item<2->
          $Q' = \underbrace{Q}_{\text{Phase 1}} \cup~~ \underbrace{\{\auf q_f,S\zu \mid q_f \in F, S \subseteq F\}}_{\text{Phase 2}}$

          \par\medskip
          \uncover<3->{%
            {\small Ph.\,1: $\Aut{A}'$ simuliert \Aut{A}, bis \Aut{A} \Emph{irgendwann} in einem $q_f\!\!\:\in\!\!\:F$ ist}

          \par\smallskip
            {\small Ph.\,2: $\Aut{A}'$ will nur noch Zustände $\in F$ sehen und \Emph{jeden} $\infty$ oft}
          }
          \uncover<4->{%
            \begin{Itemize}
              \item
                $\Aut{A}'$ wechselt in $\auf q_f,S\zu$ mit $S=\{q_f\}$
              \item
                $S$ enthält die seit dem letzten Zurücksetzen besuchten $q \in F$
              \item
                Wenn $S=F$, wird $S$ auf $\emptyset$ "`zurückgesetzt"'
              \item
                akz.\ Zustände: ein $\auf q_f,F\zu$ muss $\infty$ oft gesehen werden
            \end{Itemize}
          }
      \end{Itemize}

      \note{
        \textbf{9:23}
        
        \parI
        \textbf{TODO}~
        Vorschlag (Tryggve, WiSe 18/19):
        
        \parI
        Man kann auch die Zustände aus $F$ ordnen $(f_1,\dots, f_n)$ und dann analog zur Produktkonstruktion
        $n$ Modi verwenden, d.\,h.\ Modus $i$ bedeutet "`erwarte $f_i$"'.
        Dann sind die Zustände der Phase 2 nur Paare aus EZ und Modus,
        und $\Delta'$ hat vielleicht eine angenehmere Notation. $\leadsto$ \textbf{Ausprobieren!}
        
        \par
      }
    \end{frame}

    % ------------------------------------------------------------------------------------------
    \begin{frame}[t]
      \frametitle{Von Muller- zu Büchi-Automaten}
      
      Sei also $\Aut{A} = (Q,\Sigma,\Delta,I,\{F\})$\quad (Muller-Automat)
      
      \par\smallskip
      Konstruieren NBA $\Aut{A}' = (Q',\Sigma,\Delta',I',F')$ mit
      \begin{Itemize}
        \item
          $Q' = \underbrace{Q}_{\text{Phase 1}} \cup~~ \underbrace{\{\auf q_f,S\zu \mid q_f \in F, S \subseteq F\}}_{\text{Phase 2}}$
        \item
          $\Delta' \,=\, \Delta$\\[2pt]
          \uncover<2->{%
            \hspace*{5mm}
            ${}\cup \{(q,a,\auf q_f,\!\!\:\{q_f\}\zu) \mid (q,a,q_f) \in \Delta,~ q_f \in F\}$ \\[2pt]
          }%
          \uncover<3->{%
            \hspace*{5mm}
            \mbox{${}\cup \{(\auf q,\!\!\:S\zu, a, \auf q'\!\!\:, S \!\!\:\cup\!\!\: \{q'\}\zu) \mid (q,\!\!\:a,\!\!\:q') \!\!\:\in\!\!\: \Delta,~ q,\!\!\:q' \!\!\:\in\!\!\: F,~ S \!\!\:\neq\!\!\: F\}$} \\[2pt]
          }%
          \uncover<4->{%
            \hspace*{5mm}
            ${}\cup \{(\auf q,\!\!\:F\zu, a, \auf q'\!\!\:, \{q'\}\zu) \hspace*{6mm} \mid (q,\!\!\:a,\!\!\:q') \!\!\:\in\!\!\: \Delta,~ q,\!\!\:q' \!\!\:\in\!\!\: F\}$%
          }%
        \item<5->
        $I'=I$
        \item<6->
        $F'=\{\auf q_f,\!\!\:F\zu \mid q_f \in F\}$
      \end{Itemize}
      %
      \uncover<7->{%
        Dann gilt: $L_\omega(\Amc') = L_\omega(\Amc)$.
        \Tafel
        \qed
      }
      
      \note{
        \textbf{9:26 bis spätestens 9:56}
        
        \par
      }
    \end{frame}
    
  % ------------------------------------------------------------------------------------------
    \begin{frame}
      \frametitle{Abschlusseigenschaften}
      
%      \Bmph{Zur Erinnerung}
%      \begin{block}{Satz \ref{thm:abgeschlossenheit_v+d}}
%        Die Menge der Büchi-erkennbaren Sprachen ist abgeschlossen unter den Operationen
%        $\cup$ und $\cap$.
%      \end{block}
%      
%      \par\bigskip
      \uncover<1->{%
        \Bmph{Direkte Konsequenz aus}
        %
        \begin{Itemize}
          \item
            Satz~\ref{thm:abgeschlossenheit_v+d} (Abschlusseigenschaften der Büchi-erkennbaren Spr.)
          \item
            und Satz~\ref{thm:gleichmaechtigkeit} (Gleichmächtigkeit der Automatenmodelle):
        \end{Itemize}
        %
        \begin{Folgerung}
          Die Menge der 
          \begin{itemize}
            \item
              Muller-erkennbaren Sprachen,
            \item
              Rabin-erkennbaren Sprachen,
            \item
              Streett-erkennbaren Sprachen
          \end{itemize}
          ist abgeschlossen unter den Operationen
          $\cup$ und $\cap$.
        \end{Folgerung}
      }
      
      \par\bigskip
      \uncover<2->{%
        \Emph{Zu Komplement-Abgeschlossenheit kommen wir jetzt.}
        
        \par
        Benötigen zunächst deterministische Varianten von Muller-, Rabin-, Streett-Automaten.
      }

      \note{
        \textbf{9:56}
        
        \parI
        Tief durchatmen; wir sind so gut wie fertig für heute. :)
        
        \par
      }
    \end{frame}

  % ------------------------------------------------------------------------------------------
    \begin{frame}
      \frametitle{Deterministische Varianten}

      Deterministische Varianten sind analog zu NBA definiert:

      \par\medskip
      Ein Muller-, Rabin- oder Streett-Automat
      $\Aut{A} = (Q,\Sigma,\Delta,I,\textsl{Acc})$ \\
      ist \Bmph{deterministisch}, wenn gilt:
      \begin{Itemize}
        \item
          $|I|=1$
        \item
          $|\{q' \mid (q,a,q') \in \Delta\}| = 1$ für alle $(q,a) \in Q \times \Sigma$
      \end{Itemize}

      \par\bigskip
      \uncover<2->{%
        \Bmph{Zu Satz \ref{thm:gleichmaechtigkeit} analoge Aussage:}
        \begin{Satz}
          Für jede Sprache $L \subseteq \Sigma^\omega$ sind die folgenden Aussagen äquivalent.
          \par\smallskip
          \begin{tabular}{@{}l@{~~}l@{}}
            \ddblu{\textup{(M)}} & $L$ ist von einem deterministischen Muller-Autom.\ erkennbar. \\
            \ddblu{\textup{(R)}} & $L$ ist von einem deterministischen Rabin-Autom.\ erkennbar. \\
            \ddblu{\textup{(S)}} & $L$ ist von einem deterministischen Streett-Autom.\ erkennbar.
          \end{tabular}          
          \label{thm:gleichmaechtigkeit_deterministisch}
        \end{Satz}
      }

      \par\smallskip
      \uncover<3->{%
        Ohne Beweis (ähnlich wie Lemmas~\ref{lem:BRStoM}--\ref{lem:MtoB}).%
      }

      \note{
        \textbf{9:58 bis 10:00 $\leadsto$ Punktlandung?}
        
        \parII
        \textcolor{black!70}{\textbf{Wenn Zeit,} dann was zum Ablauf Prüfungen sagen.}
        
        \parII
        Satz~\ref{thm:gleichmaechtigkeit_deterministisch} folgt \textbf{nicht}
        unmittelbar aus den bisherigen Resultaten für \textbf{N}xAs. \\
        Er wird stückweise in Meghyns Skript bewiesen; \\
        dort sind Muller-, Rabin- und Streett-Automaten immer deterministisch.
        
%        \parII
%        % TODO!
%        \textbf{TODO:}~ Diese Folie \textbf{nach} der folgenden?
%        
        \par
      }
    \end{frame}

  % ------------------------------------------------------------------------------------------
    \begin{frame}
      \frametitle{Überblick der Automatenmodelle}
      
      \Bmph{Büchi-Automat (NBA):}~
      %
      \begin{Itemize}
        \item
          $\Amc = (Q,\Sigma,\Delta,I,F)$ mit $F \subseteq Q$
        \item
          Erfolgreicher Run $r$:~ $\textsf{Inf}(r) \cap F \neq \emptyset$
      \end{Itemize}
      
      \par\smallskip
      \Bmph{Muller-Automat (NMA):}~
      %
      \begin{Itemize}
        \item
          $\Amc = (Q,\Sigma,\Delta,I,\Fmc)$ mit $\Fmc \subseteq 2^Q$
        \item
          Erfolgreicher Run $r$:~ $\textsf{Inf}(r) \in \Fmc$
      \end{Itemize}
      
      \par\smallskip
      \Bmph{Rabin-Automat (NRA):}~
      %
      \begin{Itemize}
        \item
          $\Amc = (Q,\Sigma,\Delta,I,\Pmc)$ mit $\Pmc \subseteq 2^Q \times 2^Q$
        \item
          Erfolg:~ $\exists (E,F) \in \Pmc : \textsf{Inf}(r) \cap F \neq \emptyset \text{~und~} \textsf{Inf}(r) \cap E = \emptyset$
      \end{Itemize}
      
      \par\smallskip
      \Bmph{Streett-Automat (NSA):}~
      %
      \begin{Itemize}
        \item
          $\Amc = (Q,\Sigma,\Delta,I,\Pmc)$ mit $\Pmc \subseteq 2^Q \times 2^Q$
        \item
          Erfolg:~ $\forall (E,F) \in \Pmc : \textsf{Inf}(r) \cap F \neq \emptyset \text{~impliziert~} \textsf{Inf}(r) \cap E \neq \emptyset$
      \end{Itemize}
      \note{
        \textbf{16:00}
        
        \par
      }
    \end{frame}

  % ------------------------------------------------------------------------------------------
    \begin{frame}
      \frametitle{Determinisierung von Büchi-Automaten}

      Erinnerung an Satz \ref{thm:DBAs_schwaecher_als_NBAs}:
      Es gibt eine Büchi-erkennbare Sprache,\\
      die nicht durch einen DBA erkannt wird.

      \par\bigskip
      \uncover<2->{%
        \begin{alertblock}{Ziel}
          Prozedur zur Umwandlung eines gegebenen NBA\\
          in einen äquivalenten deterministischen \Emph{Rabin}-Automaten
        \end{alertblock}
%         \Emph{Ziel:} Prozedur, um gegebenen NBA in äquivalenten deterministischen \Emph{Rabin}-Automaten umzuwandeln
        \begin{Itemize}
          \item<3->[$\leadsto$]
            wegen Satz \ref{thm:gleichmaechtigkeit_deterministisch} erhält man daraus auch
            äquivalente deterministische Muller-/Streett-Automaten
          \item<4->
            Resultat geht auf McNaughton zurück \\
            {\footnotesize (1965 von Robert McNaughton, Philosoph/Inform., Harvard, Rensselaer)}
          \item<5->
            Wir verwenden intuitiveren Beweis von Safra \\
            {\footnotesize (1988 von Shmuel Safra, Informatiker, Tel Aviv)}
        \end{Itemize}
      }
      
      \note{
        \textbf{16:02}
        
        \par
      }
    \end{frame}

  % ------------------------------------------------------------------------------------------
    \begin{frame}
      \frametitle{Potenzmengenkonstruktion versagt}
      
      \Bmph{Zwei naheliegende Versuche:}
      \begin{Enumerate}
        \item
          NBA $\leadsto$ DBA mittels Potenzmengenkonstruktion \Emph{(PMK)}
          \begin{Itemize}
            \item[]
              muss wegen Satz \ref{thm:DBAs_schwaecher_als_NBAs} fehlschlagen -- Bsp.\ siehe Tafel \Tafel
          \end{Itemize}
        \item<2->
            NBA $\leadsto$ determ.\ Muller-(Rabin-/Streett-)Automat via PMK
            \begin{Itemize}
              \item[]
                schlägt auch fehl -- mit demselben Gegenbeispiel \Tafel
            \end{Itemize}
      \end{Enumerate}

      \par\bigskip%\bigskip
      \uncover<3->{%
        \Emph{Hauptproblem:}
        \begin{Itemize}
          \item
            Potenzautomat simuliert mehrere Runs gleichzeitig\\
          \item
            akzeptierende Zustände \Bmph{(akzZ)} müssen dabei \Emph{nicht synchron} erreicht werden
          \item
            \Emph{Bad runs:}
            \par
            Wenn DBA $\Aut{A}^d$ für $\alpha$ eine $\infty$ Folge von akzZ findet,\\
            dann können diese akzZ von \Emph{verschiedenen} Runs des NBA~$\Aut{A}$ 
            auf \Emph{Präfixen} von $\alpha$ stammen.\\
            Diese Runs müssen nicht zu einem Run auf $\alpha$ fortsetzbar sein.
        \end{Itemize}
      }

      \note{
        \textbf{16:03}
        
        \parII
        \textbf{T3.12 bis 16:10}
        
        \parII
        \textbf{T3.13 bis 16:17}

        \parII
        \textbf{insg.\ bis 16:19}

        \par
      }
    \end{frame}

  % ------------------------------------------------------------------------------------------
    \begin{frame}
      \frametitle{Abhilfe: Safras "`Tricks"'}

      \Bmph{Ziel}
      \begin{Itemize}
        \item
          Wandle NBA $\Aut{A} = (Q,\Sigma,\Delta,I,F)$\\
          in determ.\ Rabin-Automaten $\Aut{A}^d = (Q^d,\Sigma,\Delta^d,I^d,\calP^d)$ um\\
          mit $L_\omega(\Aut{A}) = L_\omega(\Aut{A}^d)$
        \item
          Vermeide ``bad runs'': \Bmph{Safras Tricks}
      \end{Itemize}


%       \begin{Itemize}
%         \item
%           \Bmph{Ziel:}\\
%           NBA $\Aut{A} = (Q,\Sigma,\Delta,I,F)$ $\leadsto$ DRA $\Aut{A}^d = (Q^d,\Sigma,\Delta^d,I^d,\calP^d)$\\
%           mit $L_\omega(\Aut{A}) = L_\omega(\Aut{A}^d)$
%         \item
%           \Bmph{Problem mit PMK:}\\
%           \Emph{bad runs} von $\Aut{A}^d$, die keinem erfolgr.\ Run von \Aut{A} entsprechen
%         \item
%           \Bmph{Safras Tricks} erweitern die PMK und vermeiden das Problem
%       \end{Itemize}

      \par\bigskip
      \uncover<2->{%
        \Bmph{Vorbetrachtungen}
        \begin{Itemize}
          \item
            \Bmph{Makrozustände}:~ Zustände der alten PMK (Mengen $M \subseteq Q$)
          \item
            \Bmph{Zustände von $\Autb{A}^d$:}\\
            $\approx$ Bäume, deren Knoten mit Makrozuständen markiert sind
          \item
            \Bmph{Startzustand:}\\
            Knoten $I$ (Menge der Anfangszust., wie bei PMK)
        \end{Itemize}

      }

      \note{
        \textbf{16:19}

        \par
      }
    \end{frame}

  % ------------------------------------------------------------------------------------------
    \begin{frame}
      \frametitle{Safras Trick 1}

      \Bmph{Trick 1:}\\
      In Makrozuständen $M$ mit $M\cap F \neq \emptyset$, initialisiere neue (Teil)Runs:
      \begin{Itemize}
        \item
          Folgezustand bekommt ein Kind mit Folgezuständen aller akzZ

          \begin{small}
            \begin{center}
%              $M$ 
              \begin{tikzpicture}[%
                baseline=-2pt,node distance=20mm,>=Latex,
                every state/.style={rectangle,rounded corners,draw=black,semithick,fill=black!5,inner sep=1mm,minimum size=6mm},
                every edge/.style={draw=black,semithick}
              ]              
                \node[state] (M) {$M$};
              \end{tikzpicture}
              \quad
              $\stackrel{a}{\longrightarrow}$
              \quad
%              \begin{tabular}{l}
%                $\{q \in Q \mid (m,a,q) \in \Delta,~ m \in M\}$ \\
%                $\{q \in Q \mid (m,a,q) \in \Delta,~ m \in M \cap F\}$~~$(X)$ \\
%              \end{tabular}
              \begin{tikzpicture}[%
                baseline=-2pt,node distance=20mm,>=Latex,
                every state/.style={rectangle,rounded corners,draw=black,semithick,fill=black!5,inner sep=1mm,minimum size=6mm},
                every edge/.style={draw=black,semithick}
              ]              
                \node[state]                (M)  {$\{q \in Q \mid (m,a,q) \in \Delta,~ m \in M\}$};
                \node[state,below=5mm of M] (M') {$\{q \in Q \mid (m,a,q) \in \Delta,~ m \in M \cap F\}$};
                \node[right=2mm of M']      (X)  {$X$};
                
                \path[-] (M) edge (M');
              \end{tikzpicture}
            \end{center}
          \end{small}

%         \item[$\Rightarrow$]
%           Zustände in $\Aut{A}^d$ sind \emph{Bäume} von Makrozuständen
        \item
          PMK wird auf jeden Knoten einzeln angewendet
        \item
          Neuer Knoten $X$ enthält alle Nachfolger von akzZ;
          \par\smallskip
          Info wird gebraucht, um aus einem erfolgreichen Run für $\Aut{A}^d$\\
          einen für $\Aut{A}$ zu konstruieren
          \hfill
          $\leadsto$ vermeidet \emph{bad runs}
      \end{Itemize}
      Beispiel: siehe Tafel \Tafel

      \note{
        \textbf{16:22}

        \parII        
        Safras Ideen bestehen aus drei Tricks, die ich jetzt halb formal, halb intuitiv vorstelle.
        
        \parI
        Anschließend präzise als Konstruktion mit 6 Schritten beschreiben.
        
        \parIII
        Kinder im Baum sind immer unten; deshalb keine Pfeilspitzen!
        
        \par
      }
    \end{frame}

  % ------------------------------------------------------------------------------------------
    \begin{frame}
      \frametitle{Konsequenzen aus Trick 1}

      \begin{Itemize}
        \item
          Organisation dieser Mengen von Makrozuständen: \\
          als geordnete Bäume -- \Bmph{Safra-Bäume}
          \par\smallskip
        \item
          Trick 1 fügt neue Kinder/Geschwister hinzu \\
          $\leadsto$ Höhe/Breite des Safra-Baums wächst
          \par\smallskip
        \item
          Zum Begrenzen der Höhe/Breite: Trick 2 und 3
      \end{Itemize}

      \note{
        \textbf{16:32}
        
        \par
      }
    \end{frame}

  % ------------------------------------------------------------------------------------------
    \begin{frame}
      \frametitle{Safras Trick 2}

      \Bmph{Trick 2:}\\
      Erkenne zusammenlaufende Teilruns und lösche überflüssige Info

      \par\medskip
      Bsp.: Betrachte Teilruns, die in demselben Zustand $q_n$ enden:
      \begin{small}%
        \begin{align*}
          r  & = q_0q_1q_2   \dots \Bmph{$f$}  \dots\dots q_{n-1}\Emph{$q_n$}  \\
          r' & = q_0q_1'q_2' \dots\dots \Bmph{$f'$} \dots q_{n-1}'\Emph{$q_n$} \qquad (\Bmph{$f$},\Bmph{$f'$} \in F)
        \end{align*}
      \end{small}%

      \par\smallskip
      Zugehörige $n$ Schritte von $\Aut{A}^d$ unter Anwendung von Trick 1:

      \begin{small}
        \begin{center}
          \begin{tikzpicture}[%
            baseline=-2pt,node distance=20mm,>=Latex,
            every state/.style={rectangle,rounded corners,draw=black,semithick,fill=black!5,inner sep=1mm,minimum size=6mm},
            every edge/.style={draw=black,semithick}
          ]              
            \node[state]                (oben)  {$\cdots q_0\cdots$};
          \end{tikzpicture}
          ~$\longrightarrow^*$~
          \begin{tikzpicture}[%
            baseline=-2pt,node distance=20mm,>=Latex,
            every state/.style={rectangle,rounded corners,draw=black,semithick,fill=black!5,inner sep=1mm,minimum size=6mm},
            every edge/.style={draw=black,semithick}
          ]              
            \node[state]                   (oben)  {$\cdots \Bmph{$f$}\cdots$};
            \node[state,below=5mm of oben] (unten) {\Bmph{$f$}};
            
            \path[-] (oben) edge (unten);
          \end{tikzpicture}
          ~$\longrightarrow^*$
          \begin{tikzpicture}[%
            baseline=-2pt,node distance=20mm,>=Latex,
            every state/.style={rectangle,rounded corners,draw=black,semithick,fill=black!5,inner sep=1mm,minimum size=6mm},
            every edge/.style={draw=black,semithick}
          ]              
            \node[state]                                 (oben)    {$\cdots \Bmph{$f'$}\cdots$};
            \node[state,below left=5mm and -5mm of oben]  (ulinks)  {$\cdots$};
            \node[state,below right=5mm and -5mm of oben] (urechts) {\Bmph{$f'$}};
            
            \path[-] (oben) edge (ulinks)
                     (oben) edge (urechts);
          \end{tikzpicture}
          $\longrightarrow^*$
          \begin{tikzpicture}[%
            baseline=-2pt,node distance=20mm,>=Latex,
            every state/.style={rectangle,rounded corners,draw=black,semithick,fill=black!5,inner sep=1mm,minimum size=6mm},
            every edge/.style={draw=black,semithick}
          ]              
            \node[state]                                  (oben)    {$\cdots \Emph{$q_n$}\cdots$};
            \node[state,below left=5mm and -5mm of oben]  (ulinks)  {\Emph{$q_n$}};
            \node[state,below right=5mm and -5mm of oben] (urechts) {\Emph{$q_n$}};
            
            \path[-] (oben) edge (ulinks)
                     (oben) edge (urechts);
          \end{tikzpicture}
        \end{center}
      \end{small}

      Trick 2 vereinigt die beiden $\{q_n\}$-Kinder ("`horizontal merge"')

      \par\medskip
      $\leadsto$ \Emph{Weite} von Safra-Bäumen wird beschränkt

      \note{
        \textbf{16:33}
        
        \par
      }
    \end{frame}

  % ------------------------------------------------------------------------------------------
    \begin{frame}
      \frametitle{Safras Trick 3}

      \Bmph{Trick 3:}\\
      Gib überflüssige Makrozustände zur Löschung frei

      \par\bigskip
      Wenn alle Kinder eines MZ $M$ bezeugen,\\
      dass \emph{jeder} Zustand in $M$ einen akz.\ Zustand als Vorgänger hat,\\
      dann können die Kinder gelöscht werden

      \par\bigskip
      Genauer:
      \mbox{wenn $M$ Kinder $M_1,\dots,M_n$ hat mit $M_1 \cup\dots\cup M_n = M$,\hspace*{-10mm}} \\
      dann werden die $M_i$ gelöscht und $M$ mit \circled{!} markiert

      \par\bigskip
      $\leadsto$ "`vertical merge"', beschränkt die \Emph{Tiefe} von Safra-Bäumen

      \note{
        \textbf{16:37 bis 16:39, dann 5min Pause}
        
        \par
      }
    \end{frame}

%   % ------------------------------------------------------------------------------------------
%     \begin{frame}
%       \frametitle{Safras Tricks}
%       
%       Beginne wie bei der PMK mit Knoten $I$
%       \par\smallskip
%       \begin{Enumerate}
%         \item
%           Von Makrozuständen mit akz.\ Zuständen, beginne neue Runs
%           \Tafel
%           \begin{Itemize}
%             \item
%               erzeuge neues Kind mit Nachfolgezuständen aller akz.\ Zustände
%             \item
%               wende zukünftig PMK auf jeden Knoten an
%           \end{Itemize}
%           \par\smallskip
%         \item<2->
%           Erkenne zusammenlaufende Runs; lösche überflüssige Info
%           \Tafel
%           \begin{Itemize}
%             \item
%               das beschränkt Weite eines Safra-Baums
%             \item
%               "`horizontal merge"'
%           \end{Itemize}
%           \par\smallskip
%         \item<3->
%           Gib überflüssige Makrozustände zur Löschung frei
%           \Tafel
%           \begin{Itemize}
%             \item
%               wenn alle Kinder eines MZ $M$ bezeugen,\\
%               dass jeder Zustand in $M$ einen akz.\ Zustand als Vorgänger hat,\\
%               dann kann $M$ gelöscht werden
%             \item
%               "`vertical merge"'
%           \end{Itemize}
%       \end{Enumerate}
%       \note{~}
%     \end{frame}
% 
  % ------------------------------------------------------------------------------------------
    \begin{frame}
      \frametitle{Definition Safra-Baum}
      
      Sei \Bmph{$Q$} Zustandsmenge des ursprünglichen NBA \\
      und \Bmph{$V$} eine nichtleere Menge von \Bmph{Knotennamen.}
      
      \parIII
      \Bmph{Makrozustand (MZ)} über $Q$:~ Teilmenge $M \subseteq Q$

      \parIII
      \Bmph{Safra-Baum} über $Q,V$:
      \begin{Itemize}
        \item
          geordneter Baum mit Knoten aus $V$
          \begin{Itemize}
            \item[]
              (der leere Baum ist erlaubt!)
          \end{Itemize}
        \item
          jeder Knoten mit einem \Emph{nichtleeren} MZ markiert\\
          und möglicherweise auch mit \circled{!}
        \item
          Wenn Knoten $v$ mit $M$ und $v$'s Kinder mit $M_1,\dots,M_n$ markiert sind, dann:
          \begin{Enumerate}
            \item
              $M_1 \cup \dots \cup M_n \subsetneq M$
            \item
              $M_i$ sind paarweise disjunkt
          \end{Enumerate}
      \end{Itemize}

      \note{
        \textbf{16:44}
        
        \parII
        \textbf{Fragen:}~ Wer ahnt, wozu die letzte Bedingung (1 und 2) wichtig ist? \\
        (Antw.: stellt sicher, dass es nicht zu viele mögliche SB gibt -- zeigen wir jetzt!)
        
        \par
      }
    \end{frame}

  % ------------------------------------------------------------------------------------------
    \begin{frame}
      \frametitle{Safra-Bäume sind beschränkt}
      
%      \Bmph{Zur Erinnerung}
%      \par\smallskip
      \begin{itshape}
        "`Wenn Knoten $v$ mit $M$ und $v$'s Kinder mit $M_1,\dots,M_n$ markiert sind, dann:
        \begin{Enumerate}
          \item
            $M_1 \cup \dots \cup M_n \subsetneq M$
          \item
            $M_i$ sind paarweise disjunkt"'
        \end{Enumerate}
      \end{itshape}
      
      \par\bigskip
      \uncover<2->{%
        \Bmph{Konsequenzen}
        \begin{Itemize}
          \item
            \emph{wegen (1):}~ Höhe jedes SB ist durch $|Q|$ beschränkt
          \item
            \emph{wegen (2):}~ Anzahl Kinder pro Knoten kleiner als $|Q|$
          \item
            \emph{sogar:}~ Jeder SB über $Q$ hat höchstens $|Q|$ Knoten\\
            {\small (Beweis per Induktion über Baumhöhe)}
            \parII
          \item<3->[$\leadsto$]
            Anzahl der möglichen SB ist beschränkt durch
            \only<3|handout:0>{\Emph{?}}%
            \only<4->{$2^{O(|Q| \cdot \text{log}|Q|)}$}
        \end{Itemize}

      }
      \note{
        \textbf{16:47}
        
        \parI
        % TODO
        \textbf{TODO:}~ Größe und Anzahl der Safra-Bäume sauber abschätzen \& erklären!
        
        \par
      }
    \end{frame}

  % ------------------------------------------------------------------------------------------
    \begin{frame}
      \frametitle{Details der Konstruktion}
      
      Sei $\Aut{A} = (Q,\Sigma,\Delta,I,F)$ ein NBA und $V=\{1,\dots,2|Q|\}$.
      \par\smallskip
      Konstruieren DRA $\Aut{A}^d = (Q^d,\Sigma,\Delta^d,I^d,\calP)$:
      \begin{Itemize}
        \item
          $Q^d = $ Menge aller Safra-Bäume über $Q,V$
        \item
          $I^d = $ Safra-Baum mit einzigem Knoten $I$
        \item
          $\Delta^d = \{(S,a,S') \mid S' \text{~wird aus $S$ wie folgt konstruiert}\}$
      \end{Itemize}

      \note{
        \textbf{16:50}
        
        \parII
        Knotennamen $1,\dots,2|Q|$ reichen wegen der Beschränkungen,
        die wir für die Knotenzahl eines SB gerade aufgestellt haben.
        
        \parII
        Übergangsrelation folgt genau Safras Tricks (in jedem \textbf{einzelnen} Übergang!).
        
        \parI
        Akzeptanzkomponente kommt am Ende.
        
        \par
      }
    \end{frame}

  % ------------------------------------------------------------------------------------------
    \begin{frame}
      \frametitle{Konstruktion von $S'$ aus $S$ in 6 Schritten}

      Sei $S$ Safra-Baum mit Knotennamen $V' \subseteq V$; sei $a \in \Sigma$

      \begin{Enumerate}
        \item<+->
          Beginne mit $S$; entferne alle Markierungen \circled{!}
          \par%\smallskip
        \item<+->
          Für jeden Knoten $v$ mit Makrozustand $M$ und $M \cap F \neq \emptyset$,\\
%           \par\smallskip
          füge neues Kind $v' \in V\setminus V'$ mit Markierung $M \cap F$ hinzu\\
          {\small (als \Bmph{jüngstes} (rechtes) Geschwister aller evtl.\ vorhandenen Kinder)}
          \par%\smallskip
        \item<+->
          Wende Potenzmengenkonstruktion auf alle Knoten $v$ an:\\
          ersetze MZ $M$ durch $\{q \in Q \mid (m,a,q) \in \Delta \text{~für ein~} m \in M\}$
          \par%\smallskip
        \item<+->
          \Bmph{Horizontales Zusammenfassen:}
          Für jeden Knoten $v$ mit MZ $M$,\\
          lösche jeden Zustand $q$, der im MZ eines älteren Geschwisters vorkommt,
          aus $M$ \emph{und aus den MZen der Kinder von $v$}
          \par%\smallskip
        \item<+->
          Entferne alle Knoten mit leeren MZen
          \par%\smallskip
        \item<+->
          \Bmph{Vertikales Zusammenfassen:}
          Für jeden Knoten $v$, dessen Markierung nur Zustände aus $v$'s Kindern enthält,\\
          lösche alle Nachfolger von $v$ und markiere $v$ mit \circled{!}
      \end{Enumerate}

      \note{
        \textbf{16:52}
        
        \parII
        Und das sind die 6 Schritte, die auf den 3 Tricks von Safra beruhen.
        
        \parII
        Schritt 2 = Trick 1 \\
        Schritt 4 = Trick 2 \\
        Schritt 6 = Trick 3
        
        \parII
        Illustration auf nächster Folie
        
        \par
      }
    \end{frame}

  % ------------------------------------------------------------------------------------------
    \begin{frame}
      \frametitle{Illustration der Schritte 2--5}

      \begin{Enumerate}
        \refstepcounter{enumi}
        \item
          ~\par\vspace*{-\baselineskip}
          \begin{small}
            \begin{tikzpicture}[%
              baseline=-2pt,node distance=20mm,>=Latex,
              every state/.style={rectangle,rounded corners,draw=black,semithick,fill=black!5,inner sep=1mm,minimum size=6mm},
              every edge/.style={draw=black,semithick}
            ]              
              \node[state]                                   (oben)    {$\cdots f_1 \cdots f_2 \cdots f_3 \cdots$};
              \node[state,below left=3mm and -14mm of oben]  (ulinks)  {$\cdots$};
              \node[state,below right=3mm and -14mm of oben] (urechts) {$\cdots$};
              
              \path[-] (oben) edge (ulinks)
                       (oben) edge (urechts);
            \end{tikzpicture}          
            \quad$\leadsto$\quad
            \begin{tikzpicture}[%
              baseline=-2pt,node distance=20mm,>=Latex,
              every state/.style={rectangle,rounded corners,draw=black,semithick,fill=black!5,inner sep=1mm,minimum size=6mm},
              every edge/.style={draw=black,semithick}
            ]              
              \node[state]                                   (oben)    {$\cdots f_1 \cdots f_2 \cdots f_3 \cdots$};
              \node[state,below left=3mm and -6mm of oben]   (ulinks)  {$\cdots$};
              \node[state,below left=3mm and -18mm of oben]  (umitte)  {$\cdots$};
              \node[state,below right=3mm and -11mm of oben] (urechts) {$f_1,f_2,f_3$};
              
              \path[-] (oben) edge (ulinks)
                       (oben) edge (umitte)
                       (oben) edge (urechts);
            \end{tikzpicture}          
          \end{small}
          \par\bigskip
        \item
          \begin{small}
            \begin{tikzpicture}[%
              baseline=-2pt,node distance=20mm,>=Latex,
              every state/.style={rectangle,rounded corners,draw=black,semithick,fill=black!5,inner sep=1mm,minimum size=6mm},
              every edge/.style={draw=black,semithick}
            ]              
              \node[state]                                   (oben)    {$\cdots q \cdots$};
              \node[state,below left=3mm and -7mm of oben]  (ulinks)  {$\cdots$};
              \node[state,below right=3mm and -7mm of oben] (urechts) {$\cdots$};
              
              \path[-] (oben) edge (ulinks)
                       (oben) edge (urechts);
            \end{tikzpicture}          
            \quad$\leadsto$\quad
            \begin{tikzpicture}[%
              baseline=-2pt,node distance=20mm,>=Latex,
              every state/.style={rectangle,rounded corners,draw=black,semithick,fill=black!5,inner sep=1mm,minimum size=6mm},
              every edge/.style={draw=black,semithick}
            ]              
              \node[state]                                   (oben)    {$\cdots q' \cdots$};
              \node[state,below left=3mm and -7mm of oben]  (ulinks)  {$\cdots$};
              \node[state,below right=3mm and -7mm of oben] (urechts) {$\cdots$};
              
              \path[-] (oben) edge (ulinks)
                       (oben) edge (urechts);
            \end{tikzpicture}          
            \quad
            wenn $(q,a,q') \in \Delta$
          \end{small}
          \par\bigskip
        \item
          \begin{small}
            \begin{tikzpicture}[%
              baseline=-2pt,node distance=20mm,>=Latex,
              every state/.style={rectangle,rounded corners,draw=black,semithick,fill=black!5,inner sep=1mm,minimum size=6mm},
              every edge/.style={draw=black,semithick}
            ]              
              \node[state]                                   (oben)    {$\cdots$};
              \node[state,below left=3mm and -2.5mm of oben]  (ulinks)  {$\cdots q \cdots$};
              \node[state,below right=3mm and -2.5mm of oben] (urechts) {$\cdots q \cdots$};
              
              \path[-] (oben) edge (ulinks)
                       (oben) edge (urechts);
            \end{tikzpicture}          
            \quad$\leadsto$\quad
            \begin{tikzpicture}[%
              baseline=-2pt,node distance=20mm,>=Latex,
              every state/.style={rectangle,rounded corners,draw=black,semithick,fill=black!5,inner sep=1mm,minimum size=6mm},
              every edge/.style={draw=black,semithick}
            ]              
              \node[state]                                   (oben)    {$\cdots$};
              \node[state,below left=3mm and -4mm of oben]  (ulinks)  {$\cdots q \cdots$};
              \node[state,below right=3mm and -1mm of oben] (urechts) {$\cdots$};
              
              \path[-] (oben) edge (ulinks)
                       (oben) edge (urechts);
            \end{tikzpicture}          
          \end{small}
          \par\bigskip
        \item
          \begin{small}
            \begin{tikzpicture}[%
              baseline=-2pt,node distance=20mm,>=Latex,
              every state/.style={rectangle,rounded corners,draw=black,semithick,fill=black!5,inner sep=1mm,minimum size=6mm},
              every edge/.style={draw=black,semithick}
            ]              
              \node[state]                                    (oben)    {$\cdots$};
              \node[state,below=3mm of oben]                  (umitte)  {$\emptyset$};
              \node[state,left=2mm of umitte]                 (ulinks)  {$\cdots$};
              \node[state,right=2mm of umitte]                (urechts) {$\cdots$};
              
              \path[-] (oben) edge (ulinks)
                       (oben) edge (umitte)
                       (oben) edge (urechts);
            \end{tikzpicture}          
            \quad$\leadsto$\quad
            \begin{tikzpicture}[%
              baseline=-2pt,node distance=20mm,>=Latex,
              every state/.style={rectangle,rounded corners,draw=black,semithick,fill=black!5,inner sep=1mm,minimum size=6mm},
              every edge/.style={draw=black,semithick}
            ]              
              \node[state]                                    (oben)    {$\cdots$};
              \node[state,below left=3mm and -2.5mm of oben]  (ulinks)  {$\cdots$};
              \node[state,below right=3mm and -2.5mm of oben] (urechts) {$\cdots$};
              
              \path[-] (oben) edge (ulinks)
                       (oben) edge (urechts);
            \end{tikzpicture}          
          \end{small}
      \end{Enumerate}

      \note{
        \textbf{16:58}
        
        \parII
        Hier wieder schematische Skizzen; als nächstes am konkreten Bsp.
        
        \parII
        nur letztes Bild an Tafel (ist hier nicht eindeutig)
        
        \par
      }
    \end{frame}

  % ------------------------------------------------------------------------------------------
    \begin{frame}
      \frametitle{Illustration von Schritt 6}

      \begin{Enumerate}
        \setcounter{enumi}{5}
        \item
          \begin{small}
            \begin{tikzpicture}[%
              baseline=-2pt,node distance=20mm,>=Latex,
              every state/.style={rectangle,rounded corners,draw=black,semithick,fill=black!5,inner sep=1mm,minimum size=6mm},
              every edge/.style={draw=black,semithick}
            ]              
              \node[state]                                       (oben)    {$M = M_1 \cup \dots \cup M_n$};
              \node[state,below=7mm of oben,draw=none,fill=none] (umitte)  {$\dots$};
              \node[state,above left=-4mm and 4mm of umitte]      (ulinks)  {$M_1$};
              \node[state,above right=-4mm and 4mm of umitte]     (urechts) {$M_n$};
              
              \node[state,below=7mm of ulinks, draw=none,fill=none]           (ulm) {$\vdots$};
              \node[state,below=7mm of urechts,draw=none,fill=none]           (urm) {$\vdots$};
              \node[state,above left =-6mm and 2mm of ulm,draw=none,fill=none] (ull) {};
              \node[state,above left =-6mm and 2mm of urm,draw=none,fill=none] (url) {};
              \node[state,above right=-6mm and 2mm of ulm,draw=none,fill=none] (ulr) {};
              \node[state,above right=-6mm and 2mm of urm,draw=none,fill=none] (urr) {};
              
              \path[-] (oben) edge (ulinks)
                       (oben) edge (umitte)
                       (oben) edge (urechts)
                       (ulinks) edge (ull)
                       (ulinks) edge (ulm)
                       (ulinks) edge (ulr)
                       (urechts) edge (url)
                       (urechts) edge (urm)
                       (urechts) edge (urr);
            \end{tikzpicture}          
%            \begin{tabular}{c}
%              $M$ \\
%              $M_1$ \quad \dots \quad $M_n$ \\
%              $\vdots$ \qquad\qquad $\vdots$
%            \end{tabular}
            \quad$\leadsto$\qquad
           \begin{tikzpicture}[%
              baseline=-2pt,node distance=20mm,>=Latex,
              every state/.style={rectangle,rounded corners,draw=black,semithick,fill=black!5,inner sep=1mm,minimum size=6mm},
              every edge/.style={draw=black,semithick}
            ]              
              \node[state] (M)      {$M$};
              \node[right=0mm of M] (excl) {\circled{!}};
            \end{tikzpicture}          
%%            \quad
%            wenn $M = M_1 \cup \dots \cup M_n$
          \end{small}
%          \Tafel
          \par\medskip
          d.\,h.\ alle Zustände in $M$ kommen im Makrozustand eines Kindes $M_i$ vor
          
          \par\smallskip
          d.\,h.\ jeder Zustand in $M$ hat einen akzZ als Vorgänger!
      \end{Enumerate}

      \note{
        \textbf{17:01}
        
        \par
      }
    \end{frame}

  % ------------------------------------------------------------------------------------------
    \begin{frame}
      \frametitle{Erläuterungen zur Konstruktion}

      \begin{Itemize}
        \item
          $S'$ ist wieder ein Safra-Baum:
          \par\smallskip
          \begin{itshape}
            Wenn Knoten $v$ mit $M$ und $v$'s Kinder mit $M_1,\dots,M_n$ markiert sind, dann:
            \begin{Enumerate}
              \item
                $M_1 \cup \dots \cup M_n \subsetneq M$ \hfill ``$\subseteq$'': Schritte 2,\,3\\
                                                       \hfill ``$\neq$'': Schritt 6
              \item
                $M_i$ sind paarweise disjunkt \hfill Schritt 4
            \end{Enumerate}
          \end{itshape}
%           Siehe Tafel \Tafel
          \par\bigskip
        \item<2->
          Beispiel: siehe Tafel \Tafel
      \end{Itemize}

      \note{
        \textbf{17:04}
        
        \parII
        Def.\ der Akzeptanzkomponente kommt nach dem Beispiel.
        
        \parII
        \textbf{Bsp. bis 17:28}
        
        \par
      }
    \end{frame}

  % ------------------------------------------------------------------------------------------
    \begin{frame}
      \frametitle{Akzeptanzkomponente von $\Aut{A}^d$}

      $\calP = \{(E_v,F_v) \mid v \in V\}$ mit
      \begin{Itemize}
        \item
          $E_v = $ alle Safra-Bäume ohne Knoten $v$
        \item
          $F_v = $ alle Safra-Bäume, in denen $v$ mit \circled{!} markiert ist
      \end{Itemize}

      \par\bigskip
      \uncover<2->{%
        $\leadsto$ d.\,h.\ Run $r = S_0 S_1 S_2 \ldots$ von $\Aut{A}^d$ ist erfolgreich,\\
        wenn es einen Knotennamen $v$ gibt, so dass
        \begin{Itemize}
          \item
            alle $S_i$, bis auf endlich viele, einen Knoten $v$ haben und
          \item
            unendlich oft auf $v$ Schritt 6 angewendet wurde,\\
            d.\,h.\ vorher kamen alle Zustände in $v$'s MZ in $v$'s Kindern vor
        \end{Itemize}
        \TafelForts
      }

      \note{
        \textbf{17:28 bis 17:30 $\leadsto$ Punktlandung?}
        
        \parII
        Jetzt müssen wir natürlich noch zeigen, dass die Konstruktion korrekt ist.
        
        \parI
        Das tun wir nächste Woche! :)
        
        \par
      }
    \end{frame}

  \newlength{\Danngilt}
  \settowidth{\Danngilt}{Dann gilt}
  % ------------------------------------------------------------------------------------------
    \begin{frame}
      \frametitle{Korrektheit und Vollständigkeit der Konstruktion}

      \begin{Lemma}
        Sei $\Aut{A} = (Q,\Sigma,\Delta,I,F)$ ein NBA
        und sei $\Aut{A}^d = (Q^d,\Sigma,\Delta^d,I^d,\calP)$ der DRA,
        den man nach Safras Konstruktion aus \Aut{A} erhält.
        \par\smallskip
        Dann gilt $L_\omega(\Aut{A}^d) = L_\omega(\Aut{A})$.
        \label{lem:soundness+completeness_safra}
      \end{Lemma}

      \par\bigskip
      \Bmph{Korrektheit}: \hfill \emph{(Soundness)}\\
      $\Aut{A}^d$ akzeptiert nur Wörter, die $\Aut{A}$ akzeptiert
      \par\smallskip
      \parbox{\Danngilt}{~} $L_\omega(\Aut{A}^d) \subseteq L_\omega(\Aut{A})$

      \par\bigskip
      \Bmph{Vollständigkeit}: \hfill \emph{(Completeness)}\\
      $\Aut{A}^d$ akzeptiert (mindestens) alle Wörter, die $\Aut{A}$ akzeptiert
      \par\smallskip
      \parbox{\Danngilt}{~} $L_\omega(\Aut{A}^d) \supseteq L_\omega(\Aut{A})$

      \par\bigskip
      \Bmph{Beweis:} Folgerung aus den nächsten beiden Lemmas
      
      \note{%
        \textbf{8:30}
        
        \par
      }
    \end{frame}

  % ------------------------------------------------------------------------------------------
    \begin{frame}
      \frametitle{Korrektheit}

      \begin{Lemma}
        Sei $\Aut{A} = (Q,\Sigma,\Delta,I,F)$ ein NBA
        und sei $\Aut{A}^d = (Q^d,\Sigma,\Delta^d,I^d,\calP)$ der DRA,
        den man nach Safras Konstruktion aus \Aut{A} erhält.
        \par\smallskip
        Dann gilt $L_\omega(\Aut{A}^d) \mathbin{\Emph{$\subseteq$}} L_\omega(\Aut{A})$.
        \label{lem:soundness_safra}
      \end{Lemma}

      \par\bigskip
      \Bmph{Beweisidee.}
      Sei $I=\{q_I\}$ und $I^d = \{S_I\}$.\quad Sei $\alpha \in L_\omega(\Aut{A}^d)$.
      \begin{Itemize}
        \item
          Betrachte erfolgreichen Run $s$ von $\Aut{A}^d$ auf $\alpha$.
        \item
%           Benutze $\mathcal{P}^d$, um einen erfolgreichen Run von $\Aut{A}$ auf $\alpha$ zu konstruieren
          "`Konstruiere"' daraus erfolgr.\ Run von $\Aut{A}$ auf $\alpha$ \emph{stückweise:}
          \[
            s = S_I \dots T_1 \dots T_2 \dots T_3 \dots, \qquad \text{(alle $T_i$ laut $\mathcalreg{P}$ gewählt)}
          \]
        \item
          Jeder Teilrun $T_i\dots T_{i+1}$ induziert Teilrun von $\Aut{A}$ auf Teilwort von $\alpha$,
          der einen akz.\ Zustand enthält
        \item
          Ordnen diese endl.\ Teilruns in einem $\infty$ Baum $\mathcalreg{T}$ an
        \item
          Gesuchter Run von $\Aut{A}$ ist ein $\infty$ Pfad in $\mathcalreg{T}$
      \end{Itemize}

%       siehe Tafel. \Tafel~~~~
%       \par
%       \vspace*{-1\baselineskip}
%       \strut \qed

      \note{%
        \textbf{8:32}
        
        \parI
        Zuerst die Beweis\textbf{idee.}
        
        \par
      }
    \end{frame}

  % ------------------------------------------------------------------------------------------
    \begin{frame}
      \frametitle{Korrektheit}

      \Bmph{Beweis.}~
      Sei also $\alpha \in L_\omega(\Aut{A}^d)$.

      \par\smallskip
      Dann gibt es erfolgreichen Run $s=S_0S_1S_2\dots$ von $\Aut{A}^d$ auf $\alpha$
      und ein Knoten $v$, der (wegen $\mathcalreg{P}^d$)
      \begin{Itemize}
        \item
          \mbox{in allen Safra-Bäumen $S_j,S_{j+1},\dots$ vorkommt, für ein $j\!\!\:\geqslant\!\!\:0$, und\hspace*{-10mm}}
        \item
          in $\infty$ vielen Safra-Bäumen mit \circled{!} markiert ist.\\
          Seien diese $T_1,T_2,\dots$ und sei $T_0=S_0$:
          \[
            s = T_0\dots T_1\dots T_2 \dots T_3 \dots
          \]
      \end{Itemize}

      \par\smallskip
      \uncover<2->{%
        Zeigen \Bmph{Hilfsaussage [HA]:}
        \begin{block}{}
          Für alle $T_i$ und alle Zustände $p$ im MZ von $v$ in $T_{i+1}$\\
          gibt es einen Zustand $q$ im MZ von $v$ in $T_{i}$\\
          und einen endlichen Run $q\dots p$ von $\Aut{A}$ auf dem zugehörigen Teilwort von $\alpha$,
          der einen akzZ enthält.
        \end{block}
      }

      \par\bigskip
      \uncover<3->{%
        Beweis der Hilfsaussage: s.\ Tafel \Tafel
      }

      \note{%
        \textbf{8:36}
        
        \parI
        Jetzt der eigentliche Beweis.
        
        \parII
        \textbf{Bis 9:10}
      }
    \end{frame}

  % ------------------------------------------------------------------------------------------
    \begin{frame}
      \frametitle{Korrektheit}

%       \Bmph{Beweis (Rest).}~
      Kombiniere nun Runs aus [HA] zu $\infty$ Run von $\Aut{A}$
      \begin{Itemize}
        \item
          Seien $0 = i_0 < i_1 < i_2 < \dots$ Positionen der $T_i$ in $s$
        \item
          Sei $M_j$ der MZ von $v$ an Positionen $i_j$, $j \geqslant 0$
      \end{Itemize}

      \par\vspace*{5.3pt}
      \uncover<2->{%
        Konstruiere Baum $\mathcalreg{T}$:
        \begin{Itemize}
          \item
            Knoten $=$ Paare $(q,j)$ mit $q \in M_j$, $j \geqslant 0$
          \item
            Jeder Knoten $(p,j+1)$ bekommt \emph{genau ein} Elternteil:\\
            beliebiger $(q,j)$ mit $q \in M_j$ und $\exists$ Run $q\dots p$ wie in \Bmph{[HA]}
%             \par\smallskip
%           \item[$\leadsto$]
%             $\mathcalreg{T}$ ist Baum mit Wurzel $(q_I,0)$\qquad $(I = \{q_I\})$
%           \item[$\leadsto$]
%             $\mathcalreg{T}$ hat $\infty$ viele Knoten, endl.\ Verzweigungsgrad $\leqslant |Q|$
          \item[$\Rightarrow$]
%             $\mathcalreg{T}$ hat 
            $\infty$ viele Knoten, Verzweigungsgrad $\leqslant |Q|$, Wurzel $(q_I,0)$
        \end{Itemize}
      }

      \par\vspace*{5.3pt}
      \uncover<3->{%
        Nach Lemma von K\H onig (nächste Folie) folgt:
        \begin{Itemize}
          \item
            $\mathcalreg{T}$ hat einen $\infty$ Pfad $(q_I,0),~ (q_1,1),~ (q_2,2),~ \dots$;
          \item
            Verkettung aller Teilruns entlang dieses Pfades ist ein Run von $\Aut{A}$ auf $\alpha$,
            der $\infty$ oft einen akzZ besucht
          \item[$\Rightarrow$]
            $\alpha \in L_\omega(\Aut{A})$ \qed
        \end{Itemize}
      }


      \note{%
        \textbf{9:10}
        
        \par
      }
    \end{frame}

  % ------------------------------------------------------------------------------------------
    \begin{frame}
      \frametitle{Im Korrektheitsbeweise benutztes Werkzeug}
      
      \begin{Lemma}[Lemma von K\H onig]
        Jeder unendliche Baum mit endlichem Verzweigungsgrad\\
        hat einen unendlichen Pfad.
      \end{Lemma}
      
      \par\bigskip
%       \uncover<2->{%
        \begin{Itemize}
          \item
            ohne Beweis
          \item
            "`endlicher Verzweigungsgrad"':\\
            jeder Knoten hat endlich viele Kinder
          \item
            1936 von D\'enes K\H onig (1884--1944, Mathematiker, Budapest)
        \end{Itemize}
%       }
      
      \note{%
          \textbf{9:15}
          
          \parII
          \textbf{einschl.\ 4min Pause bis 9:20}
          
          \par
        }
    \end{frame}

  % ------------------------------------------------------------------------------------------
    \begin{frame}
      \frametitle{Vollständigkeit}

      \begin{Lemma}
        Sei $\Aut{A} = (Q,\Sigma,\Delta,I,F)$ ein NBA
        und sei $\Aut{A}^d = (Q^d,\Sigma,\Delta^d,I^d,\calP)$ der DRA,
        den man nach Safras Konstruktion aus \Aut{A} erhält.
        \par\smallskip
        Dann gilt $L_\omega(\Aut{A}) \subseteq L_\omega(\Aut{A}^d)$.
        \label{lem:completeness_safra}
      \end{Lemma}

      \par\smallskip
      \Bmph{Beweis.} %\qquad Sei $I^d = \{S_I\}$
      \begin{Itemize}
        \item
          Sei $\alpha \in L_\omega(\Aut{A})$ und $r=q_0q_1q_2\dots$ erfolgr.\ Run von $\Aut{A}$ auf $\alpha$
        \item
          $\Aut{A}^d$ hat \emph{eindeutigen} Run $s=S_0S_1S_2\dots$ auf $\alpha$ % mit $S_0 = S_I$
        \item
          Zu zeigen: $s$ ist erfolgreich, d.\,h.:
      \end{Itemize}

      \par
      \uncover<2->{%
%         Zeigen \Bmph{Hilfsaussage [HA]:}
        \begin{block}{}
          Es gibt einen Knotennamen $v$, für den gilt:
          \vspace*{-2pt}
          \begin{Enumerate}
            \item[\Bmph{(a)}]
              $\exists m \geqslant 0$ : 
              $S_i$ enthält Knoten $v$ für alle $i \geqslant m$
              \vspace*{-1pt}
            \item[\Bmph{(b)}]
              $v$ ist in $\infty$ vielen $S_i$ mit $\circled{!}$ markiert
          \end{Enumerate}
        \end{block}
      }

%       \par\bigskip
%       \uncover<3->{%
%         \Bmph{Aus [HA] folgt:}
%         \par\medskip
%         (1) $\text{Inf}(s) \cap E_v = \emptyset$\quad und \quad
%         (2) $\text{Inf}(s) \cap F_v \neq \emptyset$
%         \par\medskip
%         $\Rightarrow$ Also ist $s$ erfolgreich
%       }

      \par\smallskip
      \uncover<3->{%
        Beweis dieser Aussage: s.\ Tafel \Tafel~~~~
        \par
        \vspace*{-.95\baselineskip}
        \qed
      }
      
      
%       siehe Tafel. \Tafel~~~~
%       \par
%       \vspace*{-1\baselineskip}
%       \strut \qed

      \note{%
        \textbf{9:20 bis 9:55}
        
        \par
      }
    \end{frame}

  \newcommand{\myto}{\only<3|handout:0>{$\to$}\only<4>{\Emph{\ding{220}}}}
  % ------------------------------------------------------------------------------------------
    \begin{frame}
      \frametitle{Konsequenz aus Safras Konstruktion}
      
      \begin{Satz}[Satz von McNaughton]
        Sei $\Aut{A}$ ein NBA.
        Dann gibt es einen DRA $\Aut{A}^d$
        mit $L_\omega(\Aut{A}^d) = L_\omega(\Aut{A})$.%
        \label{thm:mcnaughton}
      \end{Satz}
      
      \par\smallskip
%       \Bmph{Beweis.} Folgt aus Lemmas \ref{lem:completeness_safra} und \ref{lem:soundness_safra}.
%       \Bmph{Beweis.} Folgt aus Lemmas \ref{lem:soundness_safra} und \ref{lem:completeness_safra}.
      \Bmph{Beweis.} Folgt aus Lemma \ref{lem:soundness+completeness_safra}.
      
      \par\bigskip
      \uncover<2->{%
        \begin{Folgerung}
          Die Klasse der Büchi-erkennbaren Sprachen ist unter Komplement abgeschlossen.
        \end{Folgerung}
      }
      
      \par\smallskip
      \uncover<3->{%
%         \Bmph{Beweis.} Über folgende Transformationskette:
%         \par\smallskip
        \begin{tabular}[b]{@{}llcll@{}}
           \Bmph{Beweis.} & \multicolumn{4}{l}{Über folgende Transformationskette:}                                           \\[2pt]
                          & NBA für $L$ & \myto & DRA für $L$           & (gemäß Satz \ref{thm:mcnaughton})                         \\
                          &             & $\to$ & DMA für $L$           & (gemäß Satz \ref{thm:gleichmaechtigkeit_deterministisch}) \\
                          &             & $\to$ & DMA für $\overline L$ & (wie gehabt)                                              \\
                          &             & \myto & NBA für $\overline L$ & (gemäß Satz \ref{thm:gleichmaechtigkeit})%
        \end{tabular}%
        \qed
      }
      \note{%
        \textbf{9:55}
        
        \par
      }
    \end{frame}

  % ------------------------------------------------------------------------------------------
    \begin{frame}
      \label{fra:komplexitaet_komplementierung}
      \frametitle{Anmerkungen zur Komplexität}
      

      \Bmph{Determinisierung} NBA $\to$ DRA gemäß Safras Konstruktion
      \begin{Itemize}
        \item
          liefert einen \Emph{exponentiell} größeren DRA
        \item
          genauer: wenn der NBA $n$ Zustände hat,
          \begin{Itemize}
            \item
              gibt es $2^n$ mögliche Makrozustände
            \item
              und $2^{O(n \log n)}$ mögliche Safrabäume
            \item[$\leadsto$]
              DRA hat maximal $m := 2^{O(n \log n)}$ Zustände
          \end{Itemize}
        \item
          Das ist optimal (siehe Roggenbachs Kapitel in LNCS 2500)
      \end{Itemize}


      \par\bigskip
      \uncover<2->{%
        \Bmph{Komplementierung} beinhaltet auch den Schritt DMA $\to$ NBA
        \begin{Itemize}
          \item
            liefert einen nochmal \Emph{exponentiell} größeren DBA:
            \par\smallskip
            wenn der DMA $m$ Zustände hat, \\
            hat der NBA $O(m\cdot 2^m)$ Zustände
          \item[$\leadsto$]
            Resultierender NBA hat $2^{2^{O(n^2)}}$ Zustände
          \item<3->
            Alternative Prozedur erfordert nur $2^{O(n \log n)}$ Zustände
        \end{Itemize}
      }

      \note{%
        \textbf{9:57 bis Ende}
        
        \parI
        % TODO
        \textbf{TODO:}~ Größe und Anzahl der Safra-Bäume sauber abschätzen \& erklären!
        (siehe Folie $\approx{}$66)
        
        \par
      }
    \end{frame}




  % ==============================================================================================
  % ==============================================================================================
  \section[Entscheidungsprobl.]{Entscheidungsprobleme}
  
    % ------------------------------------------------------------------------------------------
    \begin{frame}
      \frametitle{Vorbetrachtungen}
      
      Betrachten 4 Standardprobleme:
      \begin{Itemize}
        \item
          Leerheitsproblem
        \item
          Wortproblem\quad {\small (Wort ist durch NBA gegeben)}
        \item
          Äquivalenzproblem
        \item
          Universalitätsproblem
      \end{Itemize}

      \par\medskip
      \uncover<2->{%
        \Emph{Beschränken} uns auf das \Emph{Leerheitsproblem} -- die anderen \dots \\
        \begin{Itemize}
          \item
            lassen sich wie üblich darauf reduzieren
          \item
            aber teils mit (doppelt) exponentiellem "`Blowup"' \\
            {\small (Determinisierung, Komplementierung, siehe Folie \ref{fra:komplexitaet_komplementierung})} \\
%             $\leadsto$ höhere, teils nicht optimale Komplexität
            $\leadsto$ höhere Komplexität
        \end{Itemize}
      }

      \par\medskip
      \uncover<3->{%
        \Emph{Beschränken} uns auf \Emph{NBA},\\
        aber Entscheidbarkeit überträgt sich auf die anderen Modelle%
      }

      \note{
        \textbf{16:00}
        
        \par
      }
    \end{frame}

  % ------------------------------------------------------------------------------------------
    \begin{frame}
      \frametitle{Das Leerheitsproblem}
      
      \Bmph{Zur Erinnerung:}
      \begin{Itemize}
        \item[]
          Gegeben: NBA $\Aut{A}$
        \item[]
          Frage: Gilt $L_\omega(\Aut{A}) = \emptyset$\,?
      \end{Itemize}
%       Mengenschreibweise: $\{\text{NBA~} \Aut{A} \mid L_\omega(\Aut{A}) = \emptyset\}$

      \uncover<2->{%
        \par %\bigskip
        \begin{Satz}
          Das Leerheitsproblem für NBAs ist entscheidbar.
        \end{Satz}
      }

      \par\medskip
      \uncover<3->{%
          \Bmph{Quiz:} Welche Komplexität hat es? \NL\ \dots\ \PT\ \dots\ höher?
      }

      \par\bigskip
      \uncover<4->{%
        \Bmph{Beweis.}
        \quad $L_\omega(\Aut{A}) \neq \emptyset$ genau dann, wenn gilt:

        \par\medskip
        \qquad Es gibt $q_0 \in I$ und $q_f \in F$ \\
        \qquad und einen Pfad von $q_0$ zu $q_f$ in $\Aut{A}$\\
        \qquad und einen Pfad von $q_f$ zu $q_f$ in $\Aut{A}$

        \par\bigskip
        \Bmph{$\Rightarrow$} Reduktion zum Leerheitsproblem für NEAs:
%         siehe Tafel. \Tafel        
      }

%       \par\bigskip\bigskip
%       \uncover<3->{%
%         \Bmph{Komplexität:} \NL-vollständig (Wegsuche in Graphen)%
%       }

      \note{
        \textbf{16:02}
        
        \par
      }
    \end{frame}

  % ------------------------------------------------------------------------------------------
    \begin{frame}
      \frametitle{Das Leerheitsproblem}

      Bezeichne $L(\Aut{A}_{q_1,q_2})$
      die von $\Aut{A}$ \Emph{als NEA} erkannte Sprache,\\
      wenn $\{q_1\}$ Anfangs- und $\{q_2\}$ Endzustandsmenge ist

      \par\bigskip
      Folgender Algorithmus entscheidet das Leerheitsproblem:
      %
      \begin{block}{}
        \renewcommand{\baselinestretch}{1.2}
        \begin{algorithm}[H]
          \DontPrintSemicolon
          \SetKwProg{Fn}{Function}{:}{end}
          \SetKwFunction{nonreach}{\myblu{nonreach}}%
          \SetKwInOut{Input}{input}\SetKwInOut{Output}{output}
          \SetKwComment{tcp}{//}{}
          \SetKw{KwRet}{return}
          \SetKw{Return}{return}

          Rate nichtdeterministisch $q_0 \in I$ und $q_F \in F$\;
          \lIf{$L(\Aut{A}_{q_0,q_f}) \subseteq \{\varepsilon\}$ oder $L(\Aut{A}_{q_f,q_f}) \subseteq \{\varepsilon\}$}{\Return "`leer"'}
          \Return "`nicht leer"'        
        \end{algorithm}
      \end{block}

      Dabei ist~ $L(\Amc_{\dots}) \subseteq \{\varepsilon\}$ ~gdw.~ $L(\Amc_{\dots}) \cap \underbrace{(\Sigma \setminus \{\varepsilon\})}_{\text{konst.\ NEA}} = \emptyset$
      
      \parI
      $\Big($"`$L(\Amc_{\dots}) = \emptyset$"' genügt nicht, denn $L_\omega($
      %
      \hspace*{-2mm}%
      \begin{tikzpicture}[%
        baseline=-2pt,node distance=20mm,>=Latex,
        initial text="",
        every state/.style={draw=black,semithick,fill=black!5,inner sep=0mm,minimum size=3mm},
        every edge/.style={draw=black,semithick}
      ]              
        \node[state,initial,accepting] {~};
      \end{tikzpicture}%
      %
      $) = \emptyset%$, aber $L(\Amc_{q_0,q_0})=\{\varepsilon\}
      .\Big)$
      \qed
      
%      \begin{Enumerate}
%        \item
%          Rate nichtdeterministisch $q_0 \in I$
%        \item
%          Rate nichtdeterministisch $q_f \in F$
%        \item
%          Wenn $L(\Aut{A}_{q_0,q_f}) = \emptyset$, dann Ausgabe "`leer"' und stop
%        \item
%          Wenn $L(\Aut{A}_{q_f,q_f}) = \emptyset$, dann Ausgabe "`leer"' und stop
%        \item
%          Ausgabe "`nicht leer"' \qed
%      \end{Enumerate}

      \par\bigskip
      \uncover<2->{%
        Das ist ein \NL-Algorithmus\\
        \begin{footnotesize}
          (eigentlich \coNL, aber \NL\ $=$ \coNL\ ist bekannt, Immerman–Szelepcsényi 1987)
        \end{footnotesize}

        \par\medskip
        Leerheit für NBAs ist \NL-vollständig
      }

      \note{
        \textbf{16:05 bis 16:07}
        
        \parII
        \textbf{TODO:}~ besser formatieren! \\
        $\leadsto$ die 2 Zeilen unter der Box als Tafelanschrieb?
        
        \par
      }
    \end{frame}

    \newlength{\sternchen}
    \settowidth{\sternchen}{${}^*$}
    \newcommand{\stNOst}{\hspace*{\sternchen}\NO{}${}^*$}
    % ------------------------------------------------------------------------------------------
    \begin{frame}
      \frametitle{Überblick Entscheidungsprobleme für NBAs}
      
%      \begin{center}
        \begin{tabular}{cccc}
          \hline\stab
%                  &               &                 & für NEAs          \\
          Problem & entscheidbar? & Komplexität     & effizient lösbar? \\
          \hline\stab
          LP      & \YES          & \NL-vollständig & \YES              \\
          WP      & \multicolumn{3}{c}{{\small \textcolor{black!70}{--- macht keinen Sinn, da Eingabewort $\infty$ ---}}} \\
          ÄP      & \YES          & \PSPACE-vollst. & \stNOst           \\
          UP      & \YES          & \PSPACE-vollst. & \stNOst           \\
          \hline
        \end{tabular}

        \par\bigskip
        \begin{tabular}{@{}l@{\,}l@{}}
          ${}^*$ & unter den üblichen komplexitätstheoretischen Annahmen \\
                 & (z.\,B.\ $\PSPACE \neq \PT$)
        \end{tabular}
%      \end{center}
      \note{
        \textbf{16:07}

        \par\medskip
        Hier nochmal für Euch als Überblick. \\
        Situation dieselbe wie für NEAs; nur WP macht hier keinen Sinn.

        \par
      }
    \end{frame}



  % ==============================================================================================
  % ==============================================================================================
  \section[\protect\emph{Model-Checking}]{\protect\emph{Anwendung: Model-Checking in Linearer Temporallogik (LTL)}}
  
    % ------------------------------------------------------------------------------------------
    \begin{frame}
      \frametitle{Reaktive Systeme und Verifikation}

      \Bmph{Reaktive Systeme}
      \begin{Itemize}
        \item
          interagieren mit ihrer Umwelt
        \item
          terminieren oft nicht
        \item
          Beispiele:
          \begin{Itemize}
            \item
              Betriebssysteme, Bankautomaten, Flugsicherungssysteme, \dots
            \item
              s.\,a.\ Philosophenproblem, Konsument-Produzent-Problem
          \end{Itemize}
      \end{Itemize}

      \par\bigskip
      \uncover<2->{%
        \Bmph{Verifikation} $=$ Prüfen von Eigenschaften eines Systems
        \begin{Itemize}          
          \item
            Eingabe-Ausgabe-Verhalten hat hier keine Bedeutung
          \item
            Andere Eigenschaften sind wichtig,\\
            z.\,B.: keine Verklemmung (deadlock) bei Nebenläufigkeit
        \end{Itemize}
      }

      \note{
        \textbf{16:08}
        
        \par
      }
    \end{frame}

  % ------------------------------------------------------------------------------------------
    \begin{frame}
      \frametitle{Repräsentation eines Systems}

      \Bmph{Bestandteile}
      \begin{Itemize}
        \item
          \Bmph{Variablen:} repräsentieren Werte, die zur Beschreibung des Systems notwendig sind
        \item
          \Bmph{Zustände:} "`Schnappschüsse"' des Systems\\
          Zustand enthält Variablenwerte zu einem bestimmten Zeitpunkt
        \item
          \Bmph{Transitionen:} erlaubte Übergänge zwischen Zuständen
      \end{Itemize}

      \par\bigskip
%      \uncover<2->{%
        \Bmph{Pfad} (Berechnung) in einem System:\\
        unendliche Folge von Zuständen entlang der Transitionen
%      }

      \note{
        \textbf{16:09}
        
        \par
      }
    \end{frame}

  % ------------------------------------------------------------------------------------------
    \begin{frame}
      \frametitle{Transitionsgraph als Kripke-Struktur${}^*$}
      
      \begin{Definition}
        Sei AV eine Menge von Aussagenvariablen.
        Eine \Bmph{Kripke-Struktur} $\calS$ über AV ist ein Quadrupel $\calS=(S,S_0,R,\ell)$, wobei
        \begin{Itemize}
          \item
            $S$ eine endliche nichtleere Menge von \Bmph{Zuständen} ist,
          \item
            $S_0 \subseteq S$ die Menge der \Bmph{Anfangszustände} ist,
          \item<2->
            $R \subseteq S \times S$ eine \Bmph{Übergangsrelation} ist,\\
            die \Bmph{total ist}:~ $\forall s \in S~ \exists s' \in S : sRs'$
          \item<3->
            $\ell : S \to 2^{\text{AV}}$ eine Funktion ist, die \Bmph{Markierungsfunktion}.\\
            \begin{small}
              $\ell(s) = \{p_1,\dots,p_m\}$ bedeutet: in $s$ sind genau $p_1,\dots,p_m$ wahr
            \end{small}
%             die jeden Zustand
%             mit der Menge von Aussagenvariablen markiert, die dort wahr sind.
        \end{Itemize}
        
        \par\medskip
        \uncover<4->{%
          Ein \Bmph{Pfad} in $\calS$ ist eine unendliche Folge $\pi = s_0s_1s_2\ldots$ von Zuständen
          mit $s_0 \in S_0$ und $s_iRs_{i+1}$ für alle $i \geqslant 0$.%
        }

      \end{Definition}

      \par\bigskip
      {\footnotesize ${}^*$\,Saul Kripke, geb.\ 1940, Philosoph und Logiker, Princeton und New York, USA}

      \note{
        \textbf{16:10}
        
        \par
      }
    \end{frame}

  % ------------------------------------------------------------------------------------------
    \begin{frame}
      \label{fra:bsp_mikrowelle}
      \frametitle{Beispiel 1: Mikrowelle}

      \begin{center}
        \begin{minipage}{.7\textwidth}
          \includegraphics[angle=270,width=\linewidth]{img/mc_microwave.jpg}\\
          {\footnotesize aus: E.\,M.\ Clarke et al., Model Checking, MIT Press 1999}
        \end{minipage}
      \end{center}

      \note{
        \textbf{16:14}
        
        \parI
        \textbf{TODO:}~ Mikrowellenbeispielbild selber tikzen; \\
        vernünftige Bezeichnungen ($\lnot$ statt $\sim$, "`offen"' statt "`Close"', \\
        keine Pfeilbeschriftungen usw.)
        
        \par
      }
    \end{frame}

  % ------------------------------------------------------------------------------------------
    \begin{frame}
      \frametitle{Beispiel 2: nebenläufiges Programm}

      \begin{tabular}{@{}l>{\footnotesize}rll@{}}
        $P$   &  0        & \textbf{cobegin}                        &                                         \\
              &  1        & \quad $P_0 \| P_1$                      &                                         \\
              &  2        & \textbf{coend}                          &                                         \\[8pt]
        \uncover<2->{%
        $P_0$ & 10        & \textbf{while}(true) \textbf{do}        &                                         \\
              & 11        & \quad \textbf{wait}($\text{turn} = 0$)  &                                         \\
              & \dred{12} & \dred{\quad $\text{turn} \leftarrow 1$} & \uncover<3->{\Emph{kritischer Bereich}} \\
              & 13        & \textbf{end while}                      &                                         \\[8pt]
        }%
        \uncover<4->{%
        $P_1$ & 20        & \textbf{while}(true) \textbf{do}        &                                         \\
              & 21        & \quad \textbf{wait}($\text{turn} = 1$)  &                                         \\
              & \dred{22} & \dred{\quad $\text{turn} \leftarrow 0$} & \uncover<5->{\Emph{kritischer Bereich}} \\
              & 23        & \textbf{end while}                      &
        }
      \end{tabular}

      \note{
        \textbf{16:17}
        
        \par
      }
    \end{frame}

  % ------------------------------------------------------------------------------------------
    \begin{frame}
      \label{fra:nebenlaeufigkeit}
      
      \frametitle{Beispiel 2: nebenläufiges Programm}

      Variablen in der zugehörigen Kripke-Struktur: $v_1, v_2, v_3$ mit
      \begin{Itemize}
        \item
          $v_1,v_2$: Werte der Programmzähler für $P_0,P_1$\\
          (einschl. $\bot$: Teilprogramm ist nicht aktiv)
        \item
          $v_3$: Werte der gemeinsamen Variable turn
      \end{Itemize}

      Kripke-Struktur: \\
      \Fig{60}
      
      \note{
        \textbf{16:20}
        
        \parI
        Hier benutzen wir für jede Programmzeile eine getrennte Aussagenvariable \\
        (auch wenn wir die wenigen möglichen Zustände mit weniger AVs kodieren könnten).
        
        \parI
        \textbf{TODO:}~ Variablen sind falsch beschrieben. Es gibt nicht nur $v_1,v_2,v_3$, \\
        sondern $10,11,12,13,\,20,21,22,23,\,\bot_0,\bot_1,t_0,t_1$ (letztere für "`turn${}_i$"'). \\
        $\leadsto$ Beschreibung und Bild anpassen
        
        \par
      }
    \end{frame}


  % ------------------------------------------------------------------------------------------
    \begin{frame}
      \frametitle{Spezifikationen}

      \dots\ sind Zusicherungen über die Eigenschaften eines Systems, z.\,B.:
      \par\smallskip
      \begin{Itemize}
        \item
          "`Wenn ein Fehler auftritt, ist er nach endlicher Zeit behoben."'
        \item
          "`Wenn die Mikrowelle gestartet wird, \\
          fängt sie immer nach endlicher Zeit an zu heizen."'
        \item
          "`Wenn die Mikrowelle gestartet wird, \\
          ist es \emph{möglich}, danach zu heizen."'
          \par\bigskip
        \item<2->
          "`Es kommt nie vor,\\
          dass beide Teilprogramme zugleich im kritischen Bereich sind."'
        \item<2->
          "`Jedes Teilprog.\ kommt beliebig oft in seinen krit.\ Bereich."'
        \item<2->
          "`Jedes Teilprogramm \emph{kann} beliebig oft in seinen kritischen Bereich gelangen."'
          \par\bigskip
        \item<3->
          \dots
      \end{Itemize}

      \note{
        \textbf{16:24}
        
        \par
      }
    \end{frame}

  % ------------------------------------------------------------------------------------------
    \begin{frame}[t]
      \frametitle{Spezifikationen für das Beispiel Mikrowelle}
      
      \begin{minipage}{.7\textwidth}
        \includegraphics[angle=270,width=\linewidth]{img/mc_microwave.jpg}
      \end{minipage}
      \hfill
      \begin{minipage}{.25\textwidth}
        \begin{footnotesize}
          aus:\\
          E.\,M.\ Clarke et al.,\\
          Model Checking,\\
          MIT Press 1999
          \par
        \end{footnotesize}
      \end{minipage}

      \par\medskip
      \only<2-3|handout:1>{\strut "`Wenn ein Fehler auftritt, ist er nach endlicher Zeit behoben."'}\only<3|handout:1>{ \NO}%
      \only<4-5|handout:2>{\strut \mbox{"`Wenn MW gestartet, beginnt sie immer nach endl.\ Zeit zu heizen."'\only<5|handout:2>{ \NO}\hspace*{-7mm}}}%
      \only<6-7|handout:3>{\strut "`Wenn MW gestartet, ist es \emph{möglich}, danach zu heizen."'}\only<7|handout:3>{ \YES}%

      \note{
        \textbf{16:26}
        
        \parII
        \textbf{Jeweils fragen:}~ Ist die Zusicherung erfüllt?
        
        \parII
        \textbf{Vorsicht:}~ Es kommt darauf an, ob gemeint ist "`in jedem Lauf"' oder "`es gibt einen Lauf"'.
        (univ.\ vs.\ exist.\ Model-Checking) -- \textbf{Diskutieren!}
        
        \par
      }
    \end{frame}

  % ------------------------------------------------------------------------------------------
    \begin{frame}[t]
      \frametitle{Spezifikationen für das Beispiel Nebenläufigkeit}
      
      \vspace*{2\baselineskip}
      \Fig{60}
      
      \par\medskip
      \only<2-3|handout:1>{\strut "`Es kommt nie vor,\\ dass beide Teilprogramme zugleich im kritischen Bereich sind."'}\only<3|handout:1>{ \YES}%
      \only<4-5|handout:2>{\strut ~\\"`Jedes $P_i$ kommt beliebig oft in seinen kritischen Bereich."'}\only<5|handout:2>{ \NO}%
      \only<6-7|handout:3>{\strut ~\\\mbox{"`Jedes $P_i$ \emph{kann} beliebig oft in seinen kritischen Bereich kommen."'\only<7|handout:3>{ \YES}\hspace*{-5mm}}}%

%       Es kommt nie vor,\\
%       dass beide Teilprogramme zugleich im kritischen Bereich sind.
%       \uncover<2->{\YES}
% 
%       \par\bigskip
%       \uncover<3->{%
%         Jedes $P_i$ kommt beliebig oft in seinen kritischen Bereich.
%         \uncover<4->{\NO}
%       }
% 
%       \par\bigskip
%       \uncover<5->{%
%         Jedes $P_i$ \emph{kann} beliebig oft in seinen kritischen Bereich kommen.
%         \uncover<6->{\YES}
%       }

      \note{
        \textbf{16:30}
        
        \parII
        \textbf{Jeweils wieder fragen \dots}
        
        \parII
        \textbf{Anmerken:}~ Diese Sätze sind mit Absicht schwammig formuliert.
        Wir werden noch Hilfsmittel kennen lernen, mit denen man sie präzise formulieren kann.
        
        \par
      }
    \end{frame}

  % ------------------------------------------------------------------------------------------
    \begin{frame}
      \frametitle{Model-Checking}

      \dots\ beantwortet die Frage,\\
      ob ein gegebenes System eine gegebene Spezifikation erfüllt

      \par\bigskip
      \uncover<2->{%
        \begin{Definition}[Model-Checking-Problem \Bmph{MCP}]
          Gegeben ein System $\calS$ und eine Spezifikation $E$,
          \begin{Itemize}
            \item
              gilt $E$ \Emph{für jeden Pfad} in $\calS$\,?\\
              \Bmph{(universelle Variante)}
            \item
              \Emph{gibt es einen Pfad} in $\calS$, der $E$ erfüllt?\\
              \Bmph{(existenzielle Variante)}
          \end{Itemize}
          \label{def:model-checking-problem}
        \end{Definition}
      }

      \par\bigskip
      \uncover<3->{%
        \Emph{Frage:} Wie kann man Model-Checking
        \begin{Itemize}
          \item
            exakt beschreiben und
          \item
            algorithmisch lösen?
        \end{Itemize}
      }

      \note{
        \textbf{16:33}
        
        \par
      }
    \end{frame}

  % ------------------------------------------------------------------------------------------
    \begin{frame}
      \frametitle{Model-Checking mittels Büchi-Automaten!}

      \Bmph{Schritt 1}
      \par\smallskip
      \begin{Itemize}
        \item
          Stellen System $\calS$ als NBA $\Aut{A}_{\calS}$ dar
          \par\smallskip
          $\leadsto$ Pfade in $\calS$ sind erfolgreiche Runs von $\Aut{A}_{\calS}$
          \par\bigskip
        \item<2->
          Stellen Spezifikation $E$ als NBA $\Aut{A}_E$ dar
          \par\smallskip
          $\leadsto$ $\Aut{A}_E$ beschreibt die Pfade, die $E$ erfüllen
          \par\bigskip
        \item<3->[$\leadsto$]
          Universelles MCP ~$=$~ "`$L(\Aut{A}_{\calS}) \subseteq L(\Aut{A}_E)$\,?"'
          \par\smallskip
          Existenzielles MCP ~$=$~ "`$L(\Aut{A}_{\calS}) \cap L(\Aut{A}_E) \neq \emptyset$\,?"'
          \par\smallskip
          {\footnotesize (beide reduzierbar zum Leerheitsproblem, benutzt Abschlusseigenschaften)}
      \end{Itemize}

      \par\bigskip
      \uncover<4->{%
        \Bmph{Schritt 2}
        \begin{Itemize}
          \item
            intuitivere Beschreibung von $E$ mittels Temporallogik
          \item
            Umwandlung von Temporallogik-Formel $\varphi_E$ in Automaten $\Aut{A}_E$
        \end{Itemize}
      }

      \note{
        \textbf{16:35}
        
        \parII
        Nun zunächst Schritt 1.
        
        \par
      }
    \end{frame}

  % ------------------------------------------------------------------------------------------
    \begin{frame}
      \frametitle{Konstruktion des NBA $\Aut{A}_{\calS}$ für das System $\calS$}

      \Bmph{Erinnerung:} $\calS$ gegeben als Kripke-Struktur $\calS = (S,S_0,R,\ell)$\\
      \hspace*{18.5mm}{\small (Zustände, Anfangszustände, Transitionen, Markierungen)}

      \par\bigskip
      \uncover<2->{%
        \Bmph{Zugehöriger Automat} $\Aut{A}_{\calS} = (Q,\Sigma,\Delta,I,F)$\,:
        \begin{Itemize}
          \item
            $\Sigma = 2^{\text{AV}}$
          \item
            $Q=S \uplus \{q_0\}$
          \item
            $I=\{q_0\}$
          \item
            $F=Q$
          \item
            \begin{tabular}[t]{@{}c@{~}c@{~}c@{~}l@{}}
              $\Delta$ & $=$ &        & $\{~(q_0, \ell(s), s)~ \mid s \in S_0\}$ \\[4pt]
                       &     & $\cup$ & $\{~(s, \ell(s'), s')~ \mid (s,s') \in R\}$
            \end{tabular}
        \end{Itemize}
      }

      \par\bigskip
      \uncover<3->{%
        \Gmph{Beispiel:} siehe Tafel. \Tafel
      }
      
      \note{
        \textbf{16:37}
        
        \parII
        Kripke-Struktur in Automaten umwandeln ist ganz einfach: \\
        im Prinzip ist die KS bereits der Automat.
        
        \par
      }
    \end{frame}

  % ------------------------------------------------------------------------------------------
    \begin{frame}
      \frametitle{Beschreibung von $E$ durch NBA $\Aut{A}_E$}

      \Bmph{Beispiel Mikrowelle}~ {\footnotesize (siehe Bild auf Folie \ref{fra:bsp_mikrowelle})}
      \begin{Itemize}
        \item[(a)]
          "`Wenn ein Fehler auftritt, ist er nach endlicher Zeit behoben."'
        \item[(b)]
          "`Wenn die Mikrowelle gestartet wird, \\
          fängt sie nach endlicher Zeit an zu heizen."'
        \item[(c)]
          "`Wenn die Mikrowelle gestartet wird, \\
          ist es \emph{möglich}, danach zu heizen."'
      \end{Itemize}

      \par\bigskip
      \Bmph{Beispiel Nebenläufigkeit}~ {\footnotesize (siehe Bild auf Folie \ref{fra:nebenlaeufigkeit})}
      \begin{Itemize}
        \item[(d)]
          "`Es kommt nie vor,\\
          dass beide Teilprog.\ zugleich im kritischen Bereich sind."'
        \item[(e)]
          "`Jedes Teilprog.\ kommt beliebig oft in seinen krit.\ Bereich."'
        \item[(f)]
          "`Jedes Teilprogramm \emph{kann} beliebig oft in seinen kritischen Bereich gelangen."'
      \end{Itemize}

      \Tafel

      \note{
        \textbf{16:43 bis 16:55 und 5min Pause}
        
        \par
      }
    \end{frame}

  % ------------------------------------------------------------------------------------------
    \begin{frame}
      \frametitle{Verifikation mittels der konstruierten NBAs}

      Gegeben sind wieder System $\calS$ und Spezifikation $E$.

      \par\bigskip
      \Bmph{Universelles MCP}
      \begin{Itemize}
        \item
          Gilt $E$ für jeden Pfad in $\calS$\,?
        \item<2->
          äquivalent:~ $L(\Aut{A}_{\calS}) \subseteq L(\Aut{A}_E)$\,?
        \item<3->
          äquivalent:~ $L(\Aut{A}_{\calS}) \cap \overline{L(\Aut{A}_E)} = \emptyset$\,?
        \item<4->[$\leadsto$]
          Komplementierung $\Aut{A}_E$, Produktautomat, Leerheitsproblem
        \item<5->
          Komplexität: \PS~
          \mbox{{\small \scalebox{.98}[1]{(exponentielle Explosion bei Komplementierung)}}\hspace*{-20mm}}
      \end{Itemize}

      \par\bigskip
      \uncover<6->{%
        \Bmph{Existenzielles MCP}
        \begin{Itemize}
          \item
            Gibt es einen Pfad in $\calS$, der $E$ erfüllt?
          \item<7->
            äquivalent: $L(\Aut{A}_{\calS}) \cap L(\Aut{A}_E) \neq \emptyset$\,?
          \item<8->[$\leadsto$]
            Produktautomat, Leerheitsproblem
          \item<9->
            Komplexität: \NL \hfill
            {\small (keine exponentielle Explosion)}
        \end{Itemize}
      }
      
      \note{
        \textbf{17:00}
        
        \parI
        "`Exponentielle Explosion"':~ wie schon gesagt, liefert der Weg über die Safra-Konstruktion
        einen \textbf{doppelt} exp.\ Blowup.
        
        \parI
        Es gibt aber direkte Verfahren zur Komplementierung mit einfach exp.\ Blowup.
        
        \parII
        Außerdem kann man natürlich nicht den exp.\ großen Komplement-Automaten im Ganzen
        erzeugen, wenn man nur Polyplatz zur Verfügung hat.
        
        \parI
        Man muss ihn also "`on the fly"' stückchenweise generieren, \\
        während man den Algo.\ für das Leerheitsproblem laufen lässt.
        
        \par
      }
    \end{frame}

  % ------------------------------------------------------------------------------------------
    \begin{frame}
      \frametitle{Bemerkung zur Implementierung}

      \Bmph{Praktisches Problem}
      \par\smallskip
      \begin{Itemize}
        \item
          Komplexität von MCP wird \Emph{bezüglich $|\Aut{A}_{\calS}| + |\Aut{A}_E|$} gemessen 
        \item
          $|\calS|$ und damit $|\Aut{A}_{\calS}|$ ist \Emph{exponentiell} in der Anzahl der Variablen:\\
          \Emph{State space explosion problem}
        \item[$\leadsto$]
          universelles bzw.\ existenzielles MCP sind eigentlich\\
          in \EXPSPACE\ bzw.\ in \PSPACE\ bezüglich Anz.\ der Variablen
      \end{Itemize}

      \par\bigskip
      \uncover<2->{%
        \Bmph{Abhilfe:}
        \begin{Itemize}
          \item
            "`On-the-fly model checking"'
          \item
            Zustände von $\Aut{A}_{\calS}$ werden während des Leerheitstests\\
            nur bei Bedarf erzeugt
        \end{Itemize}
      }

      \note{
        \textbf{17:06}
        
        \parI
        "`Abhilfe"':~ macht natürlich nicht die Komplexität kleiner, \\
        vermeidet aber, den ganzen Automaten in den Speicher schreiben zu müssen.

        \par
      }
    \end{frame}
    
  % ------------------------------------------------------------------------------------------
    \begin{frame}
      \frametitle{Spezifikationen mittels Linearer Temporallogik (LTL)}

      \Bmph{Nun zu Schritt 2.}~ Ziele:
      \begin{Itemize}
        \item
          intuitivere Beschreibung der Spezifikation $E$ durch Formel $\varphi_E$
          \par\smallskip
        \item
          Prozedur zur Umwandlung $\varphi_E$ in $\Aut{A}_E$\\
          (!) allerdings ist $|\Aut{A}_E|$ exponentiell in $|\varphi_E|$
          \par\smallskip
        \item
          dafür Explosion bei Komplementierung vermeiden:\\
          wandle $\lnot\varphi_E$ in Automaten um
          \par\smallskip
        \item[$\leadsto$]
          beide MCP für LTL sind \PSPACE-vollständig
      \end{Itemize}

      \note{
        \textbf{17:08}
        
        \parII
      }
    \end{frame}

  % ------------------------------------------------------------------------------------------
    \begin{frame}
      \frametitle{LTL im Überblick}
%       \begin{tabular}{@{}r@{~}c@{~}l@{}}
%         \Bmph{LTL} & $=$ & Aussagenlogik \\
%                    &     & plus Operatoren, die über \Emph{Pfade} sprechen:
%       \end{tabular}
      \Bmph{LTL} $=$ Aussagenlogik $+$ Operatoren, die über \Emph{Pfade} sprechen:

%       \par\smallskip
%       \hrule

      \par\smallskip
      \begin{Itemize}
        \item[$F$]
          \dblu{($F$uture)}
          \par\smallskip
          $F\varphi$ bedeutet "`$\varphi$ ist irgendwann in der Zukunft wahr"'
          \par\medskip
        \item[$G$]
          \dblu{($G$lobal)}
          \par\smallskip
          $G\varphi$ bedeutet "`$\varphi$ ist ab jetzt immer wahr"'
          \par\medskip
        \item[$X$]
          \dblu{(ne$X$t)}
          \par\smallskip          
          $X\varphi$ bedeutet "`$\varphi$ ist im nächsten Zeitpunkt wahr"'
          \par\medskip
        \item[$U$:]
          \dblu{($U$ntil)}
          \par\smallskip          
          $\varphi U\psi$ bedeutet "`$\psi$ ist irgendwann in der Zukunft wahr\\
          \hspace*{25mm} und bis dahin ist immer $\varphi$ wahr"'
      \end{Itemize}

%       \par\smallskip
%       \hrule
%       
%       \par\smallskip
%       \Bmph{Pfad} $=$ Abbildung $s : \mathbb{N} \to 2^\PROP$ \Tafel\\
%       (\PROP ist abzählbare Menge von \Bmph{Aussagenvariablen})

      \note{
        \textbf{17:09}
        
        \parII
      }
    \end{frame}

  % ------------------------------------------------------------------------------------------
    \begin{frame}
      \frametitle{LTL-Syntax}
      
      Sei \Bmph{\PROP{}} abzählbare Menge von \Bmph{Aussagenvariablen}.
      
      \begin{Definition}[LTL-Formeln]
        \begin{Itemize}
          \item
            Jede Aussagenvariable $p \in \PROP$ ist eine LTL-Formel.
          \item<2->
            Wenn $\varphi$ und $\psi$ LTL-Formeln sind,\\
            dann sind die folgenden auch LTL-Formeln.
            \begin{Itemize}
              \item
                $\lnot \varphi$       \hfill "`nicht $\varphi$"'
              \item
                $\varphi \land \psi$  \hfill "`$\varphi$ und $\psi$"'
%               \item
%                 $\varphi \lor \psi$   \hfill "`$\varphi$ oder $\psi$"'
              \item
                $F\varphi$            \hfill "`in Zukunft irgendwann $\varphi$"'
              \item
                $G\varphi$            \hfill "`in Zukunft immer $\varphi$"'
              \item
                $X\varphi$            \hfill "`im nächsten Zeitpunkt $\varphi$"'
              \item
                $\varphi \mathbin U \psi$      \hfill "`in Zukunft irgendwann $\psi$; bis dahin immer $\varphi$"'
            \end{Itemize}
        \end{Itemize}
      \end{Definition}
      
      \par\smallskip
      \uncover<3->{%
        Verwenden die üblichen Abkürzungen\qquad
        $\varphi \lor \psi = \neg(\neg\varphi \land \neg\psi)$,
        $\varphi \to \psi = \neg\varphi \lor \psi$,\qquad
        $\varphi \leftrightarrow \psi = (\varphi \to \psi) \land (\psi \to \varphi)$
      }

      \note{
        \textbf{17:11}
        
        \parI
      }
    \end{frame}

  % ------------------------------------------------------------------------------------------
    \begin{frame}
      \frametitle{LTL-Semantik}

%       \begin{tabular}{@{}l@{~~}l@{}}
%         \Bmph{Pfad:} & Abbildung $\pi : \mathbb{N} \to 2^{\PROP}$ \\
%                      & schreiben $\pi_0\pi_1\pi_2\ldots$ statt $\pi(0)\pi(1)\pi(2)\ldots$
%       \end{tabular}
      \mbox{\Bmph{Pfad:} Abbildung $\pi : \mathbb{N} \to 2^{\PROP}$\quad Schreiben $\pi_0\pi_1\dots$ statt $\pi(0)\pi(1)\dots$\hspace*{-10mm}}

      \begin{Definition}<2->
        Sei $\varphi$ eine LTL-Formel, $\pi$ ein Pfad und $i \in \mathbb{N}$.\\
        Das \Bmph{Erfülltsein} von $\varphi$ in $\pi,i$ ~\Bmph{($\pi,i \models \varphi$)}~ ist wie folgt definiert.
        \vspace*{-.1pt}
        \begin{Itemize}
          \item<3->
            $\pi,i \models p$,\quad falls $p \in \pi_i$\,, für alle $p \in \PROP$
            \vspace*{-.08pt}
          \item<4->
            $\pi,i \models \neg \psi$,\quad falls $\pi,i \not\models \psi$
            \vspace*{-.08pt}
          \item<5->
            $\pi,i \models \varphi \land \psi$,\quad falls $\pi,i \models \varphi$ und $\pi,i \models \psi$
            \vspace*{-.08pt}
          \item<6->
            $\pi,i \models F\varphi$,\quad falls $\pi,j \models \varphi$ für ein $j \geqslant i$
            \vspace*{-.08pt}
          \item<7->
            $\pi,i \models G\varphi$,\quad falls $\pi,j \models \varphi$ für alle $j \geqslant i$
            \vspace*{-.08pt}
          \item<8->
            $\pi,i \models X\varphi$,\quad falls $\pi,i\!+\!1 \models \varphi$
            \vspace*{-.08pt}
          \item<9->
            $\pi,i \models \varphi\mathbin{U}\psi$,\quad falls $\pi,j \models \psi$ für ein $j \geqslant i$\\
            \hspace*{28mm}und $\pi,k \models \varphi$ für alle $k$ mit $i \leqslant k < j$
        \end{Itemize}
      \end{Definition}

      \uncover<10->{%
        \Tafel
      }

      \note{
        \textcolor{black!70}{\textbf{17:13 bis Ende 17:40}}
        
        \par
        \textbf{8:30 -- Kurzwdhlg.:}~ Syntax+Semantik LTL; Bsp.\ $a\mathbin{U}b$ an Tafel
        
        \parII
        \textbf{Hinweisen:}~ "`nicht strikte Semantik"', also "`$\geqslant$"' statt "`$>$"';\\
        "`für ein"' ($=$ "`es gibt"') vs.\ "`für alle"'
        
        \par
      }
    \end{frame}

%   % ------------------------------------------------------------------------------------------
%     \begin{frame}
%       \frametitle{Beispiele}
% 
%       \begin{block}{}
%         \begin{Itemize}
%           \item
%             $s,i \models p$,\quad falls $p \in s_i$, für alle $p \in \PROP$
%           \item
%             $s,i \models \neg \psi$,\quad falls $s,i \not\models \psi$
%           \item
%             $s,i \models \varphi \land \psi$,\quad falls $s,i \models \varphi$ und $s,i \models \psi$
%           \item
%             $s,i \models F\varphi$,\quad falls $s,j \models \varphi$ für ein $j \geqslant i$
%           \item
%             $s,i \models G\varphi$,\quad falls $s,j \models \varphi$ für alle $j \geqslant i$
%           \item
%             $s,i \models X\varphi$,\quad falls $s,i\!+\!1 \models \varphi$
%           \item
%             $s,i \models \varphi\mathbin{U}\psi$,\quad falls $s,j \models \psi$ für ein $j \geqslant i$\\
%             \hspace*{28mm}und $s,k \models \varphi$ für alle $k$ mit $i \leqslant k < j$
%         \end{Itemize}
%       \end{block}
%       
%       \par\bigskip
%       Siehe Tafel. \Tafel
%       \note{~}
%     \end{frame}

  % ------------------------------------------------------------------------------------------
    \begin{frame}
      \frametitle{Beispiel-Spezifikationen als LTL-Formeln}
      
      \Gmph{Beispiel Mikrowelle} {\footnotesize (siehe Bild auf Folie \ref{fra:bsp_mikrowelle})}
      \begin{Itemize}
        \item%[(a)]
          "`Wenn ein Fehler auftritt, ist er nach endlicher Zeit behoben."'
          \par\smallskip
          $G(e \to F \neg e)$ \hfill {\footnotesize ($e \in \PROP$ steht für "`Error"')}
          \par\smallskip
        \item<2->%[(b)]
          "`Wenn die Mikrowelle gestartet wird, \\
          fängt sie nach endlicher Zeit an zu heizen."'
          \par\smallskip
          $G(s \to F h)$ \hfill {\footnotesize ($s,h \in \PROP$ stehen für "`Start"' bzw.\ "`Heat"')}
          \par\smallskip
        \item<3->
          "`Irgendwann ist für genau einen Zeitpunkt die Tür geöffnet."'
          \par\smallskip
          $F(c \land X(\neg c \land Xc))$ \hfill {\footnotesize ($c \in \PROP$ steht für "`Close"')}
          \par\smallskip
        \item<4->
          "`Irgendwann ist für genau einen Zeitpunkt die Tür geöffnet,\\
          und bis dahin ist sie geschlossen."'
          \par\smallskip
          $c\mathbin{U}(\neg c \land Xc)$
      \end{Itemize}

%       \par\smallskip
%       \uncover<3->{%
%         \Bmph{Beispiel Nebenläufigkeit}
%         \begin{Itemize}
%           \item[(c)]
%             Es kommt nie vor,\\
%             dass beide Teilprog.\ zugleich im kritischen Bereich sind.
%             \par\smallskip
%             $G(\neg (p_{12} \land p_{22})$ \hfill {\footnotesize ($p_i \in \PROP$ stehen für "`Programmzähler in Zeile $i$"')}
%           \item<4->[(d)]
%             Jedes Teilprog.\ kommt beliebig oft in seinen krit.\ Bereich.
%             \par\smallskip
%             $GF p_{12} \land GF p_{22}$
% %           \item<6->[(f)]
% %             Jedes Teilprogramm \emph{kann} beliebig oft in seinen kritischen Bereich gelangen. \Tafel
%         \end{Itemize}
%       }
      \note{
        \textbf{8:32}
        
        \parII
        Wir brauchen das Bild nicht zu sehen; hier geht es nur um die Eigenschaften
        und die entsprechenden LTL-Formeln.
        
        \parI
        Bedeutung der AV steht daneben.
        
        \par
      }
    \end{frame}

  % ------------------------------------------------------------------------------------------
    \begin{frame}
      \frametitle{Beispiel-Spezifikationen als LTL-Formeln}

        \Gmph{Beispiel Nebenläufigkeit} {\footnotesize (siehe Bild auf Folie \ref{fra:nebenlaeufigkeit})}
        \begin{Itemize}
          \item%[(c)]
            Es kommt nie vor,\\
            dass beide Teilprog.\ zugleich im kritischen Bereich sind.
            \par\smallskip
            $G\neg (p_{12} \land p_{22})$ \hfill {\footnotesize ($p_i \in \PROP$ stehen für "`Programmzähler in Zeile $i$"')}
            \par\smallskip
          \item<2->%[(d)]
            Jedes Teilprog.\ kommt beliebig oft in seinen krit.\ Bereich.
            \par\smallskip
            $GF p_{12} \land GF p_{22}$
        \end{Itemize}

      \note{
        \textbf{8:35}
        
        \par
      }
    \end{frame}

  % ------------------------------------------------------------------------------------------
    \begin{frame}
      \frametitle{Model-Checking mit LTL-Formeln}

      \Bmph{Zur Erinnerung:}
      \begin{block}{Definition \ref{def:model-checking-problem}: Model-Checking-Problem \Bmph{MCP}}
        Gegeben ein System $\calS$ und eine Spezifikation $E$,
        \begin{Itemize}
          \item
            gilt $E$ für jeden Pfad in $\calS$\,?\\
            \Bmph{(universelle Variante)}
          \item
            gibt es einen Pfad in $\calS$, der $E$ erfüllt?\\
            \Bmph{(existenzielle Variante)}
        \end{Itemize}
      \end{block}

      \note{
        \textbf{8:37}
        
        \par
      }
    \end{frame}

  % ------------------------------------------------------------------------------------------
    \begin{frame}
      \frametitle{Model-Checking mit LTL-Formeln}

      \Bmph{Für LTL:}
      \par\smallskip
      (jedem Pfad $s_0s_1s_2\dots$ in einer Kripke-Struktur $\calS=(S,S_0,R,\ell)$
      entspricht ein LTL-Pfad $\pi_0\pi_1\pi_2\dots$ mit $\pi_i=\ell(s_i)$)
      
      \par\bigskip
      \begin{Definition}<2->[Model-Checking-Problem]
        Gegeben Kripke-Struktur $\calS=(S,S_0,R,\ell)$ und LTL-Formel $\varphi$,
        \begin{Itemize}
          \item
            gilt $\pi,0 \models \varphi$ für alle Pfade $\pi$, die in einem $s_0 \in S_0$ starten?\\
            \Bmph{(universelle Variante)}
          \item<3->
            gibt es Pfad $\pi$, der in einem $\pi_0 \in S_0$ startet, mit $\pi,0 \models \varphi$\,?\\
            \Bmph{(existenzielle Variante)}
        \end{Itemize}
      \end{Definition}
      
      \par\bigskip
      \uncover<4->{%
        \begin{Itemize}
          \item[\YES]
            Exakte Beschreibung des Model-Checking-Problems
          \emphitem
            Algorithmische Lösung?
        \end{Itemize}
      }
      
      \note{
        \textbf{8:38}
        
        \parII
        "`Algorithmische Lösung?"' -- Büchi-Automaten zu Hilfe! :)
        
        \par
      }
    \end{frame}

  % ------------------------------------------------------------------------------------------
    \begin{frame}
      \frametitle{MCP weiterhin mittels Büchi-Automaten lösen!}

      \Bmph{Vorgehen wie gehabt:}
      \par\smallskip
      \begin{Itemize}
        \item
          Wandle Kripke-Struktur $\calS$ in NBA $\Aut{A}_{\calS}$ um
          \par\smallskip
          $\leadsto$ Pfade in $\calS$ sind erfolgreiche Runs von $\Aut{A}_{\calS}$
          \par\bigskip
        \item<2->
          Wandeln LTL-Formel $\varphi_E$ in NBA $\Aut{A}_E$ um
          \par\smallskip
          $\leadsto$ $\Aut{A}_E$ beschreibt Pfade, die $E$ erfüllen
          \par\bigskip
        \item<3->[$\leadsto$]
          Universelles MCP ~$=$~ "`$L(\Aut{A}_{\calS}) \subseteq L(\Aut{A}_E)$\,?"'
          \par\smallskip
          Existenzielles MCP ~$=$~ "`$L(\Aut{A}_{\calS}) \cap L(\Aut{A}_E) \neq \emptyset$\,?"'
      \end{Itemize}
      
      \par\bigskip
      \uncover<4->{%
        \Emph{Noch zu klären:}~ Wie wandeln wir $\varphi_E$ in $\Aut{A}_E$ um?
      }

      \note{
        \textbf{8:41}
        
        \par
      }
    \end{frame}

  % ------------------------------------------------------------------------------------------
    \begin{frame}
      \frametitle{Umwandlung von LTL-Formeln in Automaten (Überblick)}
      
      Wandeln $\varphi_E$ in \Bmph{generalisierten Büchi-Automaten (GNBA)} um:
       
      \parI
      \begin{Itemize}
        \item
          $\Aut{A}_{\varphi_E} = (Q,\Sigma,\Delta,I,\dred{\calF})$ mit $\calF \subseteq 2^Q$
          \parI
        \item
          $r = q_0q_1q_2\ldots$ ist erfolgreich:~ $\Inf(r) \cap F \neq \emptyset$ \Emph{für alle} $F \in \calF$
          \parI
        \item
          GNBAs und NBAs sind äquivalent\hfill {\footnotesize (nur quadratische Vergrößerung)}
      \end{Itemize}

%      \parI
%      Im Folgenden grobe Vorgehensweise

      \note{
        \textbf{8:43}
        
        \parII
        "`GNBAs und NBAs sind äquivalent"':~
        
        \parI
        Richtung NBA $\to$ GNBA trivial.
        
        \parI
        Richtung GNBA $\to$ NBA:~\\
        Wenn $\Fmc = \{F_1,\dots,F_n\}$, erzeuge $n$ Kopien des GNBA. \\
        Von jedem akzZ.\ der $i$-ten Kopie wechsle in $((i + 1) \bmod n)$-te Kopie. \\
        Neue akzZ:~ die bisherigen akzZ.\ \emph{einer} beliebigen Kopie.
        
        \par
      }
    \end{frame}

  % ------------------------------------------------------------------------------------------
    \begin{frame}
      \frametitle{Vorbetrachtungen}

      Sei $\varphi_E$ eine LTL-Formel, in der o.\,B.\,d.\,A.\
      \begin{Itemize}
        \item
          nur die Operatoren $\neg,\land,X,U$ vorkommen
          \par\smallskip
          \begin{small}
            \qquad Die anderen kann man mit diesen ausdrücken:\\
            \qquad $F\varphi \equiv (\lnot(p \land \lnot p)) \mathbin U \varphi$\qquad
                    $G\varphi \equiv \lnot F \lnot \varphi$
            \par
          \end{small}
          \par\smallskip
        \item
          keine doppelte Negation vorkommt
          \par\smallskip
          \begin{small}
            \qquad natürlich gilt $\lnot\lnot\psi \equiv \psi$ für alle Teilformeln $\psi$
            \par
          \end{small}
      \end{Itemize}
      \begin{small}
        (Hier steht ~$\alpha \equiv \beta$~ für ~$\forall \pi \forall i : \pi,i \models \alpha \text{ gdw.\ } \pi,i \models \beta$)
        \par
      \end{small}


      \par\bigskip
      \uncover<2->{%
        \Bmph{Etwas Notation}
        \begin{Itemize}
          \item
            $\Neg\psi = \begin{cases}
                          \vartheta & \text{falls } \psi = \lnot\vartheta \\
                          \lnot\psi & \text{sonst}
                        \end{cases}$
            \par\smallskip
          \item
            $\cl(\varphi_E) = \{\psi, \Neg\psi \mid \psi \text{ ist Teilformel von } \varphi_E\}$
            \par\smallskip
          \item
            $\Sigma=2^\PROP$
        \end{Itemize}
      }

      \note{
        \textbf{8:45}
        
        \parI
        Die Einschränkung der vorkommenden Operatoren \\
        • ist o.\,B.\,d.\,A., wie wir an den Äquivalenzen sehen; \\
        • macht die folgenden Definitionen deutlich übersichtlicher; \\
        • führt aber dazu, dass Automaten schon für kleine F-/G-Formeln riesig werden. \\
        Deshalb nur kurze Beispiele hier und in ÜS.
        
        \par
      }
    \end{frame}

  % ------------------------------------------------------------------------------------------
    \begin{frame}
      \frametitle{Intuitionen}


%       \par\bigskip
%       \uncover<2->{%
        \Bmph{Erweiterung von Pfaden}
        
        \begin{Itemize}
          \item
            Betrachten Pfade $\pi=s_0s_1s_2\ldots$ mit $s_i \subseteq \PROP$
          \item
            Erweitern jedes $s_i$ mit den $\psi \in \cl(\varphi_E)$, für die $\pi,i \models \psi$ gilt
          \item
            Resultat: Folge $\overline \pi = t_0t_1t_2\ldots$ mit $t_i \subseteq \cl(\varphi_E)$
        \end{Itemize}
%       }

      \par\smallskip
      \begin{small}
        \hspace*{.6\textwidth}
        \begin{tabular}{l}
          $\searrow$\\
          \uncover<2->{%
            $\nearrow$%
          }
        \end{tabular}
        Skizze: s.\ Tafel \Tafel
        \par
      \end{small}

      \par\vspace*{-\baselineskip}
      \uncover<2->{%
        \Bmph{Bestandteile des GNBA $\Aut{A}_{\varphi_E}$}

        \begin{Itemize}
          \item
            Zustände: $\approx$ alle $t_i$
          \item
            $\overline \pi = t_0t_1t_2\ldots$ wird ein Run von $\Aut{A}_{\varphi_E}$ für $s_0s_1s_2\ldots$ sein
          \item
            Run $\overline \pi$ wird erfolgreich sein gdw.\ $\pi,0\models \varphi_E$
          \item
            Kodieren Bedeutung der logischen Operatoren in
            \begin{Itemize}
              \item
                Zustände ($\lnot$, $\land$, teilweise $U$)
              \item
                Überführungsrelation ($X$, teilweise $U$)
              \item
                Akzeptanzbedingung (teilweise $U$)
            \end{Itemize}
        \end{Itemize}
      }

      \note{
        \textbf{8:48 bis 8:56}
        
%        \parI
%        % TODO
%        \textbf{TODO:}~ Lieber zusätzliche Bedingungen für $F,G$ bei Zuständen, Transitionen, Akzeptanzkomponente?
%        (siehe T5.4 in Kap. 5) \\
%        Dann sollte man später als Beispiel bequem $GFa$ machen können.
%        --- Nein, dann werden die Def.s unübersichtlich!
        \par
      }
    \end{frame}

  % ------------------------------------------------------------------------------------------
    \begin{frame}
      \frametitle{Zustandsmenge des GNBA $\Aut{A}_{\varphi_E}$}

      $Q$ $=$ Menge aller elementaren Formelmengen, \\
      wobei $t \subseteq \cl(\varphi_E)$ \Bmph{elementar} ist, wenn gilt:
      \par\medskip
      \begin{Enumerate}
        \item
          $t$ ist \Bmph{konsistent} bzgl.\ Aussagenlogik, d.\,h.\\
          für alle $\psi_1 \land \psi_2 \in \cl(\varphi_E)$ und $\psi \in \cl(\varphi_E)$:
          \begin{Itemize}
            \item
              $\psi_1 \land \psi_2 \in t$ gdw.\
              $\psi_1 \in t$ und $\psi_2 \in t$
            \item
              wenn $\psi \in t$, dann $\Neg\psi \notin t$
          \end{Itemize}
          \par\medskip
        \item<2->
          $t$ ist \Bmph{lokal konsistent} bzgl.\ des $U$-Operators, d.\,h.\\
          für alle $\psi_1 \mathbin{U} \psi_2 \in \cl(\varphi_E)$:
          \begin{Itemize}
            \item
              wenn $\psi_2 \in t$, dann $\psi_1 \mathbin{U} \psi_2 \in t$
            \item
              wenn $\psi_1 \mathbin{U} \psi_2 \in t$ und $\psi_2 \notin t$,
              dann $\psi_1 \in t$
          \end{Itemize}
          \par\medskip
        \item<3->
          $t$ ist \Bmph{maximal}, d.\,h.\ für alle $\psi \in \cl(\varphi_E)$:
          \begin{Itemize}
            \item[]
              wenn $\psi \notin t$, dann $\Neg\psi \in t$
          \end{Itemize}
      \end{Enumerate}

      \par\bigskip
      \uncover<4->{%
        \Gmph{Beispiel:} $a \mathbin{U} (\lnot a \land b)$, siehe Tafel \Tafel
      }

      \note{
        \textbf{8:56 bis 9:10, 5min Pause}
        
        \par
      }
    \end{frame}

  % ------------------------------------------------------------------------------------------
    \begin{frame}
      \frametitle{Überführungsrelation des GNBA $\Aut{A}_{\varphi_E}$}

%         \item
%           Zur Erinnerung: $\Sigma = 2^\PROP$
%       Betrachten Tripel $(t,s,t')$\\
% %           mit $t,t' \in Q$ (elem. FM) und $s \in \Sigma$ ($\Sigma = 2^\PROP$)
% %           mit
% %           \begin{tabular}[t]{@{}c@{}}
% %             $t,t' \in Q$ \\
% %             {\small (elementare Formelmengen)}
% %           \end{tabular}
% %           und
% %           \begin{tabular}[t]{@{}c@{}}
% %             $s \in \Sigma$ \\
% %             {\small ($\Sigma = 2^\PROP$)}
% %           \end{tabular}
%       \qquad
%       \begin{tabular}[t]{@{}l@{~}ll@{}}
%         mit & $t,t' \in Q$   & {\small (elementare Formelmengen)}\\
%         und & $s \in \Sigma$ & {\small ($\Sigma = 2^\PROP$)}
%       \end{tabular}
      Seien $t,t' \in Q$ (elementare Formelmengen) und $s \in \Sigma$ ($\Sigma = 2^\PROP$)

      \par\bigskip\bigskip
      $\Delta$ besteht aus allen Tripeln $(t,s,t')$ mit
      \begin{Enumerate}
        \item
%               $s = \text{alle Aussagenvariablen und negierte AV in $t$}$
          $s = t \cap \PROP$
          \hfill
          {\small (d.\,h.\ $s$ besteht aus allen Aussagevariablen in $t$)}
          \par\smallskip
        \item<2->
          für alle $X\psi \in \cl(\varphi_E)$:\quad
          $X\psi \in t$ gdw.\ $\psi \in t'$
          \par\smallskip
        \item<3->
          für alle $\psi_1 \mathbin{U} \psi_2 \in \cl(\varphi_E)$:\\
          $\psi_1 \mathbin{U} \psi_2 \in t$ ~gdw.~
          $\psi_2 \in t$ oder ($\psi_1 \in t$ und $\psi_1 \mathbin{U} \psi_2 \in t'$)
          \par\smallskip
%           \uncover<5->{%
            \hspace*{\fill}{\small \emph{("`Aufschieben"' von $\psi_1 \mathbin{U} \psi_2$)}} \Danger%
%           }
      \end{Enumerate}

      \par\bigskip
      \uncover<4->{%
        Skizzen: siehe Tafel \Tafel
      }
%         
%           Wenn $s \neq t \cap \PROP$, dann gibt es kein solches Tripel in $\Delta$
%           \par\smallskip
%         \item<3->
%           Sonst $(t,s,t') \in \Delta$ für alle elementaren FM $t'$ mit:
%           \begin{Itemize}
%             \item
%               für alle $X\psi \in \cl(\varphi_E)$:\quad
%               $X\psi \in t$ gdw.\ $\psi \in t'$
%             \item<4->
%               für alle $\psi_1 \mathbin{U} \psi_2 \in \cl(\varphi_E)$:\\
%               $\psi_1 \mathbin{U} \psi_2 \in t$ ~gdw.~
%               $\psi_2 \in t$ oder ($\psi_1 \in t$ und $\psi_1 \mathbin{U} \psi_2 \in t'$)
%           \end{Itemize}

      \note{
        \textbf{9:15 bis 9:26}
        
        \par
      }
    \end{frame}

  % ------------------------------------------------------------------------------------------
    \begin{frame}
      \frametitle{Anfangszustände und Akzeptanzkomponente von $\Aut{A}_{\varphi_E}$}

      \Bmph{Menge der Anfangszustände}
      \par\smallskip
      alle elementaren Formelmengen, die $\varphi_E$ enthalten

      \par\medskip
      \qquad $I = \{t \in Q \mid \varphi_E \in t\}$

      \par\bigskip\smallskip
      \uncover<2->{%
        \Bmph{Menge der akzeptierenden Zustände}
        \par\smallskip
        stellen sicher, dass kein $\psi_1 \mathbin{U} \psi_2$ für immer ``aufgeschoben'' wird

        \par\medskip
        \qquad $\calF = \{M_{\psi_1 \mathbin{U} \psi_2} \mid \psi_1 \mathbin{U} \psi_2 \in \cl(\varphi_E)\}$ mit
        \[
          M_{\psi_1 \mathbin{U} \psi_2} = \{t \in Q \mid \psi_1 \mathbin{U} \psi_2 \notin t \text{ oder } \psi_2 \in t\}
        \]
      }

%       \par\bigskip
      \par\vspace*{-\baselineskip}
      \uncover<3->{%
        \begin{small}%
%               \hspace*{\fill}\emph{(``jedes $\psi_1 \mathbin{U} \psi_2$ wird nur endlich lange aufgeschoben'')}\qquad~%
          \qquad Intuition:~
%           \par\smallskip
          Ein $t \in M_{\psi_1 \mathbin{U} \psi_2}$ kommt unendlich oft vor
%           \par
          gdw.\
          \par%\smallskip
          \qquad $\psi_1 \mathbin{U} \psi_2$ immer nur höchstens endlich lange ``aufgeschoben'' wird
          \par
        \end{small}
      }

      \par\bigskip
      \uncover<4->{%
        \Gmph{Beispiel:} $Xa$, siehe Tafel \Tafel
      }

      \par\medskip
      \uncover<4->{%
        \Gmph{Beispiel:} $(\lnot a) \mathbin{U} b$, siehe Tafel \Tafel
      }

      \note{
        \textbf{9:26 bis 9:58}
        
%        \parI
%        % TODO
%        \textbf{TODO:}~ Beispielformel $Xa$ ist zu einfach und demonstriert nicht die Akzeptanzbedingung. \\
%        Stattdessen Beispielformel $a \mathbin{U} (Xb)$ machen: \\
%        Dafür ist genug Zeit, denn T3.23 und T3.24 gehen deutlich schneller als geplant.
%        Am 21.12.17 war ich vor T3.23 ca. 5min im Rückstand (und hatte Pause weggelassen),
%        aber am Ende von T3.25 war \textbf{auch} noch Zeit für Bsp. $a \mathbin{U} b$.
%        --- erledigt, beide Bsp.e werden vorgeführt; Kombi X+U ist in Übungsserie 5.
%    
%        \parI
%        Außerdem kann man z.\,B.\ auch $GFa$ machen, wenn man $F,G$ nicht als Abkürzungen behandelt,
%        sondern zusätzliche Bedingungen für Zustände, Transitionen, Akzeptanzkomponente einführt
%        (siehe T5.4 in Kap. 5)
%        --- nö, das ist zu viel technischer Kram, (s.a. Kommentar weiter oben)

        \par
      }
    \end{frame}

  % ------------------------------------------------------------------------------------------
    \begin{frame}
      \frametitle{Abschließende Betrachtungen}

      \begin{Itemize}
        \item
          $|Q|$ ist exponentiell in $|\varphi_E|$
          \par\smallskip
        \item
          Dafür kann man jetzt beim universellen MCP auf Komplementierung $\Aut{A}_{\varphi_E}$ verzichten:\\
          Wandle $\neg\varphi_E$ in Automaten um
          \par\smallskip
        \item[$\leadsto$]
          beide MCP-Varianten in \PSPACE
          \par\smallskip
        \item
          beide MCP-Varianten sind \PSPACE-vollständig\\
          (aber für bestimmte LTL-Fragmente \NP- oder \NL-vollständig)
          \par\medskip
          \begin{footnotesize}
            A. Prasad Sistla, Edmund M. Clarke:
            \emph{The Complexity of Propositional Linear Temporal Logics}.
            Journal of the ACM 32(3): 733-749 (1985)
            \par\medskip
            Michael Bauland, Martin Mundhenk, Thomas Schneider, Henning Schnoor, Ilka Schnoor, Heribert Vollmer:
            \emph{The Tractability of Model Checking for LTL: the Good, the Bad, and the Ugly Fragments.}
            ACM Trans. Comput. Log. 12(2): 13 (2011)
            \par
          \end{footnotesize}
      \end{Itemize}

      \note{
        \textbf{9:58 bis Ende -- umblättern und evtl.\ Prüfungshinweise}
        
        \par
      }
    \end{frame}

%   % ------------------------------------------------------------------------------------------
%     \begin{frame}
%       \frametitle{\dots}
%       \dots
%       \note{~}
%     \end{frame}
% 
%   % ------------------------------------------------------------------------------------------
%     \begin{frame}
%       \frametitle{\dots}
%       \dots
%       \note{~}
%     \end{frame}
% 


%   % ------------------------------------------------------------------------------------------
%     \begin{frame}
%       \frametitle{\dots}
%       \dots
%       \note{~}
%     \end{frame}
% 

    



    

  \AtBeginSection{\relax}
  % ==============================================================================================
  % ==============================================================================================
  \section*{}

  % ------------------------------------------------------------------------------------------
    \begin{frame}
      \frametitle{Damit sind wir am Ende dieses Kapitels.}
%       \par\bigskip
      \uncover<1->{%
        \begin{center}
          \includegraphics[height=.5\textheight]{img/xkcd_1195_flowchart.png}
          \par
          \begin{footnotesize}
            {\footnotesize \url{http://xkcd.com/1195} (CC BY-NC 2.5)}
            \par
          \end{footnotesize}
          \par\bigskip\bigskip
          \begin{Huge}
            \Bmph{Vielen Dank.}
          \end{Huge}
        \end{center}
      }
      \note{~}
    \end{frame}

  % ------------------------------------------------------------------------------------------
    \begin{frame}
      \frametitle{Literatur für diesen Teil (1)}
      \begin{small}
        \begin{thebibliography}{99}
          \bibitem{Tho90}
            Wolfgang Thomas.
            \newblock
            Automata on Infinite Objects. 
            \newblock
            In J.\ van Leeuwen (Hrsg.):\\
            Handbook of Theoretical Computer Science.\\
            Volume B: Formal Models and Sematics.
            \newblock
            Elsevier, 1990, S. 133--192.
            \newblock
            SUB, Zentrale:
            \texttt{a inf 400 ad/465-2}
          \bibitem{Tho97}
            Wolfgang Thomas.
            \newblock
            Languages, automata, and logic.
            \newblock
            In G. Rozenberg and A. Salomaa (Hrsg.:)\\
            Handbook of Formal Languages. Volume 3: Beyond Words.
            \newblock
            Springer, 1997, S. 389--455.
            \newblock
            SUB, Zentrale:
            \texttt{a inf 330/168-3}
        \end{thebibliography}
        \par
      \end{small}
      \note{~}
    \end{frame}

  % ------------------------------------------------------------------------------------------
    \begin{frame}
      \frametitle{Literatur für diesen Teil (2)}
      \begin{small}
        \begin{thebibliography}{99}
          \bibitem{Rog02}
            Markus Roggenbach.
            \newblock
            Determinization of Büchi Automata.
            \newblock
            In E.\ Grädel, W.\ Thomas, T.\ Wilke (Hrsg.):\\
            Automata, Logics, and Infinite Games.
            \newblock
            LNCS 2500, Springer, 2002, S. 43--60.
            \newblock
            Erklärt anschaulich Safras Konstruktion.\\
            \url{http://www.cs.tau.ac.il/~rabinoa/Lncs2500.zip}
            \newblock
            Auch erhältlich auf Anfrage in der BB Mathematik im MZH:\\
            \texttt{19h inf 001 k/100-2500}
          \bibitem{Bie09}
            Meghyn Bienvenu.
            \newblock
            Automata on Infinite Words and Trees.
            \newblock
            Vorlesungsskript, Uni Bremen, WS 2009/10.\hfill Kapitel 2.
            \newblock
            \url{http://www.informatik.uni-bremen.de/tdki/lehre/ws09/automata/automata-notes.pdf}
        \end{thebibliography}
        \par
      \end{small}
      \note{~}
    \end{frame}

  % ------------------------------------------------------------------------------------------
    \begin{frame}
      \frametitle{Literatur für diesen Teil (3)}
      \begin{small}
        \begin{thebibliography}{99}
          \bibitem{BK08}
            Christel Baier, Joost-Pieter Katoen.
            \newblock
            Principles of Model Checking.
            \newblock
            MIT Press 2008.
            \newblock
            Abschnitt 4.3 "`Automata on Infinite Words"'\\
            Abschnitt 5.2 "`Automata-Based LTL Model Checking"'
            \newblock
            SUB, Zentrale:
            \texttt{a inf 440 ver/782},~ \texttt{a inf 440 ver/782a}
          \bibitem{CGP99}
            Edmund M.\ Clarke, Orna Grumberg, Doron A.\ Peled.
            \newblock
            Model Checking.
            \newblock
            MIT Press 1999.
            \newblock
            Abschnitt 2 "`Modeling Systems"' bis Mitte S.\ 14,\\
            Abschnitt 2.2.3\,$+$2.3 "`Concurrent Programs"' und "`Example \dots"',\\
            Abschnitt 3 "`Temporal Logics"',\\
            Abschnitt 9.1 "`Automata on Finite and Infinite Words"'.%,\\
%             Abschnitt 9.4 "`Translating LTL into Automata"'.
            \newblock
            SUB, Zentrale:
            \texttt{a\hfill inf\hfill 440\hfill ver/780(6)},\hfill\hfill  \texttt{a\hfill inf\hfill 440\hfill ver/780(6)a}
        \end{thebibliography}
        \par
      \end{small}
      \note{~}
    \end{frame}

%   % ------------------------------------------------------------------------------------------
%     \begin{frame}
%       \frametitle{Literatur für diesen Teil (2)}
%       \begin{small}
%         \begin{thebibliography}{99}
%           \bibitem{BW98}
%             Anne Br{\"u}ggemann-Klein, Derick Wood.
%             \newblock
%             One-Unambiguous Regular Languages.
%             \newblock
%             Information and Computation, 142:1998, S.\ 182--206.
%             \newblock
%             \url{http://dx.doi.org/10.1006/inco.1997.2695}
%             \newblock
%             Grundlegende Resultate für deterministische reguläre Ausdrücke.          
%         \end{thebibliography}
%         \par
%       \end{small}
%       \note{~}
%     \end{frame}
% 
% 
%   \AtBeginSection{\frame{\frametitle{Und nun \dots}\tableofcontents[currentsection]\note{~}}}
  

%     % ------------------------------------------------------------------------------------------
%     \begin{frame}
%       \frametitle{Ausblick}
% 
%       \dots
%       
%       \par\bigskip
%       \uncover<2>{%
%         \begin{center}
%           \begin{Huge}
%             \dblu{\textbf{Thank you.}}
%           \end{Huge}
%         \end{center}
%       }
%       \note{~}
%     \end{frame}

  % ==============================================================================================
  % ==============================================================================================
  \appendix
    
  % ------------------------------------------------------------------------------------------
    \begin{frame}
      \label{fra:anhang_bsp2}
      \frametitle{Anhang: Beispiel Konsument-Produzent-Problem}

      \begin{exampleblock}{}
        \begin{Itemize}
          \item
            $P$ erzeugt Produkte und legt sie einzeln in einem Lager ab
          \item
            $K$ entnimmt Produkte einzeln dem Lager
          \item
            Lager fasst maximal 3 Stück
        \end{Itemize}
      \end{exampleblock}

      \par\bigskip
      \uncover<2->{%
        \Bmph{Modellierung durch endliches Transitionssystem}
        \begin{Itemize}
          \item
            Zustände $0,1,2,3,\text{Ü},\text{U}$
            \begin{Itemize}
              \item
                0,1,2,3: im Lager liegen 0,1,2,3 Stück
              \item
                \emph{Ü}\/berschuss: $P$ will ein Stück im vollen Lager ablegen
              \item
                \emph{U}\/nterversorgung: $K$ will ein Stück aus leerem Lager nehmen
            \end{Itemize}
          \item
            Aktionen $P,K$ ($P$ legt ab oder $K$ entnimmt)
        \end{Itemize}
      }

      \note{~}
    \end{frame}

  % ------------------------------------------------------------------------------------------
    \begin{frame}
      \frametitle{Das Transitionssystem}

      \begin{center}
        \Fig{20}
      \end{center}

      \uncover<2->{%
        \begin{tabular}[t]{@{}l@{~~}l@{}}
          \Bmph{Eingaben in das System:} & unendliche Zeichenketten über $\Sigma = \{p,k\}$ \\
                                         & \Bmph{(Läufe)}
        \end{tabular}
      }

      \par\bigskip
      \uncover<3->{%
        \Bmph{Zufriedenheit:} $P$ ($K$) möchte \dots
        \begin{Itemize}
          \item
            beliebig oft Produkte produzieren (konsumieren)
          \item
            nur endlich oft \emph{Ü}\/berschuss (\emph{U}\/nterversorgung) erleiden
        \end{Itemize}
      }

      \par\bigskip
      \uncover<4->{%
        \Bmph{Lauf, der $P$ und $K$ zufrieden stellt:}\\%
      }
      \uncover<5->{%
        $p^3k^3p^3k^3\dots$\quad oder\quad $ppkpkpk\dots$\quad oder\quad $\dots$
      }
      
      \par\bigskip
      \uncover<6->{%
        \Bmph{Lauf, der weder $P$ noch $K$ zufrieden stellt:}%
      }
      \uncover<7->{%
        $p^4k^4p^4k^4\dots$
      }
      \note{~}
    \end{frame}

%     % ------------------------------------------------------------------------------------------
%     \begin{frame}
%       \frametitle{\dots}
%       \dots
%       \note{~}
%     \end{frame}



\mode<presentation>{
  {   
    \setbeamercolor{background canvas}{bg=black}
    \begin{frame}<handout:0>[plain]{}
      \note{~}
    \end{frame}
  }
}



\end{document}
