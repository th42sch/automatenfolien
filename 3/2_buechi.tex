
    % ------------------------------------------------------------------------------------------
    \begin{frame}
      \frametitle{Grundbegriffe}

      \Bmph{Unendliches Wort} über \Bmph{Alphabet} $\Sigma$
      \begin{Itemize}
        \item
          ist Funktion $\Bmph{$\alpha$} : \mathbb{N} \to \Sigma$
        \item
          \Bmph{$\alpha(n)$:}~ Symbol an $n$-ter Stelle (auch: \Bmph{$\alpha_n$})
        \item
          wird oft geschrieben als $\alpha = \alpha_0\alpha_1\alpha_2\dots$
      \end{Itemize}
%       Ab jetzt: unendliche Wörter $\alpha,\dots$; endliche Wörter $w,\dots$

      \par\bigskip
      \uncover<2->{%
        \Bmph{Weitere Notation}
        \begin{Itemize}
          \item
            \Bmph{$\alpha[m,n]$:}~ endliche Teilfolge $\alpha_m\alpha_{m+1}\dots \alpha_n$
          \item
            \begin{tabular}[t]{@{}l@{~}l@{}}
              \Bmph{$\#_w(\alpha)$:}~ & Anzahl der Vorkommen von $w$ als Teilwort in $\alpha$\\
                                      & $= \#\{(m,n) \mid \alpha[m,n] = w\}$
            \end{tabular}
          \item
            \begin{tabular}[t]{@{}l@{~}l@{}}
              \Bmph{$w^\omega$:}~ & unendliche Verkettung von $w$ \\ %\hfill(sei $n := |w|$)\\
                                  & ($\alpha$ mit $\alpha[i\!\!\:\cdot\!\!\:n,(i\!\!\:+\!\!\:1)n\!\!\;-\!\!\;1] = w \text{ f.\ alle } i \geqslant 0,~ n = |w|$)
            \end{tabular}
        \end{Itemize}

      }
      \par\bigskip
      \uncover<3->{%
        \Bmph{$\Sigma^\omega$:} Menge aller unendlichen Wörter
      }

      \par\bigskip
      \uncover<3->{%
        \Bmph{$\omega$-Sprache:} $L \subseteq \Sigma^\omega$
      }
      \note{
        \uz{9:02}
        
        \par
      }
    \end{frame}

  % ------------------------------------------------------------------------------------------
    \begin{frame}
      \frametitle{Büchi-Automaten}

      \begin{Definition}
        Ein \Bmph{nichtdeterministischer Büchi-Automat (NBA)}
        über einem \Bmph{Alphabet} $\Sigma$ ist ein 5-$\!$Tupel
        $\Aut{A} = (Q, \Sigma, \Delta, I, F)$, wobei
        \begin{Itemize}
          \item
            $Q$ eine endliche nichtleere \Bmph{Zustandsmenge} ist,
          \item
            $\Sigma$ eine endliche nichtleere Menge von Zeichen ist,
          \item
            $\Delta \subseteq Q \times \Sigma \times Q$ die \Bmph{Überführungsrelation} ist,
          \item
            $I \subseteq Q$ die Menge der \Bmph{Anfangszustände} ist,
          \item
            $F \subseteq Q$ die Menge der \Bmph{akzeptierenden Zustände} ist.
        \end{Itemize}
      \end{Definition}

      \par\bigskip
      \uncover<2->{%
        Bisher kein Unterschied zu NEAs, aber \dots
      }
      
      \note{
        \uz{9:04}
        
        \par
      }
    \end{frame}

    % ------------------------------------------------------------------------------------------
    \begin{frame}
      \frametitle{Berechnungen und Akzeptanz}

      \begin{Definition}
        Sei $\Aut{A} = (Q,\Sigma,\Delta,I,F)$ ein Büchi-Automat.
        \begin{Itemize}
          \item
            Ein \Bmph{Run} von \Aut{A} auf $\omega$-Wort $\alpha$
            ist eine Folge
            \vspace*{-.4\baselineskip}
            \[
              r = q_0q_1q_2\dots,
            \]
            \par\vspace*{-.4\baselineskip}
            so dass für alle $i\geqslant0$ gilt: $(q_i,\alpha_i,q_{i+1}) \in \Delta$.
            \par\smallskip
          \item<2->
            \Bmph{Unendlichkeitsmenge $\Inf(r)$} von $r = q_0q_1q_2\dots$:\\
            Menge der Zustände, die unendlich oft in $r$ vorkommen
            \par\smallskip
          \item<3->
            \Bmph{Erfolgreicher Run} $r = q_0q_1q_2\dots$:\hfill
            $q_0 \in I$ und $\Inf(r) \cap F \neq \emptyset$
            \par\smallskip
          \item<4->
            \Aut{A} \Bmph{akzeptiert} $\alpha$,\\
            wenn es einen erfolgreichen Run von \Aut{A} auf $\alpha$ gibt.
            \par\smallskip
          \item<5->
            Die von \Aut{A} \Bmph{erkannte Sprache} ist
            $\ddblu{L_\omega(\Aut{A})} = \{\alpha \in \Sigma^\omega \mid \text{\Aut{A} akzeptiert $\alpha$}\}$.
        \end{Itemize}
      \end{Definition}

      \note{
        \uz{9:06}
        
        \par
      }
    \end{frame}

    % ------------------------------------------------------------------------------------------
    \begin{frame}
      \frametitle{Beispiele}
%       \label{frame:automatenbsp}

      \begin{exampleblock}{}
        $\Aut{A}_1$:\quad \raisebox{-3mm}{\Fig{30}}
        \hfill
        \begin{minipage}{.55\textwidth}
          $L_\omega(\Aut{A}_1) = \uncover<2->{\{a^nb^\omega \mid n \geqslant 1\}}$
        \end{minipage}

        \par\bigskip
        \uncover<3->{$\Aut{A}_2$:\quad \raisebox{-3mm}{\Fig{31}}}
        \hfill
        \begin{minipage}{.55\textwidth}
          $\uncover<3->{L_\omega(\Aut{A}_2) = }\uncover<4->{\{\alpha\!\!\:\in\!\!\:\Sigma^\omega \mid \#_a(\alpha) < \infty\}}$
        \end{minipage}

        \par\bigskip
        \uncover<5->{$\Aut{A}_3$:\quad \raisebox{-3mm}{\Fig{32}}}
        \hfill
        \begin{minipage}{.55\textwidth}
          $\uncover<5->{L_\omega(\Aut{A}_3) = }\uncover<6->{\{\alpha \in \Sigma^\omega \mid \text{s.\ unten}\}}$
        \end{minipage}
        \par\bigskip
        \uncover<6->{%
%           Zwischen zwei aufeinanderfolgenden $a$'s in $\alpha$\\
%           -- und am Anfang von $\alpha$ -- steht eine gerade Anzahl von $b$'s.%
          \hspace*{\fill} Zwischen je zwei $a$'s in $\alpha$ sowie vor dem ersten $a$ \\
          \hspace*{\fill} steht jeweils eine gerade Anzahl von $b$'s.%
        }


%         \hfill
%         $L(\Aut{A}_1) = \uncover<2->{\{a^nb^m \mid n \geqslant 0, m \geqslant 1\}}$
% 
%         \par\bigskip
%         \uncover<3->{$\Aut{A}_2$: \raisebox{-3mm}{\Fig{10}}}
%         \par\bigskip
%         $\uncover<3->{L(\Aut{A}_2) = }\uncover<4->{\{\alpha \in \{a,b\}^* \mid \text{$\alpha$ enthält Teilwort $ab$}\}}$
% 
%         \par\bigskip
%         \uncover<5->{$\Aut{A}_3$: \raisebox{-3mm}{\Fig{11}}}
%         \par\bigskip
%         $\uncover<5->{L(\Aut{A}_3) = }\uncover<6->{\{\alpha \in \{a,b\}^* \mid \text{$\alpha$ endet auf $ab$}\}}$
      \end{exampleblock}

      \note{
        \uz{9:08 bis 9:14}
        
        \par
      }
    \end{frame}

%     % ------------------------------------------------------------------------------------------
%     \begin{frame}
%       \frametitle{Mehr Beispiele}
%       
%       \begin{exampleblock}{}
%         \begin{small}
%           $\Aut{A}_4$: \raisebox{-6mm}{\Fig{40}}
%           \hfill
%           \begin{minipage}{.557\textwidth}
%             \begin{align*}
%               L_\omega(\Aut{A}_4) & = \uncover<2->{\{\alpha\!\in\!\Sigma^\omega \mid \#_{ab}(\alpha) = \infty\}} \\
%                                   & \uncover<3->{= \{\alpha \mid \#_{a}(\alpha) = \#_{b}(\alpha) = \infty\}}
%             \end{align*}
%           \end{minipage}
% 
%           \par\bigskip
%           \uncover<4->{$\Aut{A}_5$: \raisebox{-6mm}{\Fig{41}}}
%           \hfill
%           \begin{minipage}{.54\textwidth}
%             $\uncover<4->{L_\omega(\Aut{A}_5) = }\uncover<5->{\{\alpha \mid \#_a(\alpha) = \infty\}}$
%           \end{minipage}
%           \par\smallskip
%           \uncover<6->{%
%             (Idee:~ $q_1$ kann nur erreicht werden, wenn ein $a$ gelesen wird)%
%           }
% 
%           \par\bigskip
%           \uncover<7->{$\Aut{A}_6$: \raisebox{-6mm}{\Fig{42}}}
%           \hfill
%           \begin{minipage}{.54\textwidth}
%             $\uncover<7->{L_\omega(\Aut{A}_6) = }\uncover<8->{\{\alpha \mid \#_{bb}(\alpha) = \infty\}}$
%           \end{minipage}
%           \par\smallskip
%           \uncover<9->{%
%             (Idee:~ $q_0$ nur durch $bb$ erreichbar;~ jeder Teilstring $bb$ führt zu $q_0$)%
%           }
%         \par
%         \end{small}
%       \end{exampleblock}
%       \note{~}
%     \end{frame}
% 
    % ------------------------------------------------------------------------------------------
    \begin{frame}
      \frametitle{Mehr Beispiele}

      \begin{exampleblock}{}
        \begin{small}
          $\Aut{A}_4$: \raisebox{-8mm}{\Fig{50}}
          \hfill
          \begin{minipage}[t]{.54\textwidth}
            \vspace*{-1.6\baselineskip}
            \begin{align*}
              L_\omega(\Aut{A}_4) & = \uncover<2->{\{\alpha\in\Sigma^\omega \mid \#_{a}(\alpha) < \infty} \\
                                  & \uncover<2->{\qquad\qquad \text{oder } \#_b(\alpha) < \infty\}}
            \end{align*}
          \end{minipage}

          \par\bigskip
          \uncover<3->{$\Aut{A}_5$: \raisebox{-34mm}{\Fig{51}}}
%           \hspace*{-20mm}%
          \hfill
          \begin{minipage}[t]{.54\textwidth}
            \vspace*{-1.6\baselineskip}
            \begin{align*}
              \uncover<3->{L_\omega(\Aut{A}_5) =}~ & \uncover<4->{\{\alpha\in\Sigma^\omega \mid \#_{a}(\alpha) < \infty} \\
                                                   & \uncover<4->{\qquad~~ \text{oder } \#_{aa}(\alpha) = 0\}}
            \end{align*}
            \uncover<5->{%
              \begin{footnotesize}%
                \hspace*{\fill}(Letzteres heißt:~~~~\\
                \hspace*{\fill}auf jedes $a$ in $\alpha$ folgt direkt ein $b$)~~~~%
                \par
              \end{footnotesize}%
            }

          \end{minipage}
%           \par\vspace*{-\baselineskip}
%           \uncover<5->{%
%             \hspace{\fill}{\footnotesize (letzteres heißt: auf jedes $a$ in $\alpha$ folgt direkt ein $b$)}%
%           }

%           \par\bigskip
%           \uncover<7->{$\Aut{A}_6$: \raisebox{-6mm}{\Fig{42}}}
%           \hfill
%           \begin{minipage}{.54\textwidth}
%             $\uncover<7->{L_\omega(\Aut{A}_6) = }\uncover<8->{\{\alpha \mid \#_{bb}(\alpha) = \infty\}}$
%           \end{minipage}
%           \par\smallskip
%           \uncover<9->{%
%             (Idee:~ $q_0$ nur durch $bb$ erreichbar;~ jeder Teilstring $bb$ führt zu $q_0$)%
%           }
        \par
        \end{small}
      \end{exampleblock}

      \note{
        \uz{9:14 bis 9:20; 5\,min Pause}
        
        \par
      }
    \end{frame}

    % ------------------------------------------------------------------------------------------
    \begin{frame}
      \frametitle{Erkennbare Sprache}
      
      \begin{Definition}
        Eine Sprache $L \subseteq \Sigma^\omega$ ist \Bmph{Büchi-erkennbar}, \\
        wenn es einen NBA \Aut{A} gibt mit $L = L_\omega(\Aut{A})$.
      \end{Definition}

      \note{
        \uz{9:20}
        
        \par
      }
    \end{frame}

  
