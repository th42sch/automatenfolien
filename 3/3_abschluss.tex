
  % ------------------------------------------------------------------------------------------
    \begin{frame}
      \frametitle{Operationen auf $\omega$-Sprachen}

      Zur Erinnerung: die Menge der Büchi-erkennbaren Sprachen heißt\\
      abgeschlossen unter
      \begin{Itemize}
        \item
          \Bmph{Vereinigung}, wenn gilt:
          \par\smallskip
          Falls $L_1,L_2$ Büchi-erkennbar, so auch $L_1 \cup L_2$.
        \item
          \Bmph{Schnitt}, wenn gilt:
          \par\smallskip
          Falls $L_1,L_2$ Büchi-erkennbar, so auch $L_1 \cap L_2$.
        \item
          \Bmph{Komplement}, wenn gilt: \hfill
%           \par\smallskip
          Falls $L$ Büchi-erkennbar, so auch $\overline{L}$.
      \end{Itemize}

      \par\smallskip
      \begin{alertblock}<2->{Quiz}
%       \uncover<2->{%
%         \Bmph{Quiz:}
        Unter welchen Operationen sind die Büchi-erkennbaren Sprachen abgeschlossen,
        und wie leicht ist das zu zeigen?
        \begin{center}
          \begin{tabular}{lll}
            Vereinigung? & \uncover<3->{\YES} & \uncover<3->{{\small (leicht)}} \\
            Schnitt?     & \uncover<4->{\YES} & \uncover<4->{{\small (mittel)}} \\
            Komplement?  & \uncover<5->{\YES} & \uncover<5->{{\small (schwer)}}
          \end{tabular}
        \end{center}
      \end{alertblock}
%       }
      \note{
        \uz{9:21}
        
        \par
      }
    \end{frame}

    % ------------------------------------------------------------------------------------------
    \begin{frame}
      \frametitle{Abgeschlossenheit}

      \begin{Satz}
        Die Menge der Büchi-erkennbaren Sprachen ist abgeschlossen unter den Operationen
        $\cup$ und $\cap$.
        \label{thm:abgeschlossenheit_v+d}
      \end{Satz}

      \par\medskip
      Direkte Konsequenz aus den folgenden Lemmata.
      
      \par\bigskip
      Abgeschlossenheit unter $\overline{\phantom{o}}$ : siehe Abschnitt "`Determinisierung"'

      \note{
        \uz{9:23}
        
        \par
      }
    \end{frame}

    % ------------------------------------------------------------------------------------------
    \begin{frame}
      \frametitle{Abgeschlossenheit unter Vereinigung}
      
      \begin{Lemma}
        Seien $\Aut{A}_1,\Aut{A}_2$ NBAs über $\Sigma$.\\
        Dann gibt es einen NBA $\Aut{A}_3$ mit $L_\omega(\Aut{A}_3) = L_\omega(\Aut{A}_1) \cup L_\omega(\Aut{A}_2)$.
        \label{lem:abgeschlossenheit_vereinigung}%
      \end{Lemma}
      
      \par\medskip
      \uncover<2->{
        \Bmph{Beweis.}~ analog zu NEAs und NEBAs:
        \par\medskip
        Seien $\Aut{A}_i = (Q_i, \Sigma, \Delta_i, I_i, F_i)$ für $i=1,2$.
        \par
        O.\,B.\,d.\,A.\ gelte $Q_1 \cap Q_2 = \emptyset$.
        \par\medskip
        Konstruieren $\Aut{A}_3 = (Q_3, \Sigma, \Delta_3, I_3, F_3)$ wie folgt.%
      }
      %
      \uncover<3->{
        \begin{Itemize}
          \item
            $Q_3 = Q_1 \cup Q_2$
          \item
            $\Delta_3 = \Delta_1 \cup \Delta_2$
          \item
            $I_3 = I_1 \cup I_2$
          \item
            $F_3 = F_1 \cup F_2$
        \end{Itemize}
      }
      
      \par\smallskip
      \uncover<4->{
        Dann gilt:~ $L_\omega(\Aut{A}_3) = L_\omega(\Aut{A}_1) \cup L_\omega(\Aut{A}_2)$
        \qed%
      }

      \note{
        \uz{9:24}
        
        \par
      }
    \end{frame}

    % ------------------------------------------------------------------------------------------
    \begin{frame}
      \frametitle{Abgeschlossenheit unter Schnitt}
      
      \Bmph{Für NEAs: Produktautomat}
      
      \par\smallskip
      Idee: lasse $\Aut{A}_1$ und $\Aut{A}_2$ "`gleichzeitig"' auf Eingabewort laufen.
      
      \par\smallskip
      Gegeben $\Aut{A}_1,\Aut{A}_2$,
      konstruiere $\Aut{A}_3$ mit $L(\Aut{A}_3) = L(\Aut{A}_1) \cap L(\Aut{A}_2)$:
      \begin{Itemize}
        \item
          $Q_3 = Q_1 \times Q_2$
        \item
          $\Delta_3 = \{((p,p'),a,(q,q')) \mid (p,a,q) \!\in\! \Delta_1 ~\&~ (p',a,q') \!\in\! \Delta_2\}$
        \item
          $I_3 = I_1 \times I_2$
        \item
          $F_3 = F_1 \times F_2$
          \Tafel
      \end{Itemize}
      
      \par\bigskip
      \uncover<2->{%
        \Bmph{Funktioniert das auch für Büchi-Automaten?}
      }
      
      \par\bigskip
      \uncover<3->{%
        \begin{tabular}[b]{@{}l@{~}l@{}}
          \Emph{Nein.} & $\Aut{A}_1$ und $\Aut{A}_2$ besuchen ihre akzeptierenden Zustände\\
                       & möglicherweise nicht synchron! %\\[4pt]
%                       & Beispiel siehe Tafel.
                         \TafelForts
        \end{tabular}
%        \Tafel
      }

      \note{
        \uz{9:25 bis 9:35?}
        
        \par
      }
    \end{frame}

    % ------------------------------------------------------------------------------------------
    \begin{frame}
      \frametitle{Abgeschlossenheit unter Schnitt}
      
      \Gmph{Neue Idee} für Schnitt-Automat $\Aut{A}$\,:
      %
      \begin{Itemize}
        \item
          $\Aut{A}$ simuliert $\Aut{A}_1,\Aut{A}_2$ nach wie vor parallel,
          aber mit \Bmph{2 Modi $1,2$}
          \par\smallskip
        \item
          Modus $i$ bedeutet:~ warte auf einen akz.\ Zustand $f$ von $\Aut{A}_i$
          \par\smallskip
        \item
          Sobald so ein $f$ erreicht ist, wechsle den Modus.
          \par\smallskip
        \item
          Run von $\Aut{A}$ ist erfolgreich, wenn er $\infty$ oft den Modus wechselt.
          \par\smallskip
        \item[$\leadsto$]
          Es werden genau die Wörter akzeptiert,\\
          für die $\Aut{A}_1,\Aut{A}_2$ jeweils einen erfolgreichen Run haben.
      \end{Itemize}
    
      \note{
        \uz{9:35}
        
        \par
      }
    \end{frame}

    \newcommand{\ONE}{\text{\Emph{1}}}
    \newcommand{\TWO}{\text{\Emph{2}}}
    \newcommand{\IN}[2]{\text{\Emph{$p#1\!\in\!F_{#2}$}}}
    \newcommand{\NIN}[2]{\text{\Emph{$p#1\!\notin\!F_{#2}$}}}
    % ------------------------------------------------------------------------------------------
    \begin{frame}
      \frametitle{Abgeschlossenheit unter Schnitt}
      
      \begin{Lemma}
        Seien $\Aut{A}_1,\Aut{A}_2$ NBAs über $\Sigma$.\\
        Dann gibt es einen NBA $\Aut{A}$ mit $L_\omega(\Aut{A}) = L_\omega(\Aut{A}_1) \cap L_\omega(\Aut{A}_2)$.
      \end{Lemma}
      
      \par\smallskip
      \uncover<2->{%
        \Bmph{Beweis:}~
        Seien $\Aut{A}_i = (Q_i, \Sigma, \Delta_i, I_i, F_i)$ NBAs für $i=1,2$.

        \par\smallskip
        Konstruieren $\Aut{A} = (Q, \Sigma, \Delta, I, F)$ wie folgt.
        %
        \begin{align*}
          Q      & = Q_1 \times Q_2 \times \{1,2\} \\
          \Delta & = \{((p,\!\!\;p'\!,\!\!\;\ONE),a,(q,\!\!\;q'\!,\!\!\;\ONE)) \mid \NIN{}{1} ~\&~ (p,\!\!\;a,\!\!\;q) \!\in\! \Delta_1 ~\&~ (p'\!,\!\!\;a,\!\!\;q') \!\in\! \Delta_2\} \\
          \uncover<3->{%
                 & \:\cup\!\!\: \{((p,\!\!\;p'\!,\!\!\;\ONE),a,(q,\!\!\;q'\!,\!\!\;\TWO)) \mid \IN{}{1} ~\&~ (p,\!\!\;a,\!\!\;q) \!\in\! \Delta_1 ~\&~ (p'\!,\!\!\;a,\!\!\;q') \!\in\! \Delta_2\} \\
          }%
          \uncover<4->{%
                 & \:\cup\!\!\: \{((p,\!\!\;p'\!,\!\!\;\TWO),a,(q,\!\!\;q'\!,\!\!\;\TWO)) \mid \NIN{'}{2} ~\&~ (p,\!\!\;a,\!\!\;q) \!\in\! \Delta_1 ~\&~ (p'\!,\!\!\;a,\!\!\;q') \!\in\! \Delta_2\} \\
          }%
          \uncover<5->{%
                 & \:\cup\!\!\: \{((p,\!\!\;p'\!,\!\!\;\TWO),a,(q,\!\!\;q'\!,\!\!\;\ONE)) \mid \IN{'}{2} ~\&~ (p,\!\!\;a,\!\!\;q) \!\in\! \Delta_1 ~\&~ (p'\!,\!\!\;a,\!\!\;q') \!\in\! \Delta_2\} \\
          }%
          \uncover<6->{%
            I    & = I_1 \times I_2 \times \{1\} \\
            F    & = Q_1 \times F_2 \times \{2\}
          }%
        \end{align*}
      }
      \par\vspace*{-2.4\baselineskip}
      \uncover<6->{%
        \Tafel
      }

      \par\medskip
      \uncover<7->{%
        Dann gilt $L_\omega(\Aut{A}) = L_\omega(\Aut{A}_1) \cap L_\omega(\Aut{A}_2)$. \hfill\hfill\TafelForts \qed%
      }

      \note{
        \uzz{16:00}{9:37 bis 9:59?}
        
        \parII
        Kurze Wdhlg.: haben Büchi-Automaten eingeführt, Abschlusseigenschaften behandelt.
        
        \parI
        Vereinigung war einfach; Schnitt ist komplizierter: \\
        2 Kopien des "`alten"' Produktaut., weil akz.\ Zust.\ asynchron auftreten können

        \parI
        Konstruktion auf Folie; jetzt noch 2.\ Richtung des Korrektheitsbeweises
        
        \parII
        \textbf{Beweis bis 16:20}
        
        \par
      }
    \end{frame}

    % ------------------------------------------------------------------------------------------
    \begin{frame}
      \frametitle{Abgeschlossenheit unter Komplement}

      \dots\ siehe Abschnitt\\
      "`Deterministische Büchi-Automaten und Determinisierung"'

      \note{
        \uzz{16:20}{9:59--10:00, Ende}

        \par
      }
    \end{frame}


