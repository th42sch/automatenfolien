
    % ------------------------------------------------------------------------------------------
    \begin{frame}
      \frametitle{Terminierung}

      \begin{block}{}
        \Bmph{Terminierung} von Algorithmen ist wichtig für Problemlösung.
      \end{block}

      \par\bigskip
      \Bmph{Übliches Szenario:}
      \begin{Itemize}
        \item
          Eingabe: endliche Menge von Daten
        \item
          Lasse Programm $P$ laufen, bis es \Emph{terminiert}
        \item
          Ausgabe: Ergebnis, das durch $P$ berechnet wurde
      \end{Itemize}
      Um Ausgabe zu erhalten, \Emph{muss $P$ für jede Eingabe terminieren}.

      \par\bigskip
      \uncover<2->{%
        \begin{exampleblock}{}
          \Gmph{Beispiel:} Validierung von XML-Dokumenten für gegebenes Schema
          \begin{Itemize}
            \item
              Konstruiere Automaten für Schema und Dokument \Emph{(terminiert)}
            \item
              Reduziere auf Leerheitsproblem \Emph{(terminiert)}
            \item
%               Löse Leerheitsprob.\ (sammle erreichb.\ Zustände -- \Emph{terminiert})
                Löse Leerheitsproblem\\
                \hspace*{\fill} (sammle erreichbare Zustände -- \Emph{terminiert})
          \end{Itemize}
        \end{exampleblock}
      }
      \note{
        \uz{8:41}
        
        \par
      }
    \end{frame}

  % ------------------------------------------------------------------------------------------
    \begin{frame}
      \frametitle{Terminierung unerwünscht}

      \begin{block}{}
        Von manchen Systemen/Programmen fordert man,\\
        dass sie \Bmph{nie terminieren}.
      \end{block}

      \par\bigskip
      \begin{exampleblock}{}
        \Gmph{Beispiele:}
        \par\vspace*{-6pt}
        \begin{Itemize}
          \item
            (Mehrbenutzer-)Betriebssysteme\\
            {\small sollen beliebig lange laufen ohne abzustürzen, egal was Benutzer tun}
          \item
            Bankautomaten, Flugsicherungssysteme, Netzwerkkommunikationssysteme, \dots
        \end{Itemize}
      \end{exampleblock}

      \par\bigskip
      \uncover<2->{%
        \Bmph{Gängiges Berechnungsmodell:}
        \begin{Itemize}
          \item
            endliche Automaten mit nicht-terminierenden Berechnungen
          \item
            Terminierung wird als Nicht-Akzeptanz angesehen
          \item
            ursprünglich durch Büchi entwickelt (1960)\\
            {\small Ziel: Algorithmen zur Entscheidung mathematischer Theorien}
%             z.\,B.\ bestimmte Fragmente der Arithmetik
        \end{Itemize}
      }
      \note{
        \uz{8:43}
        
        \par
      }
    \end{frame}

  % ------------------------------------------------------------------------------------------
    \begin{frame}
      \frametitle{Ziel und Vorgehen dieses Kapitels}

      \Bmph{Ziel}
      \par\smallskip
      Beschreibung von Automatenmodellen mit \Emph{unendlichen} Eingaben \\
      und \Emph{nicht-terminierenden} Berechnungen

      \par\bigskip\noindent
      \Bmph{Vorgehen}
      \begin{Itemize}
        \item
          Theorie: ausgiebiges Studium von Büchi-Automaten\\
          und der von ihnen erkannten Sprachen
          \begin{Itemize}
            \item
              Definition, Abschlusseigenschaften
            \item
              Charakterisierung mittels regulärer Sprachen
            \item
              Determinisierung
            \item
              Entscheidungsprobleme
          \end{Itemize}
        \item
          Anwendung von Büchi-Automaten:\\
          Spezifikation \& Verifikation in Linearer Temporallogik (LTL)\\
      \end{Itemize}

      \note{
        \uz{8:45}
        
        \par
      }
    \end{frame}

  % ------------------------------------------------------------------------------------------
    \begin{frame}
      \frametitle{Beispiel: Philosophenproblem \hfill {\normalsize (Dining Philosophers Problem)}}

      Erläutert Nebenläufigkeit und Verklemmung von Prozessen

      \par\medskip
      Demonstriert auch unendliche Berechnungen

      \par\medskip
      Hier: einfachste Version mit 3 Philosophen

      \begin{exampleblock}<2->{Philosophenproblem}
%         \begin{Itemize}
%           \item
%             3 Philosophen $P_1,P_2,P_3$
%           \item
%             Für alle $i$ gilt: entweder denkt $P_i$, oder $P_i$ isst.
%           \item
%             Alle $P_i$ sitzen um einen runden Tisch.
%           \item
%             Jeder $P_i$ hat einen Teller mit Essen vor sich.
%           \item
%             Zwischen je zwei Tellern liegt ein Essstäbchen.
%           \item
%             Um zu essen, muss $P_i$ beide Stäbchen neben seinem Teller benutzen.
%             \par
%             \hfill $\Rightarrow$ Keine zwei $P_i,P_j$ können gleichzeitig essen.
%         \end{Itemize}
        3 Philosophen $P_1,P_2,P_3$

        \par\medskip
        Für alle $i$ gilt: entweder denkt $P_i$, oder $P_i$ isst.

        \par\medskip
        Alle $P_i$ sitzen um einen runden Tisch.

        \par\medskip
        Jeder $P_i$ hat einen Teller mit Essen vor sich.

        \par\medskip
        Zwischen je zwei Tellern liegt ein Essstäbchen.

        \par\medskip
        Um zu essen, benötigt $P_i$ beide Stäbchen neben seinem Teller.

        \par\medskip
        \uncover<3->{%
          \hfill $\Rightarrow$ Keine zwei $P_i,P_j$ können gleichzeitig essen.%
        }
      \end{exampleblock}
      \note{
        \uz{8:46}
        
        \parI
        Weil dieses Beispiel so schräg und archaisch ist, \\
        belasse ich es mal bei der männlichen Form.
        
        \par
      }
    \end{frame}

  % ------------------------------------------------------------------------------------------
    \begin{frame}
      \frametitle{Skizze zum Philosophenproblem}
      \begin{exampleblock}{Zusammenfassung}
        \begin{Itemize}
          \item
            Für alle $i$: entweder denkt $P_i$, oder $P_i$ isst.
          \item
            Keine zwei $P_i,P_j$ können gleichzeitig essen.
        \end{Itemize}
      \end{exampleblock}

%       \par\bigskip
      \uncover<2->{%
        \begin{center}
          \begin{minipage}{.48\textwidth}
            \begin{exampleblock}{}
              \begin{center}
                \Fig{0}
                \par\medskip
                \begin{small}
                  $P_1,P_2,P_3$ denken.
                  \par
                \end{small}
              \end{center}
            \end{exampleblock}
          \end{minipage}
          \hfill
          \begin{minipage}{.48\textwidth}
            \begin{exampleblock}{}
              \begin{center}
                \Fig{1}
                \par\medskip
                \begin{small}
                  $P_1,P_3$ denken; $P_2$ isst.
                  \par
                \end{small}
              \end{center}
            \end{exampleblock}
          \end{minipage}
        \end{center}

      }
      \note{
        \uz{8:48}
        
        \par
      }
    \end{frame}

  % ------------------------------------------------------------------------------------------
    \begin{frame}
      \frametitle{Modellierung durch endliches Transitionssystem}

      \Bmph{Annahmen}
      \begin{Itemize}
        \item
          Am Anfang denken \Bmph{(d)} alle $P_i$.
          \par\smallskip
        \item
          Reihum können sich $P_1,P_2,P_3$ entscheiden,\\
          ob sie denken oder essen \Bmph{(e)} wollen.
      \end{Itemize}

      \par\bigskip
      \Bmph{Zustände des Systems}
      \begin{Itemize}
        \item
%           Anfangszustand $(d,d,d,1)$:
          Anfangszustand ddd1:
          \par%\smallskip
          alle $P_i$ denken, und $P_1$ trifft nächste Entscheidung.
          \par\smallskip
        \item
          alle zulässigen Zustände:
          \begin{center}
            \begin{tabular}{@{}l@{~~~}l@{~~~}l@{~~~}l@{}}
%               $(d,d,d,1)$ & $(e,d,d,1)$ & $(d,e,d,1)$ & $(d,d,e,1)$ \\
%               $(d,d,d,2)$ & $(e,d,d,2)$ & $(d,e,d,2)$ & $(d,d,e,2)$ \\
%               $(d,d,d,3)$ & $(e,d,d,3)$ & $(d,e,d,3)$ & $(d,d,e,3)$
              ddd1 & edd1 & ded1 & dde1 \\
              ddd2 & edd2 & ded2 & dde2 \\
              ddd3 & edd3 & ded3 & dde3
            \end{tabular}
          \end{center}

      \end{Itemize}

      \par\bigskip
      \Bmph{Zustandsüberführungen:}
      \begin{Itemize}
        \item[]
          $d$ oder $e$ -- je nach Entscheidung des $P_i$, der an der Reihe ist
      \end{Itemize}
      \note{
        \uz{8:49}
        
        \par
      }
    \end{frame}

  % ------------------------------------------------------------------------------------------
    \begin{frame}
      \frametitle{Das Transitionssystem}

      \begin{center}
        \Fig{10}
      \end{center}

      \uncover<2->{%
        \Emph{Was sind die Eingaben in das System?}

        \begin{Enumerate}
          \item<3->[]
            Endliche Zeichenketten über $\Sigma = \{d,e\}$\,?
            \par\smallskip
            \uncover<4->{Dann ist das System ein NEA.}
          \item<4->[\Gmph{$\blacktriangleright$}]
            \Gmph{Unendliche Zeichenketten über $\Sigma = \{d,e\}$\,!}
        \end{Enumerate}

      }
      \note{
        \uz{8:51}
        
        \par
      }
    \end{frame}


  % ------------------------------------------------------------------------------------------
    \begin{frame}
      \frametitle{Warum unendliche Zeichenketten?}

%       \begin{Itemize}
%         \item
%           Nehmen an, jeder $P_i$ möchte \Emph{beliebig oft} denken und essen.
%           \par\smallskip
%           Dann ist $P_i$ \Bmph{zufrieden}.
%           \par\smallskip
%         \item
%           System soll \Emph{beliebig lange} ohne Terminierung laufen.
%           \par\smallskip
%         \item[$\leadsto$]
%           \Bmph{mögliche Fragen:}
%           \begin{Enumerate}
%             \item
%               Ist es überhaupt möglich, dass das System beliebig lange läuft?
%             \item
%               Ist es zusätzlich möglich, dass $P_i$ zufrieden ist?
%             \item
%               Ist es möglich, dass $P_1,P_2$ zufrieden sind, aber $P_3$ nicht?
%             \item
%               Ist es möglich, dass alle $P_i$ zufrieden sind?
%           \end{Enumerate}
%       \end{Itemize}

      Nehmen an, jeder $P_i$ möchte \Emph{beliebig oft} denken und essen.

      \par\medskip
      System soll dazu \Emph{beliebig lange} ohne Terminierung laufen.

      \par\medskip
      Philosoph $P_i$ heißt \Bmph{zufrieden}, wenn er währenddessen \Emph{unendlich oft} denkt und isst.

      \par\bigskip
      \uncover<2->{%
        \Bmph{$\leadsto$ Mögliche Fragen:}
        \begin{Enumerate}
          \item
            Kann das System überhaupt beliebig lange laufen?
          \item
            Ist es zusätzlich möglich, dass $P_i$ zufrieden ist?
          \item
            Ist es möglich, dass $P_1,P_2$ zufrieden sind, aber $P_3$ nicht?
          \item
            Ist es möglich, dass alle $P_i$ zufrieden sind?
        \end{Enumerate}
      }

      \note{
        \uz{8:55}
        
        \par
      }
    \end{frame}

  % ------------------------------------------------------------------------------------------
    \begin{frame}[t]
      \frametitle{Frage 1}

      \begin{center}
        \Fig{10}
      \end{center}

      \Emph{Ist es überhaupt möglich, dass das System beliebig lange läuft?}%

      \par\bigskip
      \uncover<2->{%
        \begin{tabular}{@{}l@{~~}l@{}}
          \Gmph{Ja:} & jeder Zustand hat mindestens einen Nachfolgerzustand.\\
                     & $dddddd\dots$ ist ein möglicher unendlicher Lauf.
        \end{tabular}%
      }
      \note{
        \uz{8:57}
        
        \par
      }
    \end{frame}

  % ------------------------------------------------------------------------------------------
    \begin{frame}[t]
      \frametitle{Frage 2}

      \begin{center}
        \Fig{10}
      \end{center}

      \Emph{Ist es möglich, dass $P_1$ zufrieden ist?}%

      \par\bigskip
      \uncover<2->{%
        \begin{tabular}{@{}l@{~~}l@{}}
          \Gmph{Ja:} & z.\,B.\ wenn ein Lauf ddd1 und edd1 unendlich oft durchläuft:\\
                     & $ed^5ed^5\dots$
        \end{tabular}%
      }
      \note{
        \uz{8:58}
        
        \par
      }
    \end{frame}

  % ------------------------------------------------------------------------------------------
    \begin{frame}[t]
      \frametitle{Frage 3}

      \begin{center}
        \Fig{10}
      \end{center}

      \Emph{Ist es möglich, dass $P_1,P_2$ zufrieden sind, aber $P_3$ nicht?}%

      \par\bigskip
      \uncover<2->{%
        \begin{tabular}{@{}l@{~~}l@{}}
          \Gmph{Ja:} & z.\,B.\ "`ddd1, edd1, ddd2, ded2 unendlich oft, aber dde$i$ nicht"':\\
                     & $ed^3ed^4ed^3ed^4\dots$
        \end{tabular}%
      }
      \note{
        \uz{8:59}
        
        \par
      }
    \end{frame}

  % ------------------------------------------------------------------------------------------
    \begin{frame}[t]
      \frametitle{Frage 4}

      \begin{center}
        \Fig{10}
      \end{center}

      \Emph{Ist es möglich, dass alle $P_i$ zufrieden sind?}%

      \par\bigskip
      \uncover<2->{%
        \begin{tabular}{@{}l@{~~}l@{}}
          \Gmph{Ja:} & z.\,B.\ "`ddd1, edd1, ddd2, ded2, ddd3, dde3 unendlich oft"': \\
                     & $ed^3ed^3\dots$ \quad oder \quad $ed^2ed^3ed^2ed^3\dots$ \quad oder \quad $\dots$
        \end{tabular}%
      }
      \note{
        \uz{9:00}
        
        \par
      }
    \end{frame}

  % ------------------------------------------------------------------------------------------
    \begin{frame}
      \frametitle{Weiteres Beispiel}
      
      \dots\ siehe Anhang, Folie \ref{fra:anhang_bsp2} \dots
      \note{
        \uz{9:01}
        
        \par
      }
    \end{frame}


