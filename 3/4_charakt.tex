
    % ------------------------------------------------------------------------------------------
    \begin{frame}
      \frametitle{Ziel}

      \Bmph{Ziel dieses Abschnitts}
      \par\smallskip
      Charakterisierung der Büchi-erkennbaren Sprachen\\
      mittels regulärer Sprachen

      \par\bigskip
      \uncover<2->{%
        \Bmph{Etwas Notation}
        \par\smallskip
        Seien $W \subseteq \Sigma^*$ und $L \subseteq \Sigma^\omega$.
        \begin{Itemize}
          \item
            \Bmph{$W^\omega$} $=$ $\{w_0w_1w_2 \dots \mid w_i \in W\setminus\{\varepsilon\} \text{~für alle~} i \geqslant 0\}$
            \parI
            {\small (ist $\omega$-Sprache, weil $\varepsilon$ ausgeschlossen wurde)}
            \parII
          \item
            \Bmph{$WL$} $=$ $\{w\alpha \mid w \in W,~ \alpha \in L\}$
            \parI
            {\small (ist $\omega$-Sprache)}
        \end{Itemize}
      }
      \note{
        \textbf{16:20}
        
        \parI
        $W^{\omega}$ entspricht dem Kleene-Stern bei Sprachen endlicher Wörter; \\
        $WL$ entspricht der Konkatenation.

        \par
      }
    \end{frame}

  % ------------------------------------------------------------------------------------------
    \begin{frame}[t]
      \frametitle{Von regulären zu Büchi-erkennbaren Sprachen (1)}

      \begin{lemma}
        Für jede reguläre Sprache $W \subseteq \Sigma^*$ gilt:~
        $W^\omega$ ist Büchi-erkennbar.%
        \label{lem:Charakt_Buchi_hoch_omega}%
      \end{lemma}

      \parII%\bigskip
      \uncover<2->{%
        \Bmph{Beweis. (Schritt 1)}

        \par\smallskip
        Sei $\Aut{A}$ ein \Emph{NEA} mit $L(\Aut{A}) = W$.
      }
    
      \parI
      \uncover<3->{
        Dann gibt es NEA $\Aut{A}_1$ mit $L(\Aut{A}_1) = W\!\setminus\!\{\varepsilon\}$ (Abschlusseig.!)
      }
        
      \parI
      \uncover<4->{
        O.\,B.\,d.\,A.\ habe $\Aut{A}_1$ \dots
        %
        \begin{Enumerate}
          \item
            einen \Emph{einzigen} Anfangszustand $q_I$\quad und
          \item
            \Emph{keine} in $q_I$ \Emph{eingehenden} Kanten:~ keine Transitionen $(\cdot,\cdot,q_I$)
          \item
            und sei $q_I \notin F$.
        \end{Enumerate}
      }
        
      \parI
      \uncover<5->{
        Diese Form lässt sich durch Hinzufügen eines frischen Anfangs-\\
        zustandes (und der entsprechenden Transitionen) erreichen! (Ü)
      }
    
      \note{
        \textbf{16:22}
        
        \parI
        $\Big($Idee:~ Füge neuen Startzustand hinzu; dupliziere alle Kanten, die von bisherigen SZen ausgehen.
        
        \parI
        Damit sind (1) und (2) erreicht.~
        (3) ist korrekt, weil $\varepsilon \notin L(\Aut{A}_1)$.$\Big)$
        
        \parIII
        $\Big($Gegenbeispiel für die Rückrichtung:
        
        \parI
        $W = \{a^n \mid n \text{~ist prim}\}$
        
        \parI
        $\Rightarrow~ W^\omega = \{a^\omega\}$\qquad (und $W^* = \{a^n \mid n \geq 2\}$)$\Big)$

        \par        
      }
    \end{frame}

    \addtocounter{theorem}{-1}
    % ------------------------------------------------------------------------------------------
    \begin{frame}[t]
    \frametitle{Von regulären zu Büchi-erkennbaren Sprachen (1)}
    
      \begin{lemma}
        Für jede reguläre Sprache $W \subseteq \Sigma^*$ gilt:~
        $W^\omega$ ist Büchi-erkennbar.%
%        \label{lem:Charakt_Buchi_hoch_omega}
      \end{lemma}
    
      \parII
      \Bmph{Beweis. (Schritt 2a)}
      
      \parI
      Sei also $\Aut{A}_1 = (Q_1, \Sigma, \Delta_1, \{q_I\}, F)$ mit den genannten Eigenschaften
      und $L(\Aut{A}_1) = W\!\setminus\!\{\varepsilon\}$.
      
      \parII
      \uncover<2->{%
        \Bmph{Idee:}~ konstruiere \Bmph{NBA $\Amcb_2$}\,, der
        \begin{Itemize}
          \item
            $\Aut{A}_1$ simuliert, bis ein akzeptierender Zustand erreicht ist und
          \item
            dann \Emph{nichtdeterministisch entscheidet,} \\
            ob die Simulation fortgesetzt wird \\
            oder eine neue Simulation von $q_0$ aus gestartet wird
        \end{Itemize}
      }

      \note{
        \textbf{16:24}
        
        \parI
        Letztlich ist das dieselbe Idee wie bei der Kleene-Abg.\ der NEA-erkennbaren Sprachen: \\
        erzeuge Kreis, der $\infty$ oft durchlaufen werden kann.
        
        \parI
        Die kann man aber nicht so leicht auf Büchiaut.\ übertragen,
        denn sie führt $\varepsilon$-Kanten ein, und diese kann man innerhalb von Kreisen
        nicht so leicht eliminieren wie bei NEAs
        (ehemals akz.\ Zustände könnten keine ausgehenden Kanten mehr haben, also gehen erfolgr.\ Runs verloren \dots)
        

        \par
      }
    \end{frame}

    \addtocounter{theorem}{-1}
    % ------------------------------------------------------------------------------------------
    \begin{frame}[t]
      \frametitle{Von regulären zu Büchi-erkennbaren Sprachen (1)}
      
      \begin{lemma}
        Für jede reguläre Sprache $W \subseteq \Sigma^*$ gilt:~
        $W^\omega$ ist Büchi-erkennbar.%
        %        \label{lem:Charakt_Buchi_hoch_omega}
      \end{lemma}
      
      \parII
      \Bmph{Beweis. (Schritt 2b)}
      
      \parI
      Sei also $\Aut{A}_1 = (Q_1, \Sigma, \Delta_1, \{q_I\}, F)$ mit den genannten Eigenschaften
      und $L(\Aut{A}_1) = W\!\setminus\!\{\varepsilon\}$.
      
      \parII
      Definiere NBA $\Bmph{$\Amcb_2$} = (Q_1, \Sigma, \Bmph{$\Delta_2$}, \{q_I\}, \Emph{$\{q_I\}$})$ mit
      %
      \begin{center}
        \parbox{.85\linewidth}{%
          \uncover<2->{%
            $\Bmph{$\Delta_2$} \,=\, \Delta_1 \cup \{(q,a,q_I) \mid (q,a,q_f) \in \Delta_1 \text{~für ein~} q_f \in F\}$
          }%
          
          \parII
          \uncover<3->{%
            \begin{footnotesize}
              (d.\,h.\ alle Kanten, die in $\Aut{A}_1$ zu einem akz.\ Zustand führen, \\
              können in $\Aut{A}_2$ zusätzlich zu $q_I$ führen \\
              -- siehe "`nichtdeterministisch entscheidet"' auf voriger Folie!)
              \par
            \end{footnotesize}
          }
        }
      \end{center}
      %
      
      \uncover<4->{%
        \Bmph{Noch zu zeigen:}~
        $L_\omega(\Aut{A}_2) = L(\Aut{A}_1)^\omega$
        \Tafel~~~~~
        \par\vspace*{-.95\baselineskip}
        \qed
      }
      
      \note{
        \textbf{16:26 bis 16:45 $\to$ 5min Pause}
        
        \par
      }
    \end{frame}

  % ------------------------------------------------------------------------------------------
    \begin{frame}
      \frametitle{Von regulären zu Büchi-erkennbaren Sprachen (2)}

      \begin{lemma}
        \label{lem:Charakt_Buchi_Konkat}
        ~\par\vspace*{-1.4\baselineskip}
        Für jede reguläre Sprache $W \subseteq \Sigma^*$\\
        und jede Büchi-erkennbare Sprache $L \subseteq \Sigma^\omega$ gilt:
        \par\smallskip
        $WL$ ist Büchi-erkennbar.
      \end{lemma}

      \par\bigskip
      \uncover<2->{%
        \Bmph{Beweis:}
        \par\smallskip
        Wie Abgeschlossenheit der regulären Sprachen unter Konkatenation.
        \qed
      }

      \note{
        \textbf{16:50}

        \par
      }
    \end{frame}

  % ------------------------------------------------------------------------------------------
    \begin{frame}
      \frametitle{Satz von Büchi}

      \begin{Satz}
        \label{thm:Charakt_Buchi}
        ~\par\vspace*{-1.4\baselineskip}
        Eine Sprache $L \subseteq \Sigma^\omega$ ist Büchi-erkennbar
        genau dann,\\
        wenn es reguläre Sprachen $V_1,W_1,\dots,V_n,W_n$ gibt mit $n \geqslant 1$ und

        \par\medskip
        \centerline{$L = V_1W_1^\omega \cup \dots \cup V_nW_n^\omega$}
%         \[
%           L = V_1W_1^\omega \cup \dots \cup V_nW_n^\omega
%         \]
      \end{Satz}

      \par\smallskip
      \uncover<2->{%
        \Bmph{Beweisskizze:}
        \only<2|handout:0>{\hfill (Quiz: Welche der Richtungen $\Rightarrow$, $\Leftarrow$ ist leichter?)}
      }

      \par\smallskip
      \uncover<3->{%
        ``$\Leftarrow$'': folgt aus Lemmas~\ref{lem:abgeschlossenheit_vereinigung}, \ref{lem:Charakt_Buchi_hoch_omega} und~\ref{lem:Charakt_Buchi_Konkat}

        \par\medskip
        ``$\Rightarrow$'': bilden $V_i,W_i$ aus denjenigen Wörtern,
        die zum jeweils nächsten Vorkommen eines akzeptierenden Zustandes führen

        \par\medskip
        Details siehe Tafel. \Tafel~~~~
        \par\vspace*{-.95\baselineskip}
        \qed
      }
      
      \par\bigskip
      \uncover<4->{%
        \Bmph{Konsequenz:}
        \par\smallskip
        Büchi-erkennbare Sprachen durch
        \Bmph{$\omega$-reguläre Ausdrücke} darstellbar:
        \par\medskip
        \centerline{%
          $r_1s_1^\omega + \dots + r_ns_n^\omega$
          \qquad\qquad
          ($r_i,s_i$ sind reguläre Ausdrücke)%
        }%
%         \[
%           r_1s_1^\omega + \dots + r_ns_n^\omega
%           \qquad\qquad
%           \text{($r_i,s_i$ sind reguläre Ausdrücke)}%
%         \]
      }
      \note{
        \textbf{16:52 bis 17:02}
        
        \parI
        TODO: Tafelanschrieb auf Folie; hier passiert nix Anstrengendes.
        
        \par
      }
    \end{frame}


