  
  % ------------------------------------------------------------------------------------------
    \begin{frame}
      \label{fra:anhang_bsp2}
      \frametitle{Anhang: Beispiel Konsument-Produzent-Problem}

      \begin{exampleblock}{}
        \begin{Itemize}
          \item
            $P$ erzeugt Produkte und legt sie einzeln in einem Lager ab
          \item
            $K$ entnimmt Produkte einzeln dem Lager
          \item
            Lager fasst maximal 3 Stück
        \end{Itemize}
      \end{exampleblock}

      \par\bigskip
      \uncover<2->{%
        \Bmph{Modellierung durch endliches Transitionssystem}
        \begin{Itemize}
          \item
            Zustände $0,1,2,3,\text{Ü},\text{U}$
            \begin{Itemize}
              \item
                0,1,2,3: im Lager liegen 0,1,2,3 Stück
              \item
                \emph{Ü}\/berschuss: $P$ will ein Stück im vollen Lager ablegen
              \item
                \emph{U}\/nterversorgung: $K$ will ein Stück aus leerem Lager nehmen
            \end{Itemize}
          \item
            Aktionen $P,K$ ($P$ legt ab oder $K$ entnimmt)
        \end{Itemize}
      }

      \note{~}
    \end{frame}

  % ------------------------------------------------------------------------------------------
    \begin{frame}
      \frametitle{Das Transitionssystem}

      \begin{center}
        \Fig{20}
      \end{center}

      \uncover<2->{%
        \begin{tabular}[t]{@{}l@{~~}l@{}}
          \Bmph{Eingaben in das System:} & unendliche Zeichenketten über $\Sigma = \{p,k\}$ \\
                                         & \Bmph{(Läufe)}
        \end{tabular}
      }

      \par\bigskip
      \uncover<3->{%
        \Bmph{Zufriedenheit:} $P$ ($K$) möchte \dots
        \begin{Itemize}
          \item
            beliebig oft Produkte produzieren (konsumieren)
          \item
            nur endlich oft \emph{Ü}\/berschuss (\emph{U}\/nterversorgung) erleiden
        \end{Itemize}
      }

      \par\bigskip
      \uncover<4->{%
        \Bmph{Lauf, der $P$ und $K$ zufrieden stellt:}\\%
      }
      \uncover<5->{%
        $p^3k^3p^3k^3\dots$\quad oder\quad $ppkpkpk\dots$\quad oder\quad $\dots$
      }
      
      \par\bigskip
      \uncover<6->{%
        \Bmph{Lauf, der weder $P$ noch $K$ zufrieden stellt:}%
      }
      \uncover<7->{%
        $p^4k^4p^4k^4\dots$
      }
      \note{~}
    \end{frame}

%     % ------------------------------------------------------------------------------------------
%     \begin{frame}
%       \frametitle{\dots}
%       \dots
%       \note{~}
%     \end{frame}

