% ------------------------------------------------------------------------------------------
\begin{frame}
  \frametitle{Überblick}

  \Bmph{Ziel dieses Abschnitts:}
  \par\smallskip
  Lösen Komplementierung mit Hilfe eines bekannten Resultates
  über Gewinnstrategien in einer bestimmten Art (abstrakter) Spiele

  \par\medskip
  \Bmph{Vorgehen:}
  \begin{Itemize}
    \item
      \mbox{\scalebox{.98}[1]{Ordnen jedem N\Emph{P}BA \Aut{A} und Baum $t$
      ein 2-Personen-Spiel \Game{A}{t} zu}\hspace*{-10mm}} \\
%      \par\smallskip
      {\small (Beschränkung auf N\Emph{P}BAs ist unerheblich, siehe Satz~\ref{thm:muller_vs_paritaet})}
%           \par\smallskip
    \item
      Dann wird leicht zu sehen sein:
      \par\smallskip
      $\Aut{A}$ akzeptiert $t$ ~$\Leftrightarrow$~ Spielerin 1 hat Gewinnstrategie in \Game{A}{t}
%           \par\smallskip
    \item
      Ein Resultat aus der Spieltheorie impliziert:
      \par\smallskip
      In \Game{A}{t} hat immer \Emph{genau eine Spielerin eine Gewinnstrategie,}\\
      \Emph{die nicht vom bisherigen Spielverlauf abhängt}
%           \par\smallskip
    \item
%           Konstruieren $\Aut{A}'$ anhand der Gewinnstrategie von Spielerin 2;\\
%           zeigen: $L_\omega(\Aut{A}') = \overline{L_\omega(\Aut{A})}$
      Konstruieren $\Aut{A}'$, so dass gilt:
      \par\smallskip
      $\Aut{A}'$ akzeptiert $t$ ~$\Leftrightarrow$~ \Emph{Spielerin 2 hat Gewinnstrategie in \Game{A}{t}}
      \par\smallskip
      Dann folgt $L_\omega(\Aut{A}') = \overline{L_\omega(\Aut{A})}$
  \end{Itemize}
  

  \note{%
    \textbf{16:00}
    
    \par
  }
\end{frame}

% ------------------------------------------------------------------------------------------
\begin{frame}
  \frametitle{Intuitive Beschreibung des Spiels \Game{A}{t}}
  
  Zwei Spielerinnen \AUT (Automat), \PF (Pfadfinderin)
  \begin{Itemize}
    \item
      sind abwechselnd an der Reihe
    \item
      bewegen sich pro Runde 1 Schritt im Baum
      durch Markieren von Positionen mit Zuständen; zu Beginn: $(\varepsilon,q_I)$,~ $q_I \in I$
  \end{Itemize}

  \par\bigskip
  \uncover<2->{%
    In jeder Runde wählt
    \begin{Itemize}
      \item
        \mbox{\AUT eine Transition, die auf die markierte Position anwendbar ist\hspace*{-10mm}}
      \item
        \PF einen Kindknoten und verschiebt Markierung dorthin
    \end{Itemize}
  }

  \par\bigskip
  \uncover<3->{%
    Spiel läuft $\infty$ lange, erzeugt $\infty$ Folge $r=q_0q_1q_2$ von Zuständen\\
    (bestimmt durch die gewählten Transitionen)%
  }

  \par\bigskip
  \uncover<4->{%
    \AUT gewinnt, wenn $r$ der Akzeptanzbedingung von \Aut{A} entspricht;\\
    sonst gewinnt \PF

    \par%\smallskip
    \begin{footnotesize}
      \mbox{(d.\,h.\ \AUT versucht, \Aut{A} zum Akzeptieren zu bringen; \PF versucht das zu verhindern)\hspace*{-10mm}}
      \par
    \end{footnotesize}
  }

  \par\bigskip
  \uncover<5->{%
    Skizze: s.\ Tafel \Tafel
  }

  \note{%
    \textbf{16:03 bis 16:12}
    
    \parI
    $q_I$ bezeichne ab hier einen beliebigen Anfangszustand; \\
    brauchen "`$q_0$"' später noch anderweitig.
    
    \parII
    Es wird also ein Baumautomat mit \textbf{beliebiger Akzeptanzbedingung} (Büchi, Muller, whatever) zugrunde gelegt.
    
    \par
  }
\end{frame}

% ------------------------------------------------------------------------------------------
\begin{frame}
  \frametitle{Genaue Beschreibung des Spiels \Game{A}{t}}

  Spiel ist ein unendlicher Graph
  \begin{Itemize}
    \item<+->
      Knoten sind die \Bmph{Spielpositionen}:
      \begin{Itemize}
        \item
          für \AUT: $\{(p,q) \mid p \in \{0,1\}^*,~ q \in Q\}$~ (Positionen im Baum)
        \item
          für \PF: $\{(q,t(p),q_0,q_1) \in \Delta \mid p \in \{0,1\}^*\}$\quad (Transitionen)
      \end{Itemize}
      \par\smallskip
    \item<+->
      Kanten sind die möglichen \Bmph{Spielzüge}:
      \begin{Itemize}
%        \item
%%               für \AUT:
%          $(p,q) \to (q',t(p'),q_0,q_1)$, wenn $p=p'$ und $q=q'$
%        \item
%%               für \PF:
%          $(q,t(p),q_0,q_1) \to (p',q')$, wenn $q'=q_i$ und $p'=pi$ für ein $i$
        \item
          $(p,q) \to (q,t(p),q_0,q_1)$
        \item
          \begin{tabular}[t]{@{}l@{~}c@{~}l@{}}
            $(q,t(p),q_0,q_1)$ & $\to$ & $(p0,q_0)$ \\
                               & \hspace*{-12pt}\raisebox{4pt}{\turnbox{-25}{$\to$}} & $(p1,q_1)$
          \end{tabular}
      \end{Itemize}
      \par\smallskip
    \item<+->
      Startknoten: $(\varepsilon,q_I)$ für $q_I \in I$\quad (o.\,B.\,d.\,A.\ $I=\{q_I\}$)
  \end{Itemize}

  \par\bigskip
  \uncover<+->{%
    Jede mögliche $\infty$ Folge von Spielzügen entspricht\\
    einem $\infty$ Pfad im Spielbaum \Game{A}{t}
  }

  \par\bigskip
  \uncover<+->{%
    Knoten $v'$ \Bmph{erreichbar} von Knoten $v$:\\
    es gibt endliche Folge von Spielzügen von $v$ nach $v'$
  }

  \note{
    \textbf{16:12}
    
    \parII
    \textbf{Achtung:}~ Ab jetzt bezeichnet $p$ Positionen im Baum, nicht mehr Zustände!
    
    \par
  }
\end{frame}

% ------------------------------------------------------------------------------------------
\begin{frame}
  \frametitle{Spielstrategien}

  \Bmph{Strategie ab Spielposition $v$} für Spielerin $X \in \{\AUT,\PF\}$:
  \par\smallskip
  Funktion, die jeder Zugfolge $v\dots v'$ mit $v'$ Spielposition für $X$ \\
  einen in $v'$ möglichen Zug zuweist
  \par%\smallskip
  \begin{footnotesize}
    (legt fest, welchen Zug $X$ in jeder von $v$ aus erreichbaren Spielposition macht)
    \par
  \end{footnotesize}

  \par\bigskip
  \uncover<2->{%
    \Bmph{Gewinnstrategie} für Spielerin $X \in \{\AUT,\PF\}$:
    \par\smallskip
    Strategie, die sicherstellt, dass $X$ gewinnt,\\
    \emph{unabhängig} von den Zügen der Gegenspielerin \Tafel
  }

  \par\bigskip
  \uncover<3->{%
    \Bmph{gedächtnislose Strategie:}
    \par\smallskip
    Strategie, die nur von $v'$ abhängt, nicht von den vorigen Positionen
  }

  \note{
    \textbf{16:17 bis 16:26}
    
    \par
  }
\end{frame}

% ------------------------------------------------------------------------------------------
\begin{frame}
  \frametitle{Akzeptanz mittels Gewinnstrategien}

  \begin{Lemma}
    Seien $\Aut{A} = (Q,\Sigma,\Delta,\{q_I\},c)$ ein NPBA und $t$ ein $\Sigma$-Baum.\\ Dann gilt:
    \par\smallskip
    $t \in L_\omega(\Aut{A})$ ~$\Leftrightarrow$~ \AUT hat Gewinnstrategie in $\Game{A}{t}$ ab Position $(\varepsilon,q_I)$%
    \label{lem:akzeptanz_vs_gewinnstrategien}%
%         \par\vspace*{-.4\baselineskip}
  \end{Lemma}

  \par\bigskip
  \uncover<2->{%
    \Bmph{Beweis:} 

    \par\smallskip
    Konstruiere Gewinnstrategie direkt aus einem erfolgreichen Run\\
    und umgekehrt

%    \par\medskip
%    Details: s.\ Tafel \Tafel~~~~
%    \par\vspace*{-.95\baselineskip}\qed
  }

  \note{
    \textbf{16:26}
    
%    \parII
%    Nach wie vor beliebige Akzeptanzbedingung (deshalb N\textbf{x}BA)
    
    \par
  }
\end{frame}

% ------------------------------------------------------------------------------------------
\begin{frame}
  \frametitle{Akzeptanz mittels Gewinnstrategien}
  
  \begin{block}{}
    \Bmph{"`$t \in L_\omega(\Aut{A})$ ~$\Rightarrow$~ \AUT hat Gewinnstrategie in $\Game{A}{t}$ ab Position $(\varepsilon,q_I)$"'}
  \end{block}
  
  \parII
  Gelte $t \in L_\omega(\Aut{A})$ und sei $r$ erfolgreicher Run von $\Aut{A}$ auf $t$.
  
  \par
  Konstruiere Gewinnstrategie für \AUT wie folgt aus $r$.
  %
  \begin{Itemize}
    \item<2->
      in Startposition $(\varepsilon,q_I)$ wähle
      $(r(\varepsilon),t(\varepsilon),r(0),r(1))$
    \item<3->
      in allen anderen Spielpos.\ $(p,q)$ wähle
      $(q,t(p),r(p0),r(p1))$
  \end{Itemize}
  
  \parI
  \uncover<4->{%
    Wenn \AUT diese Strategie befolgt,
    dann entspricht die im Spiel erzeugte Zustandsmenge
    einem Pfad in $r$.
  }

  \parII
  \uncover<5->{%
    Da $r$ erfolgreich, gewinnt \AUT nach Definition von $G_{\Aut{A},t}$\,.
  }

  \parII
  \uncover<6->{%
    \begin{block}{}
      \Bmph{"`\AUT hat Gewinnstrategie in $\Game{A}{t}$ ab Position $(\varepsilon,q_I)$ ~$\Rightarrow$~ $t \in L_\omega(\Aut{A})$"'}
    \end{block}
    \Tafel~~~~
    \par\vspace*{-.95\baselineskip}\qed
  }
  
  \note{
    \textbf{16:28 bis 16:38}
    
    \par
  }
\end{frame}

% ------------------------------------------------------------------------------------------
\begin{frame}
  \frametitle{Determiniertheit von Paritätsspielen}

  Klassisches Resultat aus der Spieltheorie, hier nicht bewiesen:

  \begin{Satz}[Emerson \& Jutla 1991, Mostowski 1991]
    Alle Paritätsspiele sind \Bmph{gedächtnislos determiniert}:\\
    genau eine der Spielerinnen hat eine \textup{gedächtnislose} Gewinnstrategie.%
    \label{thm:gedaechtnislos_determiniert}
  \end{Satz}

%       \begin{footnotesize}
%         E.\ A.\ Emerson und C.\ S.\ Jutla 1991,\quad A.\ Mostowski 1991 %\\
%         (ohne "`gedächtnislos"': D.\ A.\ Martin 1975)
%         \par
%       \end{footnotesize}

  \par\bigskip
  "`Paritätsspiel"' bezeichnet dabei 2-Personen-Spiele, die
  %
  \begin{Itemize}
    \item
      auf Graphen gespielt werden, \\
      deren Knoten mit natürlichen Zahlen markiert sind;
    \item
      als Gewinnbedingung für unendliche Spielverläufe \\
      die Paritätsbedingung verwenden.
  \end{Itemize}
  %
  \parI
  Für alle $\Aut{A}$ und $t$ ist $\Game{A}{t}$ ein Paritätsspiel.
  
  \note{
    \textbf{16:38}
    
    \par
  }
\end{frame}

% ------------------------------------------------------------------------------------------
\begin{frame}
  \frametitle{Determiniertheit von Paritätsspielen}

  Folgerung aus Satz~\ref{thm:gedaechtnislos_determiniert}:

  \begin{Folgerung}
    Seien $\Aut{A} = (Q,\Sigma,\Delta,\{q_I\},c)$ ein NPBA und $t$ ein $\Sigma$-Baum.
    \par\smallskip
%           Dann gibt es für jede Spielposition $v$ in \Game{A}{t}
%           eine gedächtnislose Gewinnstrategie für \AUT oder \PF.
%           \par\smallskip
%           Insbesondere gibt es ab $(\varepsilon,q_I)$
%           eine solche.
    Dann gibt es für jede Spielposition $v$ in \Game{A}{t}
    --- und insbesondere für $(\varepsilon,q_I)$ ---
    eine gedächtnislose Gewinnstrategie für \AUT oder \PF.%
    \label{cor:gedaechtnislose_determiniertheit}%
%           \par\vspace*{-.4\baselineskip}
  \end{Folgerung}

  \par\bigskip
  \uncover<2->{%
%         Das heißt (mit Lemma \ref{lem:akzeptanz_vs_gewinnstrategien}):
%         \par\smallskip
%         $t \in \overline{L_\omega(\Aut{A})}$ ~$\Leftrightarrow$~ \PF hat gedächtnislose GS ab $(\varepsilon,q_I)$ in $\Game{A}{t}$
% 
%         \par\bigskip
%         Benutzen diese nun, um einen NPBA für $\overline{L_\omega(\Aut{A})}$ zu konstruieren
    \begin{Folgerung}[aus Lemma \ref{lem:akzeptanz_vs_gewinnstrategien} und Folgerung \ref{cor:gedaechtnislose_determiniertheit}]
      $t \in \overline{L_\omega(\Aut{A})}$ ~$\Leftrightarrow$~ \PF hat gedächtnislose GS ab $(\varepsilon,q_I)$ in $\Game{A}{t}$
      \label{cor:gedaechtnislose_GS_fuer_PF}%
%           \par\vspace*{-.4\baselineskip}
    \end{Folgerung}

    \par%\bigskip
    \Emph{Ziel:} konstruieren NPBA, um deren Existenz zu testen
  }

  \note{
    \textbf{16:39 bis 16:41, 5min Pause}
    
    \par
  }
\end{frame}

% ------------------------------------------------------------------------------------------
\begin{frame}
  \frametitle{Gewinnbäume}

  Betrachten gedächtnislose Gewinnstrategien für \PF
  als Menge von Funktionen
  \[
    f_p : \Delta \to \{0,1\}\qquad \text{für jede Baumposition~} p \in \{0,1\}^*
  \]
  \Bmph{Idee:} $f_p$ weist jeder Transition, die \AUT in Baumposition $p$ wählt,
  einen Spielzug (Richtung 0/1) zu

  \par\bigskip
  \uncover<2->{%
    \begin{Itemize}
      \item
        Sei $F$ die Menge dieser Funktionen
      \item
        Ordnen die $f_p$ in einem \Bmph{$F$-Baum} $s$ an\quad (Strategiebaum)
    \end{Itemize}
  }

  \par\bigskip\vspace*{-.6pt}
  \uncover<3->{%
    \Bmph{PF-Gewinnbaum} für $t$:\\
    ein $F$-Baum, der eine Gewinnstrategie für \PF in \Game{A}{t} kodiert
  }

  \par\medskip
  \uncover<4->{%
    \begin{Folgerung}[aus Folgerung \ref{cor:gedaechtnislose_GS_fuer_PF}]
      $t \in \overline{L_\omega(\Aut{A})}$ ~$\Leftrightarrow$~ es gibt einen \PF-Gewinnbaum für $t$%
      \label{cor:Gewinnbaum_fuer_PF}%
%           \par\vspace*{-.4\baselineskip}
    \end{Folgerung}

    \par%\bigskip
    \mbox{\Emph{Neues Ziel:} bauen NPBA, um Existenz \PF-Gewinnbaum zu testen\hspace*{-10mm}}
  }

  \note{
    \textbf{16:46 bis 16:49}
    
    \par
  }
\end{frame}

% ------------------------------------------------------------------------------------------
\begin{frame}
  \frametitle{Existenz von PF-Gewinnbäumen (\PF-GB)}

  Sei $\Aut{A} = (Q,\Sigma,\Delta,\{q_I\},c)$ ein NPBA und $t$ ein $\Sigma$-Baum
  \par\smallskip
  \Emph{Zwischenziel:} Prüfen, ob gegebener $F$-Baum $s$ \Emph{kein} \PF-GB ist

  \par\medskip
  \uncover<2->{%
    \Bmph{Idee:}
    \begin{Itemize}
      \item
        Benutzen NPA $\Aut{A}'$\qquad \Emph{($\omega$-Wortautomat)}
      \item
        $\Aut{A}'$ prüft für jeden Pfad $\pi$ in t und jeden möglichen Spielzug von \AUT
        separat, ob Akzeptanzbedingung von \Aut{A} erfüllt ist
      \item[$\leadsto$]
        $\Aut{A}'$ akzeptiert $\geqslant 1$ Pfad ~$\Leftrightarrow$~ $s$ ist kein \PF-Gewinnbaum für $t$ 
    \end{Itemize}
  }

  \par\bigskip
  \uncover<3->{%
    Sei $\pi \in \{0,1\}^\omega$ ein Pfad mit $\pi=\pi_1\pi_2\pi_3\dots$
    
    \parI
    $\Aut{A}'$ arbeitet auf Wörtern der folgenden Form:
    \[
%           \Big(s(\varepsilon), t(\varepsilon), \pi[1]\Big),~
%           \Big(s(\pi[1]), t(\pi[1]), \pi[2]\Big),~
%           \dots
      \Big\auf s(\varepsilon), t(\varepsilon), \pi_1\Big\zu~~
      \Big\auf s(\pi_1), t(\pi_1), \pi_2\Big\zu~~
      \Big\auf s(\pi_1\pi_2), t(\pi_1\pi_2), \pi_3\Big\zu~~
      \dots
    \]
    Sei \Bmph{\Lst} die Sprache aller dieser Wörter
  }

  \par\medskip
  \uncover<4->{%
    Beispiel: s.\ Tafel \Tafel
  }

  \note{
    \textbf{16:49 bis 16:58}
    
    \par
  }
\end{frame}

% ------------------------------------------------------------------------------------------
\begin{frame}
  \frametitle{Konstruktion des Wortautomaten für Gewinnbäume}

  Sei $\Aut{A} = (Q,\Sigma,\Delta,\{q_I\},c)$ ein NPBA und $t$ ein $\Sigma$-Baum
  \par\smallskip
  Konstruieren NPA $\Aut{A}' = (Q,\Sigma',\Delta',\{q_I\},c)$ wie folgt:
  \begin{Itemize}
    \item<2->
      $\Sigma' = \Big\{\auf f,a,i\zu ~\Big|~ f \in F,~ a \in \Sigma,~ i \in \{0,1\}\Big\}$
      \par\smallskip
    \item<3->
      $Q,c$ wie in \Aut{A}\quad {\footnotesize (wollen Akzeptanz von \Aut{A} prüfen)}
      \par\smallskip
    \item<4->
      $\Delta' = \Big\{\Big(q,~\auf f,a,i\zu,~q_i'\Big) ~\Big|~ \auf f,a,i\zu \in \Sigma',~ i \in \{0,1\},~$\\
      \hspace*{\fill} es gibt $\delta = (q,a,q_0',q_1') \in \Delta$ mit $f(\delta)=i\Big\}$
  \end{Itemize}

  \par\medskip
  \uncover<5->{%
    $\Aut{A}'$ prüft für \emph{jeden möglichen} Zug von \AUT,
    ob \AUT gewinnen kann
  }

  \par\medskip
  \uncover<6->{%
    \begin{lemma}
      $s$ ist ein \PF-Gewinnbaum für $t$
      ~$\Leftrightarrow$~
      $\Lst \cap L_\omega(\Aut{A}') = \emptyset$
      \label{lem:Gewinnbaum_mittels_Wortautomat}%
    \end{lemma}
  }

  \par\smallskip
  \uncover<7->{%
    \Bmph{Beweis:} s.\ Tafel \Tafel~~~
    \par\vspace*{-.95\baselineskip}\qed
  }

  \note{
    \textbf{16:58 $\to$ mit Beweis bis 17:30}
        
    \parII
    \textbf{Achtung:} $f \in F$ bedeutet "`Funktion aus der Menge aller Funktionen (Gewinnstrategien)"',
    nicht "`akzeptierender Zustand"'!
    
    \par
  }
\end{frame}


{
%   \setbeamertemplate{background}{{\Huge\Rewind\Rewind}}%
\usebackgroundtemplate{{\Huge\Rewind\Rewind}}%
%   \def\insertframenumber{\@arabic\c@framenumber}%
\def\insertframenumber{\relax}%
%   \AtBeginFrame{\addtocounter{framenumber}{-1}}%
%   \usebackgroundtemplate{{\Huge\Rewind\Rewind}}%
% ------------------------------------------------------------------------------------------
  \begin{frame}<handout:0>
    \frametitle{Komplementierung:~ Was bisher geschah \hfill \Rewind\Rewind}
    \addtocounter{framenumber}{-1}%

    \begin{Itemize}
      \item
        Gegeben: \Bmph{NPBA $\Autb{A}$} $= (Q,\Sigma,\Delta,\{q_I\},c)$
      \item
        Ordnen \Aut{A} und jedem \Bmph{Eingabebaum $t$} ein 2-Pers.-Spiel \Bmph{\Gameb{A}{t}} zu
      \item
        Spielerin \AUT wählt Transition für aktuelle Position in $t$;\\
        \PF wählt Kindsposition\quad ($\leadsto$ gibt schrittweise Pfad vor)
      \item
        \AUT gewinnt, wenn gespielter Pfad $c$ entspricht
    \end{Itemize}

    \begin{block}{Lemma~\ref{lem:akzeptanz_vs_gewinnstrategien}}
      $t \in L_\omega(\Aut{A})$ ~$\Leftrightarrow$~ \AUT hat Gewinnstrategie in $\Game{A}{t}$ ab Position $(\varepsilon,q_I)$%
    \end{block}

    \par\bigskip
    Mittels Resultat aus der Spieltheorie folgt:

    \begin{block}{Folgerung~\ref{cor:gedaechtnislose_GS_fuer_PF}}
      $t \in \overline{L_\omega(\Aut{A})}$ ~$\Leftrightarrow$~ \PF hat \Emph{gedächtnislose} GS ab $(\varepsilon,q_I)$ in $\Game{A}{t}$
    \end{block}

    \note{%
      \textbf{8:30}
      
      \par
    }
  \end{frame}

% ------------------------------------------------------------------------------------------
  \begin{frame}<handout:0>
    \frametitle{Komplementierung:~ Was bisher geschah \hfill \Rewind\Rewind}
    \addtocounter{framenumber}{-1}%

    \begin{Itemize}
      \item
        Betrachten gedächtnislose Gewinnstrategie für \PF als Menge $F$ von Funktionen
        \[
          f_p : \Delta \to \{0,1\}\qquad \text{für jede Baumposition~} p \in \{0,1\}^*
        \]
      \item
        Ordnen die $f_p$ in \Bmph{$F$-Baum $s$} an %(Strategiebaum)
        
        
%        \hfill ($F$ $\hat=$ alle $f_p$)
      \item
        \Bmph{\PF-Gewinnbaum:} $F$-Baum für eine Gewinnstrategie von \PF
    \end{Itemize}

    \par\bigskip
    Dann folgt sofort:

    \begin{block}{Folgerung~\ref{cor:Gewinnbaum_fuer_PF}}
      $t \in \overline{L_\omega(\Aut{A})}$ ~$\Leftrightarrow$~ es gibt einen \PF-Gewinnbaum für $t$%
    \end{block}

    \note{%
      \textbf{8:32}
      
      \parII
      $F$-Baum auch "`Strategiebaum"'; \\
      $F$ $\hat=$ alle $f_p$

      \par
    }
  \end{frame}

% ------------------------------------------------------------------------------------------
  \begin{frame}<handout:0>
    \frametitle{Komplementierung:~ Was bisher geschah \hfill \Rewind\Rewind}
    \addtocounter{framenumber}{-1}%

    Konstruieren NPA $\Aut{A}'$ ($\omega$-Wortautomat!), um \PF-Gewinnbäume zu erkennen:
    \begin{Itemize}
      \item
        Eingabewörter haben die Form
        \[
          \Big\auf s(\varepsilon), t(\varepsilon), \pi_1\Big\zu~~
          \Big\auf s(\pi_1), t(\pi_1), \pi_2\Big\zu~~
          \Big\auf s(\pi_1\pi_2), t(\pi_1\pi_2), \pi_3\Big\zu~~
          \dots
        \]
      \item
        Sei \Bmph{\Lst} die Menge aller solcher Wörter
    \end{Itemize}

    \par\bigskip
    Konstruktion von $\Aut{A}'$ stellt sicher:
%    \begin{center}
%      $s$ ist ein Gewinnbaum für $t$
%      ~$\Leftrightarrow$~
%      $\Lst \cap L_\omega(\Aut{A}') = \emptyset$
%    \end{center}
    \begin{block}{Lemma~\ref{lem:Gewinnbaum_mittels_Wortautomat}}
      $s$ ist ein \PF-Gewinnbaum für $t$
      ~$\Leftrightarrow$~
      $\Lst \cap L_\omega(\Aut{A}') = \emptyset$
    \end{block}
    
    \note{%
      \textbf{8:34}
      
      \par
    }
  \end{frame}
}



% ------------------------------------------------------------------------------------------
\begin{frame}
  \frametitle{Konstruktion des Komplementautomaten für \Aut{A}}

%       \begin{alertblock}{Gesucht {\small (siehe Folgerung \ref{cor:Gewinnbaum_fuer_PF})}}
%         NPBA $\Aut{B}$, der $t$ akzeptiert gdw.\ es einen Gewinnbaum für $t$ gibt 
%       \end{alertblock}
  \begin{alertblock}{}
    \Emph{Gesucht:}~ {\small (siehe Folgerung \ref{cor:Gewinnbaum_fuer_PF})}
    \par\smallskip
    NPBA $\Aut{B}$, der $t$ akzeptiert gdw.\ es einen \PF-Gewinnbaum für $t$ gibt 
  \end{alertblock}

  \par\smallskip
  \uncover<2->{%
    Wegen Lemma~\ref{lem:Gewinnbaum_mittels_Wortautomat} muss \Aut{B} akzeptieren ~gdw.~
    $
      \Lst \subseteq \overline{L_\omega(\Aut{A}')}
    $
  }

  \par\bigskip\smallskip
  \uncover<3->{%
    Konstruktion von \Aut{B} in 2 Schritten:
    
    \par\smallskip
    \Bmph{Schritt 1}
    \begin{Itemize}
      \item
        \mbox{Sei $\Aut{A}'' = (Q'', \Sigma', \Delta'', q_I'', c'')$ der \Emph{D}PA mit $L_\omega(\Aut{A}'') = \overline{L_\omega(\Aut{A}')}$\hspace*{-10mm}}
      \item
        $\Aut{A}''$ ist \Emph{deterministisch:}~ Safra-Konstruktion\\
        ($+$ Umwandlung zwischen den Automatentypen)
    \end{Itemize}
  }

  \par\medskip
  \uncover<4->{%
    \Bmph{Schritt 2}
    \par\smallskip
    \Aut{B} soll auf jedem Pfad von $t$
    \begin{Itemize}
      \item
        $\Aut{A}''$ laufen lassen
      \item
        und "`parallel"' dazu eine Strategie für \PF raten
    \end{Itemize}
  }

  \note{%
    \textbf{8:36 bis 8:40}
      
    \parII
    "`$\Lst \subseteq \overline{L_\omega(\Aut{A}')}$"':~
    ist äquivalent zu "`$\Lst \cap L_\omega(\Aut{A}') = \emptyset"'$ aus L.~\ref{lem:Gewinnbaum_mittels_Wortautomat}
    
    \parIII
    \textbf{Zu Schritt 1:} \\
    NPA $\to$ NMA $\to^*$ NBA ---Safra$\to^*$ DRA $\to$ DMA $\to^*$ DPA \\
    ${}^*$Resultate aus Vorlesung, exp. Blowup \\
    Es gibt effizientere direkte Konstruktion NPA $\to$ DPA (sage am Ende was dazu).
    
    \parIII
    \textbf{Zu Schritt 2:} \\
    $\Aut{A}''$ braucht ja als Eingabe Tripel, 1.\ Komponente $s(\cdot)$ aus Strategie. \\
    Diese Info wird schrittweise geraten.
    
    \par
  }
\end{frame}

% ------------------------------------------------------------------------------------------
\begin{frame}
  \frametitle{Konstruktion des Komplementautomaten für \Aut{A}}
  
  \Bmph{Idee:}~
  NPBA $\Bmc$ soll $\Amc''$ auf \Emph{jedem} Pfad simulieren, indem $\Bmc$
  %
  \begin{itemize}
    \item
      $s$ rät\quad (also pro Position $p$ ein $f_p$)
    \item
      sich ansonsten wie $\Amc''$ verhält, \\
      (also pro Position die Folgezustände $q_0,q_1$ gemäß $\Delta''$ setzt)
  \end{itemize}
  %
  $\Amc''$ deterministisch $\Rightarrow$ Zustand pro Position $p$ eindeutig bestimmt
  
  \parIII
  \uncover<2->{%
    \Bmph{Konstruktion von $\Autb{B} = (Q'', \Sigma, \Delta^{\text{neu}}, q_I'', c'')$:}
    \begin{Itemize}
      \item
        $Q'',q_I'',c''$ werden von $\Aut{A}''$ übernommen
      \item
        $\Delta^{\text{neu}} = \Big\{ (q,a,q_0,q_1) ~\Big|~ \text{es gibt } f \in F \text{ mit }$\\
        \hspace*{\fill}$\Big(q, \auf f,a,i\zu, q_i\Big) \in \Delta'' \text{ für } i=0,1\Big\}$
    \end{Itemize}
  }

  \par\vspace*{-.2\baselineskip}
  \uncover<3->{%
%    \Bmph{Es bleibt zu zeigen:}

%    \par\smallskip
    \begin{lemma}
      $t \in L_\omega(\Aut{B})$ ~gdw.~ es gibt $F$-Baum $s$ mit $\Lst \subseteq L_\omega(\Amc'')$
%      $L_\omega(\Aut{B}) = \overline{L_\omega(\Aut{A})}$
      \label{lem:Korrektheit_Komplementautomat}%
    \end{lemma}
  }

%  \parI
  %\vspace*{-\baselineskip}
  \uncover<4->{%
    \Bmph{Beweis:}~ siehe Tafel \Tafel~~~~
    \par\vspace*{-.95\baselineskip}\qed
  }
  
  \note{%
    \textbf{8:40 $\to$ bis 9:10}
    
    \parI
    DPA $\Amc''$ "`bezeugt"', dass $s$ \emph{kein} \PF-Gewinnbaum für $t$ ist
    
    \parII
    "`$\Aut{A}''$ deterministisch \dots"':~
    obwohl $\infty$ viele Pfade durch $p$ gehen!

    \parIII
    \uncover<2->{%
      $f \in F$:~
      Für \emph{jede mögliche} Funktion $f : \Delta \to \{0,1\}$~ ("`\dots\ $s$ rät"'!) \\
      und zugehörigen Übergang in $\Delta''$, ein neuer Übergang%
    }
    
    \parIII
    \uncover<3->{%
      Lemma ankündigen mit: "`es bleibt zu zeigen \dots"'%
    }
    
    \par
  }
\end{frame}

% ------------------------------------------------------------------------------------------
\begin{frame}
%       \frametitle{\dots}
  \begin{center}
    \begin{Large}
      \Gmph{\dots\ Es darf aufgeatmet werden \dots} \quad \dgre{\smiley}
    \end{Large}
  \end{center}
    
  \note{%
    \textbf{9:10}
    
    \parII
    Jetzt haben wir alles Technische geschafft und können das Hauptresultat sehr einfach beweisen.
    
    \parII
    Danach haben wir uns eine Pause verdient. :)
    
    \par
  }
\end{frame}

% ------------------------------------------------------------------------------------------
\begin{frame}
  \frametitle{Das Resultat}

  \begin{Satz}[Rabin 1969]
    Für jeden NPBA \Aut{A} gibt es einen NPBA \Aut{B}
    mit $L_\omega(\Aut{B}) = \overline{L_\omega(\Aut{A})}$.
  \end{Satz}

  \par\smallskip
  \uncover<2->{%
    \Bmph{Beweis:}
    \par\smallskip
    Für den bisher konstruierten NPBA \Aut{B} gilt:
%    Direkte Konsequenz aus Folg.~\ref{cor:Gewinnbaum_fuer_PF} und Lemmas~\ref{lem:Gewinnbaum_mittels_Wortautomat}, \ref{lem:Korrektheit_Komplementautomat} \qed
    \par\vspace*{-1.4\baselineskip}
    \begin{alignat*}{2}
      \Emph{$t \in L_\omega(\Autb{B})$} & ~~\text{gdw.}~~ \exists s \,.\, \Lst \subseteq L_\omega(\Amc'')           &\qquad  & \text{(Lemma~\ref{lem:Korrektheit_Komplementautomat})}\\
                                        & ~~\text{gdw.}~~ \exists s \,.\, \Lst \subseteq \overline{L_\omega(\Amc')} &        & \text{(Konstr.~$\Aut{A}''$)} \\
                                        & ~~\text{gdw.}~~ \exists s \,.\, \Lst \cap L_\omega(\Amc') = \emptyset     &        & \text{(Mengenlehre)} \\
                                        & ~~\text{gdw.}~~ \exists\, \text{\PF-Gewinnbaum $s$ für $t$}               &        & \text{(Lemma~\ref{lem:Gewinnbaum_mittels_Wortautomat})} \\
                                        & ~~\text{gdw.}~~ \Emph{$t \in \overline{L_\omega(\Autb{A})}$}              &        & \text{(Folg.~\ref{cor:Gewinnbaum_fuer_PF})}
    \end{alignat*}
    \qed
  }

  \note{%
    \textbf{9:10 bis 9:13}
    
    \parII
    Jetzt müssen wir nur noch alle Lemmas zusammentun und erhalten das Hauptresultat.
    
    \parII
    "`Konstr.~$\Aut{A}''$"':~ ist Komplementärautomat zu $\Aut{A}'$
    
    \parII
    \textbf{TODO:}~ Wenn Zeit ist, Beispiel für die Konstruktion vorführen \\
    (siehe Notizen in Sammlung nach T4.15, zweites$=$einfacheres Bsp.)
    
    \par
  }
\end{frame}

% ------------------------------------------------------------------------------------------
\begin{frame}
  \frametitle{Bemerkungen zur Komplexität der Konstruktion}

    Sei $n=|Q|$\quad (Anzahl der Zustände des NBPA \Aut{A}).
    
    \parI
    Dann hat der NPA $\Aut{A}'$ dieselben $n$ Zustände.
    
    \parI
    DPA $\Aut{A}''$ kann so konstruiert werden, dass $|Q''| \in O(2^{n \log n})$.
    
    \parI
    $\leadsto$ NBPA $\Aut{B}$ hat $O(2^{n \log n})$ Zustände.

  \note{%
    \textbf{9:13; auf Literatur verweisen, dann fertig? Oder noch Alternierung beginnen?}
    
    \parII
    Punkt 3 folgt \textbf{nicht} aus den Resultaten dieser Vorlesung.
    
    \parI
    Naives Vorgehen: \\
    NPA $\to$ NMA $\to^*$ NBA ---Safra$\to^*$ DRA $\to$ DMA $\to^*$ DPA \\
    ${}^*$Resultate aus Vorlesung, exp. Blowup \\
    Es gibt offenbar eine effizientere direkte Konstruktion NPA $\to$ DPA
    
    
    \par
  }
\end{frame}

%   % ------------------------------------------------------------------------------------------
%     \begin{frame}
%       \frametitle{\dots}
%       \dots
%       \note{~}
%     \end{frame}
% 
