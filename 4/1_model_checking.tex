% ------------------------------------------------------------------------------------------
\begin{frame}
  \frametitle{Erinnerung an LTL\hfill {\normalsize\emph{(Linear Temporal Logic)}}}

  \begin{Itemize}
    \item%<+->
      System gegeben als Kripke-Struktur $\calS = (S,S_0,R,\ell)$
      \par\smallskip
    \item%<+->
      LTL-Formel $\varphi_E$ beschreibt Pfade,
      die Eigenschaft $E$ erfüllen
      \par\smallskip
%         \item<+->
%           benutzt dafür Operatoren $F,G,X,U$ \hfill{\footnotesize (auch üblich: $\Diamond,\Box,\bigcirc,U$)}
%           \par\smallskip
    \item%<+->
      Beispiel:\\
      "`Wenn Fehler auftritt, ist er nach endlicher Zeit behoben."'
      \par\smallskip
      $G(e \to F \neg e)$ \hfill {\footnotesize ($e \in \PROP$ steht für "`Error"')}
      \par\smallskip
    \item%<+->
      Umwandlung $\varphi_E$ in GNBA $\Aut{A}_E$,
      der zulässige Pfade beschreibt
      \par\smallskip
    \item%<+->
      lösen damit Model-Checking-Problem:
      \begin{Itemize}
        \item
          Gilt $E$ für \emph{alle} Pfade ab $S_0$ in \calS\,?\\
          \Bmph{(universelle Variante)}
        \item
          Gilt $E$ für \emph{mindestens einen} Pfad ab $S_0$ in \calS\,?\\
          \Bmph{(existenzielle Variante)}
      \end{Itemize}
  \end{Itemize}

  \par\bigskip
%       \uncover<+->{%
    \begin{footnotesize}
      LTL 1977 eingeführt durch Amir Pnueli, 1941-2009, \\
      israelischer Informatiker (Haifa, Weizmann-Inst., Stanford, Tel Aviv, New York)
      \par
    \end{footnotesize}

%       }
  \note{
    \textbf{8:33}
    
    \par
  }
\end{frame}

% ------------------------------------------------------------------------------------------
\begin{frame}
  \frametitle{Grenzen von LTL}

  "`LTL-Formel $\varphi_E$ beschreibt Pfade, die Eigenschaft $E$ erfüllen"'

  \par\bigskip
  \Bmph{Nicht ausdrückbar:}~ zu jedem Zeitpunkt ist es immer \emph{möglich},\\
  die Berechnung auf eine gewisse Weise fortzusetzen

  \par\bigskip
%       \uncover<2->{%
    \Bmph{Beispiel:}~~
    "`Wenn ein Fehler auftritt,\\
    ist es \emph{möglich}, ihn nach endlicher Zeit zu beheben."'
    \par\smallskip
    $G(e \to F \neg e)$ ~oder~ $GF\neg e$ ~sind
    \begin{Itemize}
      \item
        zu stark in Verbindung mit universellem Model-Checking \Tafel
      \item
        zu schwach in Verbindung mit existenziellem MC \TafelForts
    \end{Itemize}
%       }

  \note{
    \textbf{8:34 bis 8:46}
    
    \par
  }
\end{frame}

% ------------------------------------------------------------------------------------------
\begin{frame}
  \frametitle{Ein Fall für CTL\hfill {\normalsize\emph{(Computation Tree Logic)}}}

  \Bmph{Abhilfe:}~ Betrachten Berechnungsbäume statt Pfaden

  \par\bigskip
  Sei also $\calS = (S,S_0,R,\ell)$ eine Kripke-Struktur

  \par\bigskip
  \Bmph{Berechnungsbaum} für $s_0 \in S_0$
  \begin{Itemize}
    \item
      entsteht durch "`Auf{}falten"' von \calS in $s_0$
    \item
      enthält \emph{alle unendlichen} Pfade, die in $s_0$ starten
      \begin{Itemize}
        \item[] d.\,h.:
          jeder Zustand $s \in S$ hat als Kinder\\
          alle seine Nachfolgerzustände aus \calS
      \end{Itemize}
  \end{Itemize}

  \par\bigskip
  $\calS$ ist eine endliche Repräsentation aller $\infty$ Berechnungsbäume

%  \par\bigskip
%  \Gmph{Beispiel:} siehe nächste Folie \& Tafel \Tafel
%
  \note{
    \textbf{8:46}
    
    \par
  }
\end{frame}

% ------------------------------------------------------------------------------------------
\begin{frame}
  \label{fra:bsp_mikrowelle}
  \frametitle{Beispielstruktur Mikrowelle}

  \begin{center}
    \begin{minipage}[b]{.7\textwidth}
      \includegraphics[angle=270,width=\linewidth]{img/mc_microwave.jpg}\\
      {\footnotesize aus: E.\,M.\ Clarke et al., Model Checking, MIT Press 1999}
    \end{minipage}
  
    \par\vspace*{-\baselineskip}
    \Tafel
  \end{center}

  \note{
    \textbf{8:47 bis 8:54}
    
    \par
  }
\end{frame}

% ------------------------------------------------------------------------------------------
\begin{frame}
  \frametitle{CTL intuitiv}

  CTL enthält \Bmph{Pfadquantoren $A$, $E$:}
  \par\smallskip
  Operatoren, die über \Bmph{alle} oder \Bmph{einige} Berechnungen sprechen, \\
  die in einem bestimmten Zustand beginnen

  \par\bigskip
  \uncover<2->{%
    \Bmph{Beispiel:}~ $AGEF\neg e$
    \par\smallskip
    \qquad Für alle Berechnungen, die hier starten ($A$), \\
    \qquad gibt es zu jedem Zeitpunkt in der Zukunft ($G$) \\
    \qquad eine Möglichkeit, die Berechnung fortzusetzen ($E$), \\
    \qquad so dass irgendwann in der Zukunft ($F$) \\
    \qquad kein Fehler auftritt ($\neg e$)
  }

  \par\bigskip
  \uncover<3->{%
    \begin{footnotesize}
      CTL 1981 eingeführt durch\\
      \quad Edmund M.\ Clarke, ${}^*$1945, Informatiker, Carnegie Mellon Univ. (Pittsburgh)\\
      \quad E.\ Allen Emerson, ${}^*$1954, Informatiker, Univ.\ of Texas, Austin, USA\\
      (beide Turing-Award-Träger 2007)
      \par
    \end{footnotesize}
  }

  \note{
    \textbf{8:54}
    
    \par
  }
\end{frame}

% ------------------------------------------------------------------------------------------
\begin{frame}
  \frametitle{CTL exakt}

  Trennung von Zustands- und Pfadformeln:

  \par\bigskip
  \Bmph{Zustandsformeln} drücken Eigenschaften eines Zustandes aus
  \[
    \zeta ::= p \mid \zeta_1 \land \zeta_2 \mid \zeta_1 \lor \zeta_2 \mid \neg \zeta \mid E\psi \mid A\psi
  \]
  \begin{small}
    \qquad
    ($p$: Aussagenvariable;\hfill
    $\zeta,\zeta_1,\zeta_2$: Zustandsformeln;\hfill
    $\psi$: Pfadformel)
    \par
  \end{small}

  \par\bigskip
  \uncover<2->{%
    \Bmph{Pfadformeln} drücken Eigenschaften eines Pfades aus
    \[
      \psi ::= F\zeta \mid G\zeta \mid X\zeta \mid \zeta_1 \mathbin{U} \zeta_2
    \]
    \begin{small}
      \qquad
      \hfill
      ($\zeta,\zeta_1,\zeta_2$: Zustandsformeln)
      \par
    \end{small}
  }

  \par\bigskip
  \uncover<3->{%
    $\leadsto$ in \Bmph{zulässigen} CTL-Formeln muss
    \begin{Itemize}
      \item
        \mbox{\scalebox{.96}[1]{jeder Pfadquantor von einem temporalen Operator gefolgt werden}\hspace*{-10mm}}
      \item
        jeder temporale Operator direkt einem Pfadquantor folgen
    \end{Itemize}

  }

  \note{
    \textbf{8:56}
    
    \par
  }
\end{frame}

% ------------------------------------------------------------------------------------------
\begin{frame}
  \frametitle{Quiz: zulässige Formeln}

  Zur Erinnerung:
  \begin{align*}
    (ZF)\quad \zeta & ::= p \mid \zeta_1 \land \zeta_2 \mid \zeta_1 \lor \zeta_2 \mid \neg \zeta \mid E\psi \mid A\psi \\
    (PF)\quad \psi  & ::= F\zeta \mid G\zeta \mid X\zeta \mid \zeta_1 \mathbin{U} \zeta_2
  \end{align*}

  \par\bigskip
  \Emph{Quizfrage:} Welche der folgenden Formeln sind zulässig?
%       \begin{Itemize}
%         \item<+->
%           $p \land q$
% %           \uncover<+->{\YES}
% %         \item<+->
%           \quad
%           $EFp$
% %           \uncover<+->{\YES}
% %         \item<+->
%           \quad
%           $AXp$
%           \quad
%           \uncover<+->{\YES}
%         \item<+->
%           $E(p \mathbin{U} q)$
%           \quad
%           \uncover<+->{\YES}
%         \item<+->
%           $A((p \lor \neg p) \mathbin{U} q)$
%           \quad
% %           \uncover<+->{\YES}
% %           \quad
% %           \uncover<+->{(äquivalent zu $AFq$)}
%           \uncover<+->{%
%             \YES
%             \quad
%             (äquivalent zu $AFq$)
%           }
%         \item<+->
%           $E(p \lor AXq)$
%           \quad
%           \uncover<+->{\NO}
%         \item<+->
%           $EX(p \lor AXq)$
%           \quad
%           \uncover<+->{\YES}
%         \item<+->
%           $EF(p \mathbin{U} q)$
%           \quad
%           \uncover<+->{\NO}
%         \item<+->
%           $EFA(p \mathbin{U} q)$
%           \quad
%           \uncover<+->{\YES}
%       \end{Itemize}
  \begin{center}
    \begin{tabular}{@{}lll@{}}
      \uncover<+->{$p \land q$ \quad $EFp$ \quad $AXp$} & \uncover<+->{\YES} &                                    \\[2pt]
      \uncover<1->{$E(p \mathbin{U} q)$}                & \uncover<+->{\YES} &                                    \\[2pt]
      \uncover<1->{$A((p \lor \neg p) \mathbin{U} q)$}  & \uncover<+->{\YES  & (äquivalent zu $AFq$)}             \\[2pt]
      \uncover<1->{$E(p \lor AXq)$}                     & \uncover<+->{\NO   & ($E$ nicht gefolgt von $F,G,X,U$)} \\[2pt]
      \uncover<1->{$EX(p \lor AXq)$}                    & \uncover<+->{\YES} &                                    \\[2pt]
      \uncover<1->{$EF(p \mathbin{U} q)$}               & \uncover<+->{\NO   & ($U$ folgt nicht $E$ oder $A$)}    \\[2pt]
      \uncover<1->{$EFA(p \mathbin{U} q)$}              & \uncover<+->{\YES} &
    \end{tabular}
  \end{center}

  \note{
    \textbf{9:00}
    
    \textbf{Als Aufgabe, 2\,min Nachdenken, 2\,min Auf"|lösung $\leadsto$ bis 9:04}
    
    \par
  }
\end{frame}

\newlength{\leftbox}
\settowidth{\leftbox}{$\Paths(s):$}
% ------------------------------------------------------------------------------------------
\begin{frame}
  \frametitle{CTL-Semantik}

  CTL-Formeln werden über \Emph{Zuständen und Pfaden} von Kripke-Strukturen
  $\calS = (S,S_0,R,\ell)$ interpretiert

  \par\bigskip
  \Bmph{Schreibweisen}
  \begin{Itemize}
    \item
      $s \models \zeta$\quad für Zustände $s \in S$ und Zustandsformeln $\zeta$
    \item
      $\pi \models \psi$\quad für Pfade $\pi$ und Pfadformeln $\psi$
  \end{Itemize}

  \par\bigskip
  \Bmph{Hilfsbegriffe}
  \begin{Itemize}
    \item
      \parbox{\leftbox}{$\Paths(s)$:}\quad Menge aller Pfade, die in Zustand $s$ beginnen
    \item
      \parbox{\leftbox}{$\pi[i]$:}\quad $i$-ter Zustand auf dem Pfad $\pi$\\
      \parbox{\leftbox}{~}\quad         d.\,h.\ wenn $\pi=s_0s_1s_2\dots$, dann $\pi[i] = s_i$
  \end{Itemize}

  \note{
    \textbf{9:04}

    \par
  }
\end{frame}

% ------------------------------------------------------------------------------------------
\begin{frame}
  \frametitle{CTL-Semantik}

  Sei $\calS = (S,S_0,R,\ell)$ eine Kripke-Struktur.

  \begin{Definition}
    \Bmph{Erfülltheit von Zustandsformeln} in Zuständen $s \in S$
    \par\medskip
    \begin{tabular}{@{\quad}lll@{}}
        $s \models p$                     & falls & $p \in \ell(s)$, für alle $p \in \PROP$             \\[2pt]
        $s \models \lnot\zeta$            & falls & $s \not\models \zeta$                               \\[2pt]
        $s \models \zeta_1 \land \zeta_2$ & falls & $s \models \zeta_1$ und $s \models \zeta_2$
                                                    \quad~\, {\small (analog für $\zeta_1 \lor \zeta_2$)} \\[2pt]
        $s \models E\psi$                 & falls & $\pi \models \psi$ für ein $\pi \in \Paths(s)$      \\[2pt]
        $s \models A\psi$                 & falls & $\pi \models \psi$ für alle $\pi \in \Paths(s)$
    \end{tabular}

    \par\medskip
    \uncover<2->{%
      \Bmph{Erfülltheit von Pfadformeln} in Pfaden $\pi$ in $\calS$
      \par\medskip
      \begin{tabular}{@{\quad}lll@{}}
          $\pi \models X\zeta$              & falls & $\pi[1] \models \zeta$
                                                    \qquad\qquad\, {\small (analog für $F\zeta$ und $G\zeta$)}                  \\[2pt]
          $\pi \models \zeta_1\,U\,\zeta_2$ & falls & $\pi[\;\!j] \models \zeta_2$ für ein $j \geqslant 0$                  \\
                                            &       & und $\pi[k] \models \zeta_1$ für alle $k$ mit $0 \leqslant k < j$
      \end{tabular}
    }
  \end{Definition}

  \par\smallskip
  \uncover<3->{%
%         (Die Fälle $\zeta_1\lor\zeta_2$, $F\zeta$, $G\zeta$ sind analog.)
    Schreiben\quad $\calS \models \zeta$\quad falls\quad $s_0 \models \zeta$ für alle $s_0 \in S_0$
  }

  \note{
    \textbf{9:06}
    
    \par
  }
\end{frame}

% ------------------------------------------------------------------------------------------
\begin{frame}
  \frametitle{Zurück zu unseren Beispielen: Spezifikationen in CTL}

  \Bmph{Beispiel Nebenläufigkeit}
  \begin{Itemize}
    \item
%           Es kommt nie vor,\\
      Beide Teilprogramme sind nie zugleich im kritischen Bereich.
      \par\smallskip
      $AG\neg (p_{12} \land p_{22})$ \hfill {\footnotesize ($p_i \in \PROP$: "`Programmzähler in Zeile $i$"')}
      \par\smallskip
    \item<2->
      Jedes Teilprog.\ kommt beliebig oft in seinen krit.\ Bereich.
      \par\smallskip
      $AGAF p_{12} \land AGAF p_{22}$
      \par\smallskip
    \item<3->
      Jedes Teilprog.\ \emph{kann} beliebig oft in seinen kB kommen.
      \par\smallskip
      $AGEF p_{12} \land AGEF p_{22}$
  \end{Itemize}

  \par\bigskip
  \uncover<4->{%
    \Bmph{Liveness properties:}
    \par\smallskip
    $AG\zeta$ besagt: "`$\zeta$ ist in allen Berechnungen immer wahr"'
    \par\smallskip
    $AGAF\zeta$ besagt: "`$\zeta$ ist in allen Berechnungen $\infty$ oft wahr"'
    \par\smallskip
    $AGEF\zeta$ besagt: "`jede begonnene Berechnung kann so fortge-\\
    \hspace*{28mm}setzt werden, dass $\zeta$ irgendwann wahr wird."'
  }

  \note{
    \textbf{9:10}
    
    \par
  }
\end{frame}

% ------------------------------------------------------------------------------------------
\begin{frame}
  \frametitle{Zurück zu unseren Beispielen: Spezifikationen in CTL}

  \Bmph{Beispiel Mikrowelle}
  \begin{Itemize}
    \item
      "`Wenn Fehler auftritt, ist er nach endlicher Zeit behoben."'
      \par\smallskip
      $AG(e \to AF \neg e)$ \hfill {\footnotesize ($e \in \PROP$ steht für "`Error"')}
      \par\smallskip
    \item<2->
      "`Wenn Fehler auftritt, \emph{kann} er nach endl.\ Z.\ behoben werden"'
      \par\smallskip
      $AG(e \to EF \neg e)$
      \par\smallskip
    \item<3->
      "`Wenn die Mikrowelle gestartet wird, \\
      beginnt sie nach endlicher Zeit zu heizen."'
      \par\smallskip
      $AG(s \to AF h)$ \hfill {\footnotesize ($s,h \in \PROP$ stehen für "`Start"' bzw.\ "`Heat"')}
      \par\smallskip
    \item<4->
      "`Wenn die Mikrowelle gestartet wird, \\
      \emph{ist es möglich}, dass sie nach endlicher Zeit zu heizen beginnt."'
      \par\smallskip
      $AG(s \to EF h)$
      \par\bigskip
%         \emphitem<5->
%           $AG(\zeta_1 \to AF \zeta_2)$,~
%           $AG(\zeta_1 \to EF \zeta_2)$:~
%           \Bmph{progress properties}
  \end{Itemize}

  \par\medskip
  \uncover<5->{%
    \Bmph{Progress properties:}~
    $AG(\zeta_1 \to AF \zeta_2)$,~
    $AG(\zeta_1 \to EF \zeta_2)$~ bedeuten: \\
    \par\smallskip
    Wann immer $\zeta_1$ eintritt, ist nach endlicher Zeit $\zeta_2$ "`garantiert"'
  }

  \note{
    \textbf{9:12}
    
    \par
  }
\end{frame}

% ------------------------------------------------------------------------------------------
\begin{frame}
  \frametitle{Ausdrucksstärke von CTL versus LTL}
  
  \begin{definition}
    Seien $\zeta$ eine CTL-Zustandsformel und $\varphi$ eine LTL-Formel.
    
    \par\smallskip
    $\zeta$ und $\varphi$ sind \Bmph{äquivalent,} geschrieben \Bmph{$\zeta \equiv \varphi$,}
    wenn für alle Kripke-Strukturen $\Smc = (S,S_0,R,\ell)$ gilt:
    \begin{center}
      $\Smc \models \zeta$
      \quad gdw.\quad
      $\Smc \models \varphi$
    \end{center}
  \end{definition}

  \par\medskip
  \Bmph{Zur Erinnerung:}
  %
  \begin{Itemize}
    \item
      $\Smc \models \zeta$, wenn $s_0 \models \zeta$ für \Emph{alle} $s_0 \in S_0$
    \item
      $\Smc \models \varphi$, wenn $\pi,0 \models \varphi$ für \Emph{alle} $\pi \in \Paths(s_0)$ und \Emph{alle} $s_0 \in S_0$
  \end{Itemize}
  
  \note{
    \textbf{9:15: 5\,min Pause, dann 2\,min für Folie}
    
    \par
  }
\end{frame}

% ------------------------------------------------------------------------------------------
\begin{frame}
  \frametitle{Ausdrucksstärke von CTL versus LTL}

  \begin{lemma}
    $AFAGp \not\equiv FGp$
    \label{lem:AFAGp_not_equiv_FGp}
  \end{lemma}

  \uncover<2->{%
    \Bmph{Beweis.}~
    Betrachte Kripke-Struktur $\calS$:
%    \raisebox{-4pt}{\Fig0}%
    \begin{tikzpicture}[%
      node distance=12mm,>=Latex,baseline=-2pt,
      initial text="", initial where=below left,
      every state/.style={draw=black,thick,fill=none,inner sep=.8mm,minimum size=5mm},
      accepting/.style={double distance=1.5pt, double=white},
      every edge/.style={draw=black,thick}
    ]
      \node[state,initial] (s0)               {$s_0$};
      \node[state]         (s1) [right of=s0] {$s_1$};
      \node[state]         (s2) [right of=s1] {$s_2$};
      
      \path[->] (s0) edge              (s1)
                (s1) edge              (s2)
                (s0) edge [loop below] ()
                (s2) edge [loop below] ();
      
      \node () [above right=-1mm and -1mm of s0] {$p$};
      \node () [above right=-1mm and -1mm of s2] {$p$};
    \end{tikzpicture}
    
    \begin{Itemize}
      \item
        alle Pfade $\pi \in \Paths(s_0)$ erfüllen $FGp$
      \item<3->
        aber $\calS \not\models AFAGp$:
        \par\medskip
        \begin{tabular}{@{\quad}ll@{\qquad}l@{}}
          \uncover<4->{              & $s_0s_1s_2^\omega \not\models Gp$ & {\small wegen $p \notin \ell(s_1)$}              \\[1pt] }
          \uncover<5->{$\Rightarrow$ & $s_0 \not\models AGp$             & {\small weil $s_0s_1s_2^\omega \in \Paths(s_0)$} \\[1pt] }
          \uncover<6->{$\Rightarrow$ & $s_0^\omega \not\models FAGp$     & {\small weil $s_0^\omega$ nur aus $s_0$ besteht} \\[1pt] }
          \uncover<7->{$\Rightarrow$ & $s_0 \not\models AFAGp$           & {\small weil $s_0^\omega \in \Paths(s_0)$}               }
        \end{tabular}
    \end{Itemize}
    \uncover<8->{%
      \qed
    }
  }

  \note{
    \textbf{9:22}
    
    \par
  }
\end{frame}

% ------------------------------------------------------------------------------------------
\begin{frame}
  \frametitle{Ausdrucksstärke von CTL versus LTL}
  
  \begin{lemma}
    Sei $\zeta$ eine CTL-Zustandsformel und $\zeta'$ die LTL-Formel,
    die man durch Entfernen aller Pfadquantoren aus $\zeta$ erhält.
    Dann gilt:
    %
    \begin{center}
      $\zeta \equiv \zeta'$ oder es gibt keine zu $\zeta$ äquivalente LTL-Formel.
    \end{center}
    \par\vspace*{-.4\baselineskip}
    \label{lem:aequiv_LTL_Fmln}
  \end{lemma}

  \par
  \Bmph{Ohne Beweis.} (Clarke, Draghicescu 1988)
  
  \parII
  \uncover<2->{%
    \begin{lemma}
      \begin{Enumerate}
        \item[\Bmph{(1)}]
          Es gibt keine zu $AFAGp$ äquivalente LTL-Formel.
        \item[\Bmph{(2)}]
          Es gibt keine zu $FGp$ äquivalente CTL-Zustandsformel.
      \end{Enumerate}
    \end{lemma}
  }
  
  \parI
  \uncover<3->{%
    \Bmph{Beweis.}
    
    \parI
    \Bmph{(1)} folgt aus Lemmas~\ref{lem:AFAGp_not_equiv_FGp} und~\ref{lem:aequiv_LTL_Fmln}
  }

  \parI
  \uncover<4->{%
    \Bmph{(2)} siehe Tafel \Tafel~~~~~
    \par\vspace*{-\baselineskip}\qed
  }
  
  \note{
    \textbf{9:26 bis 9:38}
    
    \par
  }
\end{frame}

% ------------------------------------------------------------------------------------------
\begin{frame}
  \frametitle{Ausdrucksstärke von CTL versus LTL}
  
  Auch \Bmph{progress properties} sind \Emph{nicht} in LTL ausdrückbar:
  
  \begin{lemma}
    Sei $\zeta = AG(p \to EF p')$.
    Es gibt keine zu $\zeta$ äquivalente LTL-Formel.
  \end{lemma}

  \parI
  \uncover<2->{%
    \Bmph{Beweis.}
    Angenommen, es gebe LTL-Formel $\varphi \equiv \zeta$.
    
  }

  \parII    
  \uncover<3->{%
    Betrachte Kripke-Strukturen

    \par\vspace*{-.6\baselineskip}
    \begin{center}    
      $\Smc_1$
      \begin{tikzpicture}[%
        node distance=12mm,>=Latex,baseline=-2pt,
        initial text="", initial where=below left,
        every state/.style={draw=black,thick,fill=none,inner sep=.8mm,minimum size=5mm},
        accepting/.style={double distance=1.5pt, double=white},
        every edge/.style={draw=black,thick}
      ]
        \node[state,initial] (s0)               {$s_0$};
        \node[state]         (s1) [right of=s0] {$s_1$};
        
        \path[->] (s0) edge              (s1)
                  (s0) edge [loop below] ()
                  (s1) edge [loop below] ();
                  
        \node () [above right=-1mm and -1mm of s0] {$p$};
        \node () [above right=-1mm and -1mm of s1] {$p'$};
      \end{tikzpicture}
      \qquad
      $\Smc_2$
      \begin{tikzpicture}[%
        node distance=12mm,>=Latex,baseline=-2pt,
        initial text="", initial where=below left,
        every state/.style={draw=black,thick,fill=none,inner sep=.8mm,minimum size=5mm},
        accepting/.style={double distance=1.5pt, double=white},
        every edge/.style={draw=black,thick}
      ]
        \node[state,initial] (s2)               {$s_2$};
        
        \path[->] (s2) edge [loop below] ();
        
        \node () [above right=-1mm and -1mm of s2] {$p$};
      \end{tikzpicture}
    \end{center}
  }

  \par\vspace*{-.6\baselineskip}
  \uncover<4->{%
    Dann gilt $\Smc_1 \models \zeta$.
  }

  \parII
  \uncover<5->{%
    Also auch $\Smc_1 \models \varphi$.
  }
    
  \parII
  \uncover<6->{%
    Da $\Paths(s_2) \subseteq \Paths(s_0)$, gilt auch $\Smc_2 \models \varphi$.
  }

  \parII
  \uncover<7->{%
    Aber offensichtlich $\Smc_2 \not\models \zeta$.~ \lightning
    \qed
  }

  \note{
    \textbf{9:38}

    \par
  }
\end{frame}
  
% ------------------------------------------------------------------------------------------
\begin{frame}
  \frametitle{Ausdrucksstärke von CTL versus LTL}
  
  \par\bigskip
  \uncover<+->{%
    Erweiterung von LTL und CTL: \Bmph{CTL*}
    \par\vspace*{-2pt}
    {\footnotesize CTL*: 1986 von E.\ A.\ Emerson und J.\ Y.\ Halpern (${}^*$1953, Inform., Cornell)}
  }

  \note{
    \textbf{9:42}
    
    \par
  }
\end{frame}

% ------------------------------------------------------------------------------------------
\begin{frame}
  \frametitle{Model-Checking für CTL (Skizze)}

  \Bmph{Standard-Algorithmus} ("`bottom-up labelling"', ohne Automaten):

  \par\bigskip
  Eingabe: Kripke-Str.\ \calS, Zust.\ $s_0$, CTL-Zustandsformel $\zeta$
  \par
  Frage: $s_0 \models \zeta$\,?
  
  \par\bigskip
  Vorgehen:
  \begin{Itemize}
    \item<2->
      Stelle $\zeta$ als Baum dar\hspace*{\fill} (Bsp.\ siehe Tafel) \Tafel
    \item<3->
      Gehe Baum von unten nach oben durch\\
      und markiere Zustände $s$ in $\calS$ mit der jeweiligen Teilformel,\\
      wenn sie in $s$ erfüllt ist \TafelForts
    \item<4->
      Akzeptiere gdw.\ $s_0$ mit $\zeta$ markiert ist
  \end{Itemize}

  \par\bigskip
  \uncover<5->{%
    \Bmph{Komplexität:} \PT-vollständig \quad (LTL-MC: \PSPACE-vollständig)
    
    \parII
    \scalebox{.98}[1]{Dafür ist CTL-SAT \EXP-vollständig\quad (LTL-SAT: \PSPACE-vollst.).}
  }

  \note{
    \textbf{9:42 bis 9:53}
    
    \par
  }
\end{frame}

% ------------------------------------------------------------------------------------------
\begin{frame}
  \frametitle{Model-Checking für CTL mit Baumautomaten}

  \Bmph{Automatenbasierte Entscheidungsprozedur für CTL}
  \begin{Itemize}
    \item[\dots]
      basiert auf \Bmph{alternierenden Baumautomaten}\\[.4\baselineskip]
      \begin{small}
        (Erweiterung des Begriffs der nichtdeterminist.\ Baumautomaten; \\
        siehe Teil 5 der Vorlesung)
        \par
      \end{small}
%    \item
%      voraussichtlich in Teil 5 der Vorlesung anreißen %\\
%%           {\small (siehe V\,$+$\,Ü "`Verifikation unendlicher Systeme"', WS 13/14)}
  \end{Itemize}

  \par\bigskip
  \uncover<2->{%
    \Bmph{Verwandt:}
    \par\smallskip
    Automatenbasierte Entscheidungsprozedur für CTL*-\Emph{Erfüllbarkeit}
    \begin{Itemize}
      \item
        basiert auf nichtdeterministischen Rabin-Baumautomaten
      \item
        technisch aufwändige Konstruktion
      \item
        hier nicht behandelt %\\
    \end{Itemize}
  }

  \par\bigskip
  \uncover<3->{%
    \Bmph{Es folgt:}
    \begin{Itemize}
      \item[]
        Überblick "`klassische"' nichdeterministische Baumautomaten
    \end{Itemize}
  }

  \note{
    \textbf{9:53 bis 9:55, 5\,min Reserve.}
    
    \parIII
    CTL*-MC ist PSpace-vollst., CTL*-SAT 2ExpTime-vollst.\\
    Siehe Baier \& Katoen S.\ 430.
    
    \par
  }
\end{frame}

