% ------------------------------------------------------------------------------------------
\begin{frame}
  \frametitle{Abschließende Bemerkungen}

  \Bmph{Zusammenfassung}
  \begin{Itemize}
    \item
      NBBA sind schwächer als NMBA und NPBA
    \item
      Büchi-, Muller- und paritäts-erkennbare Sprachen sind
      abgeschlossen unter $\cup,\cap$ (leicht)
    \item
      Büchi-erkennbare Sprachen sind \emph{nicht}
      abgeschlossen unter $\overline{\phantom{\circ}}$
    \item
      Muller- und paritäts-erkennbare Sprachen \emph{sind}
      abgeschlossen unter $\overline{\phantom{\circ}}$ (anspruchsvoll, Spieltheorie)
    \item
      Komplementierung ist zentral im Beweis der Entscheidbarkeit
      der monadischen Logik zweiter Stufe\hfill {\footnotesize \Lit{(Kapitel 12 in LNCS 2500)}}
  \end{Itemize}
  \note{~}
\end{frame}

% ------------------------------------------------------------------------------------------
\begin{frame}
  \frametitle{Abschließende Bemerkungen}

  \Bmph{Ausblick}
  \begin{Itemize}
    \item
      Leerheitsproblem für NPBA ist entscheidbar;\\
      beste bekannte obere Schranke: $\text{UP} \cap \text{coUP}$ ($\subseteq \NP$)
      \par%\smallskip
      \hspace*{\fill}{\footnotesize \Lit{(Kapitel 8 in LNCS 2500)}}
    \item
      Verallgemeinerung des Nichtdeterminismus:\\
      alternierende Baumautomaten\hfill {\footnotesize \Lit{(Kapitel 9 in LNCS 2500)}}
  \end{Itemize}
  \note{~}
\end{frame}

% ------------------------------------------------------------------------------------------
\begin{frame}
  \frametitle{Das war \dots}
  \dots\ das Kapitel über Automaten auf unendlichen Bäumen.

  \par\bigskip
  \begin{center}
    \includegraphics[width=.8\textwidth]{img/pythagoras_tree_col_skewed.png}
%         *** REACTIVATE IMAGE ***
    \par
    \begin{footnotesize}
      Pythagoras-Baum. Quelle: Wikipedia, User Gjacquenot (Lizenz CC BY-SA 3.0)
      \par
    \end{footnotesize}
  \end{center}
  \note{~}
\end{frame}

% ------------------------------------------------------------------------------------------
\begin{frame}
  \frametitle{Was jetzt noch fehlt \dots}

  \begin{Itemize}
    \item
      kurze Nachbesprechung Vorlesungsevaluation
    \item
      Hinweise zu Fachgesprächen/mündl.\ Prüfung
  \end{Itemize}


  \par\bigskip
  \uncover<2->{%
    \begin{center}
      \begin{huge}
        \Emph{Vielen Dank für eure Teilnahme!}
        \par
      \end{huge}
    \end{center}
  }
  \note{~}
\end{frame}

%   % ------------------------------------------------------------------------------------------
%     \begin{frame}
%       \frametitle{\dots}
%       \dots
%       \note{~}
%     \end{frame}
%
